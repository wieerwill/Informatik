\documentclass[12pt, a4paper]{article}
\usepackage[ngerman]{babel}
\usepackage[utf8]{inputenc}
\usepackage[T1]{fontenc}
\usepackage[top=1.5cm, bottom=1.5cm, left=2cm, right=2cm, columnsep=20pt]{geometry} 
\usepackage{graphics}
\usepackage[font=small, labelfont=normalfont, textfont=up,]{caption}
\usepackage[pdfencoding=auto,unicode, bookmarks=false, colorlinks=false, pdfborder={0 0 0},pdftitle={Antrag auf Kopie}, pdfauthor={github.com/wieerwill}, pdfsubject={}, pdfkeywords={Prüfungen, Einsicht, Kopie}]{hyperref}
\usepackage{fancyhdr}
\usepackage{amsmath,amsthm,amsfonts,amssymb}
\usepackage{color,graphicx,overpic}
\usepackage[dvipsnames]{xcolor}

\title{Antrag auf Kopie der Prüfungsleistung}
\author{}
\date{}

\begin{document}
\maketitle

\begin{Form}
    \centering
    \noindent Prüfung, Semester: \TextField[name=Prüfung,width=8cm, charsize=10pt, bordercolor={0 1 1}, value={}] {\mbox{}}\\
    \vspace{0.5cm}
    \noindent Matrikelnummer: \TextField[name=Matrikelnummer,width=8.4cm, charsize=10pt, bordercolor={0 1 1}, value={}] {\mbox{}}\\
    \vspace{0.5cm}
    \noindent Datum der Einsichtnahme: \TextField[name=Matrikelnummer,width=6.6cm, charsize=10pt, bordercolor={0 1 1}, value={}] {\mbox{}}\\
    \vspace{0.5cm}
\end{Form}

\section{Grundlegend}
Die Ziele der Prüfungseinsicht sind:
\begin{itemize}
    \item Aufzeigen, welche Bereiche/Themen/Aufgaben für das Bestehen gefehlt haben. Aufzeigen dieser Wissens-/ Verständnis-Lücken auch im Hinblick auf die Wiederholungsprüfung.
    \item "Lernen aus Fehlern".
    \item Vergleich eigener Lösungsansätze mit Musterlösung.
\end{itemize}
\textit{\scriptsize(Vgl. allg. Prüfungsordnung der TU Ilmenau 2019 und Thüringer Verwaltungsverfahrensgesetz 2018)}\\
Diskussionen hinsichtlich der erfolgten Bewertung sowie umfassende Erläuterungen zum Verständnis der Musterlösung sind keine Ziele. \\
Eine Einsichtnahme findet einmalig statt und bietet durch die kurze zeitliche Beschränkung keine Möglichkeit die Ergebnisse und Fehler vollständig auszuwerten oder für eigene Verbesserungen zu nutzen. Daher wird ein Anspruch auf vollständige Kopie der Prüfungsleistung gestellt und im Anhang erläutert.

\vspace{2cm}

\subitem{
    \begin{minipage}[h]{0.9\linewidth}
        \begin{Form}
            \begin{center}
                \textbf{ Vom Prüfer/Einsichtleiter auszufüllen}\\
            \end{center}
            \vspace{0.2cm}
            \noindent \CheckBox[height=0.01cm, width=0.4cm, bordercolor={0 1 1}]{Der Kopie wird zugestimmt }\hspace{3cm}
            \noindent \CheckBox[height=0.01cm, width=0.4cm, bordercolor={0 1 1}]{Die Kopie wird abgelehnt }\\
            \vspace{0.2cm}
            \noindent \CheckBox[height=0.01cm, width=0.4cm, bordercolor={0 1 1}]{Die Kopie wurde ausgehändigt/vom Prüfling selbst angefertigt }\\
            \vspace{0.2cm}
            \noindent \CheckBox[height=0.01cm, width=0.4cm, bordercolor={0 1 1}]{Die Kopie wird dem Prüfling innerhalb einer Woche nachgereicht }\\
            \vspace{0.5cm}
            \centering\rule{12cm}{1px}\\
            \noindent Unterschrift des Prüfers/Einsichtleiters: \TextField[name=Prüfer,width=7cm, charsize=10pt, bordercolor={0 1 1}, value={}] {\mbox{}}\\
            \vspace{0.5cm}
            \noindent Unterschrift des Prüflings: \TextField[name=Prüfling,width=9.4cm, charsize=10pt, bordercolor={0 1 1}, value={}] {\mbox{}}\\
        \end{Form}
    \end{minipage}
}

\newpage

\section{datenschutzrechtliche Bedenken und Anspruch}
Besteht datenschutzrechtliche Anspruch beider Seiten (Prüfer und Prüfling)? 

Grundlage des Datenschutzes bieten die Datenschutz Grundverordnung (DSGVO) und das Bundesdatenschutzgesetz (BDSG).

\subsection{Anspruchsgrundlage}
Aus DSGVO Art. 15 Abs. 3 geht ein Anspruch auf Herausgabe der personenbezogenen Daten, zu der unter anderem die Antworten der Prüfung zählen, hervor. Dieser besteht, wenn der Anwendungsbereich der Datenschutzgrundverordnung eröffnet ist und personenbezogene Daten des Prüflings verarbeitet werden, sodass sich der Prüfling mit Betroffenenrechten auf das Recht auf Auskunft und insbesondere das Recht auf Kopie dem Grunde nach berufen kann und die geforderten Kopien auf vom Recht auf Kopie abgedeckt sind (Reichweite des Rechts auf Kopie).
Das Verwaltungsgericht Gelsenkirchen entschied in einem ähnlichen Fall, dass keine Vorschriften, das Recht auf eine kostenlose Datenkopie beschränken könnten.\footnote{VG Gelsenkirchen mit Urteil vom 27. April 2020 (Az.: 20 K 6392/18)} 

Der Personenbezug ergibt sich nicht allein aus den Namen der Prüflings oder einer Prüfungsnummer, sondern zusätzlich dadurch, dass die Antworten selbst personenbezogen sind, da sie den individuellen Kenntnisstand und das Kompetenzniveau sowie gegebenenfalls Gedankengänge, Urteilsvermögen und das kritische Denken des Prüflings widerspiegeln. 

Der sachliche Anwendungsbereich der DSGVO ist für analoge Datenverarbeitungen umstritten. Durch § 2 Abs. 1 ThürDSG ist dieser Streit aber im Geltungsbereich des ThürDSG gegenstandslos. Die DSGVO ist danach selbst bei rein analoger Verarbeitung anzuwenden. Also auch im vorliegenden Fall.

\subsection{Urheberrecht}
Eine Beeinträchtigung der Rechte und Freiheiten des Prüfers durch Verletzung seines Urheberrechts setzt voraus, dass dieser an seinen Prüferanmerkungen ein Urheberrecht besitzt. Das scheitert in der Regel schon an der erforderlichen Schöpfungshöhe. Soweit (ausnahmsweise) die erforderliche Schöpfungshöhe vorliegt, kann eine Verwertung des somit geschützten Werks aufgrund des § 5 Abs. 1 UrhG in der Variante "Entscheidung" angenommen werden. Der Prüfer kann sich somit nicht auf sein Urheberrecht berufen.

\subsection{Schutz personenbezogener Daten Dritter}
Fraglich ist, ob Prüferanmerkungen personenbezogene Daten des Fachgebietsleiters darstellen, die nicht an den Prüfling übermittelt werden dürfen.

Bei der Bearbeitung des Prüflings handelt es sich nicht um ein personenbezogenes Datum des Prüfers\footnote{\href{https://dswiki.tu-ilmenau.de/wiki/user/martin_neldner/recht_auf_kopie_bei_pruefungsleistungen}{dswiki.tu-ilmenau.de}}.
Bei den Prüferanmerkungen handelt es sich dagegen um ein personenbezogenes Datum des Prüflings durch den Prüfer im Sinne von Art. 4 Nr. 1 DSGVO, wie sich schon aus der Nowak\footnote{\href{http://curia.europa.eu/juris/document/document.jsf;jsessionid=9ea7d0f130d5caea7f77b3784786a1dae42a55d82693.e34KaxiLc3eQc40LaxqMbN4PaNuKe0?text=&docid=198059&pageIndex=0&doclang=DE&mode=lst&dir=&occ=first&part=1&cid=1135575}{ECLI:EU:C:2017:994}}-Entscheidung ergibt.

Der EuGH weist darauf hin, dass die Kopie nicht die Rechte und Freiheiten anderer Personen beeinträchtigen darf. Dies spielt bei Prüfungsarbeiten jedoch keine Rolle. Denn die Herausgabe der Prüfungsarbeit unter Verweis auf die Rechte des Prüfers zu verweigern, wird in der Regel nicht zulässig sein, da die Anmerkungen des Prüfers ebenfalls personenbezogene Daten des Prüflings durch den Prüfer sind.\footnote{\href{https://www.juwiss.de/8-2018/}{juwiss.de}}

Die Aufgabenstellung der Prüfung unterliegt nicht der DSGVO/BDSG, da es sich hierbei nicht um ein personenbezogenes Datum handelt.\footnote{\href{https://ec.europa.eu/info/law/law-topic/data-protection/reform/what-personal-data_de}{https://ec.europa.eu/info/law/law-topic/data-protection/reform}}

In Papierakten archivierte Klausuren seien nach dem Nowak Fall personenbezogene Daten, die in einem Dateisystem im Sinne von Art. 4 Nr. 6 DSGVO gesammelt seien. 

Der Schutz personenbezogener Daten Dritter ist also nicht beeinträchtigt und widerspricht keiner Kopie der Prüfung.

%evtl begründung zum höheren Stellenwert der offenen Aufgabenstellung gegen Aufgabenurheber

\section{Prüfungsrechtlicher Anspruch}
Diese Sektion befasst sich mit prüfungsrechtlichen Ansprüchen an die Prüfung und Prüfer, die eine Kopie der geschriebenen Prüfung verbieten und zulassen. Grundlage für Prüfungsrecht bieten die Prüfungs- und Studienordnungen der TU Ilmenau (PO-BA-2019), das Thüringer Verwaltungsverfahrensgesetz und das bundesweite Verwaltungsverfahrensgesetz.

\subsection{TU Ilmenau Prüfungsordnung  – allgemeine Bestimmungen}

\noindent\subitem\colorbox{lightgray}{
    \begin{minipage}[h]{0.9\linewidth}
        \scriptsize %TU Ilmenau - Prüfungsordnung - Allgemeine Bestimmungen - für Studiengänge mit dem Studienabschluss "Bachelor"\\
        \textbf{§25 Einsicht in die Prüfungsakte}
        \begin{enumerate}
            \item Nach Bekanntgabe der Note für eine Prüfungsleistung hat der Studierende in angemessener Zeit die Gelegenheit zur Einsicht in die korrigierten Arbeiten oder das Protokoll der mündlichen Prüfung. Diese Möglichkeit besteht in den ersten acht Wochen nach Beginn des folgenden Vorlesungszeitraumes. […]
        \end{enumerate}
    \end{minipage}
}

\noindent\subitem\colorbox{lightgray}{
    \begin{minipage}[h]{0.9\linewidth}
        %https://www.tu-ilmenau.de/fileadmin/Bereiche/Universitaet/Dokumente/Satzungen_und_Ordnungen/Studienordnungen/PStO-AB_2019_i.d.F.2.aend._13.07.2020_LF.pdf
        \scriptsize %TU Ilmenau - Prüfungs- und Studienordnung - Allgemeine Bestimmungen - für Studiengängemit dem Abschluss "Bachelor", "Master" und "Diplom"\\
        \textbf{§ 36 Einsicht in die Prüfungsakte sowie die Dokumente zu Abschlussleistungen und zur Abschlussarbeit, Aufbewahrung}
        \begin{enumerate}
            \item Studierende haben das Recht zur Einsichtnahme in dokumentierte Abschlussleistungen und -arbeiten und deren Bewertungen. Die Einsichtnahme soll den Einblick in erbrachte Leistungen einschließlich darauf gegebenenfalls bezogener Gutachten, Korrekturvermerke des Prüfers oder eines Prüfungsprotokolls zur mündlichen Abschlussleistung gewähren. Die Einsichtnahme soll die umfassende Information über Bewertung und Ergebnisse von Abschlussleistungen und -arbeit (§ 18) ermöglichen. […]
            \item Für die Möglichkeit der Einsichtnahme in die Prüfungsakte gelten die Vorschriften des Thüringer Verwaltungsverfahrensgesetzes sowie der Datenschutzgrundverordnung. Zur Ausübung des Rechts auf Auskunft (Art. 15 DSGVO) zur Hochschulprüfung (§ 7) dürfen Studierende ausschließlich zum eigenen Gebrauch eine Kopie ihrer Bearbeitungen einschließlich der Prüferanmerkung, jedoch ohne Aufgabenstellung, anfertigen. […]
        \end{enumerate}
        %==> Gilt erst ab 27. September 2019, § 38 In-Kraft-Treten, Übergangsbestimmungen, Außer-Kraft-Treten
    \end{minipage}
}

\subsection{Der Nowak Fall}
Der Europäische Gerichtshof hat in der Rechtssache C-434/16\footnote{\href{http://curia.europa.eu/juris/document/document.jsf;jsessionid=9ea7d0f130d5caea7f77b3784786a1dae42a55d82693.e34KaxiLc3eQc40LaxqMbN4PaNuKe0?text=&docid=198059&pageIndex=0&doclang=DE&mode=lst&dir=&occ=first&part=1&cid=1135575}{ECLI:EU:C:2017:994}} (Nowak) 2018 entschieden:

Prüflinge können jederzeit verlangen, dass eine Prüfungseinrichtung ihnen vollständige Auskunft zu ihren Prüfungsarbeiten gibt. \textbf{Dieses Recht kann nicht durch Prüfungsordnungen eingeschränkt werden.} Die Auskunft muss innerhalb kurzer Fristen erfolgen und ist für die Prüflinge kostenfrei. 

\subsection{Verwaltungsverfahrensgesetz und weitere Regelungen}
Im Rahmen des Akteneinsichtsrechts ist dem Studierenden grundsätzlich die Anfertigung von Kopien und Ablichtungen zu gestatten.

Die Einsichtnahme muss gemäß § 29 VwVfG in einer Weise gewährt werden, dass sich die bzw. der Studierende unter nach Zeit, Ort und sonstigen Umständen zumutbaren Bedingungen über den Inhalt der Prüfungsarbeit vollständig informieren kann.\footnote{\href{https://www.lehren.tum.de/fileadmin/w00bmo/www/Downloads/Themen/Pruefungen/Pruefungseinsichten_Version_6_2020.pdf}{https://www.lehren.tum.de/fileadmin/w00bmo/www/Downloads/Themen/Pruefungen/}}

Grundsätzlich darf den Studierenden nicht untersagt werden, Kopien von Prüfungsarbeiten anzufertigen.  Ein  generelles  Kopierverbot  würde  das  Recht  der  Studierenden  auf effektiven Rechtsschutz (Art. 19 Absatz 4 GG) unverhältnismäßig erschweren, da sie zur Wahrnehmung der Rechtsbehelfe gegen eine fehlerhafte Korrektur substantiierte Rügen vortragen müssen. Dies gilt umso mehr, da im verwaltungsgerichtlichen Verfahren ohnehin gem. § 100 Abs. 2 Satz 1 VwGO eine Kopiermöglichkeit eingeräumt wird.

Verweigert das Prüfungsamt oder Prüfer die Einsichtnahme in die Akten, so verstößt sie gegen das in der Verfassung verankerter Gebot auf einen effektiven Rechtsschutz (Art. 19 Abs. 4 GG).

Der Prüfling darf auch Fotokopien von den Akten anfertigen, z. B. um diese für sich selbst zu archivieren oder einem Anwalt zukommen zu lassen. Der Bayerische Verwaltungsgerichtshof hat nämlich festgestellt, dass Prüfungsakten nicht geheim zu halten sind (BayVGH, BayVBl. 1978, 309). Dabei muss die Behörde allerdings nicht die Fotokopien selbst anfertigen.\footnote{\href{https://www.recht-gehabt.de/ratgeber/meine-rechte-als-student/einsicht-in-die-pruefungsakten-anfertigung-von-fotokopien-geht-das.html}{https://www.recht-gehabt.de/ratgeber/meine-rechte-als-student/einsicht-in-die-pruefungsakten-anfertigung-von-fotokopien-geht-das.html}}

Nur soweit ein Gesetz die Betroffenenrechte abschließend regle, sei ein Rückgriff auf allgemeine Regeln ausgeschlossen. Aus dem bloßen Vorhandensein einer bereichsspezifischen oder höherrangigen Regelung ergebe sich nicht zwingend deren abschließender Regelungscharakter. Dies gelte auch und gerade mit Blick auf Datenschutzrechte, die den Persönlichkeitsschutz im Hinblick auf die bei der Datenverarbeitung drohenden Gefahren erweitern und nicht bereits bestehende Rechte einschränken sollten (Erwägungsgrund 11 zur DSGVO spricht in Bezug auf das Ziel der Verordnung von einer Stärkung der Betroffenenrechte und einer Verschärfung der Pflichten der Verantwortlichen [Hervorhebungen durch den Verf.]).\footnote{\href{https://kpmg-law.de/newsservice/anspruch-des-prueflings-auf-kostenlose-kopie-der-korrigierten-examensklausuren/}{https://kpmg-law.de/newsservice/anspruch-des-prueflings-auf-kostenlose-kopie-der-korrigierten-examensklausuren/}}

Demnach enthält die Klausurkopie auch die Klausuraufgaben um entsprechend vollständige Rückschlüsse auf die Bewertung und Korrektheit der Prüfung erstellen zu können.

Das Einsichtsrecht beziehe sich auf eine Einsichtnahme in die Originalunterlagen, während das Recht auf eine Kopie nur Zugang zu einer Reproduktion vermittle. Dass es sich deshalb  – erst recht aus der Perspektive des Datenschutzrechts  – bei dem Recht auf Auskunft bzw. bei dem Recht auf Erhalt einer Kopie einerseits und dem Recht auf Einsichtnahme in Originalunterlagen andererseits um verschiedene und sich insoweit ergänzende Ansprüche handle, sei auch schon aus der früheren Regelung in § 34 Abs. 9 BDSG a.F. ersichtlich geworden.\footnote{\href{https://kpmg-law.de/newsservice/anspruch-des-prueflings-auf-kostenlose-kopie-der-korrigierten-examensklausuren/}{https://kpmg-law.de/newsservice/anspruch-des-prueflings-auf-kostenlose-kopie-der-korrigierten-examensklausuren/}} 

%Thüringer Verwaltungsverfahrensgesetzes (Nachverfolgung Frage: Korrektheit/Nachvollziehbarkeit)

\subsection{Einschränkung der Verarbeitung der angefertigten Kopie} 
Angefertigte Kopien oder Ablichtungen der Prüfungsarbeit dürfen von den Studierenden nur zur Überprüfung der Klausurbewertung verwendet werden. Eine Verbreitung der hergestellten Vervielfältigungsstücke sowie deren öffentliche Wiedergabe (z. B. im Internet) ist nicht zulässig.

Dem Prüfling ist bewusst, dass die Fragestellung der Klausur sowie die Korrekturanmerkungen der Prüfenden Urheberrechtsschutz genießen und die angefertigten Kopien oder Ablichtungen der Prüfungsaufgabe nur zum Zwecke der Klausureinsicht zu nutzen sind. Eine Weitergabe an Dritte oder Veröffentlichung im Internet ist daher verboten und kann im Falle eines Verstoßes rechtliche Konsequenzen nach sich ziehen. Das berechtigte Rechtsschutzinteresse bleibt davon unberührt (z. B. Weitergabe der Prüfungsaufgabe an eine/einen bevollmächtigte(n) Rechtsanwältin/Rechtsanwalt). 

Zusätzlich können die Kopien der Prüfungsarbeiten auf Papier mit folgender Fußzeile, Wasserzeichen oder Ähnlichem versehen werden: 
\noindent\subitem\colorbox{lightgray}{
    \begin{minipage}[h]{0.9\linewidth}
        \footnotesize Fragestellung der Klausur sowie die Korrekturanmerkungen der Prüfenden unterliegen dem Urheberrechtsschutz. Kopien oder Ablichtungen der Prüfungsaufgabe dienen nur zum Zwecke der Klausureinsicht. Eine Weitergabe an Dritte oder Veröffentlichung im Internet ist verboten und kann im Falle eines Verstoßes rechtliche Konsequenzen nach sich ziehen. Das berechtigte Rechtsschutzinteresse bleibt davon unberührt (z. B. Weitergabe der Prüfungsaufgabe an eine/einen bevollmächtigte(n) Rechtsanwältin/Rechtsanwalt).
    \end{minipage}
}

\subsection{Umfang der Klausureinsicht}
Eine Einsicht in die eigenen Klausurunterlagen umfasst die gesamte Dokumentation des Prüfungsvorganges. Dazu zählen unter anderem die Anmeldung zur Prüfung, die Zulassung und Ladung zur Prüfung, Protokolle der mündlichen Prüfung und/oder die bewerteten schriftlichen Prüfungsarbeiten. Falls der Prüfer auf Musterlösungsskizzen zurückgreift und auf diese verweist, müssen auch diese eingesehen werden können (BVerwG, 16.3.1994, Az.: 6 C 1/93). Anderenfalls besteht nämlich gar keine Möglichkeit eigene Prüfungsfehler hinreichend genau darzulegen.

Letztlich dürfen bei der Einsicht der Prüfungsunterlagen uneingeschränkt Notizen zu den Unterlagen gemacht werden, selbst das Kopieren muss im Normalfall gestattet werden. Verhindert werden kann die Anfertigung von Fotokopien nur dann, wenn das Prüfungsamt ausreichend sachliche Gründe vorbringen kann.\footnote{\href{https://www.uniturm.de/magazin/recht/darf-ich-meine-klausur-einsehen-841}{Dextra Rechtsanwälte, Mario Fröhlich, Rechtsanwalt für Hochschul- und Prüfungsrecht}}

Vor allem sei nicht festzustellen, dass eine Beschränkung auf ein Einsichtnahmerecht vor Ort und die Möglichkeit des Erhalts nur kostenpflichtiger Kopien der Sicherstellung eines oder mehrerer der in den Buchstaben a) bis j) genannten öffentlichen Ziele diene. Die abschließende Aufzählung der dort genannten Ausnahmen mache deutlich, dass die Mitgliedstaaten keine darüber hinausgehenden Beschränkungen vornehmen dürften bzw. darüber hinausgehende Beschränkungen unzulässig seien.\footnote{\href{https://kpmg-law.de/newsservice/anspruch-des-prueflings-auf-kostenlose-kopie-der-korrigierten-examensklausuren/}{https://kpmg-law.de/newsservice/anspruch-des-prueflings-auf-kostenlose-kopie-der-korrigierten-examensklausuren/}} 


\subsection{Vorzeigebeispiel: TU München}
Die Technische Universität München löst die Problemfrage selbst und erläutert diese in ihren FAQ zu Prüfungsfragen\footnote{https://www.sv.tum.de/service/faq-pruefungen/}:

\subitem\colorbox{lightgray}{
    \begin{minipage}[h]{0.9\linewidth}
        \textbf{Darf ich meine Klausur (in der Klausureinsicht) kopieren?}\\
        Dir muss die Möglichkeit eingeräumt werden deine Klausur zu kopieren. Dies kann beispielsweise durch Abfotografieren, aber auch gegen Kostenersatz durch eine Kopie am Lehrstuhl geschehen. Dass die Prüfungsaufgabe in Zukunft nicht mehr verwendet werden kann, zählt nicht als Argument um eine Kopie zu verweigern.\vspace{.5cm}

        \textbf{Darf ich meine Klausur veröffentlichen?}\\
        Die Prüfungsaufgaben fallen unter das Urheberrecht des jeweiligen Prüfungsstellers. Eine Veröffentlichung dieser ist nicht gestattet. Der Lehrstuhl kann entsprechende Vorkehrungen treffen, um eine Veröffentlichung zu verhindern. Korrekturanmerkungen fallen auch darunter. Der Urheber deiner Antwort bist du selbst.
    \end{minipage}
}

\end{document}