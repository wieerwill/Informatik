%
%
% das Paket "flashcards" erzeugt Karteikarten zum lernen
% auf der Vorderseite steht das Buzzword oder die Frage
% auf der Rückseite steht die Antwort
% beim ausdrucken auf doppelseitiges Drucken achten
%
%

\documentclass[avery5371]{flashcards}

\cardfrontstyle{headings}


\begin{document}

%%%%%%%%%%%%%%%%%%%%%%%%%%%%%%%%%%%%%


\begin{flashcard}[Definition]{Alphabet}
Ein Alphabet ist eine endliche nichtleere Menge.

Üblicherweise heißen Alphabete hier $\sum, \Gamma, \Delta$. Ist $\sum$ Alphabet, so nennen wir die Elemente oft Buchstaben. Ist $\sum$ ein Alphabet, so heißen die Elemente von $\sum*$ auch Wörter über $\sum$ (auch String/Zeichenkette).
\end{flashcard}
%%%%%%%%%%%%%%%%%%%%%%%%%%%%%%%%%%%%%


\begin{flashcard}[Definition]{Menge der endlichen Folgen}
    Für eine Menge X ist X* die Menge der endlichen Folgen über X.
\end{flashcard}
%%%%%%%%%%%%%%%%%%%%%%%%%%%%%%%%%%%%%


\begin{flashcard}[Definition]{Wort}
Sind $u=(a_1, a_2, ...a_n)$ und $v=(b_1, b_2,...,b_n)$ Wörter, so ist $u*v$ das Wort $(a_1,a_2,...a_n,b_1,b_2,...,b_n)$; es wird als Verkettung/Konkatenation von u und v bezeichnet.
An Stelle von $u*v$ schreibt man auch $uv$.
\end{flashcard}
%%%%%%%%%%%%%%%%%%%%%%%%%%%%%%%%%%%%%


\begin{flashcard}[Definition]{Sprachen}
f: Menge der mögl Eingaben $\rightarrow$ Menge der mögl Ausgaben

Spezialfall $A={0,1}$ heißt Entscheidungsproblem. Sie ist gegeben durch die Menge der Eingaben. 
\end{flashcard}
%%%%%%%%%%%%%%%%%%%%%%%%%%%%%%%%%%%%%


\begin{flashcard}[Definition]{Präfix}
Seien y,w Wörter über $\sum$. Dann heißt Präfix/Anfangsstück von w, wenn es $z\in\sum*$ gibt mit $yz=w$.
\end{flashcard}
%%%%%%%%%%%%%%%%%%%%%%%%%%%%%%%%%%%%%


\begin{flashcard}[Definition]{Infix}
Seien y,w Wörter über $\sum$. Dann heißt Infix/Faktor von w, wenn es $x,z \in \sum*$ gibt mit $xyz=w$.
\end{flashcard}
%%%%%%%%%%%%%%%%%%%%%%%%%%%%%%%%%%%%%


\begin{flashcard}[Definition]{Suffix}
Seien y,w Wörter über $\sum$. Dann heißt Suffix/Endstück von w, wenn es $x\in \sum*$ gibt mit $xy=w$.
\end{flashcard}
%%%%%%%%%%%%%%%%%%%%%%%%%%%%%%%%%%%%%


\begin{flashcard}[Definition]{formale Sprachen}
Sei $\sum$ ein Alphabet. Teilmengen von $\sum*$ werden formale Sprachen über $\sum$ genannt.

Eine Menge L ist eine formale Sprache wenn es ein Alphabet $\sum$ gibt, so dass L formale Sprache über $\sum$ ist (d.h. $L\subseteq \sum*$).
\end{flashcard}
%%%%%%%%%%%%%%%%%%%%%%%%%%%%%%%%%%%%%


\begin{flashcard}[Definition]{Kleene Abschluss}
    Sei L eine Sprache. Dann ist $L*=\bigcup_{n\geq 0} L^n$ der Kleene-Abschluss oder die Kleene-Iteration von L. Weiter ist $L+ = \bigcup_{n\geq 0} L^n$
\end{flashcard}
%%%%%%%%%%%%%%%%%%%%%%%%%%%%%%%%%%%%%
    

\begin{flashcard}[Definition]{Prioritätsregeln für Operationen auf Sprachen}
    \begin{itemize}
        \item Potenz/Iteration binden stärker als Konkatenation
        \item Konkatenation stärker als Vereinigung/Durchschnitt/Differenz
    \end{itemize}
\end{flashcard}
%%%%%%%%%%%%%%%%%%%%%%%%%%%%%%%%%%%%%
    
    

\end{document}