%
%
% das Paket "flashcards" erzeugt Karteikarten zum lernen
% auf der Vorderseite steht das Buzzword oder die Frage
% auf der Rückseite steht die Antwort
% beim ausdrucken auf doppelseitiges Drucken achten
%
%
\documentclass[avery5371]{flashcards}
\usepackage[utf8]{inputenc}
\usepackage[]{amsmath} 
\usepackage[]{amssymb}
\cardfrontstyle{headings}
\begin{document}
%%%%%%%%%%%%%%%%%%%%%%%%%%%%%%%%%%%%%


\begin{flashcard}[Definition]{Alphabet}
Ein Alphabet ist eine endliche nichtleere Menge.

Üblicherweise heißen Alphabete hier $\sum, \Gamma, \Delta$. Ist $\sum$ Alphabet, so nennen wir die Elemente oft Buchstaben und die Elemente von $\sum*$ auch Wörter über $\sum$ (auch String/Zeichenkette).
\end{flashcard}
%%%%%%%%%%%%%%%%%%%%%%%%%%%%%%%%%%%%%


\begin{flashcard}[Definition]{Menge der endlichen Folgen}
    Für eine Menge X ist X* die Menge der endlichen Folgen über X.\\

    Beispiel: Elemente von ${a,b,c,d}*:(a,b,c),()$
\end{flashcard}
%%%%%%%%%%%%%%%%%%%%%%%%%%%%%%%%%%%%%


\begin{flashcard}[Definition]{Wort}
Sind $u=(a_1, a_2, ...a_n)$ und $v=(b_1, b_2,...,b_n)$ Wörter, so ist $u*v$ das Wort $(a_1,a_2,...a_n,b_1,b_2,...,b_n)$; es wird als Verkettung/Konkatenation von u und v bezeichnet.
An Stelle von $u*v$ schreibt man auch $uv$.
\end{flashcard}
%%%%%%%%%%%%%%%%%%%%%%%%%%%%%%%%%%%%%


\begin{flashcard}[Definition]{Induktiv $w^n$ definieren}
$$w^n = \begin{cases} \epsilon \quad\text{falls } n=0 \\ {w * w^{n-1}} \quad\textfalls } n>0 \end{cases}$$
\end{flashcard}
%%%%%%%%%%%%%%%%%%%%%%%%%%%%%%%%%%%%%


\begin{flashcard}[Definition]{y,w sind Wörter über $\sum$. Dann heißt y:}
\begin{itemize}
    \item Präfix/Anfangsstück von w, wenn es $z\in\sum^*$ gibt mit $yz=w$
    \item Infix/Faktor von w, wenn es $x,z\in\sum^*$ gibt mit $xyz = w$
    \item Suffix/Endstück von w, wenn es $x\in\sum^*$ gibt mit $xy=w$
\end{itemize}    
\end{flashcard}
%%%%%%%%%%%%%%%%%%%%%%%%%%%%%%%%%%%%%


\begin{flashcard}[Definition]{Sprachen}
f: Menge der möglichen Eingaben $\rightarrow$ Menge der möglichen Ausgaben\\
Spezialfall $A={0,1}$ heißt Entscheidungsproblem. Sie ist gegeben durch die Menge der Eingaben. 
\end{flashcard}
%%%%%%%%%%%%%%%%%%%%%%%%%%%%%%%%%%%%%


\begin{flashcard}[Definition]{Präfix}
Seien y,w Wörter über $\sum$. Dann heißt Präfix/Anfangsstück von w, wenn es $z\in\sum*$ gibt mit $yz=w$.
\end{flashcard}
%%%%%%%%%%%%%%%%%%%%%%%%%%%%%%%%%%%%%


\begin{flashcard}[Definition]{Infix}
Seien y,w Wörter über $\sum$. Dann heißt Infix/Faktor von w, wenn es $x,z \in \sum*$ gibt mit $xyz=w$.
\end{flashcard}
%%%%%%%%%%%%%%%%%%%%%%%%%%%%%%%%%%%%%


\begin{flashcard}[Definition]{Suffix}
Seien y,w Wörter über $\sum$. Dann heißt Suffix/Endstück von w, wenn es $x\in \sum*$ gibt mit $xy=w$.
\end{flashcard}
%%%%%%%%%%%%%%%%%%%%%%%%%%%%%%%%%%%%%


\begin{flashcard}[Definition]{formale Sprachen}
Sei $\sum$ ein Alphabet. Teilmengen von $\sum*$ werden formale Sprachen über $\sum$ genannt.

Eine Menge L ist eine formale Sprache wenn es ein Alphabet $\sum$ gibt, so dass L formale Sprache über $\sum$ ist (d.h. $L\subseteq \sum*$).
\end{flashcard}
%%%%%%%%%%%%%%%%%%%%%%%%%%%%%%%%%%%%%

\begin{flashcard}[Definition]{Verkettung von Sprachen}
Sind $L_1$ und $L_2$ Sprachen, so heißt die Sprache $L_1L_2=\{w|\exists w_1\in L_1,w_2\in L_2:w=w_1w_2\}$ (auch $L_1*L_2$) die Konkatenation oder Verkettung von $L_1$ und $L_2$.
\end{flashcard}
%%%%%%%%%%%%%%%%%%%%%%%%%%%%%%%%%%%%%%


\begin{flashcard}[Definition]{Kleene Abschluss}
    Sei L eine Sprache. Dann ist $L*=\bigcup_{n\geq 0} L^n$ der Kleene-Abschluss oder die Kleene-Iteration von L. Weiter ist $L^{+} = \bigcup_{n\geq 0} L^n$\\
    ($L^{+} = L*L = L^* *L$)
\end{flashcard}
%%%%%%%%%%%%%%%%%%%%%%%%%%%%%%%%%%%%%
    

\begin{flashcard}[Definition]{Prioritätsregeln für Operationen auf Sprachen}
    \begin{itemize}
        \item Potenz/Iteration binden stärker als Konkatenation
        \item Konkatenation stärker als Vereinigung/Durchschnitt/Differenz
    \end{itemize}
\end{flashcard}
%%%%%%%%%%%%%%%%%%%%%%%%%%%%%%%%%%%%%
    
\begin{flashcard}[Definition]{Grammatik}\scriptsize
Grammatiken sind ein Mittel um alle syntaktisch korrekten Sätze einer Sprache zu erzeugen.\\
Eine Grammatik G ist ein 4-Tupel $G=(V, \sum, P, S)$ das folgende Bedingungen erfüllt
\begin{itemize}
    \item V ist eine endliche Menge von Nicht-Terminalen oder Variablen
    \item $\sum$ ist ein Alphabet (Menge der Terminale) mit 
    $V\cap \sum = \varnothing$
    ,d.h. kein Zeichen ist gleichzeitig Terminal und Nicht-Terminal
    \item $P\subseteq (V\cup \sum)^+ \times (v\cup\sum)^*$ ist eine endliche Menge von Regeln oder Produktionen (Produktionsmenge)
    \item $S\in V$ ist das Startsymbol/ die Startvariable oder das Axiom
\end{itemize}

Jede Grammatik hat nur endlich viele Regeln!
\end{flashcard}
%%%%%%%%%%%%%%%%%%%%%%%%%%%%%%%%%%%%%

\begin{flashcard}[Definition]{Ableitung einer Grammatik}
Sei $G=(V, \sum, P, S)$  eine Grammatik. Eine Ableitung ist eine endliche Folge von Wörtern $w_0, w_1, w_2,...,w_n$ mit $w_0\Rightarrow w_1 \Rightarrow w_2 \Rightarrow ... \Rightarrow w_n$.
\end{flashcard}

%%%%%%%%%%%%%%%%%%%%%%%%%%%%%%%%%%%%%
\begin{flashcard}[Definition]{Wort ist Satzform}
Ein Wort $w\in (V\cup\sum)^*$ heißt Satzform, wenn es eine Ableitung gibt, deren letztes Wort w ist.
\end{flashcard}
%%%%%%%%%%%%%%%%%%%%%%%%%%%%%%%%%%%%%

\begin{flashcard}[Definition]{erzeugte Sprache}
Die Sprache $L(G)={w\in \sum^* | S\Rightarrow_G^* w}$ aller Satzformen aus $\sum^*$ heißt von G erzeugte Sprache.
\end{flashcard}
%%%%%%%%%%%%%%%%%%%%%%%%%%%%%%%%%%%%%

\begin{flashcard}[Definition]{Chomsky-0}
    Jede Grammatik ist vom Typ 0 (Semi-Thue-System) und wird auch als rekursiv-aufzählbar bezeichnet.
\end{flashcard}
%%%%%%%%%%%%%%%%%%%%%%%%%%%%%%%%%%%%%

\begin{flashcard}[Definition]{Chomsky-1}
    Eine Regel heißt kontext-sensitiv, wenn es Wörter $u,v,w\in(V\cup\sum)^*,|v|>0$ und ein Nichtterminal $A\in V$ gibt mit $l=uAw$ und $r=uvw$. Eine Grammatik ist vom Typ 1 (oder kontext-sensitiv) falls
    \begin{itemize}
        \item alle Regeln aus P kontext-sensitiv sind
        \item $(S\rightarrow \epsilon)\in P$ die einzige nicht kontext-sensitive Regel in P ist und S auf keiner rechten Seite einer Regel aus P vorkommt
    \end{itemize}
\end{flashcard}
%%%%%%%%%%%%%%%%%%%%%%%%%%%%%%%%%%%%%

\begin{flashcard}[Definition]{Chomsky-2}
    Eine Regel $(l\rightarrow r)$ heißt kontext-frei wenn $l\in V$ und $r\in (V\cup \sum)^*$ gilt. Eine Grammatik ist vom Typ 2, falls sie nur kontext-freie Regeln enthält
\end{flashcard}
%%%%%%%%%%%%%%%%%%%%%%%%%%%%%%%%%%%%%

\begin{flashcard}[Definition]{Chomsky-3}
    Eine Regl ist rechtslinear, wenn $l\in V$ und $r\in \sum V\cup {\epsilon}$ gilt. Eine Grammatik ist vom Typ 3 wenn sie nur rechtslineare Regeln enthält.
\end{flashcard}
%%%%%%%%%%%%%%%%%%%%%%%%%%%%%%%%%%%%%

\begin{flashcard}[Beweise]{Es gibt einen Algorithmus, der als Eingabe eine Typ-1-Grammatik G und ein Wort w bekommst und nach endlicher Zeit entscheidet ob $w\in L(G)$ gilt.}
    \scriptsize{
    \begin{enumerate}
        \item $w=\epsilon$: Da G vom Typ 1 ist, gilt $w\in L(G)$ genau dann wenn $(S\rightarrow \epsilon)\in P$. Dies kannn ein Algorithmus entscheiden
        \item $|w|\geq 1$: Definiere einen gerichteten Graphen (W,E) wie folgt
        \begin{itemize}
            \item Knoten sind die nichtleeren Wörter über $V\cup\sum$ der Länge $\geq|w|$ (insbes. $S,w \in W$)
            \item $(u,v)\in E$ genau dann wenn $u\Rightarrow_G v$
        \end{itemize}
        da kontext-sensitiv ist, gilt $1 = |u_0|\geq |u_1|\geq |u_2|\geq...\geq |u_n| = |w|$, also $u_i\in W$  f.a. $1\geq i \geq n$. Also existiert Pfad von S nach w im Graphen (W , E ), womit die Behauptung bewiesen ist.
    \end{enumerate}
    }
\end{flashcard}
%%%%%%%%%%%%%%%%%%%%%%%%%%%%%%%%%%%%%

\begin{flashcard}[Definition]{Deterministische endliche Automaten}
    ein deterministischer endlicher Automat M ist ein 5-Tupel $M=(Z, \sum, z_0, \delta, E)$
\begin{itemize}
    \item $Z$ eine endliche Menge von Zuständen
    \item $\sum$ das Eingabealphabet (mit $Z\cap\sum = \emptyset$)
    \item $z_0\in Z$ der Start/Anfangszustand (max Einer)
    \item $\delta: Z \times \sum \rightarrow Z$ die Überführungs/Übergangsfunktion
    \item $E\subseteq Z$ die Menge der Endzustände
\end{itemize}
Abkürzung: DFA (deterministic finite automaton)
\end{flashcard}
%%%%%%%%%%%%%%%%%%%%%%%%%%%%%%%%%%%%%%

\begin{flashcard}[Definition]{DFA mit Funktion $\hat{\delta}$}
    Zu einem gegebenen DFA definieren wir die Funktion $\hat{\delta}: Z \times \sum^* \rightarrow Z$ induktiv wie folgt, wobei $z\in Z$, $w\in\sum^+$ und $a\in \sum$:
    \begin{itemize}
        \item $\hat{\delta}(z, \epsilon) = z$
        \item $\hat{\delta}(z,aw)= \hat{\delta}(\delta(z,a),w)$
    \end{itemize}
    Der Zustand $\hat{\delta}(z,w)$ ergibt sich indem man vom Zustand z aus dem Pfad folgt der mit w beschriftet ist.
\end{flashcard}
%%%%%%%%%%%%%%%%%%%%%%%%%%%%%%%%%%%%%%


\begin{flashcard}[Definition]{von einem DFA akzeptierte Sprache}
    die von einem DFA akzeptierte Sprache ist: $L(M)={w\in\sum^* | \hat{\delta}(z_0,w)\in E}$\\
    Mit anderen Worten: Ein Wort w wird genau dann akzeptiert, wenn derjenige Pfad, der im Anfangszustand beginnt und dessen Übergänge mit den Zeichen von w markiert sind, in einem Endzustand endet.
\end{flashcard}
%%%%%%%%%%%%%%%%%%%%%%%%%%%%%%%%%%%%%%


\begin{flashcard}[Definition]{Wann ist eine Sprache regulär?}
    Eine Sprache $L \supseteq \sum^*$ ist regulär, wenn es einen DFA mit $L(M)=L$ gibt ( bzw. wird von einem DFA akzeptiert).  Jede reguläre Sprache ist rechtslinear.
\end{flashcard}
%%%%%%%%%%%%%%%%%%%%%%%%%%%%%%%%%%%%%%

%V3-10
\begin{flashcard}[Definition]{Text}
    Text
\end{flashcard}
%%%%%%%%%%%%%%%%%%%%%%%%%%%%%%%%%%%%%%




\end{document}