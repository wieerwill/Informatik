\documentclass[a4paper]{article}
\usepackage[ngerman]{babel}
\usepackage{multicol}
\usepackage{calc}
\usepackage{ifthen}
\usepackage[landscape,left=1cm,top=1cm,right=1cm,nohead,nofoot]{geometry}
\usepackage{amsmath,amsthm,amsfonts,amssymb}
\usepackage{color,graphicx,overpic}
\usepackage{listings}
\usepackage[compact]{titlesec} %less space for headers
\usepackage{mdwlist} %less space for lists
\usepackage[utf8]{inputenc}
\usepackage{tikz}
\usepackage{pdflscape}
\usepackage{verbatim}
\usetikzlibrary{mindmap, arrows,shapes,positioning,shadows,trees}
\tikzstyle{every node}=[draw=black,thin,anchor=west, minimum height=2em]
\usepackage[hidelinks,pdfencoding=auto]{hyperref}

\pdfinfo{
    /Title (Automaten, Sprachen \& Komplexität - Cheatsheet)
    /Creator (TeX)
    /Producer (pdfTeX 1.40.0)
    /Author (Robert Jeutter)
    /Subject ()
}
% Information boxes
\newcommand*{\info}[4][16.3]{
  \node [ annotation, #3, scale=0.65, text width = #1em, inner sep = 2mm ] at (#2) {
  \list{$\bullet$}{\topsep=0pt\itemsep=0pt\parsep=0pt
    \parskip=0pt\labelwidth=8pt\leftmargin=8pt
    \itemindent=0pt\labelsep=2pt}
    #4
  \endlist
  };
}

% This sets page margins to .5 inch if using letter paper, and to 1cm
% if using A4 paper. (This probably isn't strictly necessary.)
% If using another size paper, use default 1cm margins.
\ifthenelse{\lengthtest { \paperwidth = 11in}}
    { \geometry{top=.5in,left=.5in,right=.5in,bottom=.5in} }
    {\ifthenelse{ \lengthtest{ \paperwidth = 297mm}}
        {\geometry{top=1cm,left=1cm,right=1cm,bottom=1cm} }
        {\geometry{top=1cm,left=1cm,right=1cm,bottom=1cm} }
    }

% Redefine section commands to use less space
\makeatletter
\renewcommand{\section}{\@startsection{section}{1}{0mm}%
                                {-1ex plus -.5ex minus -.2ex}%
                                {0.5ex plus .2ex}%x
                                {\normalfont\large\bfseries}}
\renewcommand{\subsection}{\@startsection{subsection}{2}{0mm}%
                                {-1explus -.5ex minus -.2ex}%
                                {0.5ex plus .2ex}%
                                {\normalfont\normalsize\bfseries}}
\renewcommand{\subsubsection}{\@startsection{subsubsection}{3}{0mm}%
                                {-1ex plus -.5ex minus -.2ex}%
                                {1ex plus .2ex}%
                                {\normalfont\small\bfseries}}
\makeatother

% Define BibTeX command
\def\BibTeX{{\rm B\kern-.05em{\sc i\kern-.025em b}\kern-.08em
    T\kern-.1667em\lower.7ex\hbox{E}\kern-.125emX}}

% Don't print section numbers
\setcounter{secnumdepth}{0}

\setlength{\parindent}{0pt}
\setlength{\parskip}{0pt plus 0.5ex}    
% compress space
\setlength\abovedisplayskip{0pt}
\setlength{\parskip}{0pt}
\setlength{\parsep}{0pt}
\setlength{\topskip}{0pt}
\setlength{\topsep}{0pt}
\setlength{\partopsep}{0pt}
\linespread{0.5}
\titlespacing{\section}{0pt}{*0}{*0}
\titlespacing{\subsection}{0pt}{*0}{*0}
\titlespacing{\subsubsection}{0pt}{*0}{*0}

%My Environments
\newtheorem{example}[section]{Example}

%Tikz global setting
\tikzset{
    topic/.style={
                text centered,
                text width=5cm,
                level distance=1mm,
                sibling distance=5mm,
                rounded corners=2pt
            },
        subtopic/.style={
                yshift=1.5cm,
                text centered,
                text width=3cm,
                rounded corners=2pt,
                fill=gray!10
            },
        theme/.style={
                grow=down,
                xshift=-0.6cm,
                text centered,
                text width=3cm,
                edge from parent path={(\tikzparentnode.205) |- (\tikzchildnode.west)}
            },
        description/.style={
                grow=down,
                xshift=-0.5cm,
                right,
                text centered,
                edge from parent path={(\tikzparentnode.200) |- (\tikzchildnode.west)}
            },
        level1/.style ={level distance=1cm},
        level2/.style ={level distance=2cm},
        level3/.style ={level distance=3cm},
        level4/.style ={level distance=4cm},
        level5/.style ={level distance=5cm},
        level6/.style ={level distance=6cm},
        level7/.style ={level distance=7cm},
        level8/.style ={level distance=8cm},
        level9/.style ={level distance=9cm},
        level 1/.style={sibling distance=5.5cm},
        level 1/.append style={level distance=2.5cm},
}

% Turn off header and footer
\pagestyle{empty}
\begin{document}

\begin{tikzpicture}
    \node[topic]{Automaten, Sprachen \& Komplexität}
    child{node [subtopic]{Sprache}
            child [theme, level1] { node {Chomsky Hierachie}
                    child[description, level distance=1cm] { node {Typ 0: Allgemein \\ jede Grammatik ist vom Typ 0}}
                    child[description, level distance=2cm] { node {Typ 1: Kontextsensitiv \\ wenn es Wörter $u,v,w\in(V\cup\sum)^*,|v|>0$ und ein Nichtterminal $A\in V$ gibt mit $l=uAw$ und $r=uvw$}}
                    child[description, level distance=3cm] { node {Typ 2: Kontextfrei \\ wenn $l\in V$ und $r\in (V\cup \sum)^*$ gilt}}
                    child[description, level distance=4cm] { node {Typ 3: Regulär \\ wenn $l\in V$ und $r\in \sum V\cup {\epsilon}$ gilt}}
                }
        }
    child{node [subtopic]{Wort}
            child[description, level distance=1cm] { node { y Präfix von w, wenn es $z\in\sum^*$ gibt mit $yz=w$}}
            child[description, level distance=2cm] { node { y Infix/Faktor von w, wenn es $x,z\in\sum^*$ gibt mit $xyz = w$}}
            child[description, level distance=3cm] { node { y Suffix von w, wenn es $x\in\sum^*$ gibt mit $xy=w$}}
        }
    child{node [subtopic]{Grammatik}
            child[theme, level distance=1cm] { node {Symbole}
                    child[description, level distance=1cm] { node {Nicht-Terminale (oder Variablen), aus denen noch weitere Wortbestandteile abgeleitet werden sollen}}
                    child[description, level distance=2cm] { node {Terminale (die "eigentlichen" Symbole)}}
                }
            child[theme, level distance=4cm]{ node {4-Tupel $G=(V,\sum, P, S)$}
                    child[description, level distance=1cm] { node {V ist eine endliche Menge von Nicht-Terminalen oder Variablen}}
                    child[description, level distance=2cm] { node {$\sum$ ist ein Alphabet (Menge der Terminale)}}
                    child[description, level distance=3cm] { node {$P$ ist eine endliche Menge von Regeln oder Produktionen}}
                    child[description, level distance=4cm] { node {$S\in V$ ist das Startsymbol oder das Axiom}}
                }
            child[theme, level distance=8cm]{ node {Konventionen}
                    child[description, level distance=1cm] { node {Variablen sind Großbuchstaben (Elemente aus V)}}
                    child[description, level distance=2cm] { node {Terminale sind Kleinbuchstaben (Elemente aus $\sum$)}}
                }
        };
\end{tikzpicture}

\begin{tikzpicture}[
        subtopic/.style={
                yshift=1.5cm,
                text centered,
                text width=3cm,
                rounded corners=2pt,
                fill=gray!10
            },
        level 1/.style={sibling distance=5.5cm},
        level 1/.append style={level distance=2.5cm},
    ]
    % Topic
    \node[topic]{Automaten, Sprachen \& Komplexität}
    child{node [subtopic] {intuitiv berechenbar}
            child[theme, level distance=1cm]{node{$\mu$ rekurisv}}
            child[theme, level distance=2cm]{node{while berechnenbar}}
            child[theme, level distance=3cm]{node{Turing berechenbar}}
            child[theme, level distance=4cm]{node{goto berechnenbar}}
        };
\end{tikzpicture}

Definition: Sei L eine Sprache. Dann ist $L*=\bigcup_{n\geq 0} L^n$ der Kleene-Abschluss oder die Kleene-Iteration von L. Weiter ist $L+ = \bigcup_{n\geq 0} L^n$


\begin{tikzpicture}
    \node[topic]{Rechtslineare Sprachen}
    child{node [subtopic] {endliche Automaten (Maschinen)}
    child[theme, level distance=1cm]{node{deterministischer endlicher Automat M}
    child[description, level distance=1cm]{node{5-Tupel $M=(Z, \sum, z_0, \delta, E)$}}
    child[description, level distance=1cm]{node{$Z$ eine endliche Menge von Zuständen}}
    child[description, level distance=1cm]{node{$\sum$ das Eingabealphabet (mit $Z\cap\sum = \emptyset$)}}
    child[description, level distance=1cm]{node{$z_0\in Z$ der Startzustand}}
    child[description, level distance=1cm]{node{$\delta: Z \times \sum \rightarrow Z$ die Übergangsfunktion}}
    child[description, level distance=1cm]{node{$E\subseteq Z$ die Menge der Endzustände}}
    child[description, level distance=1cm]{node{kurz: DFA (deterministic finite automaton)}}
    child[description, level distance=1cm]{node{von einem DFA akzeptierte Sprache ist: $L(M)={w\in\sum^* | \hat{\delta}(z_0,w)\in E}$}}
    child[description, level distance=1cm]{node{Eine Sprache $L \supseteq \sum^*$ ist regulär, wenn es einen DFA mit $L(M)=L$ gibt}}
    %Jede reguläre Sprache ist rechtslinear
    }
    child[theme, level distance=1cm]{node{nicht-deterministischer endlicher Automat M}
            %Jede von einem NFA akzeptierte Sprache ist regulär
            child[description, level distance=1cm]{node{kurz NFA}}
        }
    %Satz: Wenn $L_1$ und $L_2$ reguläre Sprachen sind, dann ist auch $L_1 \cup L_2$ regulär.
    %Satz: Wenn $L_1$ und $L_2$ reguläre Sprachen sind, dann ist auch $L_1 \cap L_2$ regulär.
    %Satz: Wenn $L_1$ und $L_2$ reguläre Sprachen sind, dann ist auch $L_1L_2$ regulär
    %Satz: Wenn L eine reguläre Sprache ist, dann ist auch $L^+/L^*$ regulär
    }
    child{description, level distance}{node{Reguläre Ausdrücke
                    % Definition: Die Menge $Reg(\sum)$ der **regulären Ausdrücke über dem Alphabet $\sum$** ist die kleinste Menge mit folgenden Eigenschaften:
                    % - $\varnothing \in Reg(\sum), \lambda \in Reg(\sum), \sum \subseteq Reg(\sum)$
                    % - Wenn $\alpha, \beta \in Reg(\sum)$, dann auch $(\alpha * \beta), (\alpha + \beta), (\alpha^*)\in Reg(\sum)$
                    %- für $\alpha * \beta$ schreibt man oft $\alpha\beta$
                    % für $\alpha + \beta$ schreibt man auch $\alpha|\beta$

                    %Für einen regulären Ausdruck $\alpha \in Reg(\sum)$ ist die Sprache $L(\alpha)\subseteq \sum^*$ induktiv definiert

                    %zu jedem regulären Ausdruck $\gamma$ gibt es einen NFA M mit $L(\gamma)=L(M)$
                    %zu jedem DFA M gibt es einen regulären Ausdruck $\gamma$ mit $L(M)=L(\gamma)$
                }}
    %- Rechtslineare Grammatiken
    %    - Verbindung zur Chomsky Hierarchie
    %    - erzeugen Sprachen
    %    - nicht geeignet, um zu entscheiden, ob ein gegebenes Wort zur Sprache gehört
    %- NFA
    %    - erlauben kleine Kompakte Darstellung
    %    - intuitive graphische Notation
    %    - nicht geeignet, um zu entscheiden, ob ein gegebenes Wort zur Sprache gehört
    %- DFA
    %    - für effiziente Beantwortung der Frage, ob ein Wort zur Sprache gehört
    %    - sind uU exponentiell größer als NFA
    %- Reguläre Ausdrücke
    %    - erlauben kompakte Darstellung in Textform
    child{node [subtopic] {Nicht-Reguläre Sprachen}
            % Für jedes Alphabet $\sum$ existiert eine Sprache L über $\sum$, die von keiner Grammatik G erzeugt wird.
            child[theme, level distance=2cm]{node{ Pumping Lemma}
                    %Wenn L eine reguläre Sprache ist, dann gibt es $n\leq 1$ derart, dass für alle $x\in L$ mit $|x|\geq n$ gilt: es gibt Wörter $u,v,w \in \sum^*$ mit:
                    %1. $x=uvw$
                    %2. $|uv|\leq n$
                    %3. $|v|\geq 1$
                    %4. $uv^i w\in L$ für alle $i\geq 0$

                    %Dieses Lemma spricht nicht über Automaten, sondern nur über die Eigenschaften der Sprache. Es ist geeignet, Aussagen über Nicht-Regularität zu machen. Dabei ist es aber nur eine notwendige Bedingung. Es kann nicht genutzt werden, um die Regularität einer Sprache L zu zeigen.
                }
            child[theme, level distance=3cm]{node{ Myhill-Nerode Äquivalenz}
            %Für eine Sprache $L\subseteq \sum^*$ definieren wir eine binäre Relation $R_L \subseteq \sum^* \times \sum^*$ wie folgt: Für alle $x,y\in \sum^*$ setze $(x,y)\in R_L$ genau dann, wenn $\forall z \in \sum^* :(xy\in L \leftrightarrow yz \in L)$ gilt. Wir schreiben hierfür auch $x R_L y$.
            % Definition: Für eine Sprache L und ein Wort $x\in \sum^*$ ist $[x]_L=\{y\in\sum^* | x R_L y \}$ die Äquivalenzklasse von x. Ist L klar, so schreiben wir einfacher $[x]$.
            %Satz von Myhill-Nerode: Sei L eine Sprache. L ist regulär $\leftrightarrow index(R_L)< \infty$ (d.h. nur wenn die Myhill-Nerode-Äquivalenz endliche Klassen hat)
            }
        }
    child{node [subtopic] {Minimalautomat}
        %Ein DFA M heißt reduziert, wenn es für jeden Zustand $z \in Z$ ein Wort $x_z\in \sum^*$ gibt mit $\hat{\sigma}(l, x_z)=z$
    }
    child{node [subtopic] {Entscheidbarkeit}
            child[theme, level distance=1cm]{node{Wortproblem}}
            child[theme, level distance=1cm]{node{Leerheitsproblem}}
            child[theme, level distance=1cm]{node{Endlichkeitsproblem}}
            child[theme, level distance=1cm]{node{Schnittproblem}}
            child[theme, level distance=1cm]{node{Inklusionsproblem}}
            child[theme, level distance=1cm]{node{Äquivalenzproblem}}
        };
\end{tikzpicture}

\begin{tikzpicture}
    \node[topic]{Kontextfreie Sprachen}
    child{node [subtopic] { Ableitungsbäume}}
    child{node [subtopic] {Linksableitung}}
    child{node [subtopic] {Chomsky Normalform}}
    child{node [subtopic] {Der Cocke-Younger-Kasami- oder CYK-Algorithmus}}
    child{node [subtopic] {Kellerautomaten}}
    child{node [subtopic] {die Greibach-Normalform}}
    child{node [subtopic] {PDAs mit Endzuständen}}
    child{node [subtopic] {Deterministisch kontextfreie Sprachen}}
    child{node [subtopic] {das Pumping Lemma für kontextfreie Sprachen}}
    child{node [subtopic] {das Lemma von Ogden (William Ogden)}}
    ;
\end{tikzpicture}

\begin{tikzpicture}
    \node[topic]{Berechenbarkeit}
    child{node [subtopic] {Loop-Berechenbarkeit}}
    child{node [subtopic] {While Programme}
            child[theme, level distance=1cm]{node{Gödels Vermutung}}
        }
    child{node [subtopic] {GoTo Programme}
            child[theme, level distance=1cm]{node{Kleenesche Normalform}}
        }
    child{node [subtopic] {Turing Berechenbarkeit}}
    ;
\end{tikzpicture}

\begin{tikzpicture}
    \node[topic]{Entscheidbarkeit}
    child{node [subtopic] {Halteproble}}
    child{node [subtopic] {Reduktion}}
    child{node [subtopic] {Rechnen mit Kodierungen}}
    child{node [subtopic] {Satz von Rice}}
    child{node [subtopic] {Semi Entscheidbarkeit}}
    child{node [subtopic] {Universelle Turing Maschine}}
    child{node [subtopic] {Totale berechenbare Funktionen}}
    child{node [subtopic] {Einige unentscheidbare Probleme}}
    ;
\end{tikzpicture}

\begin{tikzpicture}
    \node[topic]{Komplexitätstheorie}
    child{node [subtopic] {Berechenbarkeitstheorie}}
    child{node [subtopic] {Frage der Komplexitätstheorie}}
    child{node [subtopic] {Komplexitätsklassen}
            child[theme, level distance=1cm]{node{Deterministische Zeitklassen}}
            child[theme, level distance=1cm]{node{Deterministische Platzklassen}}
            child[theme, level distance=1cm]{node{Nichtdeterministische Zeitklassen}}
            child[theme, level distance=1cm]{node{Nichtdeterministische Platzklassen}}
        }
    child{node [subtopic] {Polynomialzeit-Reduktionen}}
    child{node [subtopic] {NP-Vollständigkeit}}
    child{node [subtopic] {Weitere NP-vollständige Probleme}
            child[theme, level distance=1cm]{node{3-SAT ist NP-vollständig}}
            child[theme, level distance=1cm]{node{3C ist NP-vollständig}}
            child[theme, level distance=1cm]{node{DHC ist NP-vollständig}}
            child[theme, level distance=1cm]{node{HC ist NP-vollständig}}
            child[theme, level distance=1cm]{node{TSP ist NP-vollständige}}
        };
\end{tikzpicture}

\end{document}
