\documentclass[10pt, a4paper]{exam}
\printanswers			    % Comment this line to hide the answers 
\usepackage[utf8]{inputenc}
\usepackage[T1]{fontenc}
\usepackage[ngerman]{babel}
\usepackage{listings}
\usepackage{float}
\usepackage{graphicx}
\usepackage{color}
\usepackage{listings}
\usepackage[dvipsnames]{xcolor}
\usepackage{tabularx}
\usepackage{geometry}
\usepackage{color,graphicx,overpic}
\usepackage{amsmath,amsthm,amsfonts,amssymb}
\usepackage{tabularx}
\usepackage{listings}
\usepackage[many]{tcolorbox}
\usepackage{multicol}
\usepackage{hyperref}
\usepackage{pgfplots}
\usepackage{bussproofs}
\usepackage{tikz}
\usetikzlibrary{automata, arrows.meta, positioning}
%\renewcommand{\solutiontitle}{\noindent}
\SolutionEmphasis{\small}
\geometry{top=1cm,left=1cm,right=1cm,bottom=1cm} 

\pdfinfo{
    /Title (Automaten, Sprachen \& Komplexität - Prüfungsvorbereitung)
    /Creator (TeX)
    /Producer (pdfTeX 1.40.0)
    /Author (Robert Jeutter)
    /Subject ()
}
\title{Automaten, Sprachen \& Komplexität - Prüfungsvorbereitung}
\author{}
\date{}

% Don't print section numbers
\setcounter{secnumdepth}{0}

\newtcolorbox{myboxii}[1][]{
  breakable,
  freelance,
  title=#1,
  colback=white,
  colbacktitle=white,
  coltitle=black,
  fonttitle=\bfseries,
  bottomrule=0pt,
  boxrule=0pt,
  colframe=white,
  overlay unbroken and first={
  \draw[red!75!black,line width=3pt]
    ([xshift=5pt]frame.north west) -- 
    (frame.north west) -- 
    (frame.south west);
  \draw[red!75!black,line width=3pt]
    ([xshift=-5pt]frame.north east) -- 
    (frame.north east) -- 
    (frame.south east);
  },
  overlay unbroken app={
  \draw[red!75!black,line width=3pt,line cap=rect]
    (frame.south west) -- 
    ([xshift=5pt]frame.south west);
  \draw[red!75!black,line width=3pt,line cap=rect]
    (frame.south east) -- 
    ([xshift=-5pt]frame.south east);
  },
  overlay middle and last={
  \draw[red!75!black,line width=3pt]
    (frame.north west) -- 
    (frame.south west);
  \draw[red!75!black,line width=3pt]
    (frame.north east) -- 
    (frame.south east);
  },
  overlay last app={
  \draw[red!75!black,line width=3pt,line cap=rect]
    (frame.south west) --
    ([xshift=5pt]frame.south west);
  \draw[red!75!black,line width=3pt,line cap=rect]
    (frame.south east) --
    ([xshift=-5pt]frame.south east);
  },
}

\begin{document}
\begin{myboxii}[Disclaimer]
  Aufgaben aus dieser Vorlage stammen aus der Vorlesung \textit{Algorithmen, Sprachen und Komplexität} und wurden zu Übungszwecken verändert oder anders formuliert! Für die Korrektheit der Lösungen wird keine Gewähr gegeben.
\end{myboxii}

%##########################################
\begin{questions}
  \question Definitionen der Automatentheorie. Vervollständige die folgenden Definitionen:
  \begin{parts}
    \part Eine Regel $(l\rightarrow r)$ einer Grammatik $G=(V,\sum,P,S)$ heißt rechtslinear, falls ...
    \begin{solution}
      immer das an der am weitesten rechts stehende Nicht-Terminal in ein Terminal umgewandelt wird.
      Dazu muss $l\in V$ und $r\in \sum V\cup {\epsilon}$.
    \end{solution}

    \part Die Menge $Reg(\sum)$ der regulären Ausdrücke über dem Alphabet ist...
    \begin{solution}
      ist die kleinste Menge mit folgenden Eigenschaften:
      \begin{itemize}
        \item $\varnothing \in Reg(\sum), \lambda \in Reg(\sum), \sum \subseteq Reg(\sum)$
        \item Wenn $\alpha, \beta \in Reg(\sum)$, dann auch $(\alpha * \beta), (\alpha + \beta), (\alpha^*)\in Reg(\sum)$
      \end{itemize}
    \end{solution}

    \part Ein NFA ist ein Tupel $M=(...)$
    \begin{solution}
      ein nichtdeterministischer endlicher Automat $M$ ist ein 5-Tupel $M=(Z,\sum,S,\delta,E)$ mit
      \begin{itemize}
        \item $Z$ ist eine endliche Menge von Zuständen
        \item $\sum$ ist das Eingabealphabet
        \item $S\subseteq Z$ die Menge der Startzustände (können mehrere sein)
        \item $\delta: Z \times \sum \rightarrow P(Z)$ ist die (Menge der) Überführungs/Übergangsfunktion
        \item $E\subseteq Z$ die Menge der Endzustände
      \end{itemize}
    \end{solution}

    \part Die von einem NFA $M=(Z,\sum,S,\delta, E)$ akzeptierte Sprache ist $L(M)=...$ (ohne Definition der Mehr-Schritt Übergangsfunktion $\delta$)
    \begin{solution}
      $L(M)={w\in \sum^* | \hat{\delta}(S,w)\cap E \not = \emptyset}$

      (Das Wort wird akzeptiert wenn es mindestens einen Pfad vom Anfangs in den Endzustand gibt)
    \end{solution}

    \part Die von einem PDA $M=(Z,\sum, \Gamma, \delta, z_0, \#)$ akzeptierten Sprache ist $L(M)=...$
    \begin{solution}
      $L(M)=\{x\in\sum^* | \text{ es gibt } z\in Z \text{ mit } (z_0, x, \#) [...] ^*(z,\epsilon, \epsilon)\}$
    \end{solution}

    \part Sei $L$ eine Sprache. Für $x,y\in\sum^*$ gilt $xR_L y$ genau dann, wenn ... ($R_L$ ist die Myhill-Nerode-Äquivalenz zu L)
    \begin{solution}
      wenn $\forall z \in \sum^* :(xy\in L \leftrightarrow yz \in L)$ gilt
    \end{solution}

    \part Sei $M=(Z,\sum,z_0,\delta, E)$ ein DFA. Die Zustände $z,z'\in Z$ heißen erkennungsäquivalent, wenn
    \begin{solution}
      Zwei Zustände $z,z'\in Z$ heißen erkennungsäquivalent ($z\equiv z'$) wenn für jedes Wort $w\in \sum^*$ gilt: $\hat{\sigma}(z,w)\in E \leftrightarrow \hat{\sigma}(z',w)\in E$.
    \end{solution}
  \end{parts}

  \question Sätze und Lemmas aus der Automatentheorie. Vervollständige die folgenden Aussagen:
  \begin{parts}
    \part Sei $L\supseteq \sum^*$ eine Sprache. Dann sind äquivalent: 1) L ist regulär (d.h. wird von einem DFA akzeptiert), 2)..., 3)...
    \begin{solution}
      \begin{enumerate}
        \item L ist regulär (d.h. von einem DFA akzeptiert)
        \item L wird von einem NFA akzeptiert
        \item L ist rechtslinear (d.h. von einer Typ-3 Grammatik erzeugt)
      \end{enumerate}
    \end{solution}

    \part Die Klasse der regulären Sprachen ist unter anderem abgeschlossen unter folgenden drei Operationen:
    \begin{solution}
      \begin{itemize}
        \item Vereinigung $L=L_0\cup L_1$
        \item Verkettung $L=L_0L_1$
        \item Abschluss $L=L_0^*$
      \end{itemize}
    \end{solution}

    \part Sei $\sum$ ein Alphabet. Die Anzahl der Grammatiken über $\sum$ ist ... und die Anzahl der Sprachen über $\sum$ ist ... .
    \begin{solution}

    \end{solution}

    \part Unter anderem sind folgende (mind. drei) Probleme für kontextfreie Sprachen entscheidbar:
    \begin{solution}

    \end{solution}

    \part Die Klasse der Kontextfreien Sprachen ist abgeschlossen unter den Operationen 1)... und 2)... . Sie ist aber nicht abgeschlossen unter 3)... und 4)... .
    \begin{solution}

    \end{solution}

    \part Der Satz von Myhill-Nerode besagt,...
    \begin{solution}
      Sei L eine Sprache. L ist regulär $\leftrightarrow index(R_L)< \infty$ (d.h. nur wenn die Myhill-Nerode-Äquivalenz endliche Klassen hat).
    \end{solution}

    \part Das Pumping-Lemma für kontextfreie Sprachen ...
    \begin{solution}
      Man versucht auszunutzen, daß eine kontextfreie Sprache von einer Grammatik mit endlich vielen Nichtterminalen erzeugt werden muss. Das bedeutet auch: wenn ein Ableitungsbaum ausreichend tief ist, so gibt es einen Ast, der ein Nichtterminal mehrfach enthält. Die durch diese zwei Vorkommen bestimmten Teilbäume werden ,,gepumpt''.

      Wenn $L$ eine kontextfreie Sprache ist, dann gibt es $n>= 1$ derart, dass für alle $z$ in $L$ mit $|z| >= n$ gilt: es gibt Wörter $u, v , w , x, y$ in SUM mit
      \begin{enumerate}
        \item $z = uvwxy$,
        \item $|vwx| <= n$,
        \item $|vx| >= 1$ und
        \item $uv^i  wx^i y \in L$ für alle $i >= 0$
      \end{enumerate}
    \end{solution}
  \end{parts}

  \question Konstruktionen der Automatentheorie
  \begin{parts}

    \part Betrachte den folgenden NFA X. Berechne einen DFA Y mit $L(X)=L(Y)$.
    \begin{center}
      \begin{tikzpicture}[node distance = 3cm, on grid, auto]
        \node (q1) [state, initial, initial text = {}] {1};
        \node (q2) [state, above right = of q1] {2};
        \node (q3) [state, accepting, below right = of q2] {3};

        \path [-stealth, thick]
        (q1) edge node {a} (q2)
        (q1) edge node {b} (q3)
        (q1) edge [loop above] node {a}()
        (q2) edge node {a} (q3)
        (q2) edge [loop above] node {b}()
        (q3) edge [bend left] node {a} (q2);
      \end{tikzpicture}
    \end{center}
    \begin{solution}
      \begin{center}
        \begin{tikzpicture}[node distance = 3cm, on grid, auto]
          \node (q1) [state, initial, initial text = {}] {1};
          \node (q3) [state, accepting, right = of q1] {3};
          \node (q12) [state, above = of q1] {\{1,2\}};
          \node (q123) [state, accepting, right = of q12] {\{1,2,3\}};

          \path [-stealth, thick]
          (q1) edge node {a} (q12)
          (q1) edge node {b} (q3)
          (q12) edge [loop above] node {b}()
          (q12) edge [bend left] node {a} (q123)
          (q123) edge [bend left] node {a} (q12)
          (q123) edge [loop above] node {b}()
          (q3) edge node {a} (q123)
          (q3) edge [loop right] node {b}()

          ;
        \end{tikzpicture}
      \end{center}
    \end{solution}

    \part Betrachte den folgenden NFA X. Berechne einen DFA Y mit $L(X)=L(Y)$.
    \begin{center}
      \begin{tikzpicture}[node distance = 3cm, on grid, auto]
        \node (q1) [state, initial, initial text = {}] {1};
        \node (q2) [state, above right = of q1] {2};
        \node (q3) [state, accepting, below right = of q2] {3};

        \path [-stealth, thick]
        (q1) edge node {b} (q2)
        (q1) edge node {a} (q3)
        (q1) edge [loop above] node {b}()
        (q2) edge node {a} (q3)
        (q2) edge [loop above] node {a}()
        (q3) edge [bend left] node {b} (q1)
        (q3) edge [bend left] node {a} (q2)
        (q3) edge [loop above] node {b}();
      \end{tikzpicture}
    \end{center}
    \begin{solution}
      \begin{center}
        \begin{tikzpicture}[node distance = 3cm, on grid, auto]
          \node (q1) [state, initial, initial text = {}] {1};
          \node (q2) [state, right = of q12] {2};
          \node (q3) [state, accepting, right = of q1] {3};
          \node (q12) [state, above = of q1] {\{1,2\}};
          \node (q13) [state, accepting, right = of q3] {\{1,3\}};
          \node (q23) [state, accepting, right = of q123] {\{2,3\}};
          \node (q123) [state, accepting, above right = of q12] {\{1,2,3\}};

          \path [-stealth, thick]
          (q1) edge node {a} (q3)
          (q1) edge node {b} (q12)
          (q2) edge node {a} (q23)
          (q2) edge [loop left] node {b}()
          (q3) edge node {a} (q2)
          (q3) edge node {b} (q13)
          (q12) edge node {a} (q123)
          (q12) edge [loop above] node {b}()
          (q13) edge [bend left] node {a} (q3)
          (q13) edge node {b} (q123)
          (q23) edge [loop above] node {a}()
          (q23) edge node {b} (q13)
          (q123) edge node {a} (q23)
          (q123) edge [loop above] node {b}()

          ;
        \end{tikzpicture}
      \end{center}
    \end{solution}

    \part Betrachte den folgenden DFA X. Berechne den minimalen DFA Y mit $L(X)=L(Y)$.
    \begin{center}
      \begin{tikzpicture}[node distance = 3cm, on grid, auto]
        \node (q1) [state, initial, initial text = {}] {1};
        \node (q2) [state, accepting, right = of q1] {2};
        \node (q3) [state, accepting, right = of q2] {3};
        \node (q4) [state, below = of q1] {4};
        \node (q5) [state, accepting, right = of q4] {5};
        \node (q6) [state, right = of q5] {6};

        \path [-stealth, thick]
        (q1) edge node {b} (q2)
        (q1) edge node {a} (q4)
        (q2) edge node {a} (q3)
        (q2) edge node {b} (q5)
        (q3) edge node {b} (q5)
        (q3) edge [loop above] node {a}()
        (q4) edge node {a} (q2)
        (q4) edge [loop left] node {b}()
        (q5) edge node {a,b} (q6)
        (q6) edge node {a} (q3)
        (q6) edge [loop right] node {b}();
      \end{tikzpicture}
    \end{center}
    \begin{solution}
      * erster Schritt: Paare mit mind. einem Endzustand markieren,
      (*) zweiter Schritt: leere Paare die in Endzustandspaar überführen markieren,
      dritter Schritt: unmarkierte Paare verschmelzen

      \begin{tabular}{c|c|c|c|c|c}
        2 & *                     \\
        3 & *   & *               \\
        4 & (*) & * & *           \\
        5 & *   & * & * & *       \\
        6 & (*) & * & * & (*) & * \\\hline
          & 1   & 2 & 3 & 4   & 5
      \end{tabular}

      Keine unmarkierten Paare $\rightarrow$ nicht minimierbar bzw schon minimal
    \end{solution}


    \part Betrachte den folgenden DFA X. Berechne den minimalen DFA Y mit $L(X)=L(Y)$.
    \begin{center}
      \begin{tikzpicture}[node distance = 3cm, on grid, auto]
        \node (q1) [state, accepting, initial, initial text = {}] {1};
        \node (q2) [state, accepting, right = of q1] {2};
        \node (q3) [state, right = of q2] {3};
        \node (q4) [state, accepting, below = of q1] {4};
        \node (q5) [state, right = of q4] {5};
        \node (q6) [state, accepting, right = of q5] {6};

        \path [-stealth, thick]
        (q1) edge node {b} (q2)
        (q1) edge node {a} (q5)
        (q2) edge [loop above] node {b}()
        (q2) edge node {a} (q3)
        (q3) edge [bend left] node {a} (q5)
        (q3) edge node {b} (q6)
        (q4) edge node {a} (q1)
        (q4) edge node {b} (q5)
        (q5) edge [bend left] node {a} (q3)
        (q5) edge [bend left] node {b} (q4)
        (q6) edge [bend left] node {a} (q2)
        (q6) edge node {b} (q5);
      \end{tikzpicture}
    \end{center}
    \begin{solution}
      \begin{tabular}{c|c|c|c|c|c}
        2 & *                   \\
        3 & * & *               \\
        4 & * & * &             \\
        5 & * & * & (*) & *     \\
        6 & * & * & *   & * & * \\\hline
          & 1 & 2 & 3   & 4 & 5
      \end{tabular}
    \end{solution}
  \end{parts}

  \question Algorithmen für reguläre Sprachen. Sei $\sum=\{a,b,c\}$. Gebe einen Algorithmus an, der bei Eingabe eines NFA X entscheidet, ob alle Wörter $\omega\in L(X)$ ungerade Länge besitzen und $abc$ als Infix enthalten.
  \begin{solution}
  \end{solution}

  \question Kontextfreie Sprachen: Sei $\sum=\{a,b,c\}$. Betrachte die Sprache $K=\{a^k b^l c^m|k\leq l \text{ oder } k\leq m\}$.
  \begin{parts}
    \part Zeige, dass $K$ eine kontextfreie Sprache ist.
    \begin{solution}
    \end{solution}
    \part Zeige, dass $L=\sum^*\backslash K$ (Komplement von $L$) nicht kontextfrei ist.
    \begin{solution}
    \end{solution}
    \part Begründe warum $K$ deterministisch kontextfrei ist oder warum nicht.
    \begin{solution}
    \end{solution}
  \end{parts}

  \question Kontextfreie Grammatiken: Sei $\sum=\{a,b,c,\}$
  \begin{parts}
    \part Sei $G$ die kontextfreie Grammatik mit Startsymbol S und der Regelmenge $S\rightarrow AB$, $A\rightarrow aBS|a$ und $B\rightarrow bBa|b|\epsilon$. Überführe G in eine äquivalente Grammatik in Chomsky Normalform.
    \begin{solution}
      Chomsky Normalform hat auf rechter Ableitungsseite nur ein Terminal oder zwei Nicht-Terminale
      \begin{enumerate}
        \item Startzustand \\$S\rightarrow AB$, $A\rightarrow aBS|a$, $B\rightarrow bBa|b|\epsilon$
        \item $\epsilon$-Regel: Menge $M=\{B\}$ der $epsilon$ Terminal-Überführungen kompensieren; \\
              $S\rightarrow AB|A$, $A\rightarrow aBS|a|aS$, $B\rightarrow bBa|b|ba$
        \item Kettenregel: Menge $M=\{(S,A), (S,S), (A,A), (B,B)\}$ von Ketten (Ableitungen auf ein Nicht-Terminal)\\
              $S\rightarrow AB|aBS|a|aS$, $A\rightarrow aBS|a|aS$, $B\rightarrow bBa|b|ba$
        \item Terminale und Nicht-Terminal trennen: \\ $S\rightarrow AB|CBS|C|CS$, $A\rightarrow CBS|C|CS$, $B\rightarrow DBC|b|DC$, $C\rightarrow a$, $D\rightarrow b$
        \item Längen verkürzen: \\$S\rightarrow AB|CX|C|CS$, $A\rightarrow CX|C|CS$, $B\rightarrow DY|b|DC$, $C\rightarrow a$, $D\rightarrow b$, $X\rightarrow BS$, $Y\rightarrow BC$
      \end{enumerate}
    \end{solution}

    \part Sei $G'$ die kontextfreie Grammatik mit Startsymbol $S$ und der Regelmenge \\\begin{center}
      $S\rightarrow AB$, $A\rightarrow CD|CF$, $F\rightarrow AD$, $B\rightarrow c|EB$, $C\rightarrow a$, $D\rightarrow b$, $E\rightarrow c$ \\\end{center} Entscheide mit dem CYK-Algorithmus, ob die Wörter $w_1=aaabbbcc$ oder $w_2=aaabbccc$ von $G'$ erzeugt werden.
    \begin{solution}

      $w_1=aaabbbcc$

      \begin{tabular}{c|c|c|c|c|c|c|c}
        a & a & a & b & b & b & c   & c   \\\hline
        C & C & C & D & D & D & B,E & B,E \\
          &   & A &   &   &   & B         \\
          &   & F &   &   &               \\
          & A &   &   &                   \\
          & F &   &                       \\
        A &   &                           \\
        S &                               \\
        S
      \end{tabular}
      $\Rightarrow w_1$ wird von $G'$ erzeugt

      $w_2=aaabbccc$

      \begin{tabular}{c|c|c|c|c|c|c|c}
        a & a & a & b & b & c   & c   & c   \\\hline
        C & C & C & D & D & B,E & B,E & B,E \\
          &   & A &   &   & B   & B         \\
          &   & F &   &   & B               \\
          & A &   &   &                     \\
          & F &   &                         \\
        A &   &                             \\
        S &                                 \\
      \end{tabular}
      $\Rightarrow w_2$ wird nicht von $G'$ erzeugt
    \end{solution}

    \part Gebe für die Wörter aus b), die von $G'$ erzeugt werden, den Ableitungsbaum an.
    \begin{solution}

      $w_1=aaabbbcc$
      \begin{center}
        \begin{tikzpicture}[node distance = 3cm, on grid, auto]
          \node (q1) [state] {S};

          \node (q2) [state, below left = of q1] {A};
          \node (q3) [state, below right=2cm and 5cm = of q1] {B};
          \node (q4) [state, below left = of q3] {E};
          \node (q6) [below = of q4] {c};
          \node (q5) [state, below right = of q3] {B};
          \node (q7) [below = of q5] {c};
          \node (q8) [state, below left = of q2] {C};
          \node (q9) [below = of q8] {a};
          \node (q10) [state, below right = of q2] {F};
          \node (q11) [state, below left = of q10] {A};
          \node (q12) [state, below right = of q10] {D};
          \node (q13) [below = of q12] {b};
          \node (q14) [state, below left = of q11] {C};
          \node (q15) [below = of q14] {a};
          \node (q16) [state, below right = of q11] {F};
          \node (q17) [state, below left = of q16] {A};
          \node (q18) [state, below right = of q16] {D};
          \node (q19) [below = of q18] {b};
          \node (q20) [state, below left = of q17] {C};
          \node (q21) [state, below right = of q17] {D};
          \node (q22) [below = of q20] {a};
          \node (q23) [below = of q21] {b};

          \path [-stealth, thick]
          (q1) edge (q2)
          (q1) edge (q3)
          (q2) edge (q8)
          (q2) edge (q10)
          (q3) edge (q4)
          (q3) edge (q5)
          (q4) edge (q6)
          (q5) edge (q7)
          (q8) edge (q9)
          (q10) edge (q11)
          (q10) edge (q12)
          (q11) edge (q14)
          (q11) edge (q16)
          (q12) edge (q13)
          (q14) edge (q15)
          (q16) edge (q17)
          (q16) edge (q18)
          (q17) edge (q20)
          (q17) edge (q21)
          (q18) edge (q19)
          (q20) edge (q22)
          (q21) edge (q23)
          ;
        \end{tikzpicture}
      \end{center}

      %$w_2=aaabbccc$
    \end{solution}
  \end{parts}

  \question Definitionen der Berechnbarkeitstheorie. Verfollständige die Definitionen
  \begin{parts}
    \part Ein While Programm ist von der Form...
    \begin{solution}
      \begin{itemize}
        \item $x_i=c, x_i=x_j+c, x_i=x_j-c$ mit $c\in\{0,1\}$ und $i,j\geq 1$ (Wertzuweisung) oder
        \item $P_1,P_2$, wobei $P_1$ und $P_2$ bereits While Programme sind (sequentielle Komposition) oder
        \item while $x_i\not = 0$ do P end, wobei P ein While Programm ist und $i\geq 1$.
      \end{itemize}
    \end{solution}

    \part Ein Loop-Programm ist von der Form
    \begin{solution}
      \begin{itemize}
        \item $x_i := c, x_i := x_j + c, x_i := x_j \div c$ mit $c\in\{0, 1\}$ und $i, j$ (Wertzuweisung) oder
        \item $P_1 ; P_2$, wobei $P_1$ und $P_2$ Loop-Programme sind (sequentielle Komposition) oder
        \item loop $x_i$ do P end, wobei P ein Loop-Programm ist und $i_1$.
      \end{itemize}
    \end{solution}

    \part Eine Turingmaschine ist ein 7-Tupel $M=(Z,\sum,\Gamma,\delta,z_0,\Box, E)$, wobei...
    \begin{solution}
      \begin{itemize}
        \item 7-Tupel $M=(Z,\sum, \Phi, \delta, z_o, \Box, E)$
        \item $\sum$ das Eingabealphabet
        \item $\Phi$ mit $\Phi\supseteq\sum$ und $\Phi\cap Z\not= 0$ das Arbeits- oder Bandalphabet,
        \item $z_0\in Z$ der Startzustand,
        \item $\delta:Z\times\Phi\rightarrow(Z\times\Phi\times\{L,N,R\})$ die Überführungsfunktion
        \item $\Box\in\Phi/\sum$ das Leerzeichen oder Blank und
        \item $E\subseteq Z$ die Menge der Endzustände ist
      \end{itemize}
    \end{solution}

    \part Die von einer Turingmaschine $M$ akzeptierte Sprache ist $L(M)=...$
    \begin{solution}
      $L(M)=\{ w\in\sum^* | \text{es gibt akzeptierte Haltekonfiguration mit } z_0w\Box\vdash_M^* k\}$.
    \end{solution}

    \part Seien $A\subseteq \sum^*$ und B$\subseteq \Gamma^*$. Eine Reduktion von A auf B ist ...
    \begin{solution}
      Eine Reduktion von A auf B ist eine totale und berechenbare Funktion $f:\sum^*\rightarrow\Gamma^*$, so dass für alle $w\in\sum^*$ gilt: $w\in A\leftrightarrow f(x)\in B$. A heißt auf B reduzierbar (in Zeichen $A\leq B$), falls es eine Reduktion von A auf B gibt.
    \end{solution}

    \part Eine Sprache $L$ heißt rekursiv aufzählbar, falls ...
    \begin{solution}
      \begin{itemize}
        \item L ist semi-entscheidbar
        \item L wird von einer Turing-Maschine akzeptiert
        \item L ist vom Typ 0 (d.h. von Grammatik erzeugt)
        \item L ist Bild berechenbarer partiellen Funktion $\sum^*\rightarrow\sum^*$
        \item L ist Bild berechenbarer totalen Funktion $\sum^*\rightarrow\sum^*$
        \item L ist Definitionsbereich einer berechenbaren partiellen Funktion $\sum^*\rightarrow\sum^*$
      \end{itemize}
    \end{solution}

    \part Sei $f:N\rightarrow N$ eine monotone Funktion. Die Klasse $TIME(f)$ besteht aus allen Sprachen L, für die es eine Turingmaschine $M$ gibt mit ...
    \begin{solution}
      \begin{itemize}
        \item M berechnet die charakteristische Funktion von L.
        \item Für jede Eingabe $w\in\sum^*$ erreicht M von der Startkonfiguration $z_0 w\Box$ aus nach höchstens $f(|w|)$ Rechenschritten eine akzeptierende Haltekonfiguration (und gibt 0 oder 1 aus, je nachdem ob $w\not\in L$ oder $w\in L$ gilt).
      \end{itemize}
    \end{solution}
  \end{parts}

  \question Sätze der Berechnbarkeitstheorie: Vervollständige die folgenden Aussagen
  \begin{parts}
    \part Zu jeder Mehrband-Turingmaschine $M$ gibt es ...
    \begin{solution}
      eine Turingmaschine M' die diesselbe Funktion löst
      \begin{itemize}
        \item Simulation mittels Einband-Turingmaschine durch Erweiterung des Alphabets: Wir fassen die übereinanderliegenden Bandeinträge zu einem Feld zusammen und markieren die Kopfpositionen auf jedem Band durch $\ast$.
        \item Alphabetsymbol der Form $(a,\ast,b,\diamond,c,\ast,...)\in(\Phi\times\{\ast,\diamond\})^k$ bedeutet: 1. und 3. Kopf anwesend ($\ast$ Kopf anwesend, $\diamond$ Kopf nicht anwesend)
      \end{itemize}
    \end{solution}

    \part Sei $f:N^k\rightarrow\mathbb{N}$ eine Funktion für ein $k\in\mathbb{N}$. Die folgenden Aussagen sind äquivalent: 1) $f$ ist Turing-berechenbar, 2)..., 3)..., 4)...
    \begin{solution}
    \end{solution}

    \part Sei $L\subseteq \sum^*$ eine Sprache. Sind $L$ und $\sum^*\backslash L$ semi-entscheidbar, dann...
    \begin{solution}
    \end{solution}

    \part Der Satz von Rice lautet...
    \begin{solution}
      dass es unmöglich ist, eine beliebige nicht-triviale Eigenschaft der erzeugten Funktion einer Turing-Maschine (oder eines Algorithmus in einem anderen Berechenbarkeitsmodell) algorithmisch zu entscheiden.

      Es sei $\mathcal{P}$ die Menge aller partiellen Turing-berechenbaren Funktionen und $\mathcal{S}\subsetneq\mathcal{P}$ eine nicht-leere, echte Teilmenge davon. Außerdem sei eine effektive Nummerierung vorausgesetzt, die einer natürlichen Zahl $n\in\mathbb{N}$ die dadurch codierte Turing-Maschine $M_{n}$ zuordnet. Dann ist die Menge $\mathcal{C}(\mathcal{S})=\{n\mid \text{die von } M_n \text{ berechnete Funktion liegt in }\mathcal{S}\}$ nicht entscheidbar.

      ,,Sei U eine nicht-triviale Eigenschaft der partiellen berechenbaren Funktionen, dann ist die Sprache $L_U=\{⟨M⟩\mid \text{ M berechnet } f\in U\}$ nicht entscheidbar.''
    \end{solution}
  \end{parts}

  \question Berechnungsmodelle
  \begin{parts}
    \part Gebe ein Loop-Programm an, das die Funktion $n\rightarrow n^2-n$ berechnet
    \begin{solution}
      \begin{lstlisting}
      h= 1
      for(i= 0; i < 2; i++) do {
        h= h * n
      }
      h= h - 1;
      return h
      \end{lstlisting}
    \end{solution}

    \part Gebe ein Loop Programm an, das die Funktion $f:\mathbb{N}\rightarrow \mathbb{N}$ mit $f(n_1,n_2)=2n_1n_2$ berechnet. Verwende nur elementare Anweisungen und keine Abkürzungen.
    \begin{solution}
      \begin{lstlisting}
        h= 1
        for (i=0; i<2; i++) do {
          h = h * n_i
        }
        h = 2 * h
        return h
      \end{lstlisting}
    \end{solution}

    \part Gebe ein GoTo Programm an, das die Funktion $g:\mathbb{N}\rightarrow\mathbb{N}$ mit $g(n_1,n_2)=|n_1-n_2|$ berechnet. Verwende nur elementare Anweisungen und keine Abkürzungen.
    \begin{solution}
      \begin{lstlisting}
        start: 
          h = n_1 - n_2
          if h > 0:
            goto end 
          h = -h
        end: 
          return h
      \end{lstlisting}
    \end{solution}

    \part Gebe eine deterministische Turingmaschine $M$ für das Eingabealphabet $\{0,1\}$ an, das folgende Funktion berechnet: Für Eingabe $a_1a_2...a_{n-1}a_n$ berechnet M die Ausgabe $a_na_1...a_{n-1}$ (letzte Symbol der Eingabe an erste Stelle).
    \begin{solution}
      $\sum=\{0,1\}$
      
      $z_0$ Zahlenende finden: $\delta(z_0,0)=(z_0,0,R), \delta(z_0,1)=(z_0,1,R), \delta(z_0,\Box)=(z_1,\Box,L)$

      $z_1$ letzte Zahl löschen: $\delta(z_1,0)=(z_2,\Box,L), \delta(z_1,1)=(z_3,\Box,L), \delta(z_1,\Box)=(z_2,\Box,N)$

      $z_2$ zurück zum Anfang bei $a_n=0$: $\delta(z_2,0)=(z_2,0,L), \delta(z_2,1)=(z_2,1,L), \delta(z_2,\Box)=(z_4,\Box,R)$

      $z_3$ zurück zum Anfang bei $a_n=1$: $\delta(z_2,0)=(z_2,0,L), \delta(z_2,1)=(z_2,1,L), \delta(z_2,\Box)=(z_5,\Box,N)$

      $z_4$ $a_n=0$ an Anfang schreiben: $\delta(z_4,\Box)=(z_e,0,N)$

      $z_5$ $a_n=1$ an Anfang schreiben: $\delta(z_4,\Box)=(z_e,1,N)$

      $z_e$ Endzustand: $\delta(z_e, 0)=(z_e,0,N), \delta(z_e,1)=(z_e,1,N), \delta(z_e,\Box)?(z_e,\Box,N)$
    \end{solution}
  \end{parts}

  \question Reduktionen
  \begin{parts}
    \part Seien $A,L\subseteq \sum^*$ nichtleere Sprachen und A entscheidbar. Gebe eine Reduktion von $L\cup A$ auf $L$ an.
    \begin{solution}
    \end{solution}
    \part Gebe eine Bedingung für A an, sodass $L\cup A\leq_p L$ für alle nichtleeren Sprachen $L\subseteq \sum^*$ gilt. Begründe.
    \begin{solution}
    \end{solution}
  \end{parts}

  \question Komplexitätsklassen. Ergänze zu den Paaren von Komplexitätsklassen das Relationssymbol zur Teilmengenbeziehung.
  \begin{parts}
    \part EXPSPACE ? EXPTIME
    \begin{solution}
      EXPSPACE $\geq$ EXPTIME
    \end{solution}
    \part NP ? P
    \begin{solution}
      NP $\geq$ P
    \end{solution}
    \part NP ? NPSPACE
    \begin{solution}
      NP $\leq$ NPSPACE
    \end{solution}
    \part NPSPACE ? PSPACE
    \begin{solution}
      NPSPACE $=$ PSPACE
    \end{solution}
  \end{parts}

  \question Unentscheidbare Probleme: Gebe (mind vier) unterscheidbare Probleme an (als Menge oder als Eingabe-Frage-Paar).
  \begin{solution}
    
    das spezielle Halteproblem

    das allgemeine Halteproblem

    das Halteproblem auf leerem Band

    das allgemeine Wortproblem $A=\{(G,w) | \text{ G ist Grammatik mit } w\in L(G)\}$

    Posts Korrespondenzproblem

    das Schnitt- und verwandte Probleme über kontextfreie Sprachen
  \end{solution}

  \question NP-vollständiges Problem: Gebe (mind. zwei) NP-vollständige Probleme an (als Menge oder Eingabe-Frage-Paar).
  \begin{solution}

    Hamilton Kreis
    \begin{itemize}
      \item Eingabe: Graph(V,E)
      \item Frage: Kann der Graph so durchlaufen werden, dass jeder Knoten genau ein mal besucht/abgelaufen wird?
    \end{itemize}

  \end{solution}

  \question Polynomialzeitreduktion: Betrachte das Problem 4C, also die Menge der ungerichteten Graphen die sich mit vier Farben färben lassen.
  \begin{parts}
    \part Gebe eine Polynomialzeitreduktion von 3C auf 4C an.
    \begin{solution}
    \end{solution}

    \part Zeige, dass wenn $4C\in P$, dann gilt $P=NP$.
    \begin{solution}
    \end{solution}
  \end{parts}
\end{questions}
\end{document}