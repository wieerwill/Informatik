\documentclass[10pt, a4paper]{article}
\usepackage[ngerman]{babel}
\usepackage{multicol}
\usepackage{enumitem,amsmath,amsthm,amsfonts,amssymb}
\usepackage[top=1in, bottom=1in, left=0.75in, right=0.75in]{geometry}
\usepackage{color,graphicx,overpic}
\usepackage{hyperref}
\usepackage{mdwlist} %less space for lists
\usepackage{tikz}
% Turn off header and footer
\pagestyle{empty}
% Don't print section numbers
\setcounter{secnumdepth}{0}

\newlist{todolist}{itemize}{2}
\setlist[todolist]{label=$\square$}

\pdfinfo{
    /Title (Computergrafik - Übungsklausur)
    /Creator (TeX)
    /Producer (pdfTeX 1.40.0)
    /Subject ()
}
\title{Computergrafik - Übungsklausur}
\date{}
\begin{document}
\maketitle
\textbf{Für die Klausur sind keinerlei Hilfsmittel wie Skript, Bücher oder Taschenrechner zulässig. \newline Bei Berechnungen reicht die Angabe der Herleitung aus, z.B. reicht $8^3$ für $512$.}

\section{Objekt- und Ansichtstransformationen}
\subsection{Aufgabe 1\newline Der Punkt P mit den Koordinaten $(x,y)$ soll erst um den Vektor $(\delta x,\delta y)$ verschoben und anschließend um den Winkel $\alpha$ um den Ursprung rotiert werden.}
\subsubsection{a) Geben Sie die beteiligten Transformationsmatrizen in allgemeiner Form an.}
\begin{center}
    \framebox(450,100){}
\end{center}

\subsubsection{b) Leiten Sie die beiden Gleichungen für die Koordinaten $(x',y')$ des transformierten Punktes $P'$ her, wenn zuerst verschoben und danach rotiert wird.}
\begin{center}
    \framebox(450,100){}
\end{center}

\subsection{Aufgabe 2\newline Die nachstehende Skizze veranschaulicht einen Abbildungsprozess, bei dem ein Punkt P auf die Projektionsebene E projiziert wird. E befindet sich im Abstand e vom Ursprung entfernt und ist parallel zur xy-Ebene.}
\begin{figure}[h]
    \centering
    \begin{tikzpicture}
        \draw[thick,->] (0,0) -- (0,4) node[anchor=north west] {X};
        \draw[thick,->] (0,0) -- (6,0) node[anchor=south east] {Z};
        \draw (0,0) node[anchor=south east] {Y};

        \draw[thick,-] (0,0) -- (5,3) node[anchor=north west] {P};
        \draw[thick,-] (0,0) -- (3,1.8) node[anchor=north west] {P'};
        \draw[thin,-] (5,3)--(0,3) node[anchor=east] {$x_{P}$};
        \draw[thin,-] (3,1.8)--(0,1.8) node[anchor=east] {$x_{P'}$};
        \draw[thin,-] (5,3)--(5,0) node[anchor=north] {$z_P$};
        \draw[thick,-] (3,3)--(3,0) node[anchor=north] {$e$};
    \end{tikzpicture}
\end{figure}
\subsubsection{a) Um welche Projektionsart handelt es sich?}
\begin{center}
    \framebox(450,100){}
\end{center}
\subsubsection{b) Geben Sie eine Formel an, mit der $x'_P$ berechnet werden kann.}
\begin{center}
    \framebox(450,100){}
\end{center}
\subsubsection{c) Geben Sie eine entsprechende Formel an, mit der $y'_P$ berechnet werden kann.}
\begin{center}
    \framebox(450,100){}
\end{center}


\section{Farbmodelle und Farbwahrnehmung}
\subsection{Aufgabe 3\newline Welche Farbe nehmen Sie wahr, wenn... \newline Begründen Sie jeweils Ihre Antwort.}
\subsubsection{a) grünes Licht auf eine gelbe Oberfläche fällt?}
\begin{center}
    \framebox(450,100){}
\end{center}
\subsubsection{b) magentafarbenes Licht auf eine gelbe Oberfläche fällt?}
\begin{center}
    \framebox(450,100){}
\end{center}
\subsubsection{c) weißes Licht auf eine gelbe Oberfläche fällt?}
\begin{center}
    \framebox(450,100){}
\end{center}


\subsection{Aufgabe 4\newline Kreuzen Sie jeweils ja oder nein an. Jede richtige Antwort gibt $0.5$ Punkte, jede falsche Antwort gibt $-0.5$ Punkte. Wenn Sie sich nicht sicher sind, lassen Sie das Feld frei oder erläutern Sie Ihre Entscheidung ausführlich. Es werden zwischen 0 und 15 Punkte vergeben.}
\subsubsection{a) Welche Räume stellen geräteabhängige Farbräume dar?}
\begin{multicols}{3}
    \begin{todolist}
        \item $CIE_{LAB}$
        \item CMYK
        \item HSI
        \item RGB
        \item XYZ
    \end{todolist}
\end{multicols}

\subsubsection{b) Welchen Farbumfang des sichtbaren Lichtes kann ein typischer Monitor darstellen?}
\begin{multicols}{3}
    \begin{todolist}
        \item $<5\%$
        \item ca. 1/6
        \item ca. 1/3
        \item ca. 80-90\%
        \item $>99\%$
    \end{todolist}
\end{multicols}

\subsubsection{c) Welches Verfahren wird angewendet, um Farbkörperunterschiede von Geräten auszugleichen?}
\begin{multicols}{3}
    \begin{todolist}
        \item Anti-Aliasing
        \item Automatischer Weißabgleich
        \item Double Buffering
        \item Ersatzfarbenbildung
        \item Gamnt Mapping
    \end{todolist}
\end{multicols}

\subsubsection{d) Das Lambert-Beersche Gesetz beschreibt den Grad der Abschwächung beim Durchgang von Strahlung durch eine lichtabsorbierende Substanz. Als Parameter geht in das Gesetz ein:}
\begin{multicols}{3}
    \begin{todolist}
        \item Einfallswinkel
        \item Konzentration
        \item Lichtgeschwindigkeit
        \item Raumtemperatur
        \item Schichtdicke
    \end{todolist}
\end{multicols}

\subsubsection{e) In welchen Fällen leigt stets maximale Sättigung (gesättigte Farbvalenz) vor? Die skalaren Werte in $F=(R,G,B)$ seien auf $[0,1]$ normiert.}
\begin{multicols}{3}
    \begin{todolist}
        \item $|F|>0$, aber mindestens ein Farbwert $=0$
        \item $R=G=B=1/3$
        \item $R=0$, $G=0.7$ und $B=0.5$
        \item Für Intensitäten $>0.5$
        \item Für Intensitäten $=1$
    \end{todolist}
\end{multicols}

\subsubsection{f) Bewerten Sie die folgenden bezüglich des CMY-Farbmodells getroffenen Aussagen.}
\begin{multicols}{3}
    \begin{todolist}
        \item Modifikation eines Parameters genügt, um einen Rotstich zu beseitigen
        \item Modifikation eines Parameters genügt, um die Farben aufzuhellen
        \item Schwarz liegt im Koordinatenursprung
        \item Die Koordinaten $(0,1,0)$ charakterisieren weiß
        \item Der Farbraum wird aus genau drei linear unabhängigen Größen gebildet
    \end{todolist}
\end{multicols}

\section{2D Rastergrafik}
\subsection{Aufgabe 5}
\subsubsection{a) Geben Sie ein Beispiel (Menge von Polygonen bzw. im einfachsten Fall Dreiecken), das vom painteralgorithmus fehlerhaft gerendert wird und erläutern Sie, warum das Problem in diesem Beispiel auftritt.}
\begin{center}
    \framebox(450,100){}
\end{center}
\subsubsection{b) Erläutern Sie die Z-Buffer-Methode. Gehen Sie dabei auch darauf ein, wie das von Ihnen unter a) gebrachte Beispiel gerendert wird.}
\begin{center}
    \framebox(450,100){}
\end{center}
\subsubsection{Zusatzaufgabe: Wie eignen sich die Painters-Algorithmus und Z-Buffer-Methode zum Rendern transparenter Polygone? Welche Probleme treten jeweils auf, und wie lassen sie sich beheben?}
\begin{center}
    \framebox(450,100){}
\end{center}

\section{3D Rendering}
\subsection{Aufgabe 6\newline erläutern Sie das Gourad-Shading-Verfahren zum Schattieren von Dreiecken.}
\subsubsection{a) Wie werden die Intensitäten der einzelnen Pixel bestimmt?}
\begin{center}
    \framebox(450,100){}
\end{center}
\subsubsection{b) Welche Möglichkeiten bestehen, um den Machband-Effekt abzuschwächen bzw. ganz zu vermeiden?}
\begin{center}
    \framebox(450,100){}
\end{center}
\subsubsection{c) In welchen Fällen werden beim Gourad-Shading Glanzlichter nicht korrekt wiedergegeben (Skizze und Erläuterung)?}
\begin{center}
    \framebox(450,100){}
\end{center}

\section{Effiziente Datenstrukturen}
\subsection{Aufgabe 7\newline Acht Punkte A bis H sollen in einem zweidimensionalen kd-Tree gespeichert werden. Jede Raumzelle enthalte maximal einen Punkt. Für zwei verschiedene Einfügestrategien sind sowohl die Raumzerlegung zu skizzieren als auch der zugehörige kd-Tree zu zeichnen.}
\subsubsection{a) Zeichnen Sie kd-Baum und Raumzerlegung bei Median-Teilung. Die erste Teilung erfolge parallel zur x-Achse. Die Teilungsachse verlaufe entlang der Koordinate des jeweiligen Punktes bzw. entlang des Mittelwerts zwischen zwei Punkten.}
\subsubsection{b) Zeichnen Sie kd-Baum und Raumzerlegung bei Einfügung der Punkte in der Reihenfolge $A,B,...,H$. Die erste Teilung erfolge parallel zur x-Achse. Die Teilungsachse verlaufe entlang der Koordinate des jeweiligen Punktes.}

\begin{tikzpicture}[
        dot/.style = {
                draw,
                fill = white,
                circle,
                inner sep = 0pt,
                minimum size = 4pt
            }
    ]
    \draw[thick,->] (0,0) -- (10,0) node[anchor=south west] {Y};
    \draw[thick,->] (0,0) -- (0,10) node[anchor=south west] {X};
    \foreach \x in {0,1,2,3,4,5,6,7,8,9,10}
    \draw (\x cm,1pt) -- (\x cm,-1pt) node[anchor=north] {$\x$};
    \foreach \y in {0,1,2,3,4,5,6,7,8,9,10}
    \draw (1pt,\y cm) -- (-1pt,\y cm) node[anchor=east] {$\y$};
    \draw[thin,-] (10,0) -- (10,10);
    \draw[thin,-] (0,10) -- (10,10);

    \draw (6,9) node[dot, label={above:$A$}]{};
    \draw (9,2) node[dot, label={above:$B$}]{};
    \draw (8,8) node[dot, label={above:$C$}]{};
    \draw (7,6) node[dot, label={above:$D$}]{};
    \draw (3,7) node[dot, label={above:$E$}]{};
    \draw (2,1) node[dot, label={above:$F$}]{};
    \draw (4,4) node[dot, label={above:$G$}]{};
    \draw (1,3) node[dot, label={above:$H$}]{};
\end{tikzpicture}

\end{document}