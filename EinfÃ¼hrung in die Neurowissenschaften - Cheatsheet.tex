\documentclass[a4paper]{article}
\usepackage[ngerman]{babel}
\usepackage[utf8]{inputenc}
\usepackage{multicol}
\usepackage{calc}
\usepackage{ifthen}
\usepackage[landscape]{geometry}
\usepackage{amsmath,amsthm,amsfonts,amssymb}
\usepackage{color,graphicx,overpic}
\usepackage{xcolor, listings}
\usepackage[compact]{titlesec} %less space for headers
\usepackage{mdwlist} %less space for lists
\usepackage{pdflscape}
\usepackage{verbatim}
\usepackage[most]{tcolorbox}
\usepackage[hidelinks,pdfencoding=auto]{hyperref}
\usepackage{bussproofs}
\usepackage{fancyhdr}
\usepackage{lastpage}
\pagestyle{fancy}
\fancyhf{}
\fancyhead[L]{Einführung in die Neurowissenschaften}
\fancyfoot[L]{\thepage/\pageref{LastPage}}
\renewcommand{\headrulewidth}{0pt} %obere Trennlinie
\renewcommand{\footrulewidth}{0pt} %untere Trennlinie

\pdfinfo{
  /Title (Einführung in die Neurowissenschaften)
  /Creator (TeX)
  /Producer (pdfTex 1.40.0)
  /Author (Robert Jeutter)
  /Subject ()
}

%%% Code Listings
\definecolor{codegreen}{rgb}{0,0.6,0}
\definecolor{codegray}{rgb}{0.5,0.5,0.5}
\definecolor{codepurple}{rgb}{0.58,0,0.82}
\definecolor{backcolour}{rgb}{0.95,0.95,0.92}
\lstdefinestyle{mystyle}{
 backgroundcolor=\color{backcolour}, 
 commentstyle=\color{codegreen},
 keywordstyle=\color{magenta},
 numberstyle=\tiny\color{codegray},
 stringstyle=\color{codepurple},
 basicstyle=\ttfamily,
 breakatwhitespace=false, 
}
\lstset{style=mystyle, upquote=true}

%textmarker style from colorbox doc
\tcbset{textmarker/.style={%
 enhanced,
 parbox=false,boxrule=0mm,boxsep=0mm,arc=0mm,
 outer arc=0mm,left=2mm,right=2mm,top=3pt,bottom=3pt,
 toptitle=1mm,bottomtitle=1mm,oversize}}

% define new colorboxes
\newtcolorbox{hintBox}{textmarker,
 borderline west={6pt}{0pt}{yellow},
 colback=yellow!10!white}
\newtcolorbox{importantBox}{textmarker,
 borderline west={6pt}{0pt}{red},
 colback=red!10!white}
\newtcolorbox{noteBox}{textmarker,
 borderline west={3pt}{0pt}{green},
 colback=green!10!white}

% define commands for easy access
\renewcommand{\note}[2]{\begin{noteBox} \textbf{#1} #2 \end{noteBox}}
\newcommand{\warning}[1]{\begin{hintBox} \textbf{Warning:} #1 \end{hintBox}}
\newcommand{\important}[1]{\begin{importantBox} \textbf{Important:} #1 \end{importantBox}}


% This sets page margins to .5 inch if using letter paper, and to 1cm
% if using A4 paper. (This probably isn't strictly necessary.)
% If using another size paper, use default 1cm margins.
\ifthenelse{\lengthtest { \paperwidth = 11in}}
 { \geometry{top=.5in,left=.5in,right=.5in,bottom=.5in} }
 {\ifthenelse{ \lengthtest{ \paperwidth = 297mm}}
 {\geometry{top=1.3cm,left=1cm,right=1cm,bottom=1.2cm} }
 {\geometry{top=1.3cm,left=1cm,right=1cm,bottom=1.2cm} }
 }

% Redefine section commands to use less space
\makeatletter
\renewcommand{\section}{\@startsection{section}{1}{0mm}%
 {-1ex plus -.5ex minus -.2ex}%
 {0.5ex plus .2ex}%x
 {\normalfont\large\bfseries}}
\renewcommand{\subsection}{\@startsection{subsection}{2}{0mm}%
 {-1explus -.5ex minus -.2ex}%
 {0.5ex plus .2ex}%
 {\normalfont\normalsize\bfseries}}
\renewcommand{\subsubsection}{\@startsection{subsubsection}{3}{0mm}%
 {-1ex plus -.5ex minus -.2ex}%
 {1ex plus .2ex}%
 {\normalfont\small\bfseries}}
\makeatother

% Don't print section numbers
\setcounter{secnumdepth}{0}

\setlength{\parindent}{0pt}
\setlength{\parskip}{0pt plus 0.5ex} 
% compress space
\setlength\abovedisplayskip{0pt}
\setlength{\parskip}{0pt}
\setlength{\parsep}{0pt}
\setlength{\topskip}{0pt}
\setlength{\topsep}{0pt}
\setlength{\partopsep}{0pt}
\linespread{0.5}
\titlespacing{\section}{0pt}{*0}{*0}
\titlespacing{\subsection}{0pt}{*0}{*0}
\titlespacing{\subsubsection}{0pt}{*0}{*0}

\begin{document}

\raggedright
\begin{multicols}{3}\scriptsize
  % multicol parameters
  % These lengths are set only within the two main columns
  %\setlength{\columnseprule}{0.25pt}
  \setlength{\premulticols}{1pt}
  \setlength{\postmulticols}{1pt}
  \setlength{\multicolsep}{1pt}
  \setlength{\columnsep}{2pt}


  Eigenschaft Nervensystem? Reizbarkeit

  andere funktionelle System zu NS? Hormonsystem (endocrines)

  Untersysteme des NS? Herz/Kreislauf, Atmung, Verdauung, Haut, Urogenital, Muskulatur

  Kommunikation mit äußerer Umwelt? Sensomotorische NS

  Kommunikation mit anderen Organsystemen? autonomes NS

  Grundbestandteile ZNS? Gehirn (Cerebrum, Pons, Cerebellum), Rückenmark (Spinal Cord, conus medullaris, region of cauda equina)

  Hauptbestandteile autonomes NS? Symphatikus, Parasympathikus, Zentraler Teil, Intramurale Plexus

  Untersystem autonomes NS Flucht/Kamp? Sympathikus

  Zelltypen im Nervengewebe? Neuronen, Glia

  Gewebe mit neuronalen Zellkörper? Ganglien, Plexus

  Nervengewebe im Rückenmark außen? weiße substanz

  funktionellen Merkmale von Neuronen? verbunden durch Nervenfasern, Informationstransfer elektrisch \& chemisch

  Geschwindigkeit Aktionspotentiale? $0,3 - 100$ m/s

  Aktionspotential ausgelöst? Membranpotential -65mV am Axonhügel

  Membran-Ruhepotential Neuron? -70mV

  Antriebskräfte für Ionentransport? Diffusion durch Konzentrationsgradienten, elektrischer Ionenstrom durch Potentialgradienten, aktiver Ionenaustausch durch Ionenpumpen

  Ionenkanäle ausgelöst? Natriumkanäle, dann Kaliumkanäle

  maximale Impulsrate auf Axonen? 500/s

  Aktionspotentiale nur in eine Richtung? Na+ Refraktärzeit, verhindern Zurücklaufen

  Abhängigkeit Aktionspotentiale? Durchmesser Axon, Myelinschicht um Axon $\rightarrow$ saltatorische Erregungsleitung

  Saltatorischer Erregungsleitung? axonale Erregungsleitung springt von Schnürring zu Schnürring

  Zelltyp Myelinscheiden im Zentral-/Perphernervensystem? Oligodendrozyten im ZNS und Schwann-Zellen in der Peripherie

  Krankheit beeinträchtigt Myelinscheiden? Multiple Sklerose

  Neurotransmitter in synaptischen Endknöpfen? Vesikeln

  informationsverarbeitende Synapse? Diodenfunktion, Transistorfunktion

  Integration von Information in Neuron? räumliche/zeitliche Dim.

  Neurotransmitter? Substanzen, an chemischen Synapsen ausgeschüttet, beeinflussen andere Zellen % (Neuronen, Muskelzellen, etc.)

  Merkmale Neurotransmittern? in präsynaptischen Endknöpfen synthetisiert, in großer Menge freigesetzt, können mechanisch entfernt werden, selbe Wirkung bei exogener Applikation

  Arten von Neurorezeptoren? ionotrope/ metabotrobe Rezeptoren

  Funktion ionotroper Rezeptor? Chemisch gesteuerte Ionenkanäle öffnet/schließt in der postsynaptischen Membran, induziert postsynaptische Potential

  Funktion metabotroper Rezeptor? langsam, variabel, bindet an Ionenkanal und löst AP oder Synthese weiterer Botenstoffe

  Häufiger Rezeptoren? metabotrope Rezeptoren

  Neurotransmitter? Dopamin, Epinephrin, Histamin, GABA, Glutamat, Serotonin, Acetylcholin

  Monoamine als Neurotransmitter? Tyrosin, Histidin, Phenylalanin

  Aminosäuren als Neurotransmitter? Glutamat, GABA, Glycin

  %Nennen Sie den wichtigsten erregenden und den wichtigsten hemmenden Neurotransmitter im Gehirn!

  %Welches sind die drei wichtigen Orte mit dopaminergen Neuronen im Gehirn?

  %Welches der drei wichtigsten dopaminergen Systeme interagiert eng mit dem neuroendokrinologischen System?

  Hirnareal enthält noradrenerge Neuronen? Locus coeruleus

  Wo serotonerge Neuronen? im Hirnstamm, in den Raphé-Kernen

  %Nennen Sie zwei wichtige Beispiele für cholinerge Übertragung!

  Gruppen cholinerger Rezeptoren? Muscarinische (metabotrop), nicotinische (ionotrop)

  Substanzen synaptische Übertragung? Inhibitor (hemmend), Aktivator (fördernd)

  Wirkmechanismen Agonisten? Steigerung NT-Freisetzung, NT Menge $\uparrow$, NT Synthese $\uparrow$, Blockierung von Abbau oder Wiederaufnahme von NT, Bindung an und Aktivierung von postsynaptischen Rezeptoren

  Wirkmechanismen Antagonisten? NT Synthese $\downarrow$, Austreten NT aus Vesikeln, Blockierung NT-Freisetzung, Bindung und Blockierung von postsynaptischen Rezeptoren

  Beispiel Antagonisten? Atropin, M1-3 Acetylcholin-Rezeptor

  %Beispiel Agonist?

  Anwendungsgebiete Atropin? Erweiterung Pupillen, Gegengift cholinerge Agonisten, Hemmung Magen/Darmaktivität, Kreislaufstillstand

  Typen von Gliazellen? Microgliazyten, Astrozyten, Ependymzellen, Oligodendrogliazyten, (Schwann-Zellen)

  Merkmale von Mikroglia? Vielfältige Formen, Amöboid beweglich, Abräum- und Abwehrfunktion

  Merkmale Astrozyten? Kurzstrahlige Astrozyten in grauer Substanz, lnagstrahlige Astrozyten in weißer Substanz, Gliafüßchen bilden geschlossene Schicht um Kapillaren, Kontrolle Ionen- und Flüssigkeitsgleichgewicht, Stütz- und Transportfunktion, Abgrenzfunktion, teilungsfähig und bilden Glianarben

  Gliazellen für Blut-Hirn-Schranke? Astrozyten

  Merkmale Oligodendrozyten? Eng an Neuronen angelagert, Stoffwechselfunktion für Neuronen, bilden Markscheide für ZNS-Neuronen

  Myelinscheide im peripheren NS gebildet? Schwann-Zellen

  Typen in motorischen Endplatten? Transmitter: ACh, Rezeptor: nikotinische ACh-Rezeptoren

  Gliazellen direkt an Informationsverarbeitung im Gehirn? Ja, 10-50 mal mehr als Neuronen%, direkt am Prozess der Informationsverarbeitung, -speicherung und -weiterleitung im Nervensystem beteiligt

  Richtungsbezeichnungen? caudal-hinten, dorsal-oben, ventral-unten, rostal- vorn, anterior-vorn, medial-innen, lateral-außen

  proximal: zum Rumpf hin gelegen

  distal: vom Körperzentrum weg gelegen

  Hauptabschnitte Gehirn? Telencephalon, Diencephalon, Mesencephalon, Metencephalon, Myelencephalon, Rückenmark

  Wieviele Hirnnervenpaare? 12 Hirnveenenpaare

  Hirnnerv entspringt im Telencephalon? N. olfactorius (sensorisch: riechen)

  Hirnnerv entspringt im Diencephalon? N. opticus (sensorisch: Sehen)

  Funktion N. trigenimus? sensorisch: Gesicht, Nase, Mund, Zunge; motorisch: kauen

  Funktion N. vestibulocochlearis? sensorisch: Gleichgewicht, Hören

  Funktion N. vagus? Motorisch (parasympathisch): Eingeweide; motorisch: Kehlkopf, Rachen; sensorisch: Kehlkopf, Rachen

  Hirnfunktion in Medulla oblongata? Atem- und Kreislaufzentrum; Zentren für Nies-, Huste-, Schluck-, Saug- und Brechreflex; formatio reticularis

  Hirnteil für Überleben unverzichtbar? Medulla

  retikuläre Formation? Zieht durch Medulla, Pons und Mesencephalon/Diencephalon

  Funktion retikulärer Formation? Zeitliche Koordination des gesamten Nervensystems; Atmung, Kreislauf, Muskeltonus; Moduation von Schmerzempfinden und Emotion, Schlaf-Wach-Rhythmus, Aufmerksamkeit

  Wo Pons? Zwischen Mesencephalon und Myelencephalon%; bildet mit Cerebellum das MEtencephalon, ist von diesem durch das (4) Ventrikel getrennt

  Was zwischen Pons und Cerebellum? Teile des 4.Hirnventrikels, Rautengrube

  %Wo befinden sich Zellkörper und Axone cerebellarer Neuronen?

  %cerebellaren Neuronentypen und ordnen Sie diese anhand der Lage ihrer Zellkörper den entsprechenden Cortexschichten zu!

  Funktion Cerebellums? Feedforward-Verarbeitung, Divergenz und Konvergenz, Modularität, Plastizität

  Symptome cerebellarer Störungen? Ataxie (Störung Bewegungskoordination), Nystagmus (Augenzittern), Rumpfataxie (Unfähigkeit aufrecht zu erhalten), Tremor, undeutliche Aussprache, Störungen im Bewegungsablauf

  Wo Mittelhirn? zwischen Pons und Diencephalon

  Hauptabschnitte Mittelhirn? Tectum, Tegmentum

  inferioren und superioren Colliculi? Tectum (Mittelhirndach, Vierhügelplatte)

  Neurotransmitter der Substantia nigra? Dopamin

  Krankheit Störungen in Substantia nigra? Morbus Parkinson

  Funktion des Thalamus? ,,Eingangskontrolle'' Großhirn, Umschaltstation sensorischer Informationen

  Funktionen Hypothalamus? Regelung Körpertemperatur, Wasser, Mineralhaushaltes, Hormonausschüttung, Appetit, Schlaf, Sexualtrieb, Aggression, Flucht

  Regulierungszentrum autonomen NS? Hypothalamus

  Quellen für Afferenzen Hypothalamus? Limbisches System, Sensorische Informationen über interne/externe Umgebung

  %Nennen Sie die 5 grundsätzlichen Efferenzen des Hypothalamus!

  funktionelles System laterale/medialen Kniehöcker? Metathalamus

  Hauptabschnitte des Großhirns? Großhirnhälften, Basalganglien

  Kommissuren verbinden? Beide Gehirnhälften

  Großhirnlappen? Frontal (Lobus frontalis), Schläfen (L. temperalis), Hinterhaupt (L. occipitalis), Scheitel (L. parietalis)

  %Was verbinden Projektions- und Assoziationsbahnen?

  %Welche histologischen und phylogenetischen Cortextypen gibt es?

  Schichten Cortex? Isocortex: 6, Allocortex: 3

  Cortexart nimmt meiste Fläche ein? Isocortex

  Strukturen Basalganglien? Nucleus caudatus, Putamen, Globus pallidus, Amygdala

  Strukturen unter Striatum? Nucleus caudatus, Putamen

  Funktionen Amygdala? wichtig bei Emotionen, insb. Angst und Furcht

  Was in Weiße Masse? Nervenfasern und Glia

  Hirnhäute? Dura mater, Arachnoidea, Pia mater

  Hirnhaut grenzt an Cortex? Pia mater

  Hirnhaut grenzt an Schädel? Dura mater

  Arterien Blutzufuhr zum Gehirn? 6 Stück

  Struktur Ausfall der zuführenden Arterien ausgleichen? Ring in Hirnbasis

  Anzahl Hirnventrikel? 5

  Struktur bildet Nervenwasser? In Ventrikeln (durch Kapillargeflechte der Pia mater)

  Wo Nervenwasser resorbiert? Arachoidalzotten im Sinus sagittalis superior

  Wo weiße/graue Masse im Rückenmark? weiße Masse außen, graue Masse innen

  Wo endet Rückenmark? Obere Lendenwirbelsäule

  Grundfunktionen Rückenmarks? Verbindung Gehirn-(größter Teil) Körper, Implementierung somatomotorischer/viszeraler Reflexe

  Wie viele Spinalnervenpaare? 31 Paare

  Dermatom? Assoziation zwischen Körperoberfläche und Spiralnerv/ Rückenmarkssegmente

  versorgenden Arterien Rückenmarks? A. spinales posterolateralis (paar), A. spinales anterior (unpaar)

  Rückenmarks über A. spinalis anterior versorgt? Vorderen zwei drittel des Rückenmarks

  Rückenmark über Arterii spinalis posteriolateralis versorgt? Hinteres drittel des Rückenmarks

  Häute des Rückenmarks? Dura Mater, Arachnoidea, Pia Mater

  Zwischen Rückenmarkshäuten Nervenwasser? Pia Mater, Arachnoidea

  Zwischen Rückenmarkshäuten venöse Blutgefäße? Epiduralraum (Knochenhaut und Dura)

  Schädigung ventrale Wurzel verursacht? schlaffe Lähmung

  schlaffer Lähmung mit Muskeln? Atropie (Rückbildung der Wurzel)

  Krankheitsmechanismus bei Amyotrophischer Lateralsklerose? Absterben der 1.+2. Motoneuronen im Vorderhorn, Tod in 5 Jahren

  Ursachen für Querschnittslähmung? Linearfraktur, Kompressionsfraktur, Trümmerfraktur

  Durchtrennung Rückenmarks bei C4? Tetraplegie, Lähmung ab Hals

  Durchtrennung des Rückenmarks bei L1? Paraplegia, paralyse ab hüfte

  Ursachen Bandscheibenvorfälle? Genetische Prädisposition, einseitige Belastung, Schwäche der paravertebralen Muskulatur, Altersbedingte Degeneration

  Wirbelsäulenabschnitt meiste Bandscheibenvorfälle? Lenden-WS

  Maßnahmen gegen Bandscheibenvorfälle! Aufbau paravertebralen Muskulatur, Rückengerechtes Heben/Sitzen%, Aufgrund genetischer Ursachen kann trotz Vorbeugung ein BS auftreten

  Grundformen von Schädel-Hirn-Traumata? Gedeckt oder offen

  Bewusstlosigkeit von 45 Minuten? Gehirnprellung

  Symptome Schädel-Hirn-Traumata? Bewusstlosigkeit, Übelkeit, Schwindel, neurologische Ausfälle, Amnesien, Kopfschmerzen

  Therapiemaßnahmen Schädel-Hirn-Traumata? Reha, Druckentlastung, Symptombehandlung, Rehabilitation

  Grundformen cerebrovaskulärer Störungen? Cerebrale Hämorrhagie, Celebrale Ischämie

  Ursachen Hämorrhagien? Arteriosklerose, Amyloidangiopathie, Gefäßveränderungen, Aneurysmen, Traume

  Risikofaktoren Hämorrhagien? Bluthochdruck, Einnahme Gerinnungshemmern, Nikotin, Alkohol

  Ursachen cerebraler Ischämien? Einengung/Verschluss von Aterien (Thrombose), Embolie, Arteriosklerose

  Faktor Schlaganfalltherapie? Zeitlich schnellstmögliche Aufnahme in Stroke Unit

  Therapiemaßnahmen Ischämien? Thrombolyse, Mechanische Thrombose Entferung, Rehabilitation, Behandlung von Ödemen, Stabilisierung der Atmung

  Hirntumorklassen? Meningeome, Gliome, Blastome, Metastasen, andere Primäre Hirntumore (Lympphome)

  häufigste Klasse von Hirntumoren? Gliome

  Symptome für Hirntumore? Kopfschmerzen nachts/morgens, Übelkeit, Erbrechen, Sehstörungen, Krampfanfälle, Neurologische Anzeichen %(Lähmungserscheinungen, Sprach- und Koordinationsstörungen, Ungeschicklichkeit), Persönlichkeitsveränderung

  neuropathologischen Befunde Alzheimer? Ausgedehnte neuronale Degeneration, Neurofibrilläre Verklumpung

  %In welchen Hirnarealen sind neuropathologische Veränderungen bei Alzheimer besonders anzutreffen?

  Art nicht von Alzheimerschen betroffen? Sensor-motorisches Lernen

  Neurotransmitter bei Parkinson besonders? Dopamin

  Hirnstruktur bei Parkinson besonders? Substania nigra

  Symptome Parkinson? Ruhetremor, Rigor, Maskenartiges Gesicht, Bradykinese, spezifischer Gang

  Behandlungsstrategien Parkinson? Medikation von L-DOPA oder Dopaminagonist, Tiefenhirnstimulation in Basalganglien

  Risiko wenn Mutter Chorea Huntington? 50\% (autosomal dominant)

  Nrvenzellen bei Amyotrophen Lateralsklerose? Motoneuronen im Cortex, Rückenmark, Hirnnervenkernen

  Krankheit greift Myelin der Axone an? Multiple Sklerose MS

  %Nennen Sie die 4 Grundprinzipien des sensomotorischen Systems!

  sensomotorischen Systeme? Eigenreflexapperat, Fremdreflexapperat, Vestibulozerebellares, Extrapyramidales, Pyramidales

  Aufgaben Eigenreflexapparates? Anpassung Muskellängen u. Muskelspannung an Schwerkraft

  Verknüpfungen zw. Sensor und Effektor Eigenreflexapparates? Monosynaptisch (eine synaptische Verbindung)

  Zellkörper der somatoafferenten Neuronen? In Spinalganglion%, keien Berührung zu anderen Axonen mit dem Zellkörper

  WO in Rückenmarks Motorneuronen? Graue Masse

  Worüber verlassen motorischen Fasern Rückenmark? Radix anterior

  Muskel versorgt? Jeder M. von nervenfasen mehrerer Rückenmarkssegmente

  motorische Einheit? Gesamtheit von Neuronen innervierten Muskelfaser

  motor. Endplatten an Muskelfaser? Jede M-Zelle nur eine Endplatte

  Größe motorischer Einheit? Von der komplexität der Motorik

  Sensoren messen Muskellänge/spannung? Muskelspindeln

  Rolle Gamma-Neuronen Eigenreflex? Längenänderung Spindelfasern

  Muskel Patellarsehnenreflex inhibiert? Beinbeuger (Bizeps)

  Funktion Fremdreflexapparates? Automatische Reaktion auf Reize außerhalb Muskulatur

  Haut- und Körperrezeptoren? Eingekapselt+organartige->Tasten, freie Nervenenden-> Schmerz, Temperatur

  Berührungs/Drucksensoren? Langsam adaptierend: Druckwahrnehmung, Schnell a.: Berührungswahrnehmung, Sehr schnell a.: Vibrationswahrnehmung

  afferenten Nervenfasern größte Geschwindigkeit? Alpha-Fasern 70-120 m/s

  sensorische Information durch C-Fasern? Temperatur und Schmerz

  Typ afferenter Nervenfasern marklos? C-Fasern

  Hinterstrangbahnen im Rückenmark ziehen? Zu Medulla oblongata

  Hinterstrangbahnen kreuzen auf kontralaterale Seite? Im Hirnstamm

  Assoziationscortexareale? Posterior-parietal Assoziationscortex, Dorsal präfrontal assotiationscortex

  Input parietale Assoziationscortex? Sensorischen Arealen: visuell, auditorisch, somato...

  %Über wie viele Neuronen wird im pyramidalen System die Information an die Muskeln übertragen?

  Stationen Sehbahn? Retina, Sehnerv, Chiasma opticum, Sehnerventrakt, Äußerer Kniehöcker, Radiatio optica, Primäre Sehrinde, Sekundäre Sehrinde

  Information Netzhaut rechten Auges? Linke Großhirnhemisphäre

  hintere Teil des Augapfels? Hornhaut, Aderhaut, Netzhaut

  Hornhaut des Auges? vordere Teil der äußeren Augenhaut, frontaler Abschluss des Augapfels

  Worauf wirkt Ziliarmuskel? Zonularfasern (Bindegewebsfasern)

  weite Pupille? weniger scharfes Bild; hohe Empfindlichkeit

  enge Pupille? empfindlichkeit gering; schärferes Bild

  Erweiterung/Verengung Pupille? durch sympathisches und parasympathisches NS

  Stress auf die Pupille? Pupille wird geweitet

  Müdigkeit auf die Pupille? Kontraktion der Pupille

  Entspannung des Ziliarmuskels? Fernakkommodation, gespannte Zonularfasern, flache Linsenkrümmung

  Linsenwölbung bewirkt Fernakkomodation? Flache Linsenkrümmung

  Fehlsichtigkeit Linsen behoben? Sammel - Weitsicht, Zerstreu - Kurzsicht

  Zelltypen Retina? Stäbchen, Zapfen, Horizontalzellen, Biolarzellen, retinale Ganglienzellen, amakrine Zellen

  Neurotransmitter schüttet Fotorezeptoren? Glutamat

  Neurotransmitter der Ganglien- und Bipolarzellen? Glutamat

  Neurotransmitter der amacrinen u. Horizontalzellen? GABA

  Zelltypen kontaktieren Fotorezeptoren? Horizontal und bipolarzellen

  synaptische Kontakte zwischen Sehnerv und zellen? 130 Mio

  Zellart der Netzhaut einfallenden Licht nahe? Axone retinaler Ganglienzellen

  Art von Fotorezeptoren in der Retina? Stäbchenzellen, Zapfenzellen

  Art Fotorezeptoren für Farbwahrnehmung? Zapfen

  Art Fotorezeptoren ist zahlreicher? Stäbchen

  Konvergenz in Retina? Geringere Auflösung, höhere Lichtempfindlichkeit

  Auswirkungen laterale Inhibition in Retina? Kontrasterhöhung

  Eintrittsstelle des Sehnervs? Blinder Fleck, keine Fotorezeptoren

  Retina Zapfendichte am höchsten? Sehgrube

  Großhirnlappen primäre Sehrinde? Primärer visueller Cortex

  Durchtrennung des rechten Sehnerves? Erblindung des Rechten Auges

  Durchtrennung optischen Tracts? Ausfall des linken/rechten Gesichtsfeldes beider Augen

  Läsionen im prim. vis. C.? Skotome: blinde Stellen im Gesichtsfeld

  Läsionen im post. Parietallappen? kann nicht mehr nach Dingen greifen aber erkennen

  Läsionen im infer. Temporallappen? kann Dinge greifen aber nicht beschreiben

  Theorie von Logothetis/Steinberg? Dorsale Bahn dient der Verhaltensinteraktion der Objekte, ventrale Bahn der bewussten Wahrnehmung

  Propagnosie? Unfähigkeit Gesichter zu erkennen

  aus rechten unteren Quadranten rechten Auges verarbeitet? primärer visueller Cortex

  Farbtheorie von Young/Helmholtz? Farbe des sichtbaren Spektrums aus drei unabhängigen Farben gemischt

  Farbtheorie von Hering? Farben lassen sich nicht beliebig mischen %(z.b. kein rötliche Grün), Schattenbilder nach Starren auf Farben

  Farbtheorien im Gehirn? Young/Helmholtz und Hering

  Farbe einer Fläche? reflektierte Wellenlänge, benutzte Lichtspektrum, umgebende Objekte

  Erklärung Blindsehen? Primärer Visueller Cortex nicht vollständig zerstört; direkte Verbindung Mittelhirn und Thalamus zu höheren viusellen Gebieten

  Abschnitte Ohr? Inneres, mittleres und äußeres Ohr

  Struktur trennt äußeres von Mittelohr? Trommelfell

  Funktion äußeres Ohr? Fokussierung Schallrichtungswahrnehmung, Schalldruckverstärkung

  Hauptfunktion des Mittelohrs? Gesamtschalldruckverstärkung

  %Mittelohr zur Schalldruckverstärkung? Flächenverhältnis Trommelfell-Steigbügelgrundplatte, Hebelarme des Gehörknöchelchen(Hammer/Amboss), Hebelarm durch die Biegung des Trommelfells und unsymmetrische Anheftung des Hammers

  Knochenstruktur Innenohr eingebettet? Felsenbein

  Struktur Hörsinneszellen? Corti-Organ

  Membran ist Corti-Organ verbunden? membrana basilaris

  Cochlea empfindlich für hohe Frequenzen? am ovalen Fenster

  Stereozilien mit Tectorialmembran verbunden? äußere Haarzellen

  Funktion äußere Haarzellen? Rückkopplung zur Regulierung von Sensoroutput

  Hörbahnen unterscheiden? dorsale und ventrale Höhrbahn

  Funktion dorsale Hörbahn? verursacht bewusste Wahrnehmung

  Funktion ventralen Hörbahn? verursacht akustische Reflexe

  Neuronen der dorsalen Hörbahn? 8er Hirnnerv(Hörnerv), Medulla(Dorsaler Cochleariskern), Mittelhirn(Colliculus inf.), Zwischenhirn(Innerer Kniehöcker)

  endet dorsale Hörbahn? Linke Hirnhälfte

  kortikale Verarbeitung auditorischer Information? Temporallapen

  Gerät Mittel-/ Innenohrtaubheit erkennen? Stimmgabel

  Ursache Mittelohrtaubheit? Riß im Trommelfell

  Ursache Innenohrtaubheit? Verletzung Cochlea

  Innenohrtaubheit therapieren? Cochlea Implantate

  Hohlräumen des Labyrinth-Organ? Sacculus, Utriculus, anterior Kanal, posterior Kanal, horizontal Kanal

  Projektionsziele vestibulärer Nervenfasern? Rückenmark, Thalamus, Retikuläre Formation, Cerebellum, auf die Kerne des 3,4,6 Hirnnervs

  vestibuläre Störungen? Neuritis Vestibularis, Gutartiger Lagerungschwindel

  gutartigen Lagerungsschwindels? Ablösung Otholiten, herumschlingern in Bogengängen

  Ursache Neuritis vestibularis? Entzündung des Vestibularnervs

  endet der Riechnerv? Riechhirn (Bulbus Olfactorius)

  Art von Neuronen im ZNS ständig erneuert? Riechzellen

  komplexe Geschmacksempfindungen? Interaktion mit anderen Sinnen

  Teil der Zunge schmecken wir süß? Zungenspitze

  kognitive Funktion mit Hippocampus? Bildung von Erinnerungen

  Wo Hippocampus? Temporallappen

  Wo grenzt limbischen Strukturen der Hippocampus? Amygdala, entohirnaler Cortex

  Haupteingangsstruktur für den Hippocampus? Entohirnaler Cortex

  struktureller Cortextyp besteht Hippocampus? Allocortex

  limbische Struktur grenzt Mandelkern? Hippocampus

  kognitive Funktion mit Amygdala? Angst und Furcht

  Ausbreitung meister Hormone? Blutkreislauf

  Wo meiste Hormone freigesetzt? Gehirn/Hypothalamus

  chemischen Gruppen von Hormonen? Peptide \& Proteine, Aminosäurederivate, Steroide

  Peptide? Ketten von Aminosäuren

  Gehirn Hormonausschüttung? Hypothalamus

  Drüse übergeordnete Rolle? Hypophyse

  Hormondrüsen? Nebenniere, Schilddrüse, Hypothalamus, Bauchspeicheldrüse, Hoden/Eierstock

  Hypophyse direkt vom Hypothalamus innerviert? Hypophysenhinterlappen

  Signalweg Information vom Hypothalamus zum Hypophysenvorderlappen? Hypothalamusneuronen zu hypothalamo-hypophysäre Pfortadersysten zu Hypophysenstiel

  Hormone durch Hypophysenhinterlappen? Oxytocin,Vasopressin

  Mechanismen Hormonfreisetzung geregelt? (meist autonom) NS, andere Hormone, nichthormonelle Substanzen

  steroide Sexualhormone? Keimdrüsen, Gonaden: Hoden,Eierstock

  Grundklassen steroiden Sexualhormonen? Androgene, Östrogene, Gestagene

  Freisetzung von Sexualhormonen? Männer = Gleichmäßig, Frauen = Zyklisch; über Hypophyse vom Hypothalamus gesteuert

  Hormon für männliche Entwicklung? Testosteron

  Hormon weibliches Sexualverhalten? Androgene

  Stresshormonen bei kurz/langfrist Stress? Kurz: Katecholamine; Lang: Glukokortikoide

  glukokortikoides Stresshormon? Cortisol

  Hormon im Nebennierenmark? Adrenalin (Epinephrin), Noradrenalin (Norepinephrin)

  Hormone in der Nebennierenrinde? Glukokortikoiden und Androgenen

  %Wirkungen von Glukokortikoiden? Neubildung von Kohlenhydraten aus Proteinen und Fetten, Beeinflussung von Wasser- und Elektrolythaushalt, Unterdrückung der Antikörperproduktion des Immunsystems, dadurch Entzündungshemmung

  chemische Elemente für Schilddrüsenhormonen? Iod und Eisen

  Hauptwirkung Schilddrüsenhormone? Regelung des Grundumsatzes

  Schilddrüsenunterfunktion? Stoffwechselverlangsamung, Verringerung der Leistungsfähigkeit

  Hormon „Wehentropf“? Oxytocin

  Ausschüttung Oxytocin? durch angenehmen Hautkontakt (Kuschelhormon)

  neuronalen Populationen (Para)Sympathikus? Sym: Ganglien Nahe der Wirbelsäule, Para: Ganglien nahe oder in den Organen

  Bestandteil des autonomen NS gehört Grenzstrang? Zentraler Teil

  Wo autonomen Ganglien des (Para)Sympathikus? Zwischen ZNS und inneren Organen

  Neurotransmitter durch präganglionären Sympaticus? Acetylcholin

  Neurotransmitter durch postganglionären Sympaticus? (Nor)Adrenalin

  Neurotransmitter durch präganglionären Sympaticus? Acetylcholin

  Neurotransmitter durch postganglionären Parasym.? Acetylcholin

  Wo präganglionären sympathischen Neuronen? Brust und Lendenmark

  Wo präganglionären parasympathischen Neuronen? Hirnstamm, Mittelhirn, Sakralmark

  Pfad Sympathikus globale Wirkung? Grenzstrang (Truncus sympathicus)

  Rolle Sympathikus? Vorbereitung Flucht und Kampf

  Rolle Parasympathikus? Entspannung und Verdauung

  Wirkungen Sympathikus? Atemfrequenz+Herzfrequenz steigern, Darmtätigkeit senken, Schwitzen, Pupillenerweiterung

  Wirkungen Parasympathikus? Atemfrequenz senken, Herzfrequenz senken, Darmtätigkeit steigern, Pupillen verengen

  Funktion Hypothalamus? Körpertemperaturregelung, Regelung Wasserhaushalt, Regelung Hormonsekretion in Hypophyse, Regelung physiologischer Reaktion auf Erregungszustände

  Phasen Energiestoffwechsels? Cephalische Phase, Absortive Phase, Fastenphase; durch Insulin und Glukagonspiegel

  Merkmale cephalischen+absorptiven Phase? niedriger Glukagonspiegel, hoher Insulinspiegel, fördert Nutzung Blutzucker(Glukose) als Energiequelle

  Merkmale Fastenphase? Hoher Glukagonspiegel, niedriger Insulinspiegel, fördert Umwandlung Fette zu Fettsäuren, Nutzung freier Fettsäuren als Energiequelle

  Argumente gegen Sollwerthypothese? Evolution, Experiment, viele Faktoren
  %Evolution: Nahrung musste aufgenommen werden, wenn sie verfügbar war, Experiment: Schwankungen in Körperfett und Blutzucker beeinflussen die Nahrungsaufnahme nur, wenn sie extrem sind, Nahrungsaufnahme wird durch vielerlei Faktoren bestimmt, wie visuelle und olfaktorische Reize, Emotionen, Stress usw.

  Alternative zu Sollwerthypothese? Positive Anreiztheorie

  Mechanismen zur Regulierung von Hunger? Magen-Darm-Trakt, Serotonin, Leptin, Insulin
  %    \begin{itemize*}
  %      \item Magen-Darm-Trakt: Freisetzung von Peptiden, die an Neurorezeptoren im Gehirn (z.B. im Hypothalamus) binden und als Sättigungssignal wirken.
  %      \item Serotonin: verringert Anziehungskraft schmackhafter Nahrung, reduziert die Aufnahme pro Mahlzeit, verlagert Präferenzen weg von fetthaltiger Nahrung. Appetitszügler sind häufig Serotoninagonisten.
  %      \item Leptin, Insulin und andere: regulieren die Anlage von Fettdepots, Leptinmangel führt zu exzessiver Nahrungsaufnahme und Fettleibigkeit. Bei Insulinmangel isst man viel und bleibt schlank, da die Nahrung nicht in Fettdepots umgewandelt werden kann.
  %    \end{itemize*}

  Schlafphasen, Slow-Wave-Sleep? 4 Phasen, 3+4 SlowWaveSleep

  Korrelate von Schlafphase 1? Schnelle Augenbewegungen und Muskeltonusverlust

  Schlafrhythmus im Verlauf? Anteil REM-Schlaf nimmt in der Nacht zu

  Notwendigkeit von Schlaf? Regenerative Theorien, Circadiane Theorien

  Auswirkungen Schlafentzug? Schlafneigung, Stimmung $\downarrow$, Aufmerskamkeit $\downarrow$

  Ursachen Insomnie? Schlafmittel, Muskelprobleme, nächtliche Myoklonien, Restless-Leg-Syndrom

  Arten Langzeitgedächtnis? explizit(deklarativ) = episodisch+semantisch; implizit=prozdeural und perzeptionell

  Grundarten Gedächtnis? Sensorisch, Kurzzeit, Langzeit

  anterograder Amnesie? Abspeicherung gestört

  retrograd Amnesie? Tendenz rezente Gedächtnisinhalte zu verlieren

  Entfernung führt zu Amnesie Langzeit? beider medialer Temporallappen

  Wo Langzeitgedächtnis gespeichert? selbe Hirnareale, wie für ursprüngliche Erfahrung

  Hebbschen Lernens? durch periodische Aktivität Langzeitveränderungen hervorrufen

  Emotion Mandelkern involviert? Angst

  Hirnhälfte meiste Menschen dominant? Linke Hirnhälfte

  Split-Brain Patienten kommunizieren? Hälften verfügen fast über gleiche Informationen

  Bestandteile Wernicke Geschwind-Modells? Broca Areal, primärer motorischer Cortex, Fasciculus arcuatus, primärer auditorischer Cortex, Wernicke Areal, Gyrus Angularis, primärer visueller Cortex

  Methoden Wernicke-Geschwind-Modells? Läsionen durch chirurgische Eingriffe, Läsionen durch Krankheit oder Unfall, Elektrische Stimulation des Cortex

  Voraussagen Wernicke-Geschwind-Modells? wichtige Rolle bei Sprache, anteriore Läsionen eher expressive und posteriore Läsionen rezeptiver Defizite

  Symptome depressive Episode? Depressive Stimmung, Geringes Interesse,  Verminderter Antrieb, Schläfrigkeit oder Schlaflosigkeit, Appetitlosigkeit, Schuldgefühle
  %Vermindertes Selbstwertgefühl und Selbstvertrauen, Entscheidungsschwäche, Konzentrationsschwäche, Selbstmordgedanken und -versuche, Pessimismus

  manische Episode? Übersteigertes Selbstbewußtsein, Verringertes Schlafbedürfnis, Erhöhtes Redebedürfnis, Sprechzwang, Ablenkbarkeit, Erhöhte zielgerichtete Aktivität, Vergnügungssucht ohne Bedenken der Konsequenzen, Euphorie, Soziale Enthemmung

  Verlaufsformen affektiver Störungen? Unipolare Depression, Bipolare Depression

  affektiver Störungen Geschlechtsunterschiede? Bipolare Depression

  pharmakologische Therapien gegen Depressionen? (Monoaminoxidase) MAO-Hemmer, Trizyklische Antidepressiva (TCAs), Selektive Wiederaufnahmehemmer

  nicht-pharmakologische antidepressive Therapie? Elektrokonvulsive Therapie

  Wirkprinzip von MAO-Hemmern? zerstört Neurotransmitter außerhalb Vesikel; Serotonin Dopamin und Noradrenalin erhöht, adaptive Änderung Repzeptordichte und Second-Messenger-Kette

  Prinzip trizyklischen Antidepressiva? Blockade präsynaptischer Transporterproteine und Hemmung der Wiederaufnahme von Serotonin und/oder Noradrenalin
  %Führt zu Veränderungen der post- und präsynaptischen Rezeptordichten
  %Daneben Wirkung auf Histamin-, Acetylcholin- und Adrenalinrezeptoren(Wirkung auf verschiedene Rezeptoren Unterschiedlich

  Nebenwirkungen MAO-Hemmern? Schlafstörungen, Blutdruckveränderungen, Heißhunger
  % Tyraminabbau in der Leber behindert = spezielle Diät notwendig
  % Interaktion mit vielen Drogen, z.B. Babiturate, Aspirin, Alkohol, Opiate, und Medikamenten $\rightarrow$ z.B. Serotonin-Syndrom

  Nebenwirkungen trizyklischen Antidepressiva? Sedierung, Verwirrung, Gedächtnis- und Sehstörungen
  % kardiovaskulare Probleme, Darmträgheit, Schwindel,

  Nebenwirkungen von Antidepressiva 2.Gen? Abhängigkeit, emotionale Abstumpfung, Nervosität, Schlafstörungen, sexuelle/Magen-Darm-Störungen
  % Potentiell gefährliche Interaktionen mit anderen Medikamenten und Drogen(Serotoninsyndrom)
  % physische Abhängigkeit möglich

  Prinzip Elektrokonvulsiven Therapie? Elektrische Reizung im Gehirn führ zu einem Epileptischen Anfall
  % Kein Bewusstes Erleben des Anfalls durch Narkose und Muskelrelaxationmedikation
  % Verstärkt Wirkung vieler Neurotransmitter(bewirkt damit Herrunterregulierung Rezeptordichte)

  Theorien affektive Störungen? Monoamin-Hypothese, Glukokortikoid-Hypothese, Neurotrophische Hypothese

  Beobachtungen Monoamin-Hypothese? reduzieren Depressions-Symptome + Mengen von Noradrenalin- und Seroton

  Glukokortikoid-Hypothese? Stress+ Angst depressiven Episoden voraus
  % Depression geht oft mit veränderten Stresshormonspiegeln einher.
  % Die Wahrscheinlichkeit, dass erhöhter Stress affektive Störungen auslöst, scheint genetisch bedingt.

  Angststörungen? Generalisierte Angststörung, Posttraumatisches Stresssyndrom, Phobien, Zwangsneurosen, Panikstörungen

  Furcht? auf konkrete Bedrohung gerichtete Angst

  Therapieform Phobien? Verhaltenstherapie (z.B. Konfrontation)

  Psychopharmaka bei Angststörungen? Bariburate, Benzodiazepine

  Prinzip Barbituraten? GABA Agonist, eingeteilt nach Fettlöslichkeit %und Pharmakinetik; je Fettlöslicher, desto schneller setzt Wirkung ein und desto kürzer hält sie an

  Nebenwirkungen Barbituraten? reduzierte REM Perioden, Benommenheit, verlangsamte Reflexe, Müdigkeit, Koma, Tod

  Prinzip Benzodiazepinen? Aktivierung Benzodiazepin Rezeptoren %(GABA-agonistischer Effekt: Wirkt nur mit GABA, Stärker an Synapsen mit wenig GABA (Aktivitätsabhängige Wirkung), verschiedene Wirkungs- und Verstoffwechlungsgeschwindigkeiten

  Symptomgruppen Schizophrenie? Positive und Negative Symptome

  positive Symptome Schizophrenie? Wahnvorstellungen, Halluzinationen, Sprachstörungen, Bizarres Verhalten, motorische Unruhe

  negative Symptome Schizophrenie?  Emotionslosigkeit, Antriebslosigkeit, sozialer Rückzug, Niedergang normaler Hirnfunktion

  Schizophrenie besser auf Neuroleptika? Positive Symptome

  Prinzip klassischer Neuroleptika? Dopaminantagonismus (D2)

  Dopaminpfade im Gehirn? Nigrostriataler, Mesolimbischer, Mesokortikaler, Tuberohypophysischer, Substantia Nigra

  Dopaminpfad Rolle Schizophrenie?
  \begin{itemize*}
    \item Extrapyramidale Nebenwirkungen (Nigrostriataler Pfad)
    \item Positive Symptome (Mesolimbischer Pfad)
    \item Negative Symptome (Tuberohypophysischer Pfad)
    \item Neuroendokrinologische Nebenwirkungen
  \end{itemize*}

  (Neben)Wirkung Neuroleptika:
  \begin{itemize*}
    \item Parkinson Symptome (Nigrostriataler Pfad)
    \item Neuroendokrinologische Nebenwirkungen (Tuberohypophysischer Pfad)
    \item autonome Störungen (Beeinflussung der cholinerger und adrenerger Neuronen)
    \item Tardive Dyskines: unwillkürliche stereotype Bewegungen
    \item Malignes neuroleptisches Syndrom: seltene, sich schnell entwickelnde und lebensbedrohliche Komplikation
  \end{itemize*}

\end{multicols}
\end{document}