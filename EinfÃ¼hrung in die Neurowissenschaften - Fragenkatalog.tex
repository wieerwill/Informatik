\documentclass[10pt, a4paper]{exam}
\printanswers			  % Comment this line to hide the answers 
\usepackage[utf8]{inputenc}
\usepackage[T1]{fontenc}
\usepackage[ngerman]{babel}
\usepackage{listings}
\usepackage{float}
\usepackage{graphicx}
\usepackage{color}
\usepackage{listings}
\usepackage[dvipsnames]{xcolor}
\usepackage{tabularx}
\usepackage{geometry}
\usepackage{color,graphicx,overpic}
\usepackage{amsmath,amsthm,amsfonts,amssymb}
\usepackage{tabularx}
\usepackage{listings}
\usepackage[many]{tcolorbox}
\usepackage{multicol}
\usepackage{hyperref}
\usepackage{pgfplots}
\usepackage{bussproofs}
\usepackage{mdwlist}
\renewcommand{\solutiontitle}{\noindent}
\SolutionEmphasis{\small}
\geometry{top=1cm,left=1cm,right=1cm,bottom=1cm} 

\setlength{\parindent}{0pt}
\setlength{\parskip}{0pt plus 0.5ex} 
% compress space
\setlength\abovedisplayskip{0pt}
\setlength{\parskip}{0pt}
\setlength{\parsep}{0pt}
\setlength{\topskip}{0pt}
\setlength{\topsep}{0pt}
\setlength{\partopsep}{0pt}
\linespread{0.5}

\newcommand\Warning{%
 \makebox[1.4em][c]{%
 \makebox[0pt][c]{\raisebox{.1em}{\small!}}%
 \makebox[0pt][c]{\color{red}\Large$\bigtriangleup$}}}%

\pdfinfo{
  /Title (Einführung in die Neurowissenschaften - Fragenkatalog)
  /Creator (TeX)
  /Producer (pdfTex 1.40.0)
  /Author (Robert Jeutter)
  /Subject ()
}
\title{Einführung in die Neurowissenschaften - Fragenkatalog}
\author{}
\date{}

% Don't print section numbers
\setcounter{secnumdepth}{0}

\newtcolorbox{myboxii}[1][]{
 breakable,
 freelance,
 title=#1,
 colback=white,
 colbacktitle=white,
 coltitle=black,
 fonttitle=\bfseries,
 bottomrule=0pt,
 boxrule=0pt,
 colframe=white,
 overlay unbroken and first={
 \draw[red!75!black,line width=3pt]
  ([xshift=5pt]frame.north west) -- 
  (frame.north west) -- 
  (frame.south west);
 \draw[red!75!black,line width=3pt]
  ([xshift=-5pt]frame.north east) -- 
  (frame.north east) -- 
  (frame.south east);
 },
 overlay unbroken app={
 \draw[red!75!black,line width=3pt,line cap=rect]
  (frame.south west) -- 
  ([xshift=5pt]frame.south west);
 \draw[red!75!black,line width=3pt,line cap=rect]
  (frame.south east) -- 
  ([xshift=-5pt]frame.south east);
 },
 overlay middle and last={
 \draw[red!75!black,line width=3pt]
  (frame.north west) -- 
  (frame.south west);
 \draw[red!75!black,line width=3pt]
  (frame.north east) -- 
  (frame.south east);
 },
 overlay last app={
 \draw[red!75!black,line width=3pt,line cap=rect]
  (frame.south west) --
  ([xshift=5pt]frame.south west);
 \draw[red!75!black,line width=3pt,line cap=rect]
  (frame.south east) --
  ([xshift=-5pt]frame.south east);
 },
}
\renewcommand{\solutiontitle}{\noindent \enspace}

\begin{document}
\begin{myboxii}[\Warning Disclaimer]
  Die Fragen die hier gezeigt werden stammen aus der Vorlesung \textit{Einführung in die Neurowissenschaften}! Für die Korrektheit der Lösungen wird keine Gewähr gegeben.\\
  Antworten mit \Warning versehen, sind unsicher!
\end{myboxii}

%##########################################
\begin{questions}
  \question Welche spezifische Eigenschaft des Organismus wird hauptsächlich durch das Nervensystem realisiert?
  \begin{solution}
    die Reizbarkeit
  \end{solution}

  \question Welches andere funktionelle System steht in besonderer Beziehung zum Nervensystem?
  \begin{solution}
    Hormonsystem (endocrines System)
  \end{solution}

  \question Nennen Sie die Untersysteme des Nervensystems!
  \begin{solution}
    Herz/ Kreislaufsystem, Atmungssystem, Verdauungssystem, Haut, Urogenitalsystem, Skelett,
    Muskulatur, ...
  \end{solution}

  \question Welches Untersystem des Nervensystems ist für die Kommunikation mit der äußeren Umwelt zuständig?
  \begin{solution}
    Sensomotorische Nervensystem
  \end{solution}

  \question Welches Untersystem des Nervensystems ist für die Kommunikation mit anderen Organsystemen zuständig?
  \begin{solution}
    autonomes Nervensystem
  \end{solution}

  \question Nennen Sie die Grundbestandteile des Zentralnervensystems!
  \begin{solution}
    Gehirn (Cerebrum, Pons, Cerebellum), Rückenmark (Spinal Cord, conus medullaris, region of
    cauda equina)
  \end{solution}

  \question Nennen Sie die vier Hauptbestandteile des autonomen Nervensystems!
  \begin{solution}
    Symphatikus, Parasympathikus, Zentraler Teil, Intramurale Plexus
  \end{solution}

  \question Welches Untersystem des autonomen Nervensystems bereitet den Organismus auf Flucht oder Kampf vor?
  \begin{solution}
    Sympathikus
  \end{solution}

  \question Nennen Sie die beiden grundsätzlichen Typen von Zellen im Nervengewebe!
  \begin{solution}
    Neuronen, Glia
  \end{solution}

  \question In welcher Art von Nervengewebe befinden sich die neuronalen Zellkörper?
  \begin{solution}
    Ganglien, Plexus
  \end{solution}

  \question Welches Nervengewebe befindet sich im Rückenmark außen?
  \begin{solution}
    Graue Substanz \Warning
  \end{solution}

  \question Was sind die wichtigsten funktionellen Merkmale von Neuronen im Unterschied zu anderen Zellen?
  \begin{solution}
    verbunden durch Nervenfasern, Informationstransfer elektrisch \& chemisch \Warning
  \end{solution}

  \question Ordnen Sie die folgenden anatomischen Merkmale einer Nervenzelle zu: Axonshügel, Ranvierscher Schnürring, Synapse, Dendrit, Axon!
  \begin{solution}

  \end{solution}

  \question Welches ist der Geschwindigkeitsbereich in dem sich Aktionspotentiale fortpflanzen?
  \begin{solution}
    $0,3 - 100$ m/s
  \end{solution}

  \question Wodurch wird ein Aktionspotential ausgelöst?
  \begin{solution}
    durch ein Membranpotential, welches die Schwelle von circa -65mV am Axonhügel überwindet \Warning
  \end{solution}

  \question Was geschieht, wenn ein Aktionspotential einen synaptischen Endknopf erreicht?
  \begin{solution}

  \end{solution}

  \question Welche Ionenarten sind im Intra- und welche im Extrazellulärraum in erhöhter Konzentration vorhanden?
  \begin{solution}

  \end{solution}

  \question Was ist ein typischer Wert für das Membran-Ruhepotential von Neuronen?
  \begin{solution}
    -70mV
  \end{solution}

  \question Nennen Sie die drei Antriebskräfte für den Ionentransport durch die Zellmembran!
  \begin{solution}
    Diffusion durch einen Konzentrationsgradienten, elektrischer Ionenstrom durch Potentialgradienten, aktiver Ionenaustausch durch Ionenpumpen
  \end{solution}

  \question Welche Ionenkanäle werden bei der Auslösung eines Aktionspotentials als erstes und welche als zweites ausgelöst?
  \begin{solution}
    als erstes: Natriumkanäle, dann Kaliumkanäle
  \end{solution}

  \question Wie hoch ist die ungefähre maximale Impulsrate auf Axonen und wodurch wird diese begrenzt?
  \begin{solution}
    500/s
  \end{solution}

  \question Weshalb breiten sich Aktionspotentiale nur in eine Richtung aus?
  \begin{solution}
    Na+ haben eine Refraktärzeit, die das Zurücklaufen der Welle verhindert
  \end{solution}

  \question Von welchen beiden Faktoren hängt die Ausbreitungsgeschwindigkeit der Aktionspotentiale hauptsächlich ab?
  \begin{solution}
    Durchmesser des Axons, Myelinschicht um Axon $\rightarrow$ saltatorische Erregungsleitung
  \end{solution}

  \question Was versteht man unter saltatorischer Erregungsleitung?
  \begin{solution}
    axonale Erregungsleitung, Erregung springt von Schnürring zu Schnürring
  \end{solution}

  \question Durch welchen Zelltyp werden die Myelinscheiden im Zentral- und im Perphernervensystem gebildet?
  \begin{solution}
    Oligodendrozyten im Zentralnervensystem und Schwann-Zellen in der Peripherie
  \end{solution}

  \question Welche Krankheit beeinträchtigt die Myelinscheiden der Axone?
  \begin{solution}
    Multiple Sklerose
  \end{solution}

  \question Worin befinden sich die Neurotransmitter in den synaptischen Endknöpfen?
  \begin{solution}
    Vesikeln
  \end{solution}

  \question Was sind die beiden informationsverarbeitenden Grundfunktionen einer Synapse?
  \begin{solution}
    Diodenfunktion, Transistorfunktion
  \end{solution}

  \question Über welche beiden Dimensionen findet die Integration von Information in einem Neuron statt?
  \begin{solution}
    räumliche und zeitliche Dimension
  \end{solution}

  \question Durch welche Potentiale werden Informationen in Neuronen digital bzw. analog repräsentiert?
  \begin{solution}

  \end{solution}

  \question Was sind Neurotransmitter?
  \begin{solution}
    ... sind Substanzen, die an chemischen Synapsen ausgeschüttet werden und andere Zellen (Neuronen, Muskelzellen, etc.) spezifisch beeinflussen
  \end{solution}

  \question Nennen Sie die 4 Merkmale von Neurotransmittern!
  \begin{solution}
    werden in präsynaptischen Endknöpfen synthetisiert und in großer Menge freigesetzt um Wirkung zu zeigen, können mechanisch entfernt werden, selbe Wirkung bei exogener Applikation
  \end{solution}

  \question Was sind die beiden Arten von Neurorezeptoren?
  \begin{solution}
    ionotrope Rezeptoren, metabotrobe Rezeptoren
  \end{solution}

  \question Wie funktioniert ein ionotroper Rezeptor?
  \begin{solution}
    Chemisch gesteuerte Ionenkanäle in der postsynaptischen Membran. Bei Bindung öffnet oder schließt sich der Ionenkanal und induziert dadurch augenblicklich das postsynaptische Potential.
  \end{solution}

  \question Wie funktioniert ein metabotroper Rezeptor?
  \begin{solution}
    \begin{itemize*}
      \item Wirkung langsamer und variabler.
      \item Bindung des NT an G-Protein - Untereinheit löst sich im Zellinneren.
      \item Bindet an Ionenkanal und löst AP aus oder Synthese eines weiteren Botenstoffes (second messenger)
    \end{itemize*}
  \end{solution}

  \question Welche Art von Neurorezeptoren ist häufiger – ionotrope oder metalotrope?
  \begin{solution}
    metabotrope Rezeptoren
  \end{solution}

  \question Nennen Sie 7 Neurotransmitter!
  \begin{solution}
    Dopamin, Epinephrin, Histamin, GABA, Glutamat, Serotonin, Acetylcholin
  \end{solution}

  \question Nennen Sie 3 Monoamine, die als Neurotransmitter fungieren!
  \begin{solution}
    Tyrosin, Histidin, Phenylalanin
  \end{solution}

  \question Nennen Sie 3 Aminosäuren, die als Neurotransmitter fungieren!
  \begin{solution}
    Glutamat, GABA, Glycin
  \end{solution}

  \question Nennen Sie den wichtigsten erregenden und den wichtigsten hemmenden Neurotransmitter im Gehirn!
  \begin{solution}

  \end{solution}

  \question Welches sind die drei wichtigen Orte mit dopaminergen Neuronen im Gehirn?
  \begin{solution}

  \end{solution}

  \question Welches der drei wichtigsten dopaminergen Systeme interagiert eng mit dem neuroendokrinologischen System?
  \begin{solution}

  \end{solution}

  \question Was ist das wichtigste Hirnareal, dass noradrenerge Neuronen enthält?
  \begin{solution}
    Locus coeruleus
  \end{solution}

  \question Wo befinden sich serotonerge Neuronen?
  \begin{solution}
    Im Hirnstamm, in den Raphé-Kernen
  \end{solution}

  \question Nennen Sie zwei wichtige Beispiele für cholinerge Übertragung!
  \begin{solution}

  \end{solution}

  \question Nennen Sie die beiden wichtigsten Gruppen cholinerger Rezeptoren!
  \begin{solution}
    Muscarinische (metabotrop), nicotinische (ionotrop) \Warning
  \end{solution}

  \question Wie werden Substanzen genannt, die die synaptische Übertragung fördern bzw. hemmen?
  \begin{solution}
    Inhibitor (hemmend), Aktivator (fördernd)
  \end{solution}

  \question Nennen Sie 5 Wirkmechanismen von Agonisten!
  \begin{solution}
    \begin{itemize*}
      \item Steigerung der NT-Freisetzung
      \item NT Menge $\uparrow$ durch Zerstörung abbauender Enzyme
      \item NT Synthese $\uparrow$ (durch Erhöhung der Menge von Vorläufersubstanzen)
      \item Blockierung von Abbau oder Wiederaufnahme von NT
      \item Bindung an und Aktivierung von postsynaptischen Rezeptoren
    \end{itemize*}
  \end{solution}

  \question Nennen Sie 5 Wirkmechanismen von Antagonisten!
  \begin{solution}
    \begin{itemize*}
      \item NT Synthese↓ (durch Zerstörung synthetisierender Enzyme)
      \item Austreten von NT aus VEsikeln, was zur Zerstörung durch Enzyme führt
      \item Blockierung der NT-Freisetzung
      \item Aktivierung von Autorezeptoren
      \item Bindung an, und Blockierung von, postsynaptischen Rezeptoren
    \end{itemize*}
  \end{solution}

  \question Nennen Sie je ein Beispiel für Antagonisten und Agonisten und nennen Sie die beeinflussten Neurotransmitter!
  \begin{solution}
    \begin{itemize*}
      \item Antagonist: Atropin, M1-3 Acetylcholin-Rezeptor
      \item Agonist:
    \end{itemize*}
  \end{solution}

  \question Nennen Sie 4 Anwendungsgebiete für Atropin!
  \begin{solution}
    Erweiterung der Pupillen, Gegengift für cholinerge Agonisten, Hemmung Magen/Darmaktivität, Kreislaufstillstand
  \end{solution}

  \question Nennen Sie die wichtigsten Typen von Gliazellen!
  \begin{solution}
    Microgliazyten, Astrozyten, Ependymzellen, Oligodendrogliazyten, (Schwann-Zellen)
  \end{solution}

  \question Nennen Sie die wichtigsten Merkmale von Mikroglia!
  \begin{solution}
    Vielfältige Formen, Amöboid beweglich, Abräum- und Abwehrfunktion
  \end{solution}

  \question Nennen Sie die wichtigsten Merkmale und Funktionen von Astrozyten!
  \begin{solution}
    Kurzstrahlige Astrozyten in grauer Substanz, lnagstrahlige Astrozyten in weißer Substanz, Gliafüßchen bilden geschlossene Schicht um Kapillaren, Kontrolle Ionen- und Flüssigkeitsgleichgewicht, Stütz- und Transportfunktion, Abgrenzfunktion, teilungsfähig und bilden Glianarben
  \end{solution}

  \question Durch welche Gliazellen wird die Blut-Hirn-Schranke realisiert?
  \begin{solution}
    Astrozyten
  \end{solution}

  \question Nennen Sie die wichtigsten Merkmale und Funktionen der Oligodendrozyten!
  \begin{solution}
    Eng an Neuronen angelagert, Stoffwechselfunktion für Neuronen, bilden Markscheide für ZNS-Neuronen
  \end{solution}

  \question Durch welche Zellen wird die Myelinscheide im peripheren Nervensystem gebildet?
  \begin{solution}
    Schwann-Zellen
  \end{solution}

  \question Nennen Sie Neurotransmitter- und Rezeptortyp in motorischen Endplatten!
  \begin{solution}
    Transmitter: ACh, Rezeptor: nikotinische ACh-Rezeptoren
  \end{solution}

  \question Sind Gliazellen auch direkt an Informationsverarbeitung im Gehirn beteiligt?
  \begin{solution}
    Ja, 10-50 mal mehr als Neuronen, direkt am Prozess der Informationsverarbeitung, -speicherung und -weiterleitung im Nervensystem beteiligt
  \end{solution}

  \question Ordnen Sie die Richtungsbezeichnungen dorsal, ventral, caudal, rostral, anterior, medial, lateral den Begriffen außen, vorn, oben, innen, unten, hinten zu!
  \begin{solution}
    caudal-hinten, dorsal-oben, ventral-unten, rostal- vorn, anterior-vorn, medial-innen, lateral-außen
  \end{solution}

  \question Was bezeichnen die Begriffe proximal und distal?
  \begin{solution}
    proximal: zum Rumpf hin gelegen, distal: vom Körperzentrum weg gelegen
  \end{solution}

  \question Nennen Sie die 6 Hauptabschnitte des Gehirns!
  \begin{solution}
    Telencephalon, Diencephalon, Mesencephalon, Metencephalon, Myelencephalon, Rückenmark \Warning
  \end{solution}

  \question Wie viele Hirnnervenpaare gibt es?
  \begin{solution}
    12 Hirnveenenpaare
  \end{solution}

  \question Welcher Hirnnerv entspringt im Telencephalon und welche Funktion hat er?
  \begin{solution}
    N. olfactorius (sensorisch: riechen)
  \end{solution}

  \question Welcher Hirnnerv entspringt im Diencephalon und welche Funktion hat er?
  \begin{solution}
    N. opticus (sensorisch: Sehen)
  \end{solution}

  \question Was ist die Funktion des N. trigenimus?
  \begin{solution}
    sensorisch: Gesicht, Nase, Mund, Zunge; motorisch: kauen
  \end{solution}

  \question Was ist die Funktion des N. vestibulocochlearis?
  \begin{solution}
    sensorisch: Gleichgewicht, Hören
  \end{solution}

  \question Was ist die Funktion des N. vagus?
  \begin{solution}
    Motorisch (parasympathisch): Eingeweide; motorisch: Kehlkopf, Rachen; sensorisch: Kehlkopf, Rachen
  \end{solution}

  \question Welche basalen Hirnfunktionen sind in der Medulla oblongata lokalisiert?
  \begin{solution}
    Atem- und Kreislaufzentrum; Zentren für Nies-, Huste-, Schluck-, Saug- und Brechreflex; formatio reticularis
  \end{solution}

  \question Welches Hirnteil ist für das Überleben des Organismus unverzichtbar?
  \begin{solution}
    Medulla
  \end{solution}

  \question Wo befindet sich die retikuläre Formation?
  \begin{solution}
    Zieht sich durch Medulla, Pons und Mesencephalon/Diencephalon
  \end{solution}

  \question Nennen sie drei wichtige Funktionen die der retikulären Formation zugeordnet werden!
  \begin{solution}
    Zeitliche Koordination des gesamten Nervensystems; Atmung, Kreislauf, Muskeltonus; Moduation von Schmerzempfinden und Emotion, Schlaf-Wach-Rhythmus, Aufmerksamkeit
  \end{solution}

  \question Wo befindet sich die Pons?
  \begin{solution}
    Zwischen Mesencephalon und Myelencephalon; bildet mit Cerebellum das MEtencephalon, ist von diesem durch das (4) Ventrikel getrennt
  \end{solution}

  \question Was befindet sich zwischen Pons und Cerebellum?
  \begin{solution}
    Teile des 4.Hirnventrikels, Rautengrube
  \end{solution}

  \question Wo befinden sich Zellkörper und Axone cerebellarer Neuronen?
  \begin{solution}

  \end{solution}

  \question Nennen Sie die beiden wichtigsten cerebellaren Neuronentypen und ordnen Sie diese anhand der Lage ihrer Zellkörper den entsprechenden Cortexschichten zu!
  \begin{solution}

  \end{solution}

  \question Nennen Sie die 4 grundsätzlichen Funktionsprinzipien des Cerebellums!
  \begin{solution}
    \begin{itemize*}
      \item Feedforward-Verarbeitung
      \item Divergenz und Konvergenz
      \item Modularität
      \item Plastizität
    \end{itemize*}
  \end{solution}

  \question Nennen Sie 5 typische Symptome cerebellarer Störungen!
  \begin{solution}
    \begin{itemize*}
      \item Ataxie, Störung in der Bewegungskoordination
      \item Nystagmus (Augenzittern)
      \item Rumpfataxie (Unfähigkeit sich im Sitzen oder TSheen aufrecht zu erhaten)
      \item Tremor
      \item Verwaschene oder undeutliche Aussprache
      \item Störungen im fließenden Bewegungsablauf
    \end{itemize*}
  \end{solution}

  \question Wo befindet sich das Mittelhirn?
  \begin{solution}
    zwischen Pons und Diencephalon
  \end{solution}

  \question Was sind die beiden Hauptabschnitte des Mittelhirns?
  \begin{solution}
    Tectum, Tegmentum \Warning
  \end{solution}

  \question Zu welchen funktionellen Systemen gehören die inferioren und die superioren Colliculi?
  \begin{solution}
    Tectum (Mittelhirndach, Vierhügelplatte)
  \end{solution}

  \question Was ist der wichtigste Neurotransmitter der Substantia nigra?
  \begin{solution}
    Dopamin
  \end{solution}

  \question Welche Krankheit ist mit Störungen in der Substantia nigra verbunden?
  \begin{solution}
    Morbus Parkinson \Warning
  \end{solution}

  \question Was ist die wichtigste Funktion des Thalamus?
  \begin{solution}
    \begin{itemize*}
      \item ,,Eingangskontrolle'' des Großhirns
      \item Umschaltstation sensorischer Informationen
    \end{itemize*}
  \end{solution}

  \question Nennen Sie 5 Funktionen des Hypothalamus!
  \begin{solution}
    \begin{itemize*}
      \item Regelung der Körpertemperatur
      \item Regelung des Wasser und Mineralhaushaltes
      \item Regelung der Hormonausschüttung der Hypophyse
      \item Regelung der physiologischen Reaktion auf Erregungszustände
      \item Appetitregelung
      \item Steuerung von Schlaf und zirkadianen Rhytmen
      \item Beeinflussung des Sexualverhalten, Aggression, Flucht
    \end{itemize*}
  \end{solution}

  \question Welches ist das oberste Regulierungszentrum des autonomen Nervensystems?
  \begin{solution}
    Hypothalamus
  \end{solution}

  \question Nennen Sie die drei grundsätzlichen Quellen für Afferenzen des Hypothalamus!
  \begin{solution}
    \begin{itemize*}
      \item Limbisches System
      \item Sensorische Informationen über interne Umgebung
      \item Sensorische Informationen über externe Umgebung
    \end{itemize*}
  \end{solution}

  \question Nennen Sie die 5 grundsätzlichen Efferenzen des Hypothalamus!
  \begin{solution}

  \end{solution}

  \question Zu welchen funktionellen Systemen gehören die lateralen und die medialen Kniehöcker?
  \begin{solution}
    Metathalamus
  \end{solution}

  \question Nennen Sie die beiden Hauptabschnitte des Großhirns!
  \begin{solution}
    Großhirnhälften, Basalganglien
  \end{solution}

  \question Was wird durch Kommissuren verbunden?
  \begin{solution}
    Beide Gehirnhälften
  \end{solution}

  \question Nennen Sie die 4 Großhirnlappen!
  \begin{solution}
    \begin{itemize*}
      \item Frontallappen (Lobus frontalis)
      \item Schläfenlappen (Lobus temperalis)
      \item Hinterhauptslappen (Lobus occipitalis)
      \item Scheitellappen (Lobus parietalis)
    \end{itemize*}
  \end{solution}

  \question Was verbinden Projektions- und Assoziationsbahnen?
  \begin{solution}

  \end{solution}

  \question Welche histologischen und phylogenetischen Cortextypen gibt es?
  \begin{solution}

  \end{solution}

  \question Wie viele Schichten unterscheidet man beim Isocortex und beim Allocortex?
  \begin{solution}
    Isocortex: 6, Allocortex: 3
  \end{solution}

  \question Welche histologische und phylogenetische Cortexart nimm die meiste Fläche ein (beim Menschen)
  \begin{solution}
    Isocortex
  \end{solution}

  \question Nennen Sie 4 wichtige Strukturen der Basalganglien!
  \begin{solution}
    Nucleus caudatus, Putamen, Globus pallidus, Amygdala
  \end{solution}

  \question Welche beiden Strukturen werden unter dem Begriff Striatum zusammengefasst?
  \begin{solution}
    Nucleus caudatus, Putamen
  \end{solution}

  \question Bei welchen Funktionen spielt die Amygdala eine herausragende Rolle?
  \begin{solution}
    Wichtige Rolle bei Emotionen, insbesondere Angst und Furcht
  \end{solution}

  \question Was enthält die Weiße Masse?
  \begin{solution}
    Nervenfasern und Glia
  \end{solution}

  \question Nennen Sie die drei Hirnhäute!
  \begin{solution}
    Dura mater, Arachnoidea, Pia mater
  \end{solution}

  \question Welche Hirnhaut grenzt unmittelbar an den Cortex?
  \begin{solution}
    Pia mater
  \end{solution}

  \question Welche Hirnhaut grenzt unmittelbar an den Schädel?
  \begin{solution}
    Dura mater
  \end{solution}

  \question Über wie viele Arterien erfolgt die Blutzufuhr zum Gehirn?
  \begin{solution}
    6
  \end{solution}

  \question Durch welche Struktur kann der Ausfall einer der zuführenden Arterien ausgeglichen werden?
  \begin{solution}
    Durch einen Ring in der Hirnbasis
  \end{solution}

  \question Wie viele Hirnventrikel gibt es?
  \begin{solution}
    5
  \end{solution}

  \question Wo und durch welche Struktur wird das Nervenwasser gebildet?
  \begin{solution}
    In den Ventrikeln( durch Kapillargeflechte der Pia mater) gebildet
  \end{solution}

  \question Wo wird das Nervenwasser wieder resorbiert?
  \begin{solution}
    Arachoidalzotten im Sinus sagittalis superior
  \end{solution}

  \question Wo befinden sich weiße und graue Masse im Rückenmark?
  \begin{solution}
    weiße Masse außen, graue Masse innen
  \end{solution}

  \question Wo endet das Rückenmark beim Erwachsenen?
  \begin{solution}
    Obere Lendenwirbelsäule
  \end{solution}

  \question Was sind die beiden wichtigsten Grundfunktionen des Rückenmarks?
  \begin{solution}
    \begin{itemize*}
      \item Verbindung zwischen Gehirn und dem größten Teil des restlichen Körpers
      \item Implementierung wichtiger somatomotorischer und viszeraler Reflexe
    \end{itemize*}
  \end{solution}

  \question Wie viele Spinalnervenpaare gibt es?
  \begin{solution}
    31 Paare
  \end{solution}

  \question Was ist ein Dermatom?
  \begin{solution}
    \begin{itemize*}
      \item Assoziation zwischen Körperoberfläche und Spiralnerv/ Rückenmarkssegmente
      \item Klinisch bedeutsam für diagnose von Schäden
    \end{itemize*}
  \end{solution}

  \question Nennen Sie die drei versorgenden Arterien des Rückenmarks!
  \begin{solution}
    \begin{itemize*}
      \item A. spinales posterolateralis (paar)
      \item A spinales anterior (unpaar)
    \end{itemize*}
  \end{solution}

  \question Welcher Anteil des Rückenmarks wird über die Arteria spinalis anterior versorgt?
  \begin{solution}
    Vorderen zwei drittel des Rückenmarks \Warning
  \end{solution}

  \question Welcher Anteil des Rückenmarks wird über die beiden Arterii spinalis posteriolateralis versorgt?
  \begin{solution}
    Hinteres drittel des Rückenmarks \Warning
  \end{solution}

  \question Nennen Sie die drei Häute des Rückenmarks!
  \begin{solution}
    Dura Mater, Arachnoidea, Pia Mater
  \end{solution}

  \question Zwischen welchen Rückenmarkshäuten befindet sich Nervenwasser?
  \begin{solution}
    Pia Mater und Arachnoidea
  \end{solution}

  \question Zwischen welchen Rückenmarkshäuten befinden sich venöse Blutgefäße?
  \begin{solution}
    Epiduralraum (zwischen Knochenhaut und Dura)
  \end{solution}

  \question Was wird durch Schädigung oder Durchtrennung der ventralen Wurzel verursacht?
  \begin{solution}
    schlaffe Lähmung
  \end{solution}

  \question Was passiert bei schlaffer Lähmung mit den betroffenen Muskeln?
  \begin{solution}
    Atropie (Rückbildung der Wurzel) der Muskeln
  \end{solution}

  \question Was ist der Krankheitsmechanismus bei Amyotrophischer Lateralsklerose?
  \begin{solution}
    Absterben der 1. und 2. Motoneuronen im Vorderhorn des Rückenmarks, Tod normalerweise
    innerhalb von 5 Jahren
  \end{solution}

  \question Nennen Sie drei wichtige Ursachen für Querschnittslähmung!
  \begin{solution}
    Linearfraktur, Kompressionsfraktur, Trümmerfraktur
  \end{solution}

  \question Was sind die Auswirkungen einer Durchtrennung des Rückenmarks bei C4?
  \begin{solution}
    Tetraplegie (Lähmung ab dem Hals an)
  \end{solution}

  \question Was sind die Auswirkungen einer Durchtrennung des Rückenmarks bei L1?
  \begin{solution}
    Paraplegia, paralysis below the waist
  \end{solution}

  \question Nennen Sie drei Ursachen für Bandscheibenvorfälle?
  \begin{solution}
    Genetische Prädisposition, einseitige Belastung, Schwäche der paravertebralen Muskulatur, Altersbedingte Degeneration
  \end{solution}

  \question In welchem Wirbelsäulenabschnitt treten die meisten Bandscheibenvorfälle auf?
  \begin{solution}
    Lenden-WS
  \end{solution}

  \question Nenne Sie zwei wirksame prophylaktische Maßnahmen gegen Bandscheibenvorfälle!
  \begin{solution}
    Aufbau der paravertebralen Muskulatur, Rückengerechtes Heben/Sitzen, Aufgrund genetischer Ursachen kann trotz Vorbeugung ein BS auftreten
  \end{solution}

  \question Welches sind die beiden Grundformen von Schädel-Hirn-Traumata?
  \begin{solution}
    Gedeckt oder offen
  \end{solution}

  \question Deutet eine Bewusstlosigkeit von 45 Minuten auf eine Gehirnerschütterung, eine Gehirnprellung oder eine Gehirnquetschung hin?
  \begin{solution}
    Gehirnprellung
  \end{solution}

  \question Nennen Sie 5 typische Symptome für Schädel-Hirn-Traumata!
  \begin{solution}
    Bewusstlosigkeit, Übelkeit, Schwindel, neurologische Ausfälle, Amnesien
    - Kopfschmerzen
  \end{solution}

  \question Nennen Sie 3 Therapiemaßnahmen bei Schädel-Hirn-Traumata!
  \begin{solution}
    Rihe, Beobachtung (Krnakenhaus), Druckentlastung, Symptombehandlung, Rehabilitation
  \end{solution}

  \question Nennen Sie die beiden Grundformen cerebrovaskulärer Störungen!
  \begin{solution}
    Cerebrale Hämorrhagie, Celebrale Ischämie
  \end{solution}

  \question Nennen Sie 3 mögliche Ursachen für Hämorrhagien!
  \begin{solution}
    Arteriosklerose, Amyloidangiopathie, Gefäßveränderungen, Aneurysmen, Traume
  \end{solution}

  \question Nennen Sie 3 wichtige Risikofaktoren für Hämorrhagien!
  \begin{solution}
    Bluthochdruck, Einnahme von Gerinnungshemmern, Nikotin, Alkohol
  \end{solution}

  \question Nennen Sie 2 mögliche unmittelbare Ursachen cerebraler Ischämien!
  \begin{solution}
    Einengung oder Verschluss von Aterien (Thrombose), Embolie, Arteriosklerose
  \end{solution}

  \question Was ist der wichtigste Faktor bei der Schlaganfalltherapie?
  \begin{solution}
    Zeitlich schnellstmögliche Aufnahme in Stroke Unit
  \end{solution}

  \question Nennen Sie 3 wichtige Therapiemaßnahmen bei Ischämien!
  \begin{solution}
    Thrombolyse, Mechanische Thrombose Entferung. Rehabilitation, Behandlung von Ödemen, Stabilisierung der Atmung
  \end{solution}

  \question Nennen Sie 4 wichtige Hirntumorklassen (nach der Gewebsart)!
  \begin{solution}
    Meningeome, Gliome, Blastome, Metastasen, andere Primäre Hirntumore (Lympphome)
  \end{solution}

  \question Welche Klasse von Hirntumoren (nach der Gewebsart) ist am häufigsten?
  \begin{solution}
    Gliome
  \end{solution}

  \question Nennen Sie 5 typische Symptome für Hirntumore!
  \begin{solution}
    Neu autretende Kopfschmerzen nachts/morgens, Übelkeit, Erbrechen, Sehstörungen, Krampfanfälle, Neurologische Anzeichen (Lähmungserscheinungen, Sprach- und Koordinationsstörungen,
    Ungeschicklichkeit), Persönlichkeitsveränderung
  \end{solution}

  \question Welches sind die beiden typischen neuropathologischen Befunde bei Alzheimer?
  \begin{solution}
    Ausgedehnte neuronale Degeneration, Neurofibrilläre Verklumpung \Warning
  \end{solution}

  \question In welchen Hirnarealen sind neuropathologische Veränderungen bei Alzheimer besonders anzutreffen?
  \begin{solution}

  \end{solution}

  \question Welche Art von Lernen/Gedächtnis ist nicht von der Alzheimerschen Krankheit betroffen?
  \begin{solution}
    Sensor-motorisches Lernen
  \end{solution}

  \question Welcher Neurotransmitter spielt eine besondere Rolle bei der Parkinsonschen Krankheit?
  \begin{solution}
    Dopamin
  \end{solution}

  \question Welche Hirnstruktur spielt eine besondere Rolle bei der Parkinsonschen Krankheit?
  \begin{solution}
    Substania nigra
  \end{solution}

  \question Nennen Sie 5 typische Symptome der Parkinsonschen Krankheit!
  \begin{solution}
    Ruhetremor, Rigor, Maskenartiges Gesicht, Bradykinese, spezifischer Gang
  \end{solution}

  \question Nennen Sie die 2 wichtigsten Behandlungsstrategien bei Parkinson!
  \begin{solution}
    Medikation von L-DOPA oder Dopaminagonist, Tiefenhirnstimulation in Basalganglien
  \end{solution}

  \question Wie hoch ist das Erkrankungsrisiko einer Person, deren Mutter an Chorea Huntington erkrankt ist?
  \begin{solution}
    50\%, da autosomal dominant vererbt
  \end{solution}

  \question Welche Nervenzellen werden bei der Amyotrophen Lateralsklerose geschädigt?
  \begin{solution}
    Motoneuronen im Cortex, im Rückenmark oder in Hirnnervenkernen
  \end{solution}

  \question Bei welcher Krankheit wird das Myelin der Axone angegriffen?
  \begin{solution}
    Multiple Sklerose MS
  \end{solution}

  \question Nennen Sie die 4 Grundprinzipien des sensomotorischen Systems!
  \begin{solution}

  \end{solution}

  \question Nennen Sie die 5 sensomotorischen Systeme!
  \begin{solution}
    Eigenreflexapperat, Fremdreflexapperat, Vestibulozerebellares System, Extrapyramidales System, Pyramidales System
  \end{solution}

  \question Was sind die beiden Aufgaben des Eigenreflexapparates?
  \begin{solution}
    Anpassung von Muskellängen, Anpassung von Muskelspannung an die Schwerkraft
  \end{solution}

  \question Wie viele synaptische Verknüpfungen befinden sich zwischen Sensor und Effektor des Eigenreflexapparates (ohne motorische Endplatten)?
  \begin{solution}
    Monosynaptisch (eine synaptische Verbindung)
  \end{solution}

  \question Wo befinden sich die Zellkörper der somatoafferenten Neuronen?
  \begin{solution}
    In den Spinalganglion, keien Berührung zu anderen Axonen mit dem Zellkörper \Warning
  \end{solution}

  \question In welchem Teil des Rückenmarks befinden sich die Motorneuronen?
  \begin{solution}
    Graue Masse
  \end{solution}

  \question Über welche Nervenwurzel verlassen die motorischen Fasern das Rückenmark?
  \begin{solution}
    Radix anterior
  \end{solution}

  \question Wird ein Muskel i.d.R. von genau einem Rückenmarkssegment versorgt?
  \begin{solution}
    Jeder Muskel wird von nervenfasen mehrerer Rückenmarkssegmente versorgt
  \end{solution}

  \question Was ist eine motorische Einheit?
  \begin{solution}
    Gesamtheit der von Neuronen innervierten Muskelfaser
  \end{solution}

  \question Wie viele motorische Endplatten kontaktieren eine Muskelfaser?
  \begin{solution}
    Jede Muskelzelle nur von einer Endplatte
  \end{solution}

  \question Wovon hängt die Größe einer motorischen Einheit ab?
  \begin{solution}
    Von der komplexität der Motorik
  \end{solution}

  \question Durch welche Sensoren werden die Muskellänge und die Muskelspannung gemessen?
  \begin{solution}
    Muskelspindeln
  \end{solution}

  \question Was ist die Rolle der Gamma-Neuronen im Eigenreflexapparat?
  \begin{solution}
    Veränderung der Länge der Spindelfasern
  \end{solution}

  \question Welcher Muskel wird beim Patellarsehnenreflex inhibiert?
  \begin{solution}
    Beinbeuger (Bizeps)
  \end{solution}

  \question Was ist die Funktion des Fremdreflexapparates?
  \begin{solution}
    Automatische Reaktion auf Reize außerhalb der Muskulatur
  \end{solution}

  \question Welche grundsätzlichen Typen von Haut- und Körperrezeptoren gibt es?
  \begin{solution}
    Eingekapselte, organartige differenzierbare Strukturen für Tastempfindlichkeit oder freie Nervenendigungen für Schmerz- und Temperaturreize
  \end{solution}

  \question Welche Typen von Berührungs/Drucksensoren gibt es?
  \begin{solution}
    \begin{itemize*}
      \item Langsam adaptierend: Druckwahrnehmung
      \item Schnell adaptierend: Berührungswahrnehmung
      \item Sehr schnell adaptierend: Vibrationswahrnehmung
    \end{itemize*}
  \end{solution}

  \question Welche afferenten Nervenfasern haben die größte Übertragungsgeschwindigkeit?
  \begin{solution}
    Aalpha-Fasern (70-120 m/s)
  \end{solution}

  \question Welche sensorische Information wird durch C-Fasern übermittelt?
  \begin{solution}
    Temperatur und Schmerz
  \end{solution}

  \question Welcher Typ afferenter Nervenfasern ist marklos?
  \begin{solution}
    C-Fasern
  \end{solution}

  \question Wohin ziehen die Hinterstrangbahnen im Rückenmark?
  \begin{solution}
    Zur Medulla oblongata
  \end{solution}

  \question Wo kreuzen die Hinterstrangbahnen auf die kontralaterale Seite?
  \begin{solution}
    Im Hirnstamm
  \end{solution}

  \question Nennen sie zwei wichtiges sensomotorische Assoziationscortexareale!
  \begin{solution}
    Posterior-parietal Assoziationscortex, Dorsal präfrontal assotiationscortex
  \end{solution}

  \question Woher erhält der parietale Assoziationscortex seinen Input?
  \begin{solution}
    Sensorischen Arealen (visuellem cortex, auditorischem Cortex, somatosensorischem Cortex, ...)
  \end{solution}

  \question Über wie viele Neuronen wird im pyramidalen System die Information an die Muskeln übertragen?
  \begin{solution}

  \end{solution}

  \question Nennen Sie die 7 Stationen der Sehbahn!
  \begin{solution}
    Retina, Sehnerv (2.Hirnnerv), Chiasma opticum, Sehnerventrakt, Äußerer Kniehöcker, Radiatio optica, Primäre Sehrinde, Sekundäre Sehrinde
  \end{solution}

  \question In welcher Hemisphäre wird die Information von der Netzhaut des rechten Auges verarbeitet?
  \begin{solution}
    Linke Großhirnhemisphäre
  \end{solution}

  \question Aus welchen drei Häuten besteht der hintere Teil des Augapfels?
  \begin{solution}
    Hornhaut, Aderhaut, Netzhaut
  \end{solution}

  \question Wo befindet sich die Hornhaut des Auges?
  \begin{solution}
    Von tränenwasser benetzt, vordere Teil der äußeren Augenhaut, frontaler Abschluss des Augapfels
  \end{solution}

  \question Worauf wirkt der Ziliarmuskel?
  \begin{solution}
    Zonularfasern (Bindegewebsfasern)
  \end{solution}

  \question Was ist der Vor- und der Nachteil einer weiten Pupille?
  \begin{solution}
    Nachteil: weniger scharfes Bild;\quad Vorteil: hohe Empfindlichkeit
  \end{solution}

  \question Was ist der Vor- und der Nachteil einer engen Pupille?
  \begin{solution}
    Nachteil: empfindlichkeit gering;\quad Vorteil: schärferes Bild
  \end{solution}

  \question Welche Teile des autonomen Nervensystems bewirken die Erweiterung bzw. die Verengung der Pupille?
  \begin{solution}
    Sympathisches NS und parasympathisches NS
  \end{solution}

  \question Wie wirkt Stress auf die Pupille?
  \begin{solution}
    Die Pupille wird geweitet
  \end{solution}

  \question Wie wirkt Müdigkeit auf die Pupille?
  \begin{solution}
    Kontraktion der Pupille
  \end{solution}

  \question Wie wirkt eine Entspannung des Ziliarmuskels auf die Linsenwölbung?
  \begin{solution}
    Fernakkommodation, gespannte Zonularfasern, flache Linsenkrümmung
  \end{solution}

  \question Welche Linsenwölbung bewirkt eine Fernakkomodation?
  \begin{solution}
    Flache Linsesnkrümmung
  \end{solution}

  \question Welche Arten von Fehlsichtigkeit werden durch Sammel- bzw. Zerstreuungslinsen behoben?
  \begin{solution}
    \begin{itemize*}
      \item Sammellinsen - Weitsichtigkeit
      \item Zerstreuungslinsen - Kurzsichtigkeit
    \end{itemize*}
  \end{solution}

  \question Nennen Sie die 5 Zelltypen der Retina!
  \begin{solution}
    Stäbchen, Zapfen, Horizontalzellen, Biolarzellen, retinale Ganglienzellen, amakrine Zellen
  \end{solution}

  \question Welchen Neurotransmitter schütten Fotorezeptoren aus?
  \begin{solution}
    Glutamat
  \end{solution}

  \question Was ist der Neurotransmitter der Ganglien- und Bipolarzellen?
  \begin{solution}
    Glutamat
  \end{solution}

  \question Was ist der Neurotransmitter der amacrinen und Horizontalzellen?
  \begin{solution}
    GABA
  \end{solution}

  \question Welche Zelltypen kontaktieren die Synapsen der Fotorezeptoren?
  \begin{solution}
    Horizontal und bipolarzellen \Warning
  \end{solution}

  \question Wie viele synaptische Kontakte befinden sich zwischen Sehnerv und Lichtsinneszellen?
  \begin{solution}
    130 Mio \Warning
  \end{solution}

  \question Welche Zellart der Netzhaut ist dem einfallenden Licht am nächsten?
  \begin{solution}
    Axone retinaler Ganglienzellen
  \end{solution}

  \question Welche beiden Arten von Fotorezeptoren gibt es in der Retina?
  \begin{solution}
    Stäbchenzellen, Zapfenzellen
  \end{solution}

  \question Welche Art von Fotorezeptoren ist für die Farbwahrnehmung zuständig?
  \begin{solution}
    Zapfen
  \end{solution}

  \question Welche der beiden Arten von Fotorezeptoren ist zahlreicher?
  \begin{solution}
    Stäbchen
  \end{solution}

  \question Welche Auswirkungen hat Konvergenz in der Retina auf die Qualität der visuellen Information?
  \begin{solution}
    Geringere Auflösung, höhere Lichtempfindlichkeit
  \end{solution}

  \question Welche Auswirkungen hat laterale Inhibition in der Retina auf die Qualität der visuellen Information?
  \begin{solution}
    Kontrasterhöhung
  \end{solution}

  \question Wie heißt die Eintrittsstelle des Sehnervs in den Augapfel und wodurch ist diese gekennzeichnet?
  \begin{solution}
    Blinder Fleck, keine Fotorezeptoren, die Lichtreize aufnehmen können
  \end{solution}

  \question An welcher Stelle der Retina ist die Zapfendichte am höchsten?
  \begin{solution}
    Sehgrube
  \end{solution}

  \question In welchem Großhirnlappen befindet sich die primäre Sehrinde?
  \begin{solution}
    Primärer visueller Cortex \Warning
  \end{solution}

  \question Welche Auswirkungen hat die Durchtrennung des rechten Sehnerves?
  \begin{solution}
    Erblindung des Rechten Auges
  \end{solution}

  \question Welche Auswirkungen hat die Durchtrennung optischen Tracts?
  \begin{solution}
    Ausfall des linken/rechten Gesichtsfeldes beider Augen
  \end{solution}

  \question Welche Auswirkungen haben Läsionen im primären visuellen Cortex?
  \begin{solution}
    Skotome: blinde Stellen im Gesichtsfeld
  \end{solution}

  \question Was sind die Auswirkungen von Läsionen im posterioren Parietallappen auf die visuelle Wahrnehmung?
  \begin{solution}
    dass Patienten nicht mehr nach Dingen greifen können, die sie problemlos erkennen
  \end{solution}

  \question Was sind die Auswirkungen von Läsionen im inferioren Temporallappen auf die visuelle Wahrnehmung?
  \begin{solution}
    dass Patienten Dinge greifen können, die sie aber nicht beschreiben können
  \end{solution}

  \question Wozu dienen nach der alternativen Theorie von Logothetis und Steinberg die ventrale und die dorsale Bahn des visuellen Systems?
  \begin{solution}
    Dorsale Bahn dient der Verhaltensinteraktion der Objekte, ventrale Bahn der bewussten Wahrnehmung
  \end{solution}

  \question Was versteht man unter Propagnosie?
  \begin{solution}
    Unfähigkeit Gesichter zu erkennen
  \end{solution}

  \question In welchem Quadranten der primären Sehrinde wird die Information aus dem rechten unteren Quadranten des Gesichtsfelds des rechten Auges verarbeitet?
  \begin{solution}
    primärer visueller Cortex
  \end{solution}

  \question Aus welchen Beobachtungen resultiert die Farbtheorie von Young und Helmholtz?
  \begin{solution}
    Jede Farbe des sichtbaren Spektrums kann aus drei beliebigen unabhängigen Farben gemischt werden
  \end{solution}

  \question Aus welchen Beobachtungen resultiert die Farbtheorie von Hering?
  \begin{solution}
    Farben lassen sich nicht beliebig mischen (z.b. kein rötliche Grün), Schattenbilder nach Starren auf Farben
  \end{solution}

  \question Welche der beiden Farbtheorien ist tatsächlich im Gehirn implementiert?
  \begin{solution}
    Beide
  \end{solution}

  \question Wovon hängt die wahrgenommene Farbe einer Fläche ab?
  \begin{solution}
    reflektierte Wellenlänge, das benutzte Lichtspektrum, falls Fläche nicht isoliert: Umgebende Objekte
  \end{solution}

  \question Was sind die beiden möglichen Erklärungen für Blindsehen?
  \begin{solution}
    Primärer Visueller Cortex nicht vollständig zerstört; direkte Verbindung Mittelhirn und Thalamus zu höheren viusellen Gebieten
  \end{solution}

  \question Was sind die drei Abschnitte des Ohres?
  \begin{solution}
    Inneres, mittleres und äußeres Ohr
  \end{solution}

  \question Welche Struktur trennt äußeres Ohr von Mittelohr?
  \begin{solution}
    Trommelfell
  \end{solution}

  \question Welchen zwei Funktionen dient das äußere Ohr?
  \begin{solution}
    Fokussierung Schallrichtungswahrnehmung, Schalldruckverstärkung
  \end{solution}

  \question Was ist die Hauptfunktion des Mittelohrs?
  \begin{solution}
    Gesamtschalldruckverstärkung
  \end{solution}

  \question Welche Strukturmerkmale des Mittelohrs tragen zur Schalldruckverstärkung bei?
  \begin{solution}
    \begin{itemize*}
      \item Flächenverhältnis Trommelfell-Steigbügelgrundplatte
      \item Hebelarme des Gehörknöchelchen(Hammer/Amboss)
      \item Hebelarm durch die Biegung des Trommelfells und unsymmetrische Anheftung des Hammers
    \end{itemize*}
  \end{solution}

  \question Wie heißt die Knochenstruktur, in die das Innenohr eingebettet ist?
  \begin{solution}
    Felsenbein
  \end{solution}

  \question In welcher Struktur befinden sich die Hörsinneszellen und wie heißen diese?
  \begin{solution}
    Corti-Organ
  \end{solution}

  \question Mit welcher Membran ist das Corti-Organ fest verbunden?
  \begin{solution}
    membrana basilaris
  \end{solution}

  \question An welchem Ende ist die Cochlea empfindlich für hohe Frequenzen – am Helicotrema oder am ovalen Fenster?
  \begin{solution}
    ovalen Fenster
  \end{solution}

  \question Die Stereozilien welcher Haarzellen sind fest mit der Tectorialmembran verbunden?
  \begin{solution}
    äußere Haarzellen
  \end{solution}

  \question Was ist die Funktion der äußeren Haarzellen?
  \begin{solution}
    Rückkopplung zur Regulierung von Sensoroutput
  \end{solution}

  \question Welche beiden Hörbahnen kann man unterscheiden?
  \begin{solution}
    dorsale und ventrale Höhrbahn
  \end{solution}

  \question Was ist die Funktion der dorsalen Hörbahn?
  \begin{solution}
    verursacht bewusste Wahrnehmung
  \end{solution}

  \question Was ist die Funktion der ventralen Hörbahn?
  \begin{solution}
    verursacht akustische Reflexe
  \end{solution}

  \question Wo befinden sich die Zellkörper der 4 Neuronen der dorsalen Hörbahn (richtige Reihenfolge in Richtung des Hauptinformationsflusses)
  \begin{solution}
    \begin{enumerate*}
      \item Neuron = 8er Hirnnerv(Hörnerv)
      \item Neuron = Medulla(Dorsaler Cochleariskern)
      \item Neuron = Mittelhirn(Colliculus inf.)
      \item Neuron = Zwischenhirn(Innerer Kniehöcker)
    \end{enumerate*}
  \end{solution}

  \question In welcher Hirnhälfte bezüglich des entsprechenden Ohres endet die dorsale Hörbahn?
  \begin{solution}
    Linke Hirnhälfte
  \end{solution}

  \question In welchem Hirnlappen findet die kortikale Verarbeitung auditorischer Information hauptsächlich statt?
  \begin{solution}
    Temporallapen
  \end{solution}

  \question Mit welchem Gerät kann man untersuchen, ob ein Patient an Mittel- oder Innenohrtaubheit leidet?
  \begin{solution}
    Stimmgabel
  \end{solution}

  \question Nennen Sie eine mögliche Ursache für Mittelohrtaubheit!
  \begin{solution}
    Riß im Trommelfell
  \end{solution}

  \question Nennen Sie eine mögliche Ursache für Innenohrtaubheit!
  \begin{solution}
    Verletzung Cochlea
  \end{solution}

  \question Womit kann Innenohrtaubheit therapiert werden?
  \begin{solution}
    Cochlea Implantate
  \end{solution}

  \question Aus welchen 5 flüssigkeitsgefüllten Hohlräumen besteht das Labyrinth-Organ?
  \begin{solution}
    Sacculus, Utriculus, anterior Kanal, posterior Kanal, horizontal Kanal
  \end{solution}

  \question Nennen Sie die 5 wichtigen Projektionsziele vestibulärer Nervenfasern!
  \begin{solution}
    Rückenmark, Thalamus, Retikuläre Formation, Cerebellum, auf die Kerne des 3,4,6 Hirnnervs
  \end{solution}

  \question Nennen Sie 2 häufige vestibuläre Störungen!
  \begin{solution}
    Neuritis Vestibularis, Gutartiger Lagerungschwindel
  \end{solution}

  \question Was ist die Ursache des gutartigen Lagerungsschwindels?
  \begin{solution}
    Ablösung Otholiten und ,,herumschlingern'' in den Bogengängen
  \end{solution}

  \question Was ist die Ursache der Neuritis vestibularis?
  \begin{solution}
    Entzündung des Vestibularnervs
  \end{solution}

  \question In welchem Teil des Gehirns endet der Riechnerv?
  \begin{solution}
    Riechhirn (Bulbus Olfactorius)
  \end{solution}

  \question Welche Arten von Neuronen im ZNS werden ständig erneuert?
  \begin{solution}
    Riechzellen
  \end{solution}

  \question Wodurch entstehen komplexe Geschmacksempfindungen?
  \begin{solution}
    Interaktion mit anderen Sinnen
  \end{solution}

  \question Auf welchem Teil der Zunge schmecken wir süß?
  \begin{solution}
    Zungenspitze
  \end{solution}

  \question Welche kognitive Funktion ist besonders mit dem Hippocampus verbunden?
  \begin{solution}
    Bildung von Erinnerungen
  \end{solution}

  \question In welchem Großhirnlappen befindet sich der Hippocampus?
  \begin{solution}
    Temporallappen
  \end{solution}

  \question An welche anderen limbischen Strukturen grenzt der Hippocampus.
  \begin{solution}
    Amygdala und entohirnaler Cortex
  \end{solution}

  \question Was ist die Haupteingangsstruktur für den Hippocampus?
  \begin{solution}
    Entohirnaler Cortex
  \end{solution}

  \question Aus welchem strukturellen Cortextyp besteht der Hippocampus?
  \begin{solution}
    Allocortex
  \end{solution}

  \question An welche andere limbische Struktur grenzt der Mandelkern unmittelbar?
  \begin{solution}
    Hippocampus \Warning
  \end{solution}

  \question Bei welcher kognitiven Funktion spielt die Amygdala eine herausragende Rolle?
  \begin{solution}
    Angst und Furcht
  \end{solution}

  \question Wie breiten sich die meisten Hormone aus?
  \begin{solution}
    Blutkreislauf
  \end{solution}

  \question Wo werden die meisten Hormone freigesetzt?
  \begin{solution}
    Gehirn/Hypothalamus \Warning
  \end{solution}

  \question Nennen Sie die drei wichtigsten chemischen Gruppen von Hormonen!
  \begin{solution}
    Peptide \& Proteine, Aminosäurederivate, Steroide
  \end{solution}

  \question Was sind Peptide?
  \begin{solution}
    Ketten von Aminosäuren
  \end{solution}

  \question Welcher Teil des Gehirns spielt eine zentrale Rolle bei der Hormonausschüttung?
  \begin{solution}
    Hypothalamus
  \end{solution}

  \question Welche Drüse spielt im hormonellen System eine übergeordnete Rolle?
  \begin{solution}
    Hypophyse
  \end{solution}

  \question Nennen Sie 5 wichtige Hormondrüsen!
  \begin{solution}
    Nebenniere, Schilddrüse, Hypothalamus, Bauchspeicheldrüse, Hoden/Eierstock
  \end{solution}

  \question Welcher Teil der Hypophyse wird direkt vom Hypothalamus innerviert?
  \begin{solution}
    Hypophysenhinterlappen
  \end{solution}

  \question Über welchen Signalweg wird die Information vom Hypothalamus zum Hypophysenvorderlappen übermittelt?
  \begin{solution}
    Hypothalamusneuronen zu hypothalamo-hypophysäre Pfortadersysten zu Hypophysenstiel
  \end{solution}

  \question Welche Hormone werden hauptsächlich durch den Hypophysenhinterlappen ausgeschüttet?
  \begin{solution}
    Oxytocin,Vasopressin
  \end{solution}

  \question Durch welche drei Mechanismen wird die Hormonfreisetzung geregelt und der Homonspiegel stabilisiert?
  \begin{solution}
    Nervensystem (=Innervierung meist durch autonomes Nervensystem), andere Hormone, nichthormonelle Substanzen
  \end{solution}

  \question Wo werden steroide Sexualhormone produziert?
  \begin{solution}
    Keimdrüsen(Gonaden:Hoden,Eierstock)
  \end{solution}

  \question Welche 3 Grundklassen von steroiden Sexualhormonen gibt es?
  \begin{solution}
    Androgene, Östrogene, Gestagene
  \end{solution}

  \question Wie erfolgt der Freisetzung von Sexualhormonen in Frauen und Männern?
  \begin{solution}
    Männer = Gleichmäßig, Frauen = Zyklisch; Verschiedene Dynamiken über Hypophyse vom Hypothalamus gesteuert
  \end{solution}

  \question Welches Hormon sorgt vor und unmittelbar nach der Geburt für eine männliche Entwicklung?
  \begin{solution}
    Testosteron
  \end{solution}

  \question Durch welche Hormone wird das weibliche Sexualverhalten beim Menschen maßgeblich gesteuert?
  \begin{solution}
    Androgene
  \end{solution}

  \question Welche Arten von Stresshormonen werden bei kurzfristigem und langfristigem Stress ausgeschüttet?
  \begin{solution}
    Kurzfristiger Stress: Katecholamine; \quad Langfristiger Stress: Glukokortikoide
  \end{solution}

  \question Nennen Sie ein typisches glukokortikoides Stresshormon!
  \begin{solution}
    Cortisol
  \end{solution}

  \question Welche beiden Hormone werden im Nebennierenmark ausgeschüttet?
  \begin{solution}
    Adrenalin (Epinephrin) und Noradrenalin (Norepinephrin)
  \end{solution}

  \question Welche beiden Gruppen von Hormonen werden in der Nebennierenrinde ausgeschüttet?
  \begin{solution}
    Glukokortikoiden und Androgenen
  \end{solution}

  \question Nennen Sie 2 wichtige Wirkungen von Glukokortikoiden!
  \begin{solution}
    \begin{itemize*}
      \item Beeinflussung des Stoffwechsels: Neubildung von Kohlenhydraten aus Proteinen und Fetten
      \item Beeinflussung von Wasser- und Elektrolythaushalt
      \item Unterdrückung der Antikörperproduktion des Immunsystems, dadurch Entzündungshemmung
    \end{itemize*}
  \end{solution}

  \question Welche beiden chemischen Elemente sind für die Bildung von Schilddrüsenhormonen von Bedeutung?
  \begin{solution}
    Iod und Eisen
  \end{solution}

  \question Was ist die Hauptwirkung der Schilddrüsenhormone?
  \begin{solution}
    Regelung des Grundumsatzes
  \end{solution}

  \question Wozu führt Schilddrüsenunterfunktion im Erwachsenenalter?
  \begin{solution}
    Stoffwechselverlangsamung, Verringerung der Leistungsfähigkeit
  \end{solution}

  \question Welches Hormon wird von der Geburtshilfemedizin im sogenannten „Wehentropf“ verwendet?
  \begin{solution}
    Oxytocin
  \end{solution}

  \question Wodurch wird die Ausschüttung von Oxytocin ausgelöst?
  \begin{solution}
    Angenehmer Hautkontakt (Kuschelhormon)
  \end{solution}

  \question Welche neuronalen Populationen haben Sympathikus und Parasympathikus und wo befinden sich diese?
  \begin{solution}
    \begin{itemize*}
      \item Sympathikus = Ganglien Nahe der Wirbelsäule  \Warning
      \item Parasympathicus = Ganglien nahe oder in den Organen  \Warning
    \end{itemize*}
  \end{solution}

  \question Zu welchem Bestandteil des autonomen Nervensystems gehört der Grenzstrang?
  \begin{solution}
    Zentraler Teil  \Warning
  \end{solution}

  \question Wo befinden sich allgemein die autonomen Ganglien des Sympathikus und des Parasympathikus?
  \begin{solution}
    Zwischen Zentralnervensystem und inneren Organen
  \end{solution}

  \question Welcher Neurotransmitter wird durch die präganglionären Neuronen des Sympathikus ausgeschüttet?
  \begin{solution}
    Acetylcholin
  \end{solution}

  \question Welcher Neurotransmitter wird durch die postganglionären Neuronen des Sympathikus ausgeschüttet?
  \begin{solution}
    (Nor)Adrenalin
  \end{solution}

  \question Welcher Neurotransmitter wird durch die präganglionären Neuronen des Parasympathikus ausgeschüttet?
  \begin{solution}
    Acetylcholin
  \end{solution}

  \question Welcher Neurotransmitter wird durch die postganglionären Neuronen des Parasympathikus ausgeschüttet?
  \begin{solution}
    Acetylcholin
  \end{solution}

  \question Wo befinden sich die Zellkörper der präganglionären sympathischen Neuronen?
  \begin{solution}
    Brust und Lendenmark
  \end{solution}

  \question Wo befinden sich die Zellkörper der präganglionären parasympathischen Neuronen?
  \begin{solution}
    Hirnstamm, Mittelhirn, Sakralmark
  \end{solution}

  \question Über welchen Pfad übt der Sympathikus eine globale Wirkung auf den Organismus aus?
  \begin{solution}
    Grenzstrang (Truncus sympathicus) \Warning
  \end{solution}

  \question Was sind die grundsätzlichen Rollen von Sympathikus und Parasympathikus?
  \begin{solution}
    \begin{itemize*}
      \item Symphatikus: Vorbereitung Flucht und Kampf
      \item Parasymphatikus: Entspannung und Verdauung
    \end{itemize*}
  \end{solution}

  \question Nennen Sie 4 Hauptwirkungen des Sympathikus!
  \begin{solution}
    Atemfrequenz steigern, Herzfrequenz steigern, Darmtätigkeit senken, Glykogenmetabolismus i.d. Leber steigern, Schwitzen, Pupillenerweiterung
  \end{solution}

  \question Nennen Sie 4 Hauptwirkungen des Parasympathikus!
  \begin{solution}
    Atemfrequenz senken, Herzfrequenz senken, Darmtätigkeit steigern, Pupillen verengen
  \end{solution}

  \question Nennen Sie 4 Funktionen des Hypothalamus!
  \begin{solution}
    Körpertemperaturregelung, Regelung Wasserhaushalt, Regelung Hormonsekretion in Hypophyse, Regelung physiologischer Reaktion auf Erregungszustände
  \end{solution}

  \question Nennen Sie die drei Phasen des Energiestoffwechsels und geben Sie an durch welche charakteristischen Hormonspiegel diese gekennzeichnet sind!
  \begin{solution}
    Cephalische Phase, Absortive Phase, Fastenphase; durch Insulin und Glukagonspiegel
  \end{solution}

  \question Nennen Sie 3 Merkmale der cephalischen und absorptiven Energiestoffwechselphasen!
  \begin{solution}
    niedriger Glukagonspiegel, hoher Insulinspiegel, fördert Nutzung Blutzucker(Glukose) als Energiequelle
  \end{solution}

  \question Nennen Sie 3 Merkmale der Fastenphase des Energiestoffwechsels!
  \begin{solution}
    Hoher Glukagonspiegel, niedriger Insulinspiegel, fördert Umwandlung Fette zu Fettsäuren, Nutzung freier Fettsäuren als Energiequelle
  \end{solution}

  \question Nennen Sie 3 Argumente die gegen die Sollwerthypothese der Nahrungsaufnahme sprechen!
  \begin{solution}
    \begin{itemize*}
      \item Evolution: Nahrung musste aufgenommen werden, wenn sie verfügbar war
      \item Experiment: Schwankungen in Körperfett und Blutzucker beeinflussen die Nahrungsaufnahme nur, wenn sie extrem sind
      \item Nahrungsaufnahme wird durch vielerlei Faktoren bestimmt, wie visuelle und olfaktorische
      Reize, Emotionen, Stress usw.
    \end{itemize*}
  \end{solution}

  \question Was ist die Alternative zur Sollwerthypothese der Nahrungsaufnahme?
  \begin{solution}
    Positive Anreiztheorie
  \end{solution}

  \question Erläuterns Sie einen der wichtigen Mechanismen zur Regulierung von Hunger und Sättigung!
  \begin{solution}
    \begin{itemize*}
      \item Magen-Darm-Trakt: Freisetzung von Peptiden, die an Neurorezeptoren im Gehirn (z.B. im Hypothalamus) binden und als Sättigungssignal wirken.
      \item Serotonin: verringert Anziehungskraft schmackhafter Nahrung, reduziert die Aufnahme pro Mahlzeit, verlagert Präferenzen weg von fetthaltiger Nahrung. Appetitszügler sind häufig Serotoninagonisten.
      \item Leptin, Insulin und andere: regulieren die Anlage von Fettdepots, Leptinmangel führt zu exzessiver Nahrungsaufnahme und Fettleibigkeit. Bei Insulinmangel isst man viel und bleibt schlank, da die Nahrung nicht in Fettdepots umgewandelt werden kann.
    \end{itemize*}
  \end{solution}

  \question Wie viele Schlafphasen unterscheidet man und welche davon bezeichnet man als Slow-Wave-Sleep?
  \begin{solution}
    4 Phasen= 3 und 4 ist SlowWaveSleep
  \end{solution}

  \question Nennen Sie die beiden wichtigen physiologischen Korrelate von Schlafphase 1!
  \begin{solution}
    Schnelle Augenbewegungen und Muskeltonusverlust
  \end{solution}

  \question Wie verändert sich der Schlafrhythmus im Verlauf der Nacht?
  \begin{solution}
    Anteil REM-Schlaf nimmt in der Nacht zu
  \end{solution}

  \question Nennen Sie die beiden grundsätzlichen Theorien zur Notwendigkeit von Schlaf!
  \begin{solution}
    Regenerative Theorien, Circadiane Theorien
  \end{solution}

  \question Nennen Sie drei wichtige Auswirkungen von Schlafentzug!
  \begin{solution}
    Schlafneigung (Müdigkeit, Sekundenschlaf), Stimmungsverschlechterung, Verschlechterung der Aufmerksamkeit
  \end{solution}

  \question Nennen Sie 3 mögliche Ursachen für Insomnie!
  \begin{solution}
    Schlafmittel, Muskelprobleme, Probleme mit Atemzentrum im Hirnstamm, nächtliche Myoklonien, Restless-Leg-Syndrom
  \end{solution}

  \question Nennen Sie die Arten und Unterarten des Langzeitgedächtnisses!
  \begin{solution}
    \begin{itemize*}
      \item explizit(deklarativ) = episodisch und semantisch
      \item implizit=prozdeural und perzeptionell
    \end{itemize*}
  \end{solution}

  \question Welche drei Grundarten von Gedächtnis unterscheiden wir?
  \begin{solution}
    Sensorisch, Kurzzeit, Langzeit
  \end{solution}

  \question Was versteht man unter anterograder und retrograder Amnesie?
  \begin{solution}
    \begin{itemize*}
      \item anterograd = Abspeicherung gestört
      \item retrograd = Tendenz rezente Gedächtnisinhalte zu verlieren
    \end{itemize*}
  \end{solution}

  \question Die Entfernung welcher Hirnstruktur führte beim Patienten H.M. zu anterograder Amnesie des expliziten Langzeitgedächtnisses?
  \begin{solution}
    beider medialer Temporallappen
  \end{solution}

  \question Wo werden, allgemein, Langzeitgedächtnisinhalte abgespeichert?
  \begin{solution}
    Langzeitgedächtnisinhalte sind in denselben Hirnarealen gespeichert, die auch für die ursprüngliche Erfahrung zuständig sind
  \end{solution}

  \question Erläutern Sie kurz das Prinzip des Hebbschen Lernens!
  \begin{solution}
    Information im Arbeitsgedächtnis gehalten; durch periodische Aktivität von Neuronetzwerken werden Langzeitveränderungen in synaptischen Verbindungen hervorgerufen
  \end{solution}

  \question In welche Emotion ist der Mandelkern besonders involviert?
  \begin{solution}
    Angst
  \end{solution}

  \question Welche Hirnhälfte ist in den meisten Menschen dominant?
  \begin{solution}
    Linke Hirnhälfte
  \end{solution}

  \question Wodurch können die Hirnhälften von Split-Brain Patienten in der Praxis kommunizieren und koordiniert agieren?
  \begin{solution}
    Hirnhälften verfügen in der Praxis fast über die Gleichen Informationen
  \end{solution}

  \question Nennen Sie die 7 wichtigen Bestandteile des Wernicke Geschwind-Modells!
  \begin{solution}
    Broca Areal, primärer motorischer Cortex, Fasciculus arcuatus, primärer auditorischer Cortex, Wernicke Areal, Gyrus Angularis, primärer visueller Cortex
  \end{solution}

  \question Nennen Sie drei Methoden mit denen die Voraussagen des Wernicke-Geschwind-Modells überprüft wurden!
  \begin{solution}
    Läsionen durch chirurgische Eingriffe, Läsionen durch Krankheit oder Unfall, Elektrische Stimulation des Cortex
  \end{solution}

  \question Welche beiden allgemeinen Voraussagen des Wernicke-Geschwind-Modells können durch die experimentellen Befunde bestätigt werden?
  \begin{solution}
    Broca- und Wernickegebiet spielen eine wichtige Rolle bei Sprache, anteriore Läsionen verursachen tendenziell eher expressive und posteriore Läsionen rezeptiver Defizite
  \end{solution}

  \question Nennen Sie 5 Symptome für eine depressive Episode!
  \begin{solution}
    Hauptsymptome
    \begin{itemize*}
      \item Depressive Stimmung während des größten Teils der meisten Tage
      \item Geringes Interesse an den meisten Aktivitäten an allen Tagen
      \item Verminderter Antrieb
    \end{itemize*}
    Zusatzsymptome:
    \begin{itemize*}
      \item Schläfrigkeit oder Schlaflosigkeit, Appetitlosigkeit, Schuldgefühle
      \item Vermindertes Selbstwertgefühl und Selbstvertrauen
      \item Entscheidungsschwäche, Konzentrationsschwäche, Selbstmordgedanken und -versuche, Pessimismus
    \end{itemize*}
  \end{solution}

  \question Nennen Sie 5 Symptome für eine manische Episode!
  \begin{solution}
    Übersteigertes Selbstbewußtsein, Verringertes Schlafbedürfnis, Erhöhtes Redebedürfnis, Sprechzwang, Gedanken und Ideen ,,rasen'', Ablenkbarkeit, Erhöhte zielgerichtete Aktivität, Vergnügungssucht ohne Bedenken der Konsequenzen (z.B. Kaufrausch, sexuelle Abenteuer), Euphorie (die schnell in Gereiztheit umschlägt), Soziale Enthemmung
  \end{solution}

  \question Welche beiden Verlaufsformen affektiver Störungen kennen wir?
  \begin{solution}
    Unipolare Depression, Bipolare Depression
  \end{solution}

  \question Bei welcher Verlaufsform affektiver Störungen gibt es keine Geschlechtsunterschiede?
  \begin{solution}
    Bipolare Depression
  \end{solution}

  \question Nennen Sie 3 pharmakologische Therapien gegen Depressionen!
  \begin{solution}
    (Monoaminoxidase) MAO-Hemmer, Trizyklische Antidepressiva (TCAs), Selektive Wiederaufnahmehemmer
  \end{solution}

  \question Welche nicht-pharmakologische antidepressive Therapie hat eine hohe Wirksamkeit?
  \begin{solution}
    Elektrokonvulsive Therapie
  \end{solution}

  \question Erläutern Sie das Wirkprinzip von MAO-Hemmern!
  \begin{solution}
    \begin{itemize*}
      \item MAO zerstört Neurotransmitter außerhalb der Vesikel
      \item Durch Hemmung Menge von Serotonin Dopamin und Noradrenalin erhöht
      \item Adaptive Änderung Repzeptordichte und Second-Messenger-Kette= Ziel erreicht
    \end{itemize*}
  \end{solution}

  \question Erläutern Sie das Wirkprinzip von trizyklischen Antidepressiva!
  \begin{solution}
    \begin{itemize*}
      \item Blockade präsynaptischer Transporterproteine und Hemmung der Wiederaufnahme von Serotonin und/oder Noradrenalin
      \item Führt zu Veränderungen der post- und präsynaptischen Rezeptordichten
      \item Daneben Wirkung auf Histamin-, Acetylcholin- und Adrenalinrezeptoren(Wirkung auf verschiedene Rezeptoren Unterschiedlich
    \end{itemize*}
  \end{solution}

  \question Nennen Sie drei wichtige Nebenwirkungen von MAO-Hemmern!
  \begin{solution}
    \begin{itemize*}
      \item Schlafstörungen, Blutdruckveränderungen, Heißhunger
      \item Tyraminabbau in der Leber behindert = spezielle Diät notwendig
      \item Interaktion mit vielen Drogen, z.B. Babiturate, Aspirin, Alkohol, Opiate, und Medikamenten $\rightarrow$ z.B. Serotonin-Syndrom
    \end{itemize*}
  \end{solution}

  \question Nennen Sie drei wichtige Nebenwirkungen von trizyklischen Antidepressiva!
  \begin{solution}
    \begin{itemize*}
      \item durch Histamin-Rezeptor-Blockade: Sedierung
      \item durch Azetylcholin-Rezeptor-Blockade: trockener Mund, Schwindel, Darmträgheit, Verwirrung, Gedächtnis- und Sehstörungen
      \item durch Blockade der $\alpha$-adrenergen Rezeptoren: kardiovaskulare Probleme
    \end{itemize*}
  \end{solution}

  \question Nennen Sie drei wichtige Nebenwirkungen von Antidepressiva der 2. Generation
  \begin{solution}
    \begin{itemize*}
      \item serotoninrelatierte Nebeneffekte: Magen-Darm-Störungen, sexuelle Störungen, emotionale Abstumpfung, Nervosität und Schlafstörungen (auch Angst).
      \item Potentiell gefährliche Interaktionen mit anderen Medikamenten und Drogen(Serotoninsyndrom)
      \item physische Abhängigkeit möglich
    \end{itemize*}
  \end{solution}

  \question Erläutern Sie das Wirkprinzip der Elektrokonvulsiven Therapie!
  \begin{solution}
    \begin{itemize*}
      \item Elektrische Reizung im Gehirn führ zu einem Epileptischen Anfall
      \item Kein Bewusstes Erleben des Anfalls durch Narkose und Muskelrelaxationmedikation
      \item Verstärkt Wirkung vieler Neurotransmitter(bewirkt damit Herrunterregulierung Rezeptordichte)
    \end{itemize*}
  \end{solution}

  \question Nennen Sie die 3 wichtigsten neurobiologischen Theorien über affektive Störungen!
  \begin{solution}
    Monoamin-Hypothese, Glukokortikoid-Hypothese, Neurotrophische Hypothese
  \end{solution}

  \question Auf welchen Beobachtungen beruht die Monoamin-Hypothese zu affektiven Störungen?
  \begin{solution}
    MAO-Hemmer reduzieren Depressions-Symptome;
    Reduzierte Mengen von Noradrenalin- und Serotoninmetaboliten in Nervenwasser, Blut und Urin von Depressiven;
    Monoamin-Agonisten produzieren manieähnliche Symptome
  \end{solution}

  \question Auf welchen Beobachtungen beruht die Glukokortikoid-Hypothese zu affektiven Störungen?
  \begin{solution}
    Stress und Angst gehen depressiven Episoden oft voraus.
    Depression geht oft mit veränderten Stresshormonspiegeln einher.
    Die Wahrscheinlichkeit, dass erhöhter Stress affektive Störungen auslöst, scheint genetisch bedingt.
  \end{solution}

  \question Nennen Sie die 5 Klassen von Angststörungen!
  \begin{solution}
    Generalisierte Angststörung, Posttraumatisches Stresssyndrom, Phobien, Zwangsneurosen, Panikstörungen
  \end{solution}

  \question Was ist Furcht?
  \begin{solution}
    auf konkrete Bedrohung gerichtete Angst
  \end{solution}

  \question Was ist eine effektive Therapieform für Phobien?
  \begin{solution}
    Verhaltenstherapie (z.B. Konfrontationsverfahren)
  \end{solution}

  \question Nennen Sie 2 Gruppen von Psychopharmaka, die bei Angststörungen eingesetzt wurden bzw. werden!
  \begin{solution}
    Bariburate, Benzodiazepine
  \end{solution}

  \question Erläutern Sie das Wirkprinzip von Barbituraten!
  \begin{solution}
    GABA Agonist, Eingeteilt nach Fettlöslichkeit und Pharmakinetik; je Fettlöslicher, desto schneller setzt Wirkung ein und desto kürzer hält sie an
  \end{solution}

  \question Nennen Sie 4 wichtige Nebenwirkungen von Barbituraten!
  \begin{solution}
    \begin{itemize*}
      \item Barbiturat-induzierter Schlaf ist suboptimal mit reduzierten REM-Perioden
      \item Benommenheit, verlangsamte Reflexe, Müdigkeit
      \item Bei Überdosierung: Symptome wie bei Alkohol
      \item Starke Überdosierung: Coma und Tod
    \end{itemize*}
  \end{solution}

  \question Erläutern Sie das Wirkprinzip von Benzodiazepinen!
  \begin{solution}
    Aktivierung Benzodiazepin Rezeptoren (GABA-agonistischer Effekt: Wirkt nur mit GABA,
    Stärker an Synapsen mit wenig GABA(Aktivitätsabhängige Wirkung), verschiedene Wirkungs- und Verstoffwechlungsgeschwindigkeiten
  \end{solution}

  \question Welches sind die beiden Symptomgruppen bei Schizophrenie?
  \begin{solution}
    Positive und Negative Symptome
  \end{solution}

  \question Nennen Sie 3 positive Symptome von Schizophrenie!
  \begin{solution}
    Wahnvorstellungen und Halluzinationen, Sprachstörungen, Bizarres Verhalten, motorische Unruhe
  \end{solution}

  \question Nennen Sie 3 negative Symptome von Schizophrenie!
  \begin{solution}
    Niedergang normaler Hirnfunktion (wie reduzierte Sprache - Alogie), Emotionslosigkeit, Antriebslosigkeit, sozialer Rückzug, intellektuelle Behinderung
  \end{solution}

  \question Welche Gruppe von Symptomen der Schizophrenie spricht besser auf Neuroleptika an?
  \begin{solution}
    Positive Symptome
  \end{solution}

  \question Was ist das wichtigste Wirkprinzip klassischer Neuroleptika?
  \begin{solution}
    Dopaminantagonismus (besonders D2)
  \end{solution}

  \question Nennen Sie die 5 wichtigsten Dopaminpfade im Gehirn und deren Rolle bei Schizophrenie und der Wirkung von Neuroleptika
  \begin{solution}
    Nigrostriataler Pfad,
    Mesolimbischer Pfad,
    Mesokortikaler Pfad,
    Tuberohypophysischer Pfad,
    \Warning Substantia Nigra (hoher Dopamingehalt vorhanden )

    Rolle Schizophrenie
    \begin{itemize*}
      \item Extrapyramidale Nebenwirkungen (Nigrostriataler Pfad)
      \item Positive Symptome (Mesolimbischer Pfad)
      \item Negative Symptome (Tuberohypophysischer Pfad)
      \item Neuroendokrinologische Nebenwirkungen
    \end{itemize*}

    (Neben)Wirkung Neuroleptika :
    \begin{itemize*}
      \item Parkinson Symptome (Tremor, Rigor, Akinese, Mimikverlust) (Nigrostriataler Pfad)
      \item Neuroendokrinologische Nebenwirkungen (Brustvergrößerungen, sexuelle Störungen, Wachstumsstörungen, Gewichtszunahme)(Tuberohypophysischer Pfad)
      \item autonome Störungen (Mundtrockenheit, Verdauungsprobleme, Sehstörungen, Schwindel, Sedierung)(Beeinflussung der cholinerger und adrenerger Neuronen)
      \item Tardive Dyskines: unwillkürliche stereotype Bewegungen (besonders Kau-, Schnalz- und Saugbewegungen, auch Arm-, Bein- und Rumpfbewegungen)
      \item Malignes neuroleptisches Syndrom: seltene, sich schnell entwickelnde und lebensbedrohliche Komplikation; mit extrapyramidalen Symptomen, autonomer Entgleisung, psychischen Störungen und schließlich Multiorganversagen.
    \end{itemize*}
  \end{solution}

\end{questions}
\end{document}