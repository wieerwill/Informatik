\documentclass[10pt, a4paper]{exam}
\printanswers			  % Comment this line to hide the answers 
\usepackage[utf8]{inputenc}
\usepackage[T1]{fontenc}
\usepackage[ngerman]{babel}
\usepackage{listings}
\usepackage{float}
\usepackage{graphicx}
\usepackage{color}
\usepackage{listings}
\usepackage[dvipsnames]{xcolor}
\usepackage{tabularx}
\usepackage{geometry}
\usepackage{color,graphicx,overpic}
\usepackage{amsmath,amsthm,amsfonts,amssymb}
\usepackage{tabularx}
\usepackage{listings}
\usepackage[many]{tcolorbox}
\usepackage{multicol}
\usepackage{hyperref}
\usepackage{pgfplots}
\usepackage{bussproofs}

\pdfinfo{
  /Title (Einführung in die Neurowissenschaften - Fragenkatalog)
  /Creator (TeX)
  /Producer (pdfTex 1.40.0)
  /Author (Robert Jeutter)
  /Subject ()
}
\title{Einführung in die Neurowissenschaften - Fragenkatalog}
\author{}
\date{}

% Don't print section numbers
\setcounter{secnumdepth}{0}

\newtcolorbox{myboxii}[1][]{
 breakable,
 freelance,
 title=#1,
 colback=white,
 colbacktitle=white,
 coltitle=black,
 fonttitle=\bfseries,
 bottomrule=0pt,
 boxrule=0pt,
 colframe=white,
 overlay unbroken and first={
 \draw[red!75!black,line width=3pt]
  ([xshift=5pt]frame.north west) -- 
  (frame.north west) -- 
  (frame.south west);
 \draw[red!75!black,line width=3pt]
  ([xshift=-5pt]frame.north east) -- 
  (frame.north east) -- 
  (frame.south east);
 },
 overlay unbroken app={
 \draw[red!75!black,line width=3pt,line cap=rect]
  (frame.south west) -- 
  ([xshift=5pt]frame.south west);
 \draw[red!75!black,line width=3pt,line cap=rect]
  (frame.south east) -- 
  ([xshift=-5pt]frame.south east);
 },
 overlay middle and last={
 \draw[red!75!black,line width=3pt]
  (frame.north west) -- 
  (frame.south west);
 \draw[red!75!black,line width=3pt]
  (frame.north east) -- 
  (frame.south east);
 },
 overlay last app={
 \draw[red!75!black,line width=3pt,line cap=rect]
  (frame.south west) --
  ([xshift=5pt]frame.south west);
 \draw[red!75!black,line width=3pt,line cap=rect]
  (frame.south east) --
  ([xshift=-5pt]frame.south east);
 },
}
\renewcommand{\solutiontitle}{\noindent \enspace}

\begin{document}
\begin{myboxii}[Disclaimer]
  Die Fragen die hier gezeigt werden stammen aus der Vorlesung \textit{Einführung in die Neurowissenschaften}! Für die Korrektheit der Lösungen wird keine Gewähr gegeben.
\end{myboxii}

%##########################################
\begin{questions}
  \question Welche spezifische Eigenschaft des Organismus wird hauptsächlich durch das Nervensystem realisiert?
  \begin{solution}
  \end{solution}
  \question Welches andere funktionelle System steht in besonderer Beziehung zum Nervensystem?
  \begin{solution}
  \end{solution}
  \question Nennen Sie die Untersysteme des Nervensystems!
  \begin{solution}
  \end{solution}
  \question Welches Untersystem des Nervensystems ist für die Kommunikation mit der äußeren Umwelt zuständig?
  \begin{solution}
  \end{solution}
  \question Welches Untersystem des Nervensystems ist für die Kommunikation mit anderen Organsystemen zuständig?
  \begin{solution}
  \end{solution}
  \question Nennen Sie die Grundbestandteile des Zentralnervensystems!
  \begin{solution}
  \end{solution}
  \question Nennen Sie die vier Hauptbestandteile des autonomen Nervensystems!
  \begin{solution}
  \end{solution}
  \question Welches Untersystem des autonomen Nervensystems bereitet den Organismus auf Flucht oder Kampf vor?
  \begin{solution}
  \end{solution}
  \question Nennen Sie die beiden grundsätzlichen Typen von Zellen im Nervengewebe!
  \begin{solution}
  \end{solution}
  \question In welcher Art von Nervengewebe befinden sich die neuronalen Zellkörper?
  \begin{solution}
  \end{solution}
  \question Welches Nervengewebe befindet sich im Rückenmark außen?
  \begin{solution}
  \end{solution}
  \question Was sind die wichtigsten funktionellen Merkmale von Neuronen im Unterschied zu anderen Zellen?
  \begin{solution}
  \end{solution}
  \question Ordnen Sie die folgenden anatomischen Merkmale einer Nervenzelle zu: Axonshügel, Ranvierscher Schnürring, Synapse, Dendrit, Axon!
  \begin{solution}
  \end{solution}
  \question Welches ist der Geschwindigkeitsbereich in dem sich Aktionspotentiale fortpflanzen?
  \begin{solution}
  \end{solution}
  \question Wodurch wird ein Aktionspotential ausgelöst?
  \begin{solution}
  \end{solution}
  \question Was geschieht, wenn ein Aktionspotential einen synaptischen Endknopf erreicht?
  \begin{solution}
  \end{solution}
  \question Welche Ionenarten sind im Intra- und welche im Extrazellulärraum in erhöhter Konzentration vorhanden?
  \begin{solution}
  \end{solution}
  \question Was ist ein typischer Wert für das Membran-Ruhepotential von Neuronen?
  \begin{solution}
  \end{solution}
  \question Nennen Sie die drei Antriebskräfte für den Ionentransport durch die Zellmembran!
  \begin{solution}
  \end{solution}
  \question Welche Ionenkanäle werden bei der Auslösung eines Aktionspotentials als erstes und welche als zweites ausgelöst?
  \begin{solution}
  \end{solution}
  \question Wie hoch ist die ungefähre maximale Impulsrate auf Axonen und wodurch wird diese begrenzt?
  \begin{solution}
  \end{solution}
  \question Weshalb breiten sich Aktionspotentiale nur in eine Richtung aus?
  \begin{solution}
  \end{solution}
  \question Von welchen beiden Faktoren hängt die Ausbreitungsgeschwindigkeit der Aktionspotentiale hauptsächlich ab?
  \begin{solution}
  \end{solution}
  \question Was versteht man unter saltatorischer Erregungsleitung?
  \begin{solution}
  \end{solution}
  \question Durch welchen Zelltyp werden die Myelinscheiden im Zentral- und im Perphernervensystem gebildet?
  \begin{solution}
  \end{solution}
  \question Welche Krankheit beeinträchtigt die Myelinscheiden der Axone?
  \begin{solution}
  \end{solution}
  \question Worin befinden sich die Neurotransmitter in den synaptischen Endknöpfen?
  \begin{solution}
  \end{solution}
  \question Was sind die beiden informationsverarbeitenden Grundfunktionen einer Synapse?
  \begin{solution}
  \end{solution}
  \question Über welche beiden Dimensionen findet die Integration von Information in einem Neuron statt?
  \begin{solution}
  \end{solution}
  \question Durch welche Potentiale werden Informationen in Neuronen digital bzw. analog repräsentiert?
  \begin{solution}
  \end{solution}
  \question Was sind Neurotransmitter?
  \begin{solution}
  \end{solution}
  \question Nennen Sie die 4 Merkmale von Neurotransmittern!
  \begin{solution}
  \end{solution}
  \question Was sind die beiden Arten von Neurorezeptoren?
  \begin{solution}
  \end{solution}
  \question Wie funktioniert ein ionotroper Rezeptor?
  \begin{solution}
  \end{solution}
  \question Wie funktioniert ein metabotroper Rezeptor?
  \begin{solution}
  \end{solution}
  \question Welche Art von Neurorezeptoren ist häufiger – ionotrope oder metalotrope?
  \begin{solution}
  \end{solution}
  \question Nennen Sie 7 Neurotransmitter!
  \begin{solution}
  \end{solution}
  \question Nennen Sie 3 Monoamine, die als Neurotransmitter fungieren!
  \begin{solution}
  \end{solution}
  \question Nennen Sie 3 Aminosäuren, die als Neurotransmitter fungieren!
  \begin{solution}
  \end{solution}
  \question Nennen Sie den wichtigsten erregenden und den wichtigsten hemmenden Neurotransmitter im Gehirn!
  \begin{solution}
  \end{solution}
  \question Welches sind die drei wichtigen Orte mit dopaminergen Neuronen im Gehirn?
  \begin{solution}
  \end{solution}
  \question Welches der drei wichtigsten dopaminergen Systeme interagiert eng mit dem neuroendokrinologischen System?
  \begin{solution}
  \end{solution}
  \question Was ist das wichtigste Hirnareal, dass noradrenerge Neuronen enthält?
  \begin{solution}
  \end{solution}
  \question Wo befinden sich serotonerge Neuronen?
  \begin{solution}
  \end{solution}
  \question Nennen Sie zwei wichtige Beispiele für cholinerge Übertragung!
  \begin{solution}
  \end{solution}
  \question Nennen Sie die beiden wichtigsten Gruppen cholinerger Rezeptoren!
  \begin{solution}
  \end{solution}
  \question Wie werden Substanzen genannt, die die synaptische Übertragung fördern bzw. hemmen?
  \begin{solution}
  \end{solution}
  \question Nennen Sie 5 Wirkmechanismen von Agonisten!
  \begin{solution}
  \end{solution}
  \question Nennen Sie 5 Wirkmechanismen von Antagonisten!
  \begin{solution}
  \end{solution}
  \question Nennen Sie je ein Beispiel für Antagonisten und Agonisten und nennen Sie die beeinflussten Neurotransmitter!
  \begin{solution}
  \end{solution}
  \question Nennen Sie 4 Anwendungsgebiete für Atropin!
  \begin{solution}
  \end{solution}
  \question Nennen Sie die wichtigsten Typen von Gliazellen!
  \begin{solution}
  \end{solution}
  \question Nennen Sie die wichtigsten Merkmale von Mikroglia!
  \begin{solution}
  \end{solution}
  \question Nennen Sie die wichtigsten Merkmale und Funktionen von Astrozyten!
  \begin{solution}
  \end{solution}
  \question Durch welche Gliazellen wird die Blut-Hirn-Schranke realisiert?
  \begin{solution}
  \end{solution}
  \question Nennen Sie die wichtigsten Merkmale und Funktionen der Oligodendrozyten!
  \begin{solution}
  \end{solution}
  \question Durch welche Zellen wird die Myelinscheide im peripheren Nervensystem gebildet?
  \begin{solution}
  \end{solution}
  \question Nennen Sie Neurotransmitter- und Rezeptortyp in motorischen Endplatten!
  \begin{solution}
  \end{solution}
  \question Sind Gliazellen auch direkt an Informationsverarbeitung im Gehirn beteiligt?
  \begin{solution}
  \end{solution}
  \question Ordnen Sie die Richtungsbezeichnungen dorsal, ventral, caudal, rostral, anterior, medial, lateral den Begriffen außen, vorn, oben, innen, unten, hinten zu!
  \begin{solution}
  \end{solution}
  \question Was bezeichnen die Begriffe proximal und distal?
  \begin{solution}
  \end{solution}
  \question Nennen Sie die 6 Hauptabschnitte des Gehirns!
  \begin{solution}
  \end{solution}
  \question Wie viele Hirnnervenpaare gibt es?
  \begin{solution}
  \end{solution}
  \question Welcher Hirnnerv entspringt im Telencephalon und welche Funktion hat er?
  \begin{solution}
  \end{solution}
  \question Welcher Hirnnerv entspringt im Diencephalon und welche Funktion hat er?
  \begin{solution}
  \end{solution}
  \question Was ist die Funktion des N. trigenimus?
  \begin{solution}
  \end{solution}
  \question Was ist die Funktion des N. vestibulocochlearis?
  \begin{solution}
  \end{solution}
  \question Was ist die Funktion des N. vagus?
  \begin{solution}
  \end{solution}
  \question Welche basalen Hirnfunktionen sind in der Medulla oblongata lokalisiert?
  \begin{solution}
  \end{solution}
  \question Welches Hirnteil ist für das Überleben des Organismus unverzichtbar?
  \begin{solution}
  \end{solution}
  \question Wo befindet sich die retikuläre Formation?
  \begin{solution}
  \end{solution}
  \question Nennen sie drei wichtige Funktionen die der retikulären Formation zugeordnet werden!
  \begin{solution}
  \end{solution}
  \question Wo befindet sich die Pons?
  \begin{solution}
  \end{solution}
  \question Was befindet sich zwischen Pons und Cerebellum?
  \begin{solution}
  \end{solution}
  \question Wo befinden sich Zellkörper und Axone cerebellarer Neuronen?
  \begin{solution}
  \end{solution}
  \question Nennen Sie die beiden wichtigsten cerebellaren Neuronentypen und ordnen Sie diese anhand der Lage ihrer Zellkörper den entsprechenden Cortexschichten zu!
  \begin{solution}
  \end{solution}
  \question Nennen Sie die 4 grundsätzlichen Funktionsprinzipien des Cerebellums!
  \begin{solution}
  \end{solution}
  \question Nennen Sie 5 typische Symptome cerebellarer Störungen!
  \begin{solution}
  \end{solution}
  \question Wo befindet sich das Mittelhirn?
  \begin{solution}
  \end{solution}
  \question Was sind die beiden Hauptabschnitte des Mittelhirns?
  \begin{solution}
  \end{solution}
  \question Zu welchen funktionellen Systemen gehören die inferioren und die superioren Colliculi?
  \begin{solution}
  \end{solution}
  \question Was ist der wichtigste Neurotransmitter der Substantia nigra?
  \begin{solution}
  \end{solution}
  \question Welche Krankheit ist mit Störungen in der Substantia nigra verbunden?
  \begin{solution}
  \end{solution}
  \question Was ist die wichtigste Funktion des Thalamus?
  \begin{solution}
  \end{solution}
  \question Nennen Sie 5 Funktionen des Hypothalamus!
  \begin{solution}
  \end{solution}
  \question Welches ist das oberste Regulierungszentrum des autonomen Nervensystems?
  \begin{solution}
  \end{solution}
  \question Nennen Sie die drei grundsätzlichen Quellen für Afferenzen des Hypothalamus!
  \begin{solution}
  \end{solution}
  \question Nennen Sie die 5 grundsätzlichen Efferenzen des Hypothalamus!
  \begin{solution}
  \end{solution}
  \question Zu welchen funktionellen Systemen gehören die lateralen und die medialen Kniehöcker?
  \begin{solution}
  \end{solution}
  \question Nennen Sie die beiden Hauptabschnitte des Großhirns!
  \begin{solution}
  \end{solution}
  \question Was wird durch Kommissuren verbunden?
  \begin{solution}
  \end{solution}
  \question Nennen Sie die 4 Großhirnlappen!
  \begin{solution}
  \end{solution}
  \question Was verbinden Projektions- und Assoziationsbahnen?
  \begin{solution}
  \end{solution}
  \question Welche histologischen und phylogenetischen Cortextypen gibt es?
  \begin{solution}
  \end{solution}
  \question Wie viele Schichten unterscheidet man beim Isocortex und beim Allocortex?
  \begin{solution}
  \end{solution}
  \question Welche histologische und phylogenetische Cortexart nimm die meiste Fläche ein (beim Menschen)
  \begin{solution}
  \end{solution}
  \question Nennen Sie 4 wichtige Strukturen der Basalganglien!
  \begin{solution}
  \end{solution}
  \question Welche beiden Strukturen werden unter dem Begriff Striatum zusammengefasst?
  \begin{solution}
  \end{solution}
  \question Bei welchen Funktionen spielt die Amygdala eine herausragende Rolle?
  \begin{solution}
  \end{solution}
  \question Was enthält die Weiße Masse?
  \begin{solution}
  \end{solution}
  \question Nennen Sie die drei Hirnhäute!
  \begin{solution}
  \end{solution}
  \question Welche Hirnhaut grenzt unmittelbar an den Cortex?
  \begin{solution}
  \end{solution}
  \question Welche Hirnhaut grenzt unmittelbar an den Schädel?
  \begin{solution}
  \end{solution}
  \question Über wie viele Arterien erfolgt die Blutzufuhr zum Gehirn?
  \begin{solution}
  \end{solution}
  \question Durch welche Struktur kann der Ausfall einer der zuführenden Arterien ausgeglichen werden?
  \begin{solution}
  \end{solution}
  \question Wie viele Hirnventrikel gibt es?
  \begin{solution}
  \end{solution}
  \question Wo und durch welche Struktur wird das Nervenwasser gebildet?
  \begin{solution}
  \end{solution}
  \question Wo wird das Nervenwasser wieder resorbiert?
  \begin{solution}
  \end{solution}
  \question Wo befinden sich weiße und graue Masse im Rückenmark?
  \begin{solution}
  \end{solution}
  \question Wo endet das Rückenmark beim Erwachsenen?
  \begin{solution}
  \end{solution}
  \question Was sind die beiden wichtigsten Grundfunktionen des Rückenmarks?
  \begin{solution}
  \end{solution}
  \question Wie viele Spinalnervenpaare gibt es?
  \begin{solution}
  \end{solution}
  \question Was ist ein Dermatom?
  \begin{solution}
  \end{solution}
  \question Nennen Sie die drei versorgenden Arterien des Rückenmarks!
  \begin{solution}
  \end{solution}
  \question Welcher Anteil des Rückenmarks wird über die Arteria spinalis anterior versorgt?
  \begin{solution}
  \end{solution}
  \question Welcher Anteil des Rückenmarks wird über die beiden Arterii spinalis posteriolateralis versorgt?
  \begin{solution}
  \end{solution}
  \question Nennen Sie die drei Häute des Rückenmarks!
  \begin{solution}
  \end{solution}
  \question Zwischen welchen Rückenmarkshäuten befindet sich Nervenwasser?
  \begin{solution}
  \end{solution}
  \question Zwischen welchen Rückenmarkshäuten befinden sich venöse Blutgefäße?
  \begin{solution}
  \end{solution}
  \question Was wird durch Schädigung oder Durchtrennung der ventralen Wurzel verursacht?
  \begin{solution}
  \end{solution}
  \question Was passiert bei schlaffer Lähmung mit den betroffenen Muskeln?
  \begin{solution}
  \end{solution}
  \question Was ist der Krankheitsmechanismus bei Amyotrophischer Lateralsklerose?
  \begin{solution}
  \end{solution}
  \question Nennen Sie drei wichtige Ursachen für Querschnittslähmung!
  \begin{solution}
  \end{solution}
  \question Was sind die Auswirkungen einer Durchtrennung des Rückenmarks bei C4?
  \begin{solution}
  \end{solution}
  \question Was sind die Auswirkungen einer Durchtrennung des Rückenmarks bei L1?
  \begin{solution}
  \end{solution}
  \question Nennen Sie drei Ursachen für Bandscheibenvorfälle?
  \begin{solution}
  \end{solution}
  \question In welchem Wirbelsäulenabschnitt treten die meisten Bandscheibenvorfälle auf?
  \begin{solution}
  \end{solution}
  \question Nenne Sie zwei wirksame prophylaktische Maßnahmen gegen Bandscheibenvorfälle!
  \begin{solution}
  \end{solution}
  \question Welches sind die beiden Grundformen von Schädel-Hirn-Traumata?
  \begin{solution}
  \end{solution}
  \question Deutet eine Bewusstlosigkeit von 45 Minuten auf eine Gehirnerschütterung, eine Gehirnprellung oder eine Gehirnquetschung hin?
  \begin{solution}
  \end{solution}
  \question Nennen Sie 5 typische Symptome für Schädel-Hirn-Traumata!
  \begin{solution}
  \end{solution}
  \question Nennen Sie 3 Therapiemaßnahmen bei Schädel-Hirn-Traumata!
  \begin{solution}
  \end{solution}
  \question Nennen Sie die beiden Grundformen cerebrovaskulärer Störungen!
  \begin{solution}
  \end{solution}
  \question Nennen Sie 3 mögliche Ursachen für Hämorrhagien!
  \begin{solution}
  \end{solution}
  \question Nennen Sie 3 wichtige Risikofaktoren für Hämorrhagien!
  \begin{solution}
  \end{solution}
  \question Nennen Sie 2 mögliche unmittelbare Ursachen cerebraler Ischämien!
  \begin{solution}
  \end{solution}
  \question Was ist der wichtigste Faktor bei der Schlaganfalltherapie?
  \begin{solution}
  \end{solution}
  \question Nennen Sie 3 wichtige Therapiemaßnahmen bei Ischämien!
  \begin{solution}
  \end{solution}
  \question Nennen Sie 4 wichtige Hirntumorklassen (nach der Gewebsart)!
  \begin{solution}
  \end{solution}
  \question Welche Klasse von Hirntumoren (nach der Gewebsart) ist am häufigsten?
  \begin{solution}
  \end{solution}
  \question Nennen Sie 5 typische Symptome für Hirntumore!
  \begin{solution}
  \end{solution}
  \question Welches sind die beiden typischen neuropathologischen Befunde bei Alzheimer?
  \begin{solution}
  \end{solution}
  \question In welchen Hirnarealen sind neuropathologische Veränderungen bei Alzheimer besonders anzutreffen?
  \begin{solution}
  \end{solution}
  \question Welche Art von Lernen/Gedächtnis ist nicht von der Alzheimerschen Krankheit betroffen?
  \begin{solution}
  \end{solution}
  \question Welcher Neurotransmitter spielt eine besondere Rolle bei der Parkinsonschen Krankheit?
  \begin{solution}
  \end{solution}
  \question Welche Hirnstruktur spielt eine besondere Rolle bei der Parkinsonschen Krankheit?
  \begin{solution}
  \end{solution}
  \question Nennen Sie 5 typische Symptome der Parkinsonschen Krankheit!
  \begin{solution}
  \end{solution}
  \question Nennen Sie die 2 wichtigsten Behandlungsstrategien bei Parkinson!
  \begin{solution}
  \end{solution}
  \question Wie hoch ist das Erkrankungsrisiko einer Person, deren Mutter an Chorea Huntington erkrankt ist?
  \begin{solution}
  \end{solution}
  \question Welche Nervenzellen werden bei der Amyotrophen Lateralsklerose geschädigt?
  \begin{solution}
  \end{solution}
  \question Bei welcher Krankheit wird das Myelin der Axone angegriffen?
  \begin{solution}
  \end{solution}
  \question Nennen Sie die 4 Grundprinzipien des sensomotorischen Systems!
  \begin{solution}
  \end{solution}
  \question Nennen Sie die 5 sensomotorischen Systeme!
  \begin{solution}
  \end{solution}
  \question Was sind die beiden Aufgaben des Eigenreflexapparates?
  \begin{solution}
  \end{solution}
  \question Wie viele synaptische Verknüpfungen befinden sich zwischen Sensor und Effektor des Eigenreflexapparates (ohne motorische Endplatten)?
  \begin{solution}
  \end{solution}
  \question Wo befinden sich die Zellkörper der somatoafferenten Neuronen?
  \begin{solution}
  \end{solution}
  \question In welchem Teil des Rückenmarks befinden sich die Motorneuronen?
  \begin{solution}
  \end{solution}
  \question Über welche Nervenwurzel verlassen die motorischen Fasern das Rückenmark?
  \begin{solution}
  \end{solution}
  \question Wird ein Muskel i.d.R. von genau einem Rückenmarkssegment versorgt?
  \begin{solution}
  \end{solution}
  \question Was ist eine motorische Einheit?
  \begin{solution}
  \end{solution}
  \question Wie viele motorische Endplatten kontaktieren eine Muskelfaser?
  \begin{solution}
  \end{solution}
  \question Wovon hängt die Größe einer motorischen Einheit ab?
  \begin{solution}
  \end{solution}
  \question Durch welche Sensoren werden die Muskellänge und die Muskelspannung gemessen?
  \begin{solution}
  \end{solution}
  \question Was ist die Rolle der Gamma-Neuronen im Eigenreflexapparat?
  \begin{solution}
  \end{solution}
  \question Welcher Muskel wird beim Patellarsehnenreflex inhibiert?
  \begin{solution}
  \end{solution}
  \question Was ist die Funktion des Fremdreflexapparates?
  \begin{solution}
  \end{solution}
  \question Welche grundsätzlichen Typen von Haut- und Körperrezeptoren gibt es?
  \begin{solution}
  \end{solution}
  \question Welche Typen von Berührungs/Drucksensoren gibt es?
  \begin{solution}
  \end{solution}
  \question Welche afferenten Nervenfasern haben die größte Übertragungsgeschwindigkeit?
  \begin{solution}
  \end{solution}
  \question Welche sensorische Information wird durch C-Fasern übermittelt?
  \begin{solution}
  \end{solution}
  \question Welcher Typ afferenter Nervenfasern ist marklos?
  \begin{solution}
  \end{solution}
  \question Wohin ziehen die Hinterstrangbahnen im Rückenmark?
  \begin{solution}
  \end{solution}
  \question Wo kreuzen die Hinterstrangbahnen auf die kontralaterale Seite?
  \begin{solution}
  \end{solution}
  \question Nennen sie zwei wichtiges sensomotorische Assoziationscortexareale!
  \begin{solution}
  \end{solution}
  \question Woher erhält der parietale Assoziationscortex seinen Input?
  \begin{solution}
  \end{solution}
  \question Über wie viele Neuronen wird im pyramidalen System die Information an die Muskeln übertragen?
  \begin{solution}
  \end{solution}
  \question Nennen Sie die 7 Stationen der Sehbahn!
  \begin{solution}
  \end{solution}
  \question In welcher Hemisphäre wird die Information von der Netzhaut des rechten Auges verarbeitet?
  \begin{solution}
  \end{solution}
  \question Aus welchen drei Häuten besteht der hintere Teil des Augapfels?
  \begin{solution}
  \end{solution}
  \question Wo befindet sich die Hornhaut des Auges?
  \begin{solution}
  \end{solution}
  \question Worauf wirkt der Ziliarmuskel?
  \begin{solution}
  \end{solution}
  \question Was ist der Vor- und der Nachteil einer weiten Pupille?
  \begin{solution}
  \end{solution}
  \question Was ist der Vor- und der Nachteil einer engen Pupille?
  \begin{solution}
  \end{solution}
  \question Welche Teile des autonomen Nervensystems bewirken die Erweiterung bzw. die Verengung der Pupille?
  \begin{solution}
  \end{solution}
  \question Wie wirkt Stress auf die Pupille?
  \begin{solution}
  \end{solution}
  \question Wie wirkt Müdigkeit auf die Pupille?
  \begin{solution}
  \end{solution}
  \question Wie wirkt eine Entspannung des Ziliarmuskels auf die Linsenwölbung?
  \begin{solution}
  \end{solution}
  \question Welche Linsenwölbung bewirkt eine Fernakkomodation?
  \begin{solution}
  \end{solution}
  \question Welche Arten von Fehlsichtigkeit werden durch Sammel- bzw. Zerstreuungslinsen behoben?
  \begin{solution}
  \end{solution}
  \question Nennen Sie die 5 Zelltypen der Retina!
  \begin{solution}
  \end{solution}
  \question Welchen Neurotransmitter schütten Fotorezeptoren aus?
  \begin{solution}
  \end{solution}
  \question Was ist der Neurotransmitter der Ganglien- und Bipolarzellen?
  \begin{solution}
  \end{solution}
  \question Was ist der Neurotransmitter der amacrinen und Horizontalzellen?
  \begin{solution}
  \end{solution}
  \question Welche Zelltypen kontaktieren die Synapsen der Fotorezeptoren?
  \begin{solution}
  \end{solution}
  \question Wie viele synaptische Kontakte befinden sich zwischen Sehnerv und Lichtsinneszellen?
  \begin{solution}
  \end{solution}
  \question Welche Zellart der Netzhaut ist dem einfallenden Licht am nächsten?
  \begin{solution}
  \end{solution}
  \question Welche beiden Arten von Fotorezeptoren gibt es in der Retina?
  \begin{solution}
  \end{solution}
  \question Welche Art von Fotorezeptoren ist für die Farbwahrnehmung zuständig?
  \begin{solution}
  \end{solution}
  \question Welche der beiden Arten von Fotorezeptoren ist zahlreicher?
  \begin{solution}
  \end{solution}
  \question Welche Auswirkungen hat Konvergenz in der Retina auf die Qualität der visuellen Information?
  \begin{solution}
  \end{solution}
  \question Welche Auswirkungen hat laterale Inhibition in der Retina auf die Qualität der visuellen Information?
  \begin{solution}
  \end{solution}
  \question Wie heißt die Eintrittsstelle des Sehnervs in den Augapfel und wodurch ist diese gekennzeichnet?
  \begin{solution}
  \end{solution}
  \question An welcher Stelle der Retina ist die Zapfendichte am höchsten?
  \begin{solution}
  \end{solution}
  \question In welchem Großhirnlappen befindet sich die primäre Sehrinde?
  \begin{solution}
  \end{solution}
  \question Welche Auswirkungen hat die Durchtrennung des rechten Sehnerves?
  \begin{solution}
  \end{solution}
  \question Welche Auswirkungen hat die Durchtrennung optischen Tracts?
  \begin{solution}
  \end{solution}
  \question Welche Auswirkungen haben Läsionen im primären visuellen Cortex?
  \begin{solution}
  \end{solution}
  \question Was sind die Auswirkungen von Läsionen im posterioren Parietallappen auf die visuelle Wahrnehmung?
  \begin{solution}
  \end{solution}
  \question Was sind die Auswirkungen von Läsionen im inferioren Temporallappen auf die visuelle Wahrnehmung?
  \begin{solution}
  \end{solution}
  \question Wozu dienen nach der alternativen Theorie von Logothetis und Steinberg die ventrale und die dorsale Bahn des visuellen Systems?
  \begin{solution}
  \end{solution}
  \question Was versteht man unter Propagnosie?
  \begin{solution}
  \end{solution}
  \question In welchem Quadranten der primären Sehrinde wird die Information aus dem rechten unteren Quadranten des Gesichtsfelds des rechten Auges verarbeitet?
  \begin{solution}
  \end{solution}
  \question Aus welchen Beobachtungen resultiert die Farbtheorie von Young und Helmholtz?
  \begin{solution}
  \end{solution}
  \question Aus welchen Beobachtungen resultiert die Farbtheorie von Hering?
  \begin{solution}
  \end{solution}
  \question Welche der beiden Farbtheorien ist tatsächlich im Gehirn implementiert?
  \begin{solution}
  \end{solution}
  \question Wovon hängt die wahrgenommene Farbe einer Fläche ab?
  \begin{solution}
  \end{solution}
  \question Was sind die beiden möglichen Erklärungen für Blindsehen?
  \begin{solution}
  \end{solution}
  \question Was sind die drei Abschnitte des Ohres?
  \begin{solution}
  \end{solution}
  \question Welche Struktur trennt äußeres Ohr von Mittelohr?
  \begin{solution}
  \end{solution}
  \question Welchen zwei Funktionen dient das äußere Ohr?
  \begin{solution}
  \end{solution}
  \question Was ist die Hauptfunktion des Mittelohrs?
  \begin{solution}
  \end{solution}
  \question Welche Strukturmerkmale des Mittelohrs tragen zur Schalldruckverstärkung bei?
  \begin{solution}
  \end{solution}
  \question Wie heißt die Knochenstruktur, in die das Innenohr eingebettet ist?
  \begin{solution}
  \end{solution}
  \question In welcher Struktur befinden sich die Hörsinneszellen und wie heißen diese?
  \begin{solution}
  \end{solution}
  \question Mit welcher Membran ist das Corti-Organ fest verbunden?
  \begin{solution}
  \end{solution}
  \question An welchem Ende ist die Cochlea empfindlich für hohe Frequenzen – am Helicotrema oder am ovalen Fenster?
  \begin{solution}
  \end{solution}
  \question Die Stereozilien welcher Haarzellen sind fest mit der Tectorialmembran verbunden?
  \begin{solution}
  \end{solution}
  \question Was ist die Funktion der äußeren Haarzellen?
  \begin{solution}
  \end{solution}
  \question Welche beiden Hörbahnen kann man unterscheiden?
  \begin{solution}
  \end{solution}
  \question Was ist die Funktion der dorsalen Hörbahn?
  \begin{solution}
  \end{solution}
  \question Was ist die Funktion der ventralen Hörbahn?
  \begin{solution}
  \end{solution}
  \question Wo befinden sich die Zellkörper der 4 Neuronen der dorsalen Hörbahn (richtige Reihenfolge in Richtung des Hauptinformationsflusses)
  \begin{solution}
  \end{solution}
  \question In welcher Hirnhälfte bezüglich des entsprechenden Ohres endet die dorsale Hörbahn?
  \begin{solution}
  \end{solution}
  \question In welchem Hirnlappen findet die kortikale Verarbeitung auditorischer Information hauptsächlich statt?
  \begin{solution}
  \end{solution}
  \question Mit welchem Gerät kann man untersuchen, ob ein Patient an Mittel- oder Innenohrtaubheit leidet?
  \begin{solution}
  \end{solution}
  \question Nennen Sie eine mögliche Ursache für Mittelohrtaubheit!
  \begin{solution}
  \end{solution}
  \question Nennen Sie eine mögliche Ursache für Innenohrtaubheit!
  \begin{solution}
  \end{solution}
  \question Womit kann Innenohrtaubheit therapiert werden?
  \begin{solution}
  \end{solution}
  \question Aus welchen 5 flüssigkeitsgefüllten Hohlräumen besteht das Labyrinth-Organ?
  \begin{solution}
  \end{solution}
  \question Nennen Sie die 5 wichtigen Projektionsziele vestibulärer Nervenfasern!
  \begin{solution}
  \end{solution}
  \question Nennen Sie 2 häufige vestibuläre Störungen!
  \begin{solution}
  \end{solution}
  \question Was ist die Ursache des gutartigen Lagerungsschwindels?
  \begin{solution}
  \end{solution}
  \question Was ist die Ursache der Neuritis vestibularis?
  \begin{solution}
  \end{solution}
  \question In welchem Teil des Gehirns endet der Riechnerv?
  \begin{solution}
  \end{solution}
  \question Welche Arten von Neuronen im ZNS werden ständig erneuert?
  \begin{solution}
  \end{solution}
  \question Wodurch entstehen komplexe Geschmacksempfindungen?
  \begin{solution}
  \end{solution}
  \question Auf welchem Teil der Zunge schmecken wir süß?
  \begin{solution}
  \end{solution}
  \question Welche kognitive Funktion ist besonders mit dem Hippocampus verbunden?
  \begin{solution}
  \end{solution}
  \question In welchem Großhirnlappen befindet sich der Hippocampus?
  \begin{solution}
  \end{solution}
  \question An welche anderen limbischen Strukturen grenzt der Hippocampus.
  \begin{solution}
  \end{solution}
  \question Was ist die Haupteingangsstruktur für den Hippocampus?
  \begin{solution}
  \end{solution}
  \question Aus welchem strukturellen Cortextyp besteht der Hippocampus?
  \begin{solution}
  \end{solution}
  \question An welche andere limbische Struktur grenzt der Mandelkern unmittelbar?
  \begin{solution}
  \end{solution}
  \question Bei welcher kognitiven Funktion spielt die Amygdala eine herausragende Rolle?
  \begin{solution}
  \end{solution}
  \question Wie breiten sich die meisten Hormone aus?
  \begin{solution}
  \end{solution}
  \question Wo werden die meisten Hormone freigesetzt?
  \begin{solution}
  \end{solution}
  \question Nennen Sie die drei wichtigsten chemischen Gruppen von Hormonen!
  \begin{solution}
  \end{solution}
  \question Was sind Peptide?
  \begin{solution}
  \end{solution}
  \question Welcher Teil des Gehirns spielt eine zentrale Rolle bei der Hormonausschüttung?
  \begin{solution}
  \end{solution}
  \question Welche Drüse spielt im hormonellen System eine übergeordnete Rolle?
  \begin{solution}
  \end{solution}
  \question Nennen Sie 5 wichtige Hormondrüsen!
  \begin{solution}
  \end{solution}
  \question Welcher Teil der Hypophyse wird direkt vom Hypothalamus innerviert?
  \begin{solution}
  \end{solution}
  \question Über welchen Signalweg wird die Information vom Hypothalamus zum Hypophysenvorderlappen übermittelt?
  \begin{solution}
  \end{solution}
  \question Welche Hormone werden hauptsächlich durch den Hypophysenhinterlappen ausgeschüttet?
  \begin{solution}
  \end{solution}
  \question Durch welche drei Mechanismen wird die Hormonfreisetzung geregelt und der Homonspiegel stabilisiert?
  \begin{solution}
  \end{solution}
  \question Wo werden steroide Sexualhormone produziert?
  \begin{solution}
  \end{solution}
  \question Welche 3 Grundklassen von steroiden Sexualhormonen gibt es?
  \begin{solution}
  \end{solution}
  \question Wie erfolgt der Freisetzung von Sexualhormonen in Frauen und Männern?
  \begin{solution}
  \end{solution}
  \question Welches Hormon sorgt vor und unmittelbar nach der Geburt für eine männliche Entwicklung?
  \begin{solution}
  \end{solution}
  \question Durch welche Hormone wird das weibliche Sexualverhalten beim Menschen maßgeblich gesteuert?
  \begin{solution}
  \end{solution}
  \question Welche Arten von Stresshormonen werden bei kurzfristigem und langfristigem Stress ausgeschüttet?
  \begin{solution}
  \end{solution}
  \question Nennen Sie ein typisches glukokortikoides Stresshormon!
  \begin{solution}
  \end{solution}
  \question Welche beiden Hormone werden im Nebennierenmark ausgeschüttet?
  \begin{solution}
  \end{solution}
  \question Welche beiden Gruppen von Hormonen werden in der Nebennierenrinde ausgeschüttet?
  \begin{solution}
  \end{solution}
  \question Nennen Sie 2 wichtige Wirkungen von Glukokortikoiden!
  \begin{solution}
  \end{solution}
  \question Welche beiden chemischen Elemente sind für die Bildung von Schilddrüsenhormonen von Bedeutung?
  \begin{solution}
  \end{solution}
  \question Was ist die Hauptwirkung der Schilddrüsenhormone?
  \begin{solution}
  \end{solution}
  \question Wozu führt Schilddrüsenunterfunktion im Erwachsenenalter?
  \begin{solution}
  \end{solution}
  \question Welches Hormon wird von der Geburtshilfemedizin im sogenannten „Wehentropf“ verwendet?
  \begin{solution}
  \end{solution}
  \question Wodurch wird die Ausschüttung von Oxytocin ausgelöst?
  \begin{solution}
  \end{solution}
  \question Welche neuronalen Populationen haben Sympathikus und Parasympathikus und wo befinden sich diese?
  \begin{solution}
  \end{solution}
  \question Zu welchem Bestandteil des autonomen Nervensystems gehört der Grenzstrang?
  \begin{solution}
  \end{solution}
  \question Wo befinden sich allgemein die autonomen Ganglien des Sympathikus und des Parasympathikus?
  \begin{solution}
  \end{solution}
  \question Welcher Neurotransmitter wird durch die präganglionären Neuronen des Sympathikus ausgeschüttet?
  \begin{solution}
  \end{solution}
  \question Welcher Neurotransmitter wird durch die postganglionären Neuronen des Sympathikus ausgeschüttet?
  \begin{solution}
  \end{solution}
  \question Welcher Neurotransmitter wird durch die präganglionären Neuronen des Parasympathikus ausgeschüttet?
  \begin{solution}
  \end{solution}
  \question Welcher Neurotransmitter wird durch die postganglionären Neuronen des Parasympathikus ausgeschüttet?
  \begin{solution}
  \end{solution}
  \question Wo befinden sich die Zellkörper der präganglionären sympathischen Neuronen?
  \begin{solution}
  \end{solution}
  \question Wo befinden sich die Zellkörper der präganglionären parasympathischen Neuronen?
  \begin{solution}
  \end{solution}
  \question Über welchen Pfad übt der Sympathikus eine globale Wirkung auf den Organismus aus?
  \begin{solution}
  \end{solution}
  \question Was sind die grundsätzlichen Rollen von Sympathikus und Parasympathikus?
  \begin{solution}
  \end{solution}
  \question Nennen Sie 4 Hauptwirkungen des Sympathikus!
  \begin{solution}
  \end{solution}
  \question Nennen Sie 4 Hauptwirkungen des Parasympathikus!
  \begin{solution}
  \end{solution}
  \question Nennen Sie 4 Funktionen des Hypothalamus!
  \begin{solution}
  \end{solution}
  \question Nennen Sie die drei Phasen des Energiestoffwechsels und geben Sie an durch welche charakteristischen Hormonspiegel diese gekennzeichnet sind!
  \begin{solution}
  \end{solution}
  \question Nennen Sie 3 Merkmale der cephalischen und absorptiven Energiestoffwechselphasen!
  \begin{solution}
  \end{solution}
  \question Nennen Sie 3 Merkmale der Fastenphase des Energiestoffwechsels!
  \begin{solution}
  \end{solution}
  \question Nennen Sie 3 Argumente die gegen die Sollwerthypothese der Nahrungsaufnahme sprechen!
  \begin{solution}
  \end{solution}
  \question Was ist die Alternative zur Sollwerthypothese der Nahrungsaufnahme?
  \begin{solution}
  \end{solution}
  \question Erläuterns Sie einen der wichtigen Mechanismen zur Regulierung von Hunger und Sättigung!
  \begin{solution}
  \end{solution}
  \question Wie viele Schlafphasen unterscheidet man und welche davon bezeichnet man als Slow-Wave-Sleep?
  \begin{solution}
  \end{solution}
  \question Nennen Sie die beiden wichtigen physiologischen Korrelate von Schlafphase 1!
  \begin{solution}
  \end{solution}
  \question Wie verändert sich der Schlafrhythmus im Verlauf der Nacht?
  \begin{solution}
  \end{solution}
  \question Nennen Sie die beiden grundsätzlichen Theorien zur Notwendigkeit von Schlaf!
  \begin{solution}
  \end{solution}
  \question Nennen Sie drei wichtige Auswirkungen von Schlafentzug!
  \begin{solution}
  \end{solution}
  \question Nennen Sie 3 mögliche Ursachen für Insomnie!
  \begin{solution}
  \end{solution}
  \question Nennen Sie die Arten und Unterarten des Langzeitgedächtnisses!
  \begin{solution}
  \end{solution}
  \question Welche drei Grundarten von Gedächtnis unterscheiden wir?
  \begin{solution}
  \end{solution}
  \question Was versteht man unter anterograder und retrograder Amnesie?
  \begin{solution}
  \end{solution}
  \question Die Entfernung welcher Hirnstruktur führte beim Patienten H.M. zu anterograder Amnesie des expliziten Langzeitgedächtnisses?
  \begin{solution}
  \end{solution}
  \question Wo werden, allgemein, Langzeitgedächtnisinhalte abgespeichert?
  \begin{solution}
  \end{solution}
  \question Erläutern Sie kurz das Prinzip des Hebbschen Lernens!
  \begin{solution}
  \end{solution}
  \question In welche Emotion ist der Mandelkern besonders involviert?
  \begin{solution}
  \end{solution}
  \question Welche Hirnhälfte ist in den meisten Menschen dominant?
  \begin{solution}
  \end{solution}
  \question Wodurch können die Hirnhälften von Split-Brain Patienten in der Praxis kommunizieren und koordiniert agieren?
  \begin{solution}
  \end{solution}
  \question Nennen Sie die 7 wichtigen Bestandteile des Wernicke Geschwind-Modells!
  \begin{solution}
  \end{solution}
  \question Nennen Sie drei Methoden mit denen die Voraussagen des Wernicke-Geschwind-Modells überprüft wurden!
  \begin{solution}
  \end{solution}
  \question Welche beiden allgemeinen Voraussagen des Wernicke-Geschwind-Modells können durch die experimentellen Befunde bestätigt werden?
  \begin{solution}
  \end{solution}
  \question Nennen Sie 5 Symptome für eine depressive Episode!
  \begin{solution}
  \end{solution}
  \question Nennen Sie 5 Symptome für eine manische Episode!
  \begin{solution}
  \end{solution}
  \question Welche beiden Verlaufsformen affektiver Störungen kennen wir?
  \begin{solution}
  \end{solution}
  \question Bei welcher Verlaufsform affektiver Störungen gibt es keine Geschlechtsunterschiede?
  \begin{solution}
  \end{solution}
  \question Nennen Sie 3 pharmakologische Therapien gegen Depressionen!
  \begin{solution}
  \end{solution}
  \question Welche nicht-pharmakologische antidepressive Therapie hat eine hohe Wirksamkeit?
  \begin{solution}
  \end{solution}
  \question Erläutern Sie das Wirkprinzip von MAO-Hemmern!
  \begin{solution}
  \end{solution}
  \question Erläutern Sie das Wirkprinzip von trizyklischen Antidepressiva!
  \begin{solution}
  \end{solution}
  \question Nennen Sie drei wichtige Nebenwirkungen von MAO-Hemmern!
  \begin{solution}
  \end{solution}
  \question Nennen Sie drei wichtige Nebenwirkungen von trizyklischen Antidepressiva!
  \begin{solution}
  \end{solution}
  \question Nennen Sie drei wichtige Nebenwirkungen von Antidepressiva der 2.  Generation!
  \begin{solution}
  \end{solution}
  \question Erläutern Sie das Wirkprinzip der Elektrokonvulsiven Therapie!
  \begin{solution}
  \end{solution}
  \question Nennen Sie die 3 wichtigsten neurobiologischen Theorien über affektive Störungen!
  \begin{solution}
  \end{solution}
  \question Auf welchen Beobachtungen (3) beruht die Monoamin-Hypothese zu affektiven Störungen?
  \begin{solution}
  \end{solution}
  \question Auf welchen Beobachtungen (2) beruht die Glukokortikoid-Hypothese zu affektiven Störungen?
  \begin{solution}
  \end{solution}
  \question Nennen Sie die 5 Klassen von Angststörungen!
  \begin{solution}
  \end{solution}
  \question Was ist Furcht?
  \begin{solution}
  \end{solution}
  \question Was ist eine effektive Therapieform für Phobien?
  \begin{solution}
  \end{solution}
  \question Nennen Sie 2 Gruppen von Psychopharmaka, die bei Angststörungen eingesetzt wurden bzw. werden!
  \begin{solution}
  \end{solution}
  \question Erläutern Sie das Wirkprinzip von Barbituraten!
  \begin{solution}
  \end{solution}
  \question Nennen Sie 4 wichtige Nebenwirkungen von Barbituraten!
  \begin{solution}
  \end{solution}
  \question Erläutern Sie das Wirkprinzip von Benzodiazepinen!
  \begin{solution}
  \end{solution}
  \question Welches sind die beiden Symptomgruppen bei Schizophrenie?
  \begin{solution}
  \end{solution}
  \question Nennen Sie 3 positive Symptome von Schizophrenie!
  \begin{solution}
  \end{solution}
  \question Nennen Sie 3 negative Symptome von Schizophrenie!
  \begin{solution}
  \end{solution}
  \question Welche Gruppe von Symptomen der Schizophrenie spricht besser auf Neuroleptika an?
  \begin{solution}
  \end{solution}
  \question Was ist das wichtigste Wirkprinzip klassischer Neuroleptika?
  \begin{solution}
  \end{solution}
  \question Nennen Sie die 5 wichtigsten Dopaminpfade im Gehirn und deren Rolle bei Schizophrenie und der Wirkung von Neuroleptika!
  \begin{solution}
  \end{solution}
\end{questions}
\end{document}