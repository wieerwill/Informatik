\documentclass[10pt, a4paper]{exam}
\printanswers			 % Comment this line to hide the answers 
\usepackage[utf8]{inputenc}
\usepackage[T1]{fontenc}
\usepackage[ngerman]{babel}
\usepackage{listings}
\usepackage{float}
\usepackage{graphicx}
\usepackage{color}
\usepackage{listings}
\usepackage[dvipsnames]{xcolor}
\usepackage{tabularx}
\usepackage{geometry}
\usepackage{color,graphicx,overpic}
\usepackage{amsmath,amsthm,amsfonts,amssymb}
\usepackage{tabularx}
\usepackage{listings}
\usepackage[many]{tcolorbox}
\usepackage{multicol}
\usepackage{hyperref}
\usepackage{pgfplots}
\usepackage{bussproofs}
\usepackage{tikz}
\usetikzlibrary{automata, arrows.meta, positioning}
\renewcommand{\solutiontitle}{\noindent\textbf{Antwort}: }
\SolutionEmphasis{\small}
\geometry{top=1cm,left=1cm,right=1cm,bottom=1cm} 

\pdfinfo{
 /Title (Grundlagen der Biosignalverarbeitung - Prüfungsvorbereitung)
 /Creator (TeX)
 /Producer (pdfTeX 1.40.0)
 /Author (Robert Jeutter)
 /Subject ()
}
\title{Grundlagen der Biosignalverarbeitung - Prüfungsvorbereitung}
\author{}
\date{}

% Don't print section numbers
\setcounter{secnumdepth}{0}

\newtcolorbox{myboxii}[1][]{
 breakable,
 freelance,
 title=#1,
 colback=white,
 colbacktitle=white,
 coltitle=black,
 fonttitle=\bfseries,
 bottomrule=0pt,
 boxrule=0pt,
 colframe=white,
 overlay unbroken and first={
 \draw[red!75!black,line width=3pt]
 ([xshift=5pt]frame.north west) -- 
 (frame.north west) -- 
 (frame.south west);
 \draw[red!75!black,line width=3pt]
 ([xshift=-5pt]frame.north east) -- 
 (frame.north east) -- 
 (frame.south east);
 },
 overlay unbroken app={
 \draw[red!75!black,line width=3pt,line cap=rect]
 (frame.south west) -- 
 ([xshift=5pt]frame.south west);
 \draw[red!75!black,line width=3pt,line cap=rect]
 (frame.south east) -- 
 ([xshift=-5pt]frame.south east);
 },
 overlay middle and last={
 \draw[red!75!black,line width=3pt]
 (frame.north west) -- 
 (frame.south west);
 \draw[red!75!black,line width=3pt]
 (frame.north east) -- 
 (frame.south east);
 },
 overlay last app={
 \draw[red!75!black,line width=3pt,line cap=rect]
 (frame.south west) --
 ([xshift=5pt]frame.south west);
 \draw[red!75!black,line width=3pt,line cap=rect]
 (frame.south east) --
 ([xshift=-5pt]frame.south east);
 },
}

\begin{document}
\begin{myboxii}[Disclaimer]
  Aufgaben aus dieser Vorlage stammen aus der Vorlesung \textit{Grundlagen der Biosignalverarbeitung} und wurden zu Übungszwecken verändert oder anders formuliert! Für die Korrektheit der Lösungen wird keine Gewähr gegeben.
\end{myboxii}

%##########################################
\begin{questions}
  \question Signalflussgraph
  \begin{parts}
    \part Zeitdiskretes vs. Analog?
    \part Rekursionsgleichung aus Signalflussgraph ermitteln
    \part Übertragungsfunktion Gz(Z) im z-Bereich?
    \part Ermitteln aller Pol und Nullstellen
    \part Zeichne Pol-Nullstellendiagramm mit b=1
    \part Welcher Filtertyp? IIR oder FIR
    \part Ist das System stabil? Begründe
    \part wie muss man b wählen, damit aus dem System ein Allpass wird (mit b>1)?
  \end{parts}

  \question Blockschaltbild
  \begin{parts}
    \part rekursive Gleichung aus Blockschaltbild bestimmen,
    \part Übertragungsfunktion ermitteln
    \part Pol~/Nullstellen bestimmen
    \part Pol~/Nullstellen-Diagramm zeichnen
    \part für welchen Parameter a wird aus rekursiver Gleichung ein FIR-Filter?
    \part in welchem Intervall für a wird Filter stabil, wann/warum instabil?
    \part Eingangssignal als Graph gegeben, Ausgangssignal als Graph bestimmen (Faltung durchführen)
  \end{parts}

  \question Gradiometer
  \begin{parts}
    \part Was ist ein Gradiometer?
    \part Aufbau, Funktionsweise?
    \part Warum stört Erdmagnetfeld nicht? Gradiometerprinzip reicht nicht aus für Erklärung
  \end{parts}

  \question EKG
  \begin{parts}
    \part Wie sieht EKG aus? Zeichnen
    \part Was bedeutet welche Zacke?
    \part Wie und wo abgeleitet (Standardableitung)?
    \part Standardsysteme - Arten der Ableitung (typische Frequenzen/Phasen)?
    \part Anwendungsbeispiele
  \end{parts}

  \question Biosignale
  \begin{parts}
    \part Welche Arten von Signalen existieren?
    \part Störung duch Biosignale, wie beheben?
  \end{parts}

  \question Guarding-Technik beschreiben

  \question Messverstärker
  \begin{parts}
    \part Welcher Phasenfrequenzgang bei Messverstärker?
    \part Warum? Was passiert bei Nichteinhaltung?
    \part Eigenrauschen qualitativ beschreiben, aus welchen Komponenten besteht es
  \end{parts}

  \question EEG
  \begin{parts}
    \part Welche Frequenzbänder?
    \part Zuordnung Kanal zu Spektrum mit Begründung
    \part Nach welchem Prinzip ableiten?
    \part Spektralanalyse EEG (Leitungsspektrum, Auflösung berechnen)
    \part Fenster festlegen, wenn Arzt 0,4Hz Auflösung will
  \end{parts}

  \question AKF periodisch $\leftrightarrow$ aperiodisch (FFT-Shift)

  \question Signal mit $f(t)=2*cos(t*2*\pi *f)$, die Frequenz war 9Hz, das Signal war im Bereich von 0 bis 4,5s gegeben. Gegeben TM,...?
  \begin{parts}
    \part $f=fs:df:fe\rightarrow fs,df,fe$ berechnen
    \part Welcher Effekt verursacht viele Spektralteile? Leckeffekt erklären
    \part Welche Eigenschaft muss eine Fensterfunktion haben damit dieser Effekt verringert wird?
    \part Wie muss man die Eigenschaft der Fensterfunktion wählen, damit 1. Dynamischer Amplitudengang und 2. Gute Auflösung im Spektralbereich entsteht?
  \end{parts}

  \question Adaptive Noise Cancelller
  \begin{parts}
    \part Erläutern Sie anhand eines Blockschaltbilds die Funktionsweise eines adaptiven noise Cancellers
    \part Was muss für das Rauschen gelten, damit das LMS Prinzip angewendet werden kann? Welche Signale müssen korellieren und welche dürfen nicht in Korrelationsbeziehung stehen?
  \end{parts}

  \question Filter
  \begin{parts}
    \part Anhand welcher Merkmale kann man einen Filter klassifizieren, ob er FIR oder IIR ist?
    \part Können FIR instabil werden? Begründen Sie Ihre Vermutung.
    \part Wie kann man schnell die Koeffizienten eines FIR Filters ausrechnen?
  \end{parts}

  \question Übertragung
  \begin{parts}
    \part Warum muss man bei der Übertragung von Biosignale über größere Distanz das Signal modulieren?
    \part Welche Form der analogen Modulation ist besonders Störungsresistent und warum?
  \end{parts}

  \question Abtasttheorem: Welche notwendige und hinreichende Bedingung benötigt man für die Abtastung bei fc=100Hz und USB 0,1...1kHz

  \question Filtertypen (nicht für Informatiker)
  \begin{parts}
    \part Filtertypen anhand ihres Amplitudengangs klassifizieren.
    \part Welcher ist am besten für Biosignale geeignet und warum?
    \part Filtertypen die nach Namen ihres Erfinders heißen? Tscheby, Butter, Bessel
  \end{parts}

  \question LTI System
  \begin{parts}
    \part Blockschaltbild -> daraus die Übertragungsfunktion ableiten,
    \part Z-Transformation,
    \part Stabilität $y(n)=x(n)+2x(n-1)-3(n-1)$?
  \end{parts}

  \question chemische Vorgänge an Elektroden, Probleme bei Signalauswertung

  \question cos Funktion
  \begin{parts}
    \part cos-Fkt. gegeben (Gleichung, Graph zu Original-Fkt. + ihrer DFT), Sample-Freq. 10Hz, Abtastung für 2sec:
    \part Matlab-Befehl ermitteln für Parameter der $DFT(f_s, d_f, f_e)$
    \part Warum im DFT-Graph soviele Freq-Anteile?
    (vmtl. Leck-Effekt)
    \part Ursachen dieses Effekts im Zeit~ und Freq-Bereich (Zeit: Signal nicht genau an Periodengrenze abgeschnitten)
    \part Wie durch Fensterung beheben?
    \part Wie muss Spektrum des Fenster beschaffen sein um:
    \begin{itemize}
      \item hohe spektrale Auflösung
      \item hohe Amplitudendynamik
    \end{itemize}
    zu erreichen?
  \end{parts}

  \question Elektroden: Elektroden auf Haut erzeugen Gleichspannung
  \begin{parts}
    \part wie entsteht diese Gleichspannung?
    \part wie kann man sie reduzieren?
  \end{parts}

  \question Störungen
  \begin{parts}
    \part Welche Arten von Störungen existieren? Mit Erklärungen
    \begin{solution}
      transiente, periodische und biologische
    \end{solution}

    \part Entstehung von biologischen Störungen
    \part Methoden zur Eindämmung von Störungen
  \end{parts}

  \question Berechne mit $t_{ab}=10s$, $f_{s1}=9kHz$ und $f_{s2}=10kHz$, peaks gegeben
  \begin{parts}
    \part Abtasttheorem
    \part Welche Frequenzbereiche der Signale
    \part Fehler - Aliasing, graphisch erklären, wie vermeiden?
    \part wie sieht analoges Signal im Spektrum aus?
  \end{parts}

  \question $H(z)=\frac{1-0,8z^{-1}}{1+0,3z^{-1}}$
  \begin{parts}
    \part Signalflussdiagram zeichnen
    \part IIR oder FIR?
    \part Differenzgleichung?
    \part Pol/Nullstellendiagramm zeichnen?
    \part Stabilität im Z- und Zeitbereich?
  \end{parts}

\end{questions}
\end{document}