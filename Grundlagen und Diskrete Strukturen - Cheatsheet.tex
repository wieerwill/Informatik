\documentclass[a4paper]{article}
\usepackage[ngerman]{babel}
\usepackage[utf8]{inputenc}
\usepackage{multicol}
\usepackage{calc}
\usepackage{ifthen}
\usepackage[landscape]{geometry}
\usepackage{amsmath,amsthm,amsfonts,amssymb}
\usepackage{color,graphicx,overpic}
\usepackage{listings}
\usepackage[compact]{titlesec} %less space for headers
\usepackage{mdwlist} %less space for lists
\usepackage{pdflscape}
\usepackage{verbatim}
\usepackage[hidelinks,pdfencoding=auto]{hyperref}
\usepackage{fancyhdr}
\usepackage{lastpage}
\pagestyle{fancy}
\fancyhf{}
\fancyhead[L]{Grundlagen und Diskrete Strukturen}
\fancyfoot[L]{\thepage/\pageref{LastPage}}
\renewcommand{\headrulewidth}{0pt} %obere Trennlinie
\renewcommand{\footrulewidth}{0pt} %untere Trennlinie

\pdfinfo{
    /Title (Grundlagen und Diskrete Strukturen - Cheatsheet)
    /Creator (TeX)
    /Producer (pdfTeX 1.40.0)
    /Author (Robert Jeutter)
    /Subject (Grundlagen und Diskrete Strukturen)
}

% This sets page margins to .5 inch if using letter paper, and to 1cm
% if using A4 paper. (This probably isn't strictly necessary.)
% If using another size paper, use default 1cm margins.
\ifthenelse{\lengthtest { \paperwidth = 11in}}
    { \geometry{top=.5in,left=.5in,right=.5in,bottom=.5in} }
    {\ifthenelse{ \lengthtest{ \paperwidth = 297mm}}
    {\geometry{top=1.3cm,left=1cm,right=1cm,bottom=1.2cm} }
    {\geometry{top=1.3cm,left=1cm,right=1cm,bottom=1.2cm} }
    }

% Redefine section commands to use less space
\makeatletter
\renewcommand{\section}{\@startsection{section}{1}{0mm}%
                                {-1ex plus -.5ex minus -.2ex}%
                                {0.5ex plus .2ex}%x
                                {\normalfont\large\bfseries}}
\renewcommand{\subsection}{\@startsection{subsection}{2}{0mm}%
                                {-1explus -.5ex minus -.2ex}%
                                {0.5ex plus .2ex}%
                                {\normalfont\normalsize\bfseries}}
\renewcommand{\subsubsection}{\@startsection{subsubsection}{3}{0mm}%
                                {-1ex plus -.5ex minus -.2ex}%
                                {1ex plus .2ex}%
                                {\normalfont\small\bfseries}}
\makeatother

% Don't print section numbers
\setcounter{secnumdepth}{0}

\setlength{\parindent}{0pt}
\setlength{\parskip}{0pt plus 0.5ex}
% compress space
\setlength\abovedisplayskip{0pt}
\setlength{\parskip}{0pt}
\setlength{\parsep}{0pt}
\setlength{\topskip}{0pt}
\setlength{\topsep}{0pt}
\setlength{\partopsep}{0pt}
\linespread{0.5}
\titlespacing{\section}{0pt}{*0}{*0}
\titlespacing{\subsection}{0pt}{*0}{*0}
\titlespacing{\subsubsection}{0pt}{*0}{*0}

\begin{document}
\raggedright
\footnotesize
\begin{multicols}{3}
    
    % multicol parameters
    % These lengths are set only within the two main columns
    %\setlength{\columnseprule}{0.25pt}
    \setlength{\premulticols}{1pt}
    \setlength{\postmulticols}{1pt}
    \setlength{\multicolsep}{1pt}
    \setlength{\columnsep}{2pt}
    
    \section{Aussagen}
    Aussagen sind Sätze die wahr oder falsch sind, d.h. der Wahrheitswert ist wahr oder falsch.
    
    \paragraph{Verknüpfungen von Aussagen}
    Seien p und q Aussagen, dass sind folgende Sätze auch Aussagen
    - $p \wedge q$ "und"
    - $p \vee q$ "oder"
    - $\neg p$ "nicht"
    - $p \rightarrow q$ "impliziert"
    - $p \leftrightarrow q$ "genau dann wenn"
    
    \paragraph{Wahrheitswerteverlauf}
    \begin{tabular}{ l | c | c | c | c | c | c }
        p & q & $p\wedge q$ & $p\vee q$ & $\neg q$ & $p\rightarrow q$ & $p\leftrightarrow q$ \\
        \hline
        f & f & f           & f         & w        & w                & w                    \\
        f & w & f           & w         & w        & w                & f                    \\
        w & f & f           & w         & f        & f                & f                    \\
        w & w & w           & w         & f        & w                & w                    \\
    \end{tabular}
    
    \begin{description}
        \item[Aussagenlogische Variablen] Variable die den Wert w oder f annimmt
        \item[Aussagenlogische Formel] Verknüpfung aussagenloser Variablen nach obigen Muster
        \item[Belegung] Zuordnung von w/f an jede Variable einer aussagenlogischer Formel
        \item[Wahrheitswerteverlauf] Wahrheitswert der Aussagenformel in Abhängigkeit von der Belegung der Variable
        \item[Tautologie] Formel deren Wahrheitswerteverlauf konstant w ist
        \item[Kontradiktion] Formel deren Wahrheitswerteverlauf konstant f ist
        \item[Kontraposition] $(p\rightarrow q)\leftrightarrow (\neg q \rightarrow p)$ ist eine Tautologie
        \item[Modus Potens] $(p\vee (p\rightarrow q))\rightarrow q$ ist eine Tautologie
        \item[Äquivalenz] Zwei Formeln p,q sind äquivalent (bzw logisch äquivalent) wenn $p\leftrightarrow$ Tautologie ist. Man schreibt $p \equiv q$. Die Formel p impliziert die Formel q, wenn $p\rightarrow q$ eine Tautologie ist
    \end{description}
    
    \paragraph{Regeln}
    \begin{itemize}
        \item $p\wedge q \equiv q \wedge p$ (Kommutativ)
        \item $p\vee q \equiv q \vee p$ (Kommutativ)
        \item $p\wedge (q \wedge r) \equiv (p \wedge q) \wedge r$ (Assoziativ)
        \item $p\vee ( q \vee r) \equiv (p \vee q) \vee$ (Assoziativ)
        \item $p\wedge (q\vee r) \equiv (p\wedge q) \vee (p\wedge r)$ (Distributiv)
        \item $p\vee (q\wedge r) \equiv (p\vee q) \wedge (p\vee r)$ (Distributiv)
        \item $\neg(\neg q) \equiv q$ (Doppelte Verneinung)
        \item $\neg(p\wedge q) \equiv (\neg p) \wedge (\neg q)$ (de Morgansche)
    \end{itemize}
    
    Aussagenformen in einer Variable x aus dem Universum U heißen Prädikate von U. Aussagenformen in n Variablen $x_1,...,x_n$ aus dem Universum U heißen "n-stellige Prädikate" von U.
    
    Seien p,q Prädikate über U
    \begin{itemize}
        \item $(\forall x: (p(x) \wedge q(x)))\leftrightarrow (\forall x: p(x) \wedge \forall x: q(x))$
        \item $\exists x: (p(x) \vee q(x)) \leftrightarrow (\exists x: p(x) \vee \exists x: q(x))$
        \item $\neg (\forall x:p(x))\leftrightarrow \exists x: \neg p(x)$
        \item $\neg(\exists x:p(x))\leftrightarrow \forall x:\neg p(x)$
    \end{itemize}
    Achtung: Verschiedenartige Quantoren dürfen nicht getauscht werden! gleichartige Quantoren dürfen getauscht werden
    
    \section{Mengen}
    "Eine Menge ist eine Zusammenfassung bestimmter, wohlunterschiedener Objekte unserer Anschauung oder unseres Denkens" ~ Cantor
    Von jedem Objekt steht fest, ob es zur Menge gehört oder nicht.
    
    \paragraph{Wunsch 0}
    Es gibt eine Menge. Ist A irgendeine Menge, so ist ${x \in A: \neg (x=x)}$ eine Menge ohne Elemente, die sogenannte leere Menge $\emptyset$.
    
    \paragraph{Wunsch 1}
    "$x\in y$" soll Aussagenform über dem Universum U aller Mengen sein. D.h. für je zwei Mengen x und y ist entweder x ein Element von y oder nicht. D.h. "$x\in y$" ist ein 2-stelliges Prädikat über U.
    
    \paragraph{Wunsch 2}
    Ist p(x) ein Prädikat über U, so soll es eine Menge geben, die aus genau denjenigen Mengen x besteht, für die p(x) wahr ist. Bezeichnung $\{x:p(x) "ist-wahr" \}$.
    Danach gäbe es eine Menge M, die aus genau denjenigen Mengen x mit $x\not\in x$ besteht: $M=\{x:x\not \in x\}$.
    
    \paragraph{Wunsch 2'}
    Ist A eine Menge und p(x) ein Prädikat über U, dann gilt es eine Menge B die aus genau denjenigen Mengen x aus A besteht, für die p(x) wahr ist. Bezeichnung: $B={x\in A:p(x) wahr}$.
    Folgerung: die Gesamtheit aller Mengen ist selbst keine Menge, sonst findet man einen Widerspruch wie oben.
    
    \paragraph{Wunsch 3}
    Zwei Mengen x,y sind genau dann gleich wenn sie diesselben Elemente enthalten. D.h. $x=y: \leftrightarrow \forall z:(z\in x \leftrightarrow z\in y)$. Somit gilt für zwei Prädikate p(x), q(x) über U und jede Menge A: ${x\in A: p(x) wahr} = {x\in A: q(x) wahr}$ genau dann, wen q(x), p(x) den gleichen Wahrheitswert für jedes x aus A haben.
    
    \paragraph{Wunsch 4}
    Zu jeder Menge A gibt es eine Menge B, die aus genau denjenigen Mengen besteht, die Teilmengen von A sind. Dabei ist x eine Teilmenge von $y: \leftrightarrow \forall z:(z\in x \rightarrow z \in y) [x \subseteq y]$\
    $B={x:x\subseteq A}=\wp(A)$ B heißt Potentmenge von A\
    
    \subsection{Teilmengen}
    A Teilmenge von B $\leftrightarrow \forall x: (x\in A \rightarrow x \in B):\Rightarrow A\subseteq B$\
    A Obermenge von B $\leftrightarrow \forall x: (x\in B \rightarrow x \in A):\Rightarrow A\supseteq B$\
    Folglich $A=B \leftrightarrow A\subseteq B \wedge B\subseteq A$\
    Schnittmenge von A und B: $A\cap B = {x: x\in A \wedge x\in B}$\
    Vereinigungsmenge von A und B: $A\cup B = {x: x\in A \vee x\in B}$
    
    Sei eine Menge (von Mengen) dann gibt es eine Menge die aus genau den Mengen besteht, die in jeder Menge von A enthalten sind (außer $A=\emptyset$).
    Ebenso gibt es Mengen die aus genau den Mengen besteht, die in wenigstens einer Menge aus A liegen. Die Existenz dieser Menge wird axiomatisch gefordert in ZFC:$ UA = {x: \exists z \in A: x \in z}$\
    
    Seien A,B Mengen, dann sei $A/B:={x\in A: x\not \in B } = A\bigtriangleup B$\
    De Moorgansche Regel: $\overline{A \cup B} = \overline{A} \cap \overline{B}$ und $\overline{A\cap B}=\overline{A}\cup \overline{B}$\
    Das geordnete Paar (x,y) von Mengen x,y ist definiert durch ${{x},{x,y}}:={x,y}$\
    A und B Mengen: $A x B:={(x,y):x\in A \wedge y \in B}$
    
    \section{Relationen}
    $A={Peter, Paul, Marry}$ und $B={C++, Basic, Lisp}: R\subseteq AxB$, etwa {(Peter,c++),(Paul, C++), (Marry,Lisp)}. Seien A,B Mengen: Eine Relation von A nach B ist eine Teilmenge R von AxB.\
    $(x,y)\in R:$ x steht in einer Relation R zu y; auch xRy\
    Ist A=B, so heißt R auch binäre Relation auf A
    
    \paragraph{binäre Relation}
    \begin{itemize}
        \item Allrelation $R:=AxA \subseteq AxA$
        \item Nullrelation $R:=\emptyset \subseteq AxA$
        \item Gleichheitsrelation $R:={(x,y)... x=y}$
        \item $A=R; R:=((x,y)\in \mathbb{R} x \mathbb{R}, x \leq y)$
        \item $A=\mathbb{Z}; R:={(x,y)\in \mathbb{Z} x \mathbb{Z}: \text{x ist Teiler von y} }$ kurz: x|y
    \end{itemize}
    
    \paragraph{Eigenschaften von Relationen}
    Sei $R\in AxA$ binäre Relation auf A
    \begin{itemize}
        \item Reflexiv $\leftrightarrow$ xRx $\forall x \in A$
        \item symmetrisch $\leftrightarrow xRy \rightarrow yRx$
        \item Antisymmetrisch $\leftrightarrow xRy \wedge yRx \rightarrow x=y$
        \item Transitiv $\leftrightarrow xRy \wedge yRz \rightarrow xRz$
        \item totale Relation $\leftrightarrow xRy \vee yRx  \forall x,y \in A$
    \end{itemize}
    \begin{itemize}
        \item R heißt Äquivalenzrelation $\leftrightarrow$ R reflexiv, symmetrisch und transitiv
        \item R heißt Ordnung $\leftrightarrow$ R reflexiv, antisymmetrisch und transitiv
        \item R heißt Totalordnung $\leftrightarrow$ R Ordnung und total
        \item R heißt Quasiordnung $\leftrightarrow$ R reflexiv und transitiv
    \end{itemize}
    
    \paragraph{Äqivalenzrelation}
    Sei A Menge, $C\wp (A)$ Menge von Teilmengen von A. C heißt Partition von A, falls gilt:
    1. $UC=A$ d.h. jedes $x\in A$ liegt in (wenigstens) einem $y\in C$
    2. $\emptyset \not \in C$ d.h. jedes $y\in C$ enthält (wenigstens) ein Element von A
    3. $X \cap Y = \emptyset$ f.a. $X\not \in Y$ aus C
    
    Zwei Mengen $X\cap Y = \emptyset$ heißten disjunkt.\
    Satz: Sei $\sim$ Äquivalenzrelation auf A. Für $x\in A$ betrachtet $[x]_{/ \sim }:={y\in A: y \sim x}$. Dann ist ${[x]_{/ \sim }:x\in A}= C_{/ \sim }$ Partition von A. Die Elemente $[x]_{/ \sim }$ von $C_{/ \sim }$ heißen Äquivalenzklassen. Die Elemente von C heißten Teile, Klassen oder Partitionen.
    
    Somit ist $\equiv(mod m)$ eine Äquivalenzrelation. Ihre Äquivalenzklassen heißen Restklassen mod m
    
    Ein Graph $G=(V,E)$ ist ein Paar bestehend aus einer Menge V und $E\subseteq (x,y: x \not = y \text{ aus V} )$.
    Zu $a,b\in V$ heißt eine Folge $P=x_1,..,x_n$ von paarweise verschiedenen Ebenen mit $a=x_0, b=x_j; x_{j-1},x_i \in E{a*i \in b*j}$ ein a,b-Weg der Länge l oder Weg a nach b. Durch $a\sim b$ gibt es einen a,b-Weg in G, wird eine Äquivalenzrelation auf V definiert, denn:
    \begin{itemize}
        \item "$\sim$ reflexiv": es ist $x\sim x$, denn $P=x$ ist ein x,x-Weg in G
        \item "$\sim$ symmetrisch": aus $x\sim y$ folgt, es gibt einen x,y-Weg $\rightarrow$ es gibt einen y,x-Weg $y\sim x$
        \item "$\sim$ transitiv": aus $x\sim y$ und $y\sim x$ folgt, es gibt einen x,y-Weg und einen y,x-Weg
    \end{itemize}
    Die Äquivalenzklassen von $\sim _G$ erzeugen die Zusammenhangskomponenten von G
    
    Satz: Sei C eine Partition von A, dann wird durch $x\sim _G y \leftrightarrow$ es gibt ein $X\in C$ mit $x,y\in X$ eine Äquivalenzrelation auf A definiert.
    
    \paragraph{(Halb) Ordnungen}
    Sei also $leq$ eine Ordnung auf X. Seo $A\subseteq X, b\in X$
    \begin{itemize}
        \item b minimal in A $\leftrightarrow b\in A$ und $(c\leq b \rightarrow c=b f.a. c\in A)$
        \item b maximal in A $\leftrightarrow b\in A$ und $(b\leq c \rightarrow b=c f.a. c\in A)$
        \item b kleinstes Element in A $\leftrightarrow b\in A$ und $(b\leq c f.a. c\in A)$
        \item b größtes Element in A $\leftrightarrow b\in A$ und $(c\leq b f.a. c\in A)$
        \item b untere Schranke von A $\leftrightarrow b\leq c f.a. c\in A$
        \item b obere Schranke von A $\leftrightarrow c\leq b f.a. c\in A$
        \item b kleinste obere Schranke von A $\leftrightarrow$ b ist kleinstes Element von $(b'\in X: \text{b' obere Schranke von A})$ auch Supremum von A: $\lor A = b$
        \item b größte untere Schranke von A $\leftrightarrow$ b ist das größte Element von $(b'\in X: \text{ b' untere Schranke von A} )$ auch Infinum von A; $\land A = b$
    \end{itemize}
    kleinstes und größtes Element sind jew. eindeutig bestimmt (falls existent)
    
    Satz: Sei X Menge. $\subseteq$ ist Ordnung auf $\wp(X)$. Ist $O\subseteq \wp(X)$, so ist $sup O=\bigcup O$ und $inf O=\bigcap O$
    
    Satz: Die Teilbarkeitrelation | ist Ordnung auf den natürlichen Zahlen $\mathbb{N}$. Es gibt $sup(a,b)=kgV(a,b)$ (kleinstes gemeinsames Vielfaches) und $inf(a,b)=ggT(a,b)$ (größtes gemeinsames Vielfaches)
    
    \paragraph{Hesse Diagramm}
    Darstellung einer Ordnung $\subseteq$ auf X
    1. Im Fall $x\subseteq y$ zeichne x "unterhalb" von y in die Ebene
    2. Gilt $x\subseteq y (x\not = y)$ und folgt aus $x \subseteq z \subseteq y$ stets $x=z$ oder $y=z$ so wird x mit y "verbunden"
    
    \paragraph{Zoonsche Lemma}
    Zu jeder Menge und für jede Ordnung $\leq$ auf X mit der Eigenschaft, dass jede nicht-leere Kette nach der beschränkt ist, gibt es ein maximales Element.
    
    \paragraph{Wohlordnungssatz}
    Jede Menge lässt sich durch eine Ordnung $\subseteq$ so ordnen, dass jede nichtleere Teilmenge von X darin ein kleinstes Element ist
    
    \section{Induktion}
    X ist eine Menge, $X:=X\vee {X}$\
    M Menge heißt induktiv $:\leftrightarrow \emptyset \in M \wedge \forall X \in M$  $X^+ \in M$.
    
    Ist O eine Menge von induktiven Mengen, $O\pm O$ dann ist auch $\bigcap O$ induktiv. Insbesondere ist der Durchschnitt zweier induktiver Mengen induktiv. Es gibt eine induktive Menge M: $M =\bigcap {A \in \wp(M): A induktiv}$.
    Sei M' irgendeine (andere) induktive Menge $\rightarrow M \cap M'$ ist induktive Teilmenge von M. $\mathbb{N}_M$ ist der Durchschnitt über alle induktiven Teilmengen von M $\mathbb{N}_M \subseteq M \cap M' \subseteq M'$. Folglich ist $\mathbb{N}_m$ Teilmenge jeder induktiven Menge.
    
    \paragraph{Satz I (Induktion I)}
    Sei $p(n)$ ein Prädikat über $\mathbb{N}$. Gelte $p(0)$ und $p(n)\rightarrow p(n^{+})$ f.a. $n\in \mathbb{N}$ dann ist $p(n)$ wahr f.a. $n \in \mathbb{N}$. Schreibe $x=y:\leftrightarrow x\in y \vee x=y$
    
    \paragraph{Satz II (Induktion II)}
    Sei $p(n)$ ein Prädikat über $\mathbb{N}$, gelte ($\forall x < n: p(x)) \rightarrow p(n)$ f.a. $n\in \mathbb{N}$. Damit ist $p(n)$ wahr für alle $n\in \mathbb{N}$.
    
    \section{Funktionen}
    Seien A,B Mengen: Eine Relation $f\subseteq A x B$ heißt Funktion. A nach B ("$f:A\rightarrow B$") falls es zu jedem $x\in A$ genau ein $y\in B$ mit $(x,y)\in f$ gibt. Dieses y wird mit $f(x)$ bezeichnet.
    
    Satz: $f:A\rightarrow B, g:A\rightarrow B$; dann gilt $f=g \leftrightarrow f(x)=g(x)$. Sei $f:A\rightarrow B$ Funktion
    \begin{itemize}
        \item f heißt injektiv $\leftrightarrow$ jedes y aus B hat höchstens ein Urbild
        \item f heißt subjektiv $\leftrightarrow$ jedes y aus B hat wenigstens ein Urbild
        \item f heißt bijektiv $\leftrightarrow$ jedes y aus B hat genau ein Urbild
    \end{itemize}
    
    Ist $f:A\rightarrow B$ bijektive Funktion, dann ist auch $f^{-1}\subseteq BxA$ bijektiv von B nach A, die Umkehrfunktion von f.
    Man nennt f dann Injektion, Surjektion bzw Bijektion
    \begin{itemize}
        \item f injektiv $\leftrightarrow (f(x)=f(y)\rightarrow x=y)$ f.a. $x,y\in A$ oder $(x\not = y \rightarrow f(x)\not = f(y))$
        \item f surjektiv $\leftrightarrow$ Zu jedem $x\in B$ existiert ein $x\in A$ mit $f(x)=y$
        \item f bijektiv $\leftrightarrow$ f injektiv und surjektiv
    \end{itemize}
    
    Sind $f:A\rightarrow B$ und $g:B\rightarrow C$ Funktionen, so wird durch $(g \circ f)(x):=g(f(x))$ eine Funktion $g \circ f: A \rightarrow C$ definiert, die sog. Konkatenation/Hintereinanderschaltung/Verkettung/Verkopplung von f und g (gesprochen "g nach f").
    
    Satz: $f:A\rightarrow B, g:B\rightarrow C$ sind Funktionen. Sind f,g bijektiv, so ist auch $g \circ f: A\rightarrow C$ bijektiv
    
    Satz: ist $f:A\rightarrow B$ bijektiv, so ist $f^{-1}$ eine Funktion B nach A. Mengen A,B, heißen gleichmächtig ($|A|=|B| \equiv A\cong B$) falls Bijektion von A nach B. $\cong$ ist auf jeder Menge von Mengen eine Äquivalenzrelation
    \begin{itemize}
        \item "$\cong$ reflexiv": $A\cong A$, denn $f:A\rightarrow A, f(x)=X$, ist Bijektion von A nach A
        \item "$\cong$ symmetrisch": Aus $A\cong B$ folgt Bijektion von A nach B $\rightarrow B\cong A$
        \item "$\cong$ transitiv": Aus $A\cong B$ und $B\cong C$ folgt $A\cong C$
    \end{itemize}
    
    $|A|=|A|:|A|$ ist die Kordinalität von A, d.h. die kleinste zu A gleichmächtige Ordinalzahl. Eine Ordinalzahl ist eine e-transitive Menge von e-transitiven Mengen. Eine Menge X heißt e-transitiv, wenn aus $a\in b$ und $b\in c$ stets $a\in c$ folgt.
    Sei $A:=\mathbb{N}$ und $B={0,2,4,...}={n\in \mathbb{N}: 2|n}$, dann sind A und B gleichmächtig, denn $f:A\rightarrow B, f(x)=2x$ ist Bijektion von A nach B.
    Eine Menge A heißt endlich, wenn sie gleichmächtig zu einer natürlichen Zahl ist; sonst heißt A unendlich.
    Eine Menge A heißt Deckend-unendlich, falls es eine Injektion $f:A\rightarrow B$ gibt die nicht surjektiv ist.
    
    Satz: A unendlich $\leftrightarrow$ A deckend-unendlich
    A,B sind Mengen. A heißt höchstens so mächtig wie B, falls es eine Injektion von A nach B gibt. $|A|\leq |B|$ bzw $A\preceq B$. $\preceq$ ist Quasiordnung auf jeder Menge von Mengen.
    \begin{itemize}
        \item "$\preceq$ reflexiv": Injektion von A nach A
        \item "$\preceq$ transitiv": $A\preceq B$ und $B\preceq C$ folgt Injektion $f:A\rightarrow B$ und $g:B\rightarrow C$. Verkopplung $g \circ f \rightarrow A \preceq C$
    \end{itemize}
    
    Satz (Vergleichbarkeitssatz):
    Für zwei Mengen A,B gilt $|A|\leq |B|$ oder $|B| \leq |A|$. Eine Relation f von A nach B heißt partielle Bijektion (oder Matching), falls es Teilmengen $A'\subseteq A$ und $B'\subseteq B$ gibt sodass f eine Bijektion von A' nach B' gibt.
    
    Sei M die Menge aller Matchings von A nach B und wie jede Menge durch $\subseteq$ geordnet. Sei $K\subseteq M$ eine Kette von Matchings. K besitzt eine obere Schranke ($\bigcup K$) in M. Seien $(x,y);(x',y')$ zwei Zuordnungspfeile aus $\bigcup K$, zeige $x\not = x'$ und $y\not = y'$ dann folgt Matching.
    Jede Kette von Matchings benutzt eine obere Schranke, die ebenfalls ein Matching ist $\rightarrow$ es gibt ein maximales Matching von A nach B, etwa h. Im Fall ($x\in A, y\in B$ mit $(x,y)\in h$) ist h eine Injektion von A nach B, d.h. $|A| \subseteq |B|$ andernfalls $y\in B, x\in A$ mit $x,y\in h$ ist $h^{-1}$ eine Injektion von B nach A, d.h. $|B| \subseteq |A|$.
    
    Satz (Cantor/Schröder/Bernstein): 
    Für zwei Mengen A,B gilt: Aus $|A|\subseteq |B|$ und $|B| \subseteq |A|$ folgt $|A| = |B|$.
    
    Satz (Cantor):
    Für jede Menge X gilt: $|X| \leq \wp(X)$ und $|X|\not= |\wp (X)|$. Z.B. ist $|\mathbb{N}|<|\mathbb{R}|$; zu $|\mathbb{N}|$ gleichmächtige Mengen nennt man abzählbar; unendliche nicht-abzählbare Mengen nennt man überzählbar.
    
    \paragraph{Kontinuitätshypothese}
    Aus $|\mathbb{N}|\leq |A| \leq |\mathbb{R}|$ folgt $|A|=|\mathbb{N}|$ oder $|A|=|\mathbb{R}|$ (keine Zwischengrößen).
    
    Seien M,I zwei Mengen. Eine Funktion $f:I\rightarrow M$ von I nach M heißt auch Familie über der Indexmenge I auf M. Schreibweise $(m_i)_{i\in I}$ wobei $m_i=f(i)$. Familien über $I=\mathbb{N}$ heißen Folgen (bzw. unendliche Folgen).
    Eine (endliche) Folge ist eine Familie über einer endlichen Indexmenge I. Funktionen von ${1,...,n}$ in einer Menge A ($a_q,...,a_n\in A$) heißen n-Tupel. Für eine Mengenfamilie $(A_i)_{i\in A}$ sei ihr Produkt durch $\prod A_i=(f: \text{ Funktion von I nach}\bigcup A_i \text{ mit } f(i)\in A_i \text{ f.a. } i\in I)$. Ist allgemein $A_i=A$ konstant, so schreibe $\prod A_i=A^I={f:I\rightarrow R}$. Bezeichnung auch $2^{\mathbb{N}}$.
    
    \section{Gruppen, Ringe, Körper}
    Eine Operation auf eine Menge A ist eine Funktion $f:AxA\rightarrow A$; schreibweise $xfy$. EIne Menge G mit einer Operation $\circ$ auf G heißt Gruppe, falls gilt:
    \begin{itemize}
        \item $a\circ (b\circ c) = (a\circ b)\circ c$ freie Auswertungsfolge
        \item es gibt ein $e\in G$ mit $a\circ e=a$ und $e\circ a=a$ f.a. $a\in G$. e heißt neutrales Element von G und ist eindeutig bestimmt
        \item zu jedem $a\in G$ existiert ein $b\in G$ mit $a\circ b=e$ und $b\circ a=e$; wobei e ein neutrales Element ist. b ist durch a eindeutig bestimmt, denn gäbe es noch ein $c\in G$ mit $a\circ c=e$ folgt $b=b\circ e$. Schreibweise für dieses eindeutig durch a bestimmte b: $a^{-1}$
    \end{itemize}
    
    Eine Gruppe G mit $\circ$ wird auch mit $(G, \circ)$ bezeichnet. Sie heißt kommutativ bzw abelsch, falls neben 1.,2. und 3. außerdem gilt:
    \begin{itemize}
        \item $a\circ b = b\circ a$ f.a. $a,b \in G$
    \end{itemize}
    
    Das neutrale Element aus 2. wird mit 1 bezeichnet. Im Fall der abelschen Gruppe benutzt man gerne "additive Schreibung": "+" statt "$\circ$" und "0" statt "1" (Bsp: $1*a = a*1 = a$).
    Eine Bijektion von X nach X heißt Permutation von X. $(S_X, \circ)$ ist eine Gruppe.
    
    Zwei Gruppen $(G, \circ_G)$ und $(H,\circ_H)$ heißen isomorph, falls es einen Isomorphismus von $(G,\circ_G)$ nach $(H,\circ_H)$ gibt (bzw. von G nach H). Schreibweise $(G,\circ_G)\cong (H,\circ_H)$
    \begin{itemize}
        \item "$\cong$ reflexiv": $G\cong G$, denn $id_G$ ist ein Isomorphismus
        \item "$\cong$ symmetrisch": aus $G\cong G$ folgt: es existiert ein bijektiver Homomorphismus
        \item "$\cong$ transitiv": sei $G\cong H$ und $H\cong J \rightarrow$ es gibt einen Isomorphismus $\phi:G\rightarrow H$ und $\psi:H\rightarrow J \rightarrow \phi\circ \psi :G\rightarrow J \rightarrow$ J ist bijektiv. $\phi\circ G$ ist Homomorphismus von G nach J und bijektiv also Isomorph
    \end{itemize}
    Satz: Jede Gruppe $(G,\circ)$ ist zu einer Untergruppe von $(S_G, \circ)$ isomorph
    
    \paragraph{Arithmetik von $\mathbb{N}$}
    $+: \mathbb{N} x \mathbb{N} \rightarrow \mathbb{N}$ wird definiert durch:
    \begin{itemize}
        \item $m+0:=m$ f.a. $m\in \mathbb{N}$ (0 ist neutral)
        \item $m+n$ sei schon definiert f.a. $m\in \mathbb{N}$ und ein gutes $n\in \mathbb{N}$
        \item $m+n^+:=(m+n)^+$ f.a. $m,n \in \mathbb{N}$
    \end{itemize}
    
    Satz: $m+n=n+m$ f.a. $m,n\in\mathbb{N}$ (Beweis induktiv über m)
    
    Satz: $l+(m+n)=(l+m)+n$ f.a. $l,m,n\in\mathbb{N}$ (Klammern sind neutral bzgl +)
    
    Satz (Streichungregel): aus $a+n=b+n$ folgt $a=b$ f.a. $a,b,n\in\mathbb{N}$
    
    \paragraph{Analog: Multiplikation}
    $*: \mathbb{N} x \mathbb{N} \rightarrow \mathbb{N}$ wird definiert durch:
    \begin{itemize}
        \item $m*0:=0$ f.a. $m\in \mathbb{N}$
        \item $m*n^+=m*n+m$ f.a. $n\in\mathbb{N}$
    \end{itemize}
    Es gilt:
    \begin{itemize}
        \item $m*n=n*m$ f.a. $n\in\mathbb{N}$
        \item $m*(n*l)=(m*n)*l$ f.a. $m,n\in\mathbb{N}$
        \item $m*1 = 1*m =m$ f.a. $m\in\mathbb{N}$
        \item $a*n=b*n \rightarrow a=b$ f.a. $a,b\in\mathbb{N}, n\in\mathbb{N}/{0}$
        \item $a*(b+c)=a*b+a*c$ (Distributivgesetz)
    \end{itemize}
    
    \paragraph{Die ganzen Zahlen $\mathbb{Z}$}
    Durch $(a,b)\sim (c,d)\leftrightarrow a+d=b+c$ wird eine Äquivalenzrelation auf $\mathbb{N} x\mathbb{N}$ definiert.
    Die Äquivalenzklassen bzgl $\sim$ heißen ganze Zahlen (Bezeichnung $\mathbb{Z}$, Bsp $17=[(17,0)]_{/\sim }$).
    Wir definieren Operationen +, * auf $\mathbb{Z}$ durch:
    \begin{itemize}
        \item $[(a,b)]_{/\sim } + [(c,d)]_{/\sim } = [(a+c, b+d)]_{/\sim }$
        \item $[(a,b)]_{/\sim } * [(c,d)]_{/\sim } = [(ac+bd, ad+bc)]_{/\sim }$
    \end{itemize}
    Zu zeigen ist: Die auf der rechten Seite definierten Klassen hängen nicht von der Wahl der "Repräsentanten" der Klassen auf der linken Seite ab (Wohldefiniert).
    
    Formal (für +): $[(a,b)]_{/\sim } = [(a',b')]_{/\sim }$ und $[(c,d)]_{/\sim } = [(c',d')]_{/\sim }$ impliziert $[(a,b)]_{/\sim } + [(c,d)]_{/\sim } = [(a'+c', b'+d')]_{/\sim }$. Aus der Vss konstant kommt $a+b'=b+a'$ und $c+d'=c'+d$. Dann folgt $a+c+b'+d'=b+d+a'+c'$, also $(a+c, b+d)\sim (a'+c',b'+d')$.
    
    Satz: $\mathbb{Z}$ ist eine abelsche Gruppe (+ assoziativ, enthält neutrales Element, additiv Invers).
    $[(a,0)]_{/\sim }$ wird als a notiert. $-[(a,0)]_{/\sim }=[(0,a)]_{/\sim }$ wird als -a notiert.
    Anordnung: $[(a,b)]_{/\sim } \subseteq [(c,d)]_{/\sim } \leftrightarrow a+d\leq b+c$
    
    Ein Ring R ist eine Menge mit zwei Operationen $+,*: \mathbb{R} x \mathbb{R} \rightarrow \mathbb{R}$ mit:
    \begin{itemize}
        \item $a+(b+c) = (a+b)+c$ f.a. $a,b,c\in \mathbb{R}$
        \item Es gibt ein neutrales Element $O\in \mathbb{R}$ mit $O+a=a+O=O$ f.a. $a\in\mathbb{R}$
        \item zu jedem $a\in \mathbb{R}$ gibt es ein $-a\in \mathbb{R}$ mit $a+(-a)=-a+a=0$
        \item $a+b=b+a$ f.a. $a,b\in\mathbb{R}$
        \item $a*(b*c)=(a*b)*c)$ f.a. $a,b,c\in\mathbb{R}$
        \item $a*(b+c)=a*b+a*c$ f.a. $a,b,c\in\mathbb{R}$
    \end{itemize}
    R heißt Ring mit 1, falls:
    \begin{itemize} 
        \item es gibt ein $1\in\mathbb{R}$ mit $a*1=1*a=a$ f.a. $a\in\mathbb{R}$
    \end{itemize}
    R heißt kommutativ, falls:
    \begin{itemize}
        \item $a*b=b*a$ f.a. $a,b\in\mathbb{R}$
    \end{itemize}
    Ein kommutativer Ring mit $1\not=O$ heißt Körper, falls:
    \begin{itemize}
        \item zu jedem $a\in\mathbb{R}$ gibt es ein $a^{-1}\in\mathbb{R}$ mit $a*a^{-1}=a^{-1}*a=1$
    \end{itemize}
    
    Bemerkung: $O$ kann kein multiplikativ inverses haben.
    \begin{itemize}
        \item Ist $\mathbb{R}$ ein Körper, so ist $\mathbb{R}*=\mathbb{R} /(0)$ mit $*$ eine abelsche Gruppe.
        \item $\mathbb{Z}$ mit + und * ist ein kommutativer Ring mit $1 \not= 0$ aber kein Körper
        \item $\mathbb{Q}, \mathbb{C}, \mathbb{R}$ mit + und * ist ein Körper
    \end{itemize}
    
    \paragraph{Division mt Rest in $\mathbb{Z}$}
    Satz: Zu $a,b\in\mathbb{Z}, b \not= 0$, gibt es eindeutig bestimmte $q,r\in \mathbb{Z}$ mit $a=q*b+r$ und $0\leq q <|b|$ (d.h. $\mathbb{Z}$ ist ein euklidischer Ring). (Beweis über Induktion)
    
    \paragraph{Zerlegen in primäre Elemente}
    Satz: Jede ganze Zahl $n>0$ lässt sich bis auf die Reihenfolge der Faktoren eindeutig als Produkt von Primzahlen darstellen.
    
    Beweis-Existenz mit Annahme: Der Satz gilt nicht, dann gibt es eine kleinste Zahl n die sich nicht als Produkt von Primzahlen schreiben lässt $\rightarrow$ n weder Primzahl noch 1 $\rightarrow n=m*l$ für $m,l>1 \rightarrow$ m und l sind Produkte von Primzahlen $\rightarrow m*l=$ Produkt von Primzahlen.
    
    Eindeutigkeit mit Annahme: es gibt ein $n>0$ ohne eindeutige Primfaktorzerlegung (PFZ)$\rightarrow$ es gibt ein kleinstes $n>0$ ohne eindeutige PFZ. Kommt eine Primzahl p in beiden Zerlegungen vor, so hat auch $\frac{n}{p}$ zwei verschiedene PFZen. Man erhält die PFZ von $n'=(1_1-p_1)*b$ aus den PFZen von $q_1-p_1$ und b.. -> Eindeutig bestimmbar.
    
    \paragraph{Arithmetik im Restklassenring in $\mathbb{Z}$}
    Sei $m > 1$ gegeben, $a\equiv \text{b mod m} \leftrightarrow m|a-b$ def. Relation auf $\mathbb{Z}$. Die Äquivalenzklasse zu a wird mit $\bar{a}$ bezeichnet, d.h. $\bar{a}=[a]_{\text{mod m}}={x\in \mathbb{Z}: x\equiv \text{a mod m}}$, $\mathbb{Z}_m={\bar{a}:a\in \mathbb{Z}}$. Sei dazu $\bar{a}\in \mathbb{Z}_m$ beliebig.
    
    Division mit Rest $\rightarrow$ es gibt eindeutig bestimmt q,r mit $a=q*m+r$ und $0\leq r < m \rightarrow a-r=q*m \rightarrow m| a-r \rightarrow a\equiv \text{r mod m } \rightarrow \bar{a}=\bar{r}$. Also tritt $\bar{a}$ in der Liste $\bar{0},\bar{1},...,\bar{m-1}$ auf. Aus $0\leq i < j \leq m-1$ folgt $\bar{i}\not=\bar{j}$. In der Liste $\bar{0},\bar{1},...,\bar{m-1}$ gibt es daher keine Wiederholungen $\rightarrow |\mathbb{Z}_M|=m$.
    
    Wir definieren Operationen +,* auf $\mathbb{Z}_m$ durch $\bar{a}+\bar{b}:= \bar{a+b}$ und $\bar{a}*\bar{b}:=\bar{a*b}$ für $a,b\in \mathbb{Z}$. 
    Wohldefiniert: aus $\bar{a}=\bar{a'}$ und $\bar{b}=\bar{b'}$ folgt $\bar{a+b}=\bar{a'+b'}$. Analog für Multiplikation.
    
    Eigenschaften von $\mathbb{Z}$ mit +,* werden auf $\mathbb{Z}$ mit +,* "vererbt", z.B. Distributivgesetz.
    
    Satz: Sei $m\geq 2$ dann ist $\mathbb{Z}_m$ mit +,* ein kommutativer Ring mit $\bar{1}\not=\bar{0}$. Genau dann ist $\mathbb{Z}_m$ sogar ein Körper, wenn m eine Primzahl ist.
    
    Satz: Genau dann gibt es einen Körper mit n ELementen, wenn n eine Primzahl ist. D.h.. wenn $n=p^a$ ist für eine Primzahl p und $a\geq 1$.
    
    \paragraph{Konstruktion von $\mathbb{Q}$ aus $\mathbb{Z}$}
    Sei $M=\mathbb{Z} x(\mathbb{Z} /{0}$ die Menge von Brüchen. Durch $(a,b)\sim (c,d)\leftrightarrow ad=bc$ wird Äquivalenzrelation auf M durchgeführt. Schreibweise für die Äquivalenzklassen $\frac{a}{b}$ Die Elemente von $\mathbb{Q} :{\frac{a}{b}:a,b\in\mathbb{Z}, b\not=0}$ heißten rationale Zahlen.
    Definiere Operationen +,* auf $\mathbb{Q}$ wie folgt:
    \begin{itemize}
        \item $\frac{a}{b}+\frac{c}{d} = \frac{ad+bc}{b*d}$ (wohldefiniert)
        \item $\frac{a}{b}*\frac{c}{d} = \frac{a*c}{b*d}$
    \end{itemize}
    
    Satz: $\mathbb{Q}$ mit +,* ist ein Körper.
    
    Durch $\frac{a}{b}\leq\frac{c}{d}$ wird eine totale Ordnung auf $\mathbb{Q}$ definiert. Konstruktion von $\mathbb{R}$ aus $\mathbb{Q}$ mit Dedchin-Schnitten.
    
    \paragraph{Ring der formalen Potenzreihe}
    Sei k ein Körper (oder nur ein Ring mit 1+0). Eine Folge $(a_0, a_1,...,a:n)\in K^{\mathbb{N}}$ mit Einträgen aus K heißt formale Potenzreihe. Die Folge (0,1,0,0,...) wird mit x bezeichnet. Statt $K^{\mathbb{N}}$ schreibt man $K[[x]]$. $(0_0,a_1,a_2,...)$ heißt Polynom in x, falls es ein $d\in \mathbb{N}$ gibt mit $a_j=0$ f.a. $j<n$. Die Menge aller Polynome wird mit $K[x]$ bezeichnet.
    
    Satz: $K[[x]]$ wird mit +,* wie folgt zu einem kommutativen Ring mit $1\not=0$
    \begin{itemize}
        \item +: $(a_0,a_1,...) + (b_0,b_1,...) = (a_o+b_0, a_1+b_1, ...)$
        \item *: $(a_0,a_1,...) + (b_0,b_1,...) = (c_0, c_1,...)$ mit $c_K=\sum_{j=a}^{k} a_j*b_{k-j}$
    \end{itemize}
    Die formale Potenzreihe $(a,0,0,0,...)$ wird ebenfalls mit a bezeichnet.
    
    Die bzgl $\leq$ minimalen Elemente von $B /\perp$ heißen Atom von B.
    Satz: Sei $b\in B /\perp$ und $a_1,...,a_k$ diejenigen Atome a mit $a\leq b$, dann ist $b= a_1 \vee a_2 \vee ... \vee a_k$.
    
    B mit $\vee, \wedge, \bar{ }$ und $\dot{B}$ mit $\dot{\vee}, \dot{\wedge}, \dot{\bar{}}$ seien boolesche Algebren. Sie heißen isomorph, falls es einen Isomorphismus von B nach $\dot{B}$ gibt, d.h. eine Bijektion $\phi: B \rightarrow \dot{B}$ mit:
    \begin{itemize}
        \item $\phi(a\vee b) =\phi(a)\dot{\vee}\phi(b)$ f.a. $a,b \in B$
        \item $\phi(a\wedge b)=\phi(a)\dot{\wedge}\phi(b)$ f.a. $a,b\in B$
        \item $\phi(\bar{a}) = \dot{\bar{\phi(a)}}$ f.a. $a\in B$
    \end{itemize}
    
    Satz (Stone): Ist B mit $\vee, \wedge, \bar{}$ eine boolesche Algebra, B endlich und A die Menge ihrer Atome, so ist B isomorph zur booleschen Algebra $\wp(A)$ mit $\cap,\cup,\dot{\bar{}}$, wobei $\dot{\bar{X}}=A/X$.
    Also ist in jeder Teilmenge X von A Bild eines Elements von B unter $\phi$.
    
    Satz: $\perp$, T sind durch die Bedingung 3 eindeutig bestimmt.
    
    Satz: $\bar{a}$ ist durch die Bedingung 1,2,4 eindeutig bestimmt.
    
    Lemma: Sei B mit $\vee, \wedge, \bar{}$ eine boolesche Algebra, dann gilt:
    \begin{itemize}
        \item Dominanz
              \begin{itemize}
                  \item $a\vee T = T$ f.a. $a\in B$
                  \item $a\wedge \perp = \perp$ f.a. $a\in B$
              \end{itemize}
        \item Absorption
              \begin{itemize}
                  \item $a\vee(a\wedge b)= a$ f.a. $a,b\in B$
                  \item $a\wedge(a\vee b)= a$ f.a. $a,b\in B$
              \end{itemize}
        \item Streichungsregel
              \begin{itemize}
                  \item $a\wedge x = b\wedge x \rightarrow a=b$ f.a. $a,b,c \in B$
                  \item $a\wedge \bar{x} = b\wedge\bar{x} \rightarrow a=b$ f.a. $a,b,x \in B$
              \end{itemize}
        \item Assoziativität
              \begin{itemize}
                  \item $a\vee(b\vee c)=(a\vee b)\vee c$ f.a. $a,b,c\in B$
                  \item $a\wedge(b\wedge c)=(a\wedge b)\wedge c$ f.a. $a,b,c \in B$
              \end{itemize}
        \item De Moorgansche Regel
              \begin{itemize}
                  \item $\bar{a\vee b} = \bar{a}\wedge \bar{b}$ f.a. $a,b\in B$
                  \item $\bar{a\wedge b} = \bar{a}\vee \bar{b}$ f.a. $a,b\in B$
              \end{itemize}
    \end{itemize}
    
    Satz: Durch $a\leq b:\leftrightarrow a\vee b=b$ wird eine Ordnung auf der booleschen Algebra B mit $\vee, \wedge, \bar{}$ definiert ($a\vee b = sup{a,b}$; $a\wedge b = inf{a,b}$)
    
    Es gilt $a\vee b= b \rightarrow a\wedge b = a\wedge(a\vee b)= a$
    \begin{itemize}
        \item $a\vee b$ ist obere Schranke von ${a,b}$, d.h. $a\leq a\vee b$, dann $a\vee(a\vee b)=a\vee b$
        \item $a\vee b$ ist kleinste obere Schranke, d.h. $a\leq z$ und $b\leq z$ folgt $a\vee b \leq z$
    \end{itemize}
    
    Sind $B, \dot{B}$ isomorph, so schreibe $B \cong \dot{B}$. Daraus folgt $\dot{B} \cong B$ und aus $B \cong \dot{B}$ und $\dot{B} \cong \ddot{B}$ folgt $B \cong \ddot{B}$.
    Weiterhin besitzt jede boolesche Algebra mit genau n Atomen genau $2^n$ viele Elemente (denn sie ist isomorph zur booleschen Algebra).
    
    Beispiel: Sei X eine endliche Menge von Variablen. Eine aussagenlogische Formel F in X ist:
    \begin{itemize}
        \item atomar: "x" mit $x\in X$ oder "f" oder "w" oder
        \item zusammengesetzt: $(P\vee Q), (P \wedge Q), (\neg P)$ aus den Formeln P,Q
    \end{itemize}
    Der Wahrheitswert von F unter der Belegung $\beta: X\rightarrow {f,w}$ ergibt sich wie in Kapitel 1. Bezeichnung für den Wahrheitswert von F unter $\beta: W_F(\beta)$. Es gibt $2^{|x|}$ Belegungen.
    Der Wahrheitswerteverlauf ist die so definierte Funktion $W_F:{f,w}^X\rightarrow{f,w}$. Folglich gibt es $2^ {2^{|x|}}$ verschiedene Wahrheitswertverläufe für logische Formeln. Formeln F, F' heißen äquivalent, falls $W_F=W_{F'} \rightarrow$ es gibt $2^ {2^{|x|}}$ verschiedene Äquivalenzklassen aussagenlogischer Formeln in X. Die Äquivalenzklassen werden mit $[F]_{/\equiv}$ bezeichnet.
    
    Sei $B:=([F]_{/\equiv }: \text{F aussagenlogische Formel in X} )$ die Menge aller Äquivalenzklassen aussagenlogischer Formeln in X.
    \begin{itemize}
        \item $[P]_{/\equiv} \vee [Q]_{/\equiv} = [(P\vee Q)]_{/\equiv}$
        \item $[P]_{/\equiv} \wedge [Q]_{/\equiv} = [(P\wedge Q)]_{/\equiv}$
        \item $\bar{[P]_{/\equiv}} = [-(P)]_{/\equiv}$
    \end{itemize}
    liefert die boolesche Algebra auf B
    \begin{itemize}
        \item $\perp = [f]_{/\equiv}$ = Menge der Kontradiktionen von X
        \item $T = [w]_{/\equiv}$ = Menge der Tautologien von X
    \end{itemize}
    
    Ordnung $\leq$ auf B: $[P]_{/\equiv} \leq [Q]_{/\equiv} \leftrightarrow [P]_{/\equiv} \wedge [Q]_{/\equiv} \rightarrow$ Die Atome von B sind genau die Klassen zu Formel P mit $W_p^{-1}({w})=1$. Kanonische Repräsentanten für diese Atome sind die Min-Terme.
    Zu jeder aussagenlogischen Formel f kann man die Atome $[P]_{/\equiv}$ mit $[P]_{/\equiv} \leq [F]_{/\equiv}$ betrachten, wobei P Min-Terme sind.
    
    Satz: Jede Formel ist äquivalent zu einer Formel in DNF (disjunkte normal Form)
    
    Coatome der booleschen Algebra B mit $\vee, \wedge, \bar{}$ := Atome der dualen booleschen Algebra B mit $\vee, \wedge, \bar{}$
    
    Ist $b\in B$ und $a_1,...,a_k$ die Coatome a mit $b\leq a$ so gibt $b=a_1 \wedge ... \wedge a_k$. Max-Terme sind "$x_1\vee ... \vee x_k$" und alle j die durch Ersetzung einiger $x_j$ durch $\neg x_j$ daraus hervorgehen und sind die kanonische Repräsentation der Coatome von B.
    
    Satz: Jede aussagenlogische Formel ist äquivalent zu einer Formel in konjunktiver Normalform (KNF), d.h. zu einer Formel $P_1\wedge ... \wedge P_n$, worin die $P_j$ Max-Terme sind.
    
    \section{Diskrete Wahrscheinlichkeitsräume}
    Ein (endlicher, diskreter) Wahrscheinlichkeitsraum ist ein Paar $(\Omega, p)$ bestehend aus einer endlichen Menge $\Omega$ und einer Funktion $p:\Omega \rightarrow [0,1]\in \mathbb{R}$ mit $\sum_{\omega \in \Omega} p(\omega)=1$. Jeder derartige p heißt (Wahrscheinlichkeits-) Verteilung auf $\Omega$. Die Elemente aus $\Omega$ heißen Elementarereignis, eine Teilmenge A von $\Omega$ heißt ein Ereignis; seine Wahrscheinlichkeit ist definiert durch $p(A):=\sum_{\omega in A} p(\omega)$.\
    $A=\emptyset$ und jede andere Menge $A\subseteq \Omega$ mit $p(A)=0$ heißt unmöglich (unmögliches Ereignis).\
    $A=\Omega$ und jede andere Menge $A\subseteq \Omega$ mit $p(A)=1$ heißt sicher (sicheres Ereignis).\
    Es gilt für Ereignisse $A,B,A_1,...,A_k$:
    \begin{itemize}
        \item $A\subseteq B \rightarrow p(A)\leq p(B)$ denn $p(A)=\sum p(\omega) \leq \sum p(\omega) = p(B)$
        \item $p(A\cup B) \rightarrow p(A)+p(B)-p(A\cap B)$
        \item Sind $A_1,...,A_k$ paarweise disjunkt (d.h. $A_i \cap A_J=\emptyset$ für $i\not =j$) so gilt $p(A_1 \cup ... cup A_k)= p(A_1)+...+p(A_k)$
        \item $p(\Omega / A):=$ Gegenereignis von $A=1-p(A)$
        \item $p(A_1,...,A_k) \leq p(A_1)+...+p(A_k)$
    \end{itemize}
    
    \paragraph{Beispiel: Würfelwurf}
    \begin{itemize}
        \item ungezinkt:
              \begin{itemize}
                  \item $\Omega = {1,2,3,4,5,6}$
                  \item $p=(\frac{1}{6},\frac{1}{6},\frac{1}{6},\frac{1}{6},\frac{1}{6},\frac{1}{6})$
                  \item d.h. $p(\omega)=\frac{1}{6}$ f.a. $\omega \in \Omega$
              \end{itemize}
        \item gezinkt:
              \begin{itemize}
                  \item $\Omega = {1,2,3,4,5,6}$
                  \item $p=(\frac{1}{4}, \frac{1}{10}, \frac{1}{5}, \frac{1}{4}, \frac{1}{10}, \frac{1}{10})=(25\%, 10\%, 20\%, 25\%, 10\%, 10\%)$
                  \item $p({\omega \in \Omega: \omega gerade})=p({2,4,6})=p(2)+p(4)+p(6)= \frac{1}{10}+ \frac{1}{4}+ \frac{1}{10} = \frac{9}{20}$
              \end{itemize}
    \end{itemize}
    
    Satz: Sind $(\Omega, p_1),...,(\Omega, p_m)$ Wahrscheinlichkeitsräume so ist durch $p((\omega_1,...,\omega_m))=\prod p_i(\omega_i)$ eine Verteilung auf $\Omega = \Omega_1 x ... x \Omega_m = {(\omega_1,...,\omega_m): \omega \in \Omega, f.a. i\in{1,...,m}}$. Für $A_1\subseteq \Omega_1, A_2\subseteq \Omega_2,...,A_m\subseteq \Omega_m$ gilt $p(A_1x...xA_m)=\prod p_i(A_i)$.
    $(\Omega, p)$ heißt Produktraum von $(\Omega_1, p_1),...$.
    
    $(\Omega, p)$ Wahrscheinlichkeitsraum; $A,B\in \Omega$ heißen (stochastisch) unabhängig, falls $p(A\cap B) = p(A)*p(B)$.
    Beispiel: $p(A\cap B) = p({i,j}) =p_1{i}*p_2{j} = p(A)*p(B)$ für das Ereignis "der 1. Würfel zeigt i, der 2. Würfel zeigt j"
    
    \paragraph{Bedingte Wahrscheinlichkeiten}
    $(\Omega, p)$ Wahrscheinlichkeitsraum, $B\subseteq \Omega$ ("bedingtes Ereignis") mit $p(B)>0$, dann ist $p_B:B\rightarrow [0,1]; p_B(\omega)=\frac{p(\omega)}{p(B)}$ eine Verteilung auf B, denn $\sum p_b(\omega)=\sum \frac{p(\omega)}{p(B)}=\frac{1}{p(B)} \sum p(\omega)= \frac{1}{p(B)} p(B)= 1$.
    $p_B$ ist die durch B bedingte Verteilung. Für $A\subseteq \Omega$ gilt $p_B(A\cap B)=\sum p_B(\omega)=\sum\frac{p(\omega)}{p(B)}= \frac{p(A\cap B)}{p(B)}:= p(A|B)$ ("p von A unter B") bedingte Wahrscheinlichkeit von A unter B.
    
    Satz (Bayer): $p(A|B)=\frac{p(B|A)*p(A)}{p(B)}$ wobei $p_A, p_B \geq 0$
    
    Satz (Totale Wahrscheinlichkeit): Seien $A_1, ...,A_k$ paarweise disjunkt, $\bigcup A_j=\Omega, p(A_i)>0, B\subseteq \Omega$, dann gilt $p(B)=\sum p(B|A_i)*p(A_i)$.
    
    Satz (Bayer, erweitert): $A_1,...,A_k,B$ wie eben, $p(B)>0$. Für $i\in {1,...,k}$ gilt $p(A_i|B)=\frac{p(B|A_i)*p(A_i)}{\sum p(B|A_j)*p(A_j)}$
    
    Beispiel: In einem Hut liegen drei beidseitig gefärbte Karten. Jemand zieht ("zufällig") eine Karte und leg sie mit einer ("zufälligen") Seite auf den Tisch. Karten rot/rot, rot/blau und blau/blau. Gegeben er sieht rot, wie groß ist die Wahrscheinlichkeit, dass die andere Seite auch rot ist?
    p(unten rot | oben rot) = p(unten rot und oben rot)/p(oben rot) = $\frac{p\binom{r}{r}}{p(\binom{r}{r}\binom{r}{b})}=\frac{\frac{2}{6}}{\frac{3}{6}}=\frac{2}{3}$
    
    Eine Funktion $X:\Omega \rightarrow \mathbb{R}$ heißt (reellwertige) Zufallsvariable. Weil $\Omega$ endlich ist, ist auch $X(\Omega)={X(\omega): \omega \in \Omega}\subseteq \mathbb{R}$ endlich. Durch $p_x(x):=p(X=x):=p({\omega \in \Omega: X(\omega)=x})$ wird ein Wahrscheinlichkeitsraum $(X(\Omega),p_x)$ definiert; denn $\sum p_x(x)=p(\Omega)=1$. $p_x$ heißt die von X induzierte Verteilung. $X(\Omega)$ ist meist erheblich kleiner als $\Omega$.
    Beispiel: Augensumme beim Doppelwurf: $X:\Omega\rightarrow \mathbb{R}, X((i,j))=i+j \rightarrow X(\Omega)={2,3,4,...,12}$
    
    Satz: Seien $(\Omega_1, p_1),(\Omega_2, p_2)$ Wahrscheinlichkeitsräume und $(\Omega, p)$ ihr Produktraum. Sei $X:\Omega_1\rightarrow\mathbb{R},Y:\Omega_2\rightarrow \mathbb{R}$, fasse X,Y als ZVA in $\Omega$ zusammen $X((\omega_1,\omega_2))=X(\omega_1)$ und $Y((\omega_1,\omega_2))=Y(\omega_2)$; d.h. X,Y werden auf $\Omega$ "fortgesetzt". Dann sind X,Y stochastisch unabhängig in $(\Omega, p)$ (und $p(X=x)=p_1(X=x), p(Y=y)=p_2(Y=y)$).
    
    \paragraph{Erwartungswert, Varianz, Covarianz}
    Sei $X:\Omega\rightarrow \mathbb{R}$ ZVA im Wahrscheinlichkeitsraum $(\Omega, p)$. $E(X)=\sum_{x\in X(\Omega)}x p(X=x)=\sum_{\omega in Omega} X(\omega)p(\omega)$ "E verhält sich wie Integral"; E(x) heißt Erwartungswert von x.
    
    Linearität des Erwartungswertes: $E(x+y)=E(x)+E(y)$ und $E(\alpha x)=\alpha E(x)$.\
    Ist $X:\Omega\rightarrow \mathbb{R}$ konstant gleich c, so ist $E(x)=\sum x*p(X=x)=c*p(X=x)=c*1=c$.\
    Die Varianz von X: $Var(X)=E((X-E(X))^2)$ heißt Varianz von X (um E(X)).\
    Die Covarianz: $Cov(X,Y)=E((X-E(X))*(Y-E(Y)))$ heißt Covarianz von X und Y.\
    Der Verschiebungssatz: $Cov(X,Y)=E(X*Y)-E(X)*E(Y)$\
    $Var(X)=Cov(X,X)=E(X*X)-E(X)E(X)=E(X^2)-(E(X))^2$
    
    Seien X,Y stochastisch unabhängig ($\leftrightarrow p(X=x \wedge Y=y)=p(X=x)*p(Y=y)$)
    $E(X)*E(Y)=\sum_{x\in X(\Omega)} x*p(X=x)* \sum_{y\in Y(\Omega)} y*p(Y=y)=\sum_{x\in X(\Omega)} \sum_{y\in Y(\Omega)} xy*p(X=x)p(Y=y)=\sum_{Z\in\mathbb{R}} z*p(X*Y=Z) = E(X*Y)$. 
    Sind X,Y stochastisch unabhängig ZVA, so ist $E(X)*E(Y)=E(X*Y)$; folglich $Cov(X,Y)=0$
    
    Satz: Seien X,Y ZVA, dann gilt $Var(X+Y)=Var(x)+Var(Y)+2*Cov(X,Y)$. Sind insbesondere X,Y unabhängig gilt: $Var(X+Y)=Var(X)+Var(Y)$.
    
    Sei $(\Omega, p)$ Wahrscheinlichkeitsraum, $X:\Omega\rightarrow \mathbb{R}$ Zufallsvariable heißt Bernoulliverteilt im Parameter p falls $p(X=1)=p$ und $p(X=0)=1-p$, $p\in [0,1]$. $E(X)=\sum x*p(X=x)= 1*p(X=1)=p$
    Für $X:\Omega\rightarrow {0,1}$ ist $X^2=X$: $Var(X)=E(X^2)-E(X)^2 = p-p^2 = p(1-p)=p*q$
    
    \paragraph{Binominalkoeffizienten}
    Sei N eine Menge, dann ist $\binom{N}{k} := (x \subseteq N: \text{x hat genau k Elemente } (|x|=k) )$ für $k\in \mathbb{N}$. Für $n\in \mathbb{N}$ sei $\binom{n}{k}:=|(\binom{1,...,k}{k})$.
    
    Satz: $\binom{n}{0}={n}{n}=1$ f.a. $n\geq 0$ $\binom{n}{k}=\binom{n-1}{k-1}+\binom{n-1}{k}$ f.a. $n\geq 0,k\geq 1, k\geq n-1$
    
    Jede n-elementige Menge N ist $\binom{N}{0}=(\emptyset), \binom{N}{n}={N}\rightarrow \binom{n}{0}=\binom{n}{n}=1$. Den zweiten Teil der Behauptung zeigt man induktiv über n.
    
    \paragraph{Binominalsatz}
    $(a+b)^n = \sum_{k=0}^n a^k b^{n-k}$ für $a,b\in \mathbb{R}$. 
    Für $n\in \mathbb{N}$ sei $n!=n(n-1)(n-2)...*3*2*1=\prod i$; für $n\in\mathbb{N}$ und $k\geq 0$ sei $[\binom{n}{k}]=\frac{n!}{k!(n-k)!}$
    
    Satz: $\binom{n}{0}=\binom{n}{n}=1$ für jedes $n\in\mathbb{N}$, $\binom{n}{k}=\binom{n-1}{k}+\binom{n-1}{k-1}$ für $k\geq 1$ und $k\leq n-1$.
    Zweiter Teil: $[\binom{n-1}{k}]+[\binom{n-1}{k-1}]=\frac{n!}{k!(n-k)!} = [\binom{n}{k}]$. Also stimmen die Rekursionsgleichungen von $\binom{n}{k}$ und $[\binom{n}{k}]$ überein sowie $\binom{n}{k}=[\binom{n}{k}]$. Folglich ist die Anzahl k-elementiger Teilmengen eine n-elementige Menge gleich $\frac{n!}{k!(n-k)!}$.
    
    Seien $X_1,...,X_n$ unabhängige ZVAen, alle $X_i$ seien Bernoulli-Verteilt im Parameter $p[0,1]$, d.h. $p(X_1=1)=p$, $p(X_i=0)=(1-p)$. Dann ist $X_i=X_1+X_2+...+X_n$ ebenfalls reellwertige ZVA. Im Fall $X_i:\Omega\rightarrow {0,1}$ ist $X:\Omega\rightarrow {0,1,...,n}$. Die Verteilung von X ergibt sich wie folgt, für $k\in {0,1,...,n}$: $p(X=k)=\binom{n}{k}*p^k(1-p)^{n-k}$
    
    Eine ZVA heißt binominalverteilt in den Parametern n und p falls gilt: $p(X=k)=\binom{n}{k}p^k (1-p)^{n-k}$ für $k\in{0,1,...,n}$; schreibe $X\sim L(n,p)$. Sonst ist X Bernoulliverteilt (genau dann wenn $X\sim L(1,p)$).
    
    \paragraph{Erwartungswert und Varianz}
    Sei $X\sim L(n,p)$ OBdA $X=X_1,+...+X_n$ wobei $X_i$ unabhängig und Bernoulliverteilt.\
    $E(X)=n*p$, $E(X_i)=p$\
    $Var(X)=n*p*(1-p)$, $Var(X_i)=p*(1-p)$
    
    \paragraph{Multinominalverteilung}
    $\binom{N}{k_1,...,k_n}$ sei Menge der Abbildungen $f:N\rightarrow {1,...,r}$ mit $k1,...,k_r\geq 0$, $k_1+...+k_r=|\mathbb{N}|$ und $f^{-1}[{j}]=k_j \binom{n}{k_1,...,k_r} = |\binom{N}{k_1,...,k_r}$.
    
    \paragraph{Hypergeometrische Verteilung}
    Beispiel: Urne mit zwei Sorten Kugeln; N Gesamtzahl der Kugeln, M Gesamtzahl Kugeln Sorte 1, N-M Gesamtzahl Kugeln Sorte 2, $n\leq N$ Anzahl Elemente einer Stichprobe. X Anzahl der Kugeln Sorte 1 in einer zufälligen n-elementigen Stichprobe.
    $p(X=k)=\frac{\binom{M}{k}\binom{N-M}{n-k}}{\binom{N}{n}}$
    Eine ZVA $X:\Omega\rightarrow \mathbb{R}$ heißt hypergeometrisch Verteilt in den Parametern M,N,n falls $p(X=k)$ für alle $k\geq 0, k\geq M$ gilt.
    
    $E(X)=\sum_{x=0}^M \frac{\binom{M}{k}\binom{N-M}{n-k}}{\binom{N}{n}}=...=n*\frac{M}{N}$
    
    $Var(X)=E(X^2)-E(X)^2 =...= n*\frac{M}{N}(1-\frac{M}{N})(\binom{N-n}{N-1})$
    
    \section{Elementare Graphentheorie}
    $G=(V,E)$ heißt Graph mit Eckenmenge $V(G)=V$ und Kantenmenge $E(G)=E\subseteq {{x,y}:x\not=y \in V}$. Veranschaulichung als Punkte in der Ebene (V) mit "Verknüpfunglinien" von x nach y. Bsp $G=({1,2,3,4},{12,13,14,15,16})$.
    
    $P=x_0,...,x_e$ Folge pw verschiedener Ecken mit $x_{i-1},...,x_i \in E(k)$ für $i\in{1,...,l}$ heißt ein Weg von $x_0$ nach $x_e$ der Länge l. Für $(a,b)\in V(G)$ heißt $d_G(a,b)=min(l: \text{ es gibt einen a,b-Weg der Länge l} )$ Abstand von a nach b. Falls es keinen a,b-Weg gibt, definiere $d_G(a,b)=+\infty$.
    
    $a\sim b \leftrightarrow$ es gibt einen a,b-Weg in G wird eine Äquivalenzrelation auf V(G) definiert. DIe Äquivalenzklassen heißen (Zusammenhangs-) Komponenten von G.
    
    G heißt zusammenhängend, wenn G höchstens eine Komponente besitzt. $d_G: V(G) x V(G) \leftrightarrow \mathbb{R}_{\geq 0}$ ist eine Matrix
    \begin{itemize}
        \item $d_G(x,y)=0 \leftrightarrow x=y$ f.a. $x,y \in V(G)$
        \item $d_G(x,y)=d_G(y,x)$ f.a. $x,y\in V(F)$
        \item $d_G(x,z)\leq d_G(x,y) + d_G(y,z))$ f.a. $x,y,z \in V(G)$
    \end{itemize}
    
    Für $A\subseteq V(G)$ sei $G[A]:= (A, {x,y\in E(G):x,y\in A})$. Für $F\subseteq E(G)$ sei $G[F]:=(V(G), F)$. $G[A]$ bzw $G[F]$ heißt von A bzw F induzierte Teilgraph. Ein Graph H mit $V(H)\subseteq V(G)$ und $E(H)\subseteq E(G)$ heißt Teilgraph von G, schreibweise $H\leq G$. $\leq$ ist Ordnung, denn:
    \begin{itemize}
        \item $G\leq G$
        \item $H\leq G \wedge G\leq H \rightarrow H=G$
        \item $H\leq G \wedge G=L \rightarrow H\leq L$
    \end{itemize}
    
    Ist $P=x_0,...,x_p$ Weg, so heißt auch der Teilgraph ein Weg von $x_0$ nach $x_e$.
    Graphen G, H heißen isomorph, falls es einen Isomorphismus von V(G) nach V(H) gibt. Das heißt eine Bijektion.
    $V(G)\rightarrow V(H)$ mit $f(x)f(y)\in E(H)\leftrightarrow x,y \in E(G)$. Es gilt:
    \begin{itemize}
        \item $G\cong G$
        \item $G\cong H \rightarrow H \cong G$
        \item $G\cong H \wedge H\cong L \rightarrow G\cong L$
    \end{itemize}
    
    Eine Folge $C=x_0,x_1,...,x_{l-1}$ von Ecken mit $x_i,x_{i+1}\in E(G)$ für $i\in {0,...,l-2}$ und $x_{l-1}x_0 \in E(G)$ heißt Kreis in G der Länge l, falls $x_0,...,x_{l-1}$ pw verschieden sind. Bsp: Kreis der Länge 5.
    
    Ein Teilgraph H des Graphen G (also $H\leq G$) heißt aufspannend, falls $V(H)=V(G)$. Für eine Ecke $x\in V(G)$ sei $d_G(x)=|{x,y\in E(G), y\in V(G)}|$ die Anzahl der mit x indizierten Kanten, der sogenannte Grad von x in G.
    
    Weiter $N_G(x):={x\in V(G): xy \in E(G)}$ die Menge der nachbarn von x in G. Hier gilt: $|N_G(x)=d_G(x)|$.
    
    In jedem Graph G gilt $\sum_{x\in V(G)} d_G(x)=2|E(G)|$. Der Durchschnittsgrad von G ist somit $\bar{d(G)}=\frac{1}{|V(G)|}\sum d_G(x)=\frac{2|E(G)|}{|V(G)|}$.
    
    Ein Graph ist ein Baum wenn G zusammenhängend und G-e nicht zusammenhängend für jedes $e\in E(G)$ "G ist minimal zusammenhängend"
    Graph G ist ein Baum wenn G kreisfrei und Graph G+xy nicht kreisfrei für jedes $xy \not\in E(G)$
    G ist Baum, wenn
    \begin{itemize}
        \item G ist kreisfrei und zusammenhängend
        \item G kreisfrei und $|E(G)|=|V(G)|-1$
        \item G zusammenhängend und $|E(G)|=|V(G)|-1$
    \end{itemize}
    
    Jeder Baum mit wenigstens einer Ecke besitzt eine Ecke vom Grad $\leq 1$, ein sog. Blatt ("jeder Baum besitzt ein Blatt").
    $\rightarrow E(G)=|V(G)|-1$ für jeden Baum also $d(G)=\frac{2|V(G)| -2}{|V(G)|}<2$.
    
    G Wald $\leftrightarrow$ die Komponenten von G sind Bäume
    
    G Baum $\leftrightarrow$ G ist zusammenhängender Wald
    
    Ein Teilgraph H von G heißt Teilbaum von G, falls H ein Baum ist. Ein aufspannender Teilbaum von G heißt Spannbaum von G. G zusammenhängend $\leftrightarrow$ G Spannbaum.
    
    Ein Spannbaum T von G heißt Breitensuchbaum von G bei $x\in V(G)$ falls $d_F(z,x)=d_G(z,x)$ f.a. $z\in V(G)$.
    
    Ein Spannbaum T von G heißt Tiefensuchbaum von G bei $x\in V(G)$ falls für jede Kante zy gilt: z liegt auf dem y,x-Weg in T oder y liegt auf dem z,t-Weg in T.
    
    Satz: Sei G zusammenhängender Graph $x\in V(G)$.
    (X) sind $x_0,...,x_{e-1}$ schon gewählt und gibt es ein $+ \in (0,..., e-1)$ so, dass $x_{+}$ einen Nachbarn y in $V(G)\ (x_0,...,x_{e-1} )$, so setze $x_e=y$ und $f(e):=t$; iteriere mit $e+1$ statt e.
    Dann ist $T:=({x_0,...,x_e},{x_j*x_{f(j)}: j\in {1,...,e}})$ ein Spannbaum
    \begin{itemize}
        \item (X) wird in + stets kleinstmöglich gewählt, so ist T ein Breitensuchbaum
        \item wird in (X) + stets größtmöglich gewählt, so ist T ein Tiefensuchbaum
    \end{itemize}
    
    \paragraph{Spannbäume minimaler Gewichte}
    G Graph, $F \subseteq E(G)$ heißt kreisfrei, falls G(F) kreisfrei ist.
    
    Lemma (Austauschlemma für Graphen):
    Seien F, F' zwei kreisfreie Kantenmengen in Graph G und $|F|<|F'|$, dann gibt es ein $e \in F'/F$ so, dass $F\vee {e}$ kreisfrei ist.
    
    G, $\omega:E(G)\rightarrow \mathbb{R}$ (Gewichtsfunktion an den Kanten). Für $F\subseteq E(G)$ sei $\omega (F)=\sum \omega (e)$, speziell $\omega (\emptyset)=0$.
    
    Für einen Teilgraphen H von G sei $\omega (G)=\omega (E(G))$. Ein Spannbaum minimalen Gewichts ist ein Spannbaum T von G mit $\omega (T)\leq \omega (S)$ für jeden Spannbaum S von G.
    
    Satz (Kruskal): Sei G zuständiger Graph, $\omega:E(G)\rightarrow \mathbb{R}$; Setze $F=\emptyset$. Solange es eine Kante $e\in E(G)/F$ gibt so, dass $F \vee (e)$ kreisfrei ist, wähle e mit minimalem Gewicht $\omega(e)$, setzte $F=F\vee {e}$, iterieren. Das Verfahren endet mit einem Spannbaum $T=G(F)$ minimalen Gewichts.
    
    Beweis: Weil G endlich ist endet das Verfahren mit einem maximal kreisfreien Graphen T. Seien $e_1,...,e_n$ die Kanten von T in der Reihenfolge ihres Erscheinens, sei S Spannbaum minimalen Gewichts und $f_1,...,f_m$ die Kanten in Reihenfolge aufsteigenden Gewichts. Angenommen (redactio ad absurdum) $\omega(T)>\omega(S)$. Dann gibt es ein $i\in{1,...,m}$ mit $\omega(e_i)>\omega(f_i)$. Wähle i kleinstmöglich, dann ist $F={e_1,...,e_{i-1}}$ und $F'={f_1,...,f_i}$ kreisfrei. Nach Austauschlemma gibt es ein $f\in F'/F$ so, dass $F\vee {f}$ kreisfrei ist. Also ist f ein Kandidat bei der Auswahl von $e_i$ gewesen, also $\omega(e_i)\leq \omega(f)$ (Fehler!). Folglich ist $\omega(T)\leq \omega(S) \Rightarrow \omega(T)=\omega(S)$ also T und S Spannbaum mit minimalen Gewichten.
    
    \subsection{Das Traveling Salesman Problem}
    G sei Graph (vollständig) auf n Ecken, d.h. $xy\in E(G) \forall x\not =y$ aus V(G) und $\omega*E(G)\rightarrow \mathbb{R}$. Finde aufspannenden Kreis C von G minimalen Gewichts. Zusatzannahme (metrische TSP) $\omega(xz)\leq \omega(xy)+\omega(yz)$. 
    Finde einen aufspannenden Kreis C, der um einen Faktor von höchstens zwei von einem aufspannenden Kreis D minimalen Gewichts abweicht ($\omega(C)\leq 2 \omega(D)$) sog. Approximationsalgorithmus mit Gütefaktor $\leq$.
    
    Konstruiere eine Folge$x_0,...,x_m$ mit der Eigenschaft, dass jede Kante von T genau zweimal zum Übergang benutzt wird, d.h. zu $e\in E(T)$ existieren $i\not = j$ mit $e=x_i x_{i+1}$ und $e=x_j x_{j+1}$ und zu jedem k existieren $e\in E(T)$ mit $e=x_k x_{k+1}$. Das Gewicht dieser Folge sei $\sum \omega(x_i x_{i+1})= 2\omega(T)$.
    
    Eliminiere Mehrfachnennungen in der Folge. Gibt es $i\not= j$ mit $x_j=x_i$ so streiche x aus der Folge. Das Gewicht der neuen Folge ist maximal so groß wie das Gewicht der alten. Durch iteration erhält man einen aufspannenden Kreis mit $\omega(X) \leq 2 \omega(T)$. Sei e Kante von D $\rightarrow D-e=S$ ist aufspanndender Weg $\rightarrow \omega(T) \leq w(D-e) \leq \omega (D)$.
    
    G Graph, $k\geq 0$. Eine Funktion $f:V(G)\rightarrow C$ mit $|C|\leq k$ heißt k-Färbung, falls $f(x)\not = f(y)$ für $xy\in E(G)$. G heißt k-färbbar, falls G eine k-Färbung besitzt. Das kleinste $k\geq 0$ für das G k-färbbar ist heißt dramatische Zahl von G, Bezeichnung $X(G)$.
    
    Satz (Tuga): Sei $k\geq 2$ und G ein Graph ohne Kreise eine Lösung $l\equiv 1 mod k$, dann ist G k-faltbar. G 2-färbbar $\leftrightarrow$ G hat keine Kreise ungerader Länge. Ein Graph heißt bipartit mit den Klassen A,B falls $(x\in A \wedge y\in B)\vee (x\in B \wedge y\in A)$ für alle $xy \in E(G)$ gilt. Genau dann ist G bipartit mit gewissen Klassen A,B wenn G 2-färbbar ist.
    
    Satz (Hall): Sei G bipartit mit Klassen A,B. Dann gilt G hat ein Matching von A $\leftrightarrow |N_G(X)|\leq |X|$ für alle $X\subseteq A$.
    
    Satz: "$\rightarrow$" sei M Matching von A in G $\rightarrow |N_G(X)| \leq N_{G[M]}(X)=|X|$. "$\leftarrow$" Induktiv über $|V(G)|$.
    Ein schneller Zeuge für die Existenz eines Matchings von A im bipartiten Graphen G mit Klassen A,B ist das Matching selbst. Ein schneller Zeuge für die nicht-existenz eines Matchings ist ein $X\subseteq A$ mit $|N_G(X)| < |X|$.
    
    Das Entscheidungsproblem "hat ein bipartiter Graph ein Matching?" ist im NP und zugleich in co-NP. Also ist auch Problem "ist ein Graph 2-färbbar?" in NP und co-NP. Das Problem "ist ein Graph 3-färbbar" ist in NP. Es ist sogar NP-vollständig, d.h. jedes Problem in NP (jedes Entscheidungsproblem mit schnellen Zeugen für Ja) lässt sich in Polynomalzeit in dieses Färbungsproblem überführen.
    
\end{multicols}
\end{document}