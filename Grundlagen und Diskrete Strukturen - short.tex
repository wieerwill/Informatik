\documentclass[10pt,landscape]{article}
\usepackage{multicol}
\usepackage{calc}
\usepackage{ifthen}
\usepackage[landscape]{geometry}
\usepackage{amsmath,amsthm,amsfonts,amssymb}
\usepackage{color,graphicx,overpic}
\usepackage{hyperref}

\pdfinfo{
    /Title (Grundlagen und Diskrete Strukturen - Short Script)
    /Creator (TeX)
    /Producer (pdfTeX 1.40.0)
    /Author (Robert Jeutter)
    /Subject (Grundlagen und Diskrete Strukturen)
}

% This sets page margins to .5 inch if using letter paper, and to 1cm
% if using A4 paper. (This probably isn't strictly necessary.)
% If using another size paper, use default 1cm margins.
\ifthenelse{\lengthtest { \paperwidth = 11in}}
    { \geometry{top=.5in,left=.5in,right=.5in,bottom=.5in} }
    {\ifthenelse{ \lengthtest{ \paperwidth = 297mm}}
        {\geometry{top=1cm,left=1cm,right=1cm,bottom=1cm} }
        {\geometry{top=1cm,left=1cm,right=1cm,bottom=1cm} }
    }

% Turn off header and footer
\pagestyle{empty}

% Redefine section commands to use less space
\makeatletter
\renewcommand{\section}{\@startsection{section}{1}{0mm}%
                                {-1ex plus -.5ex minus -.2ex}%
                                {0.5ex plus .2ex}%x
                                {\normalfont\large\bfseries}}
\renewcommand{\subsection}{\@startsection{subsection}{2}{0mm}%
                                {-1explus -.5ex minus -.2ex}%
                                {0.5ex plus .2ex}%
                                {\normalfont\normalsize\bfseries}}
\renewcommand{\subsubsection}{\@startsection{subsubsection}{3}{0mm}%
                                {-1ex plus -.5ex minus -.2ex}%
                                {1ex plus .2ex}%
                                {\normalfont\small\bfseries}}
\makeatother

% Define BibTeX command
\def\BibTeX{{\rm B\kern-.05em{\sc i\kern-.025em b}\kern-.08em
    T\kern-.1667em\lower.7ex\hbox{E}\kern-.125emX}}

% Don't print section numbers
\setcounter{secnumdepth}{0}


\setlength{\parindent}{0pt}
\setlength{\parskip}{0pt plus 0.5ex}

%My Environments
\newtheorem{example}[section]{Example}
% -----------------------------------------------------------------------

\begin{document}
\raggedright
\footnotesize
\begin{multicols}{3}


% multicol parameters
% These lengths are set only within the two main columns
%\setlength{\columnseprule}{0.25pt}
\setlength{\premulticols}{1pt}
\setlength{\postmulticols}{1pt}
\setlength{\multicolsep}{1pt}
\setlength{\columnsep}{2pt}

\section{Aussagen}
Aussagen sind Sätze die wahr oder falsch sind.
\begin{tabular}{ c | c | c | c | c | c | c }
    p & q & $p\wedge q$ & $p\vee q$ & $\neg q$ & $p\rightarrow q$ & $p\leftrightarrow q$\\
    f & f & f & f & w & w & w \\
    f & w & f & w & w & w & f \\
    w & f & f & w & f & f & f \\
    w & w & w & w & f & w & w \\
\end{tabular}

\begin{description}
    \item[Tautologie] Wahrheitswerteverlauf konstant w
    \item[Kontradiktion] Wahrheitswerteverlauf konstant f
    \item[Äquivalenz] wenn $p\leftrightarrow$ Tautologie ist. $p \equiv q$
    \item[Disjunkt] Zwei Mengen $X\cap Y = \emptyset$ 
    \item[Atom] die bzgl $\leq$ minimalen Elemente von $B /\perp$  
\end{description}

\section{Mengen}
"Eine Menge ist eine Zusammenfassung bestimmter, wohlunterschiedener Objekte unserer Anschauung oder unseres Denkens" ~ Cantor
Von jedem Objekt steht fest, ob es zur Menge gehört oder nicht.

\paragraph{Wunsch 0}
${x \in A: \neg (x=x)}=\emptyset$ die leere Menge

\paragraph{Wunsch 1}
"$x\in y$" x ein Element von y oder nicht.

\paragraph{Wunsch 2}
$B={x\in A:p(x) wahr}$ B aus wahren p(x) aus A

\paragraph{Wunsch 3}
$x=y: \leftrightarrow \forall z:(z\in x \leftrightarrow z\in y)$.

\paragraph{Wunsch 4}
$y: \leftrightarrow \forall z:(z\in x \rightarrow z \in y) [x \subseteq y]$

\paragraph{Teilmengen}
A Teilmenge von B $\leftrightarrow \forall x: (x\in A \rightarrow x \in B):\Rightarrow A\subseteq B$\
A Obermenge von B $\leftrightarrow \forall x: (x\in B \rightarrow x \in A):\Rightarrow A\supseteq B$\
Folglich $A=B \leftrightarrow A\subseteq B \wedge B\subseteq A$\
Schnittmenge von A und B: $A\cap B = {x: x\in A \wedge x\in B}$\
Vereinigungsmenge von A und B: $A\cup B = {x: x\in A \vee x\in B}$\
Seien A,B Mengen, dann sei $A/B:={x\in A: x\not \in B } = A\bigtriangleup B$

\section{Relationen}
Eine Relation von Mengen A nach B ist eine Teilmenge R von AxB.\
$(x,y)\in R:$ x steht in einer Relation R zu y; auch xRy\

\paragraph{binäre Relation}
\begin{itemize}
    \item Allrelation $R:=AxA \subseteq AxA$
    \item Nullrelation $R:=\emptyset \subseteq AxA$
    \item Gleichheitsrelation $R:={(x,y)... x=y}$
    \item $A=R; R:=((x,y)\in \mathbb{R} x \mathbb{R}, x \leq y)$
    \item $A=\mathbb{Z}; R:=\{(x,y)\in \mathbb{Z} x \mathbb{Z}: \text{x ist Teiler von y} \}$ kurz: x|y
\end{itemize}

\paragraph{Eigenschaften von Relationen}
Sei $R\in AxA$ binäre Relation auf A
\begin{itemize}
    \item Reflexiv $\leftrightarrow \text{ xRx } \forall x \in A$
    \item symmetrisch $\leftrightarrow \text{ xRy } \rightarrow \text{ yRx }$
    \item Antisymmetrisch $\leftrightarrow \text{ xRy } \wedge yRx \rightarrow x=y$
    \item Transitiv $\leftrightarrow \text{ xRy } \wedge \text{ yRz } \rightarrow \text{ xRz }$
    \item totale Relation $\leftrightarrow \text{ xRy } \vee \text{ yRx }  \forall x,y \in A$
\end{itemize}
R heißt:
\begin{itemize}
    \item Äquivalenzrelation $\leftrightarrow$ R reflexiv, symmetrisch und transitiv
    \item Ordnung $\leftrightarrow$ R reflexiv, antisymmetrisch und transitiv
    \item Totalordnung $\leftrightarrow$ R Ordnung und total
    \item Quasiordnung $\leftrightarrow$ R reflexiv und transitiv
\end{itemize}

\paragraph{Äqivalenzrelation $\sim$}
Sei $C\wp (A)$. C heißt Partition/Klasse von A, falls gilt:
\begin{itemize}
    \item $\bigcup C=A$ d.h. jedes $x\in A$ liegt in (min) einem $y\in C$
    \item $\emptyset \not \in C$ d.h. jedes $y\in C$ enthält (min) ein Element von A
    \item $x \cap y = \emptyset$ f.a. $x\not \in y$ aus C
\end{itemize}

Ein Graph $G=(V,E)$ ist ein Paar bestehend aus einer Menge V und $E\subseteq (x,y: x \not = y \text{ aus V} )$.
Zu $a,b\in V$ heißt eine Folge $P=x_1,..,x_n$ von paarweise verschiedenen Ebenen mit $a=x_0, b=x_j; x_{j-1},x_i \in E{a*i \in b*j}$ ein a,b-Weg der Länge l oder Weg a nach b. Durch $a\sim b$ gibt es einen a,b-Weg in G, wird eine Äquivalenzrelation auf V definiert, denn:
\begin{itemize}
    \item "$\sim$ reflexiv": es ist $x\sim x$, denn $P=x$ ist ein x,x-Weg in G
    \item "$\sim$ symmetrisch": aus $x\sim y$ folgt, es gibt einen x,y-Weg $\rightarrow$ es gibt einen y,x-Weg $y\sim x$
    \item "$\sim$ transitiv": aus $x\sim y$ und $y\sim x$ folgt, es gibt einen x,y-Weg und einen y,x-Weg
\end{itemize}

\paragraph{(Halb) Ordnungen}
Sei $leq$ eine Ordnung auf X. Sei $A\subseteq X, b\in X$
\begin{itemize}
    \item b minimal in A $\leftrightarrow b\in A$ und $(c\leq b \rightarrow c=b f.a. c\in A)$
    \item b maximal in A $\leftrightarrow b\in A$ und $(b\leq c \rightarrow b=c f.a. c\in A)$
    \item b kleinstes Element in A $\leftrightarrow b\in A$ und $(b\leq c f.a. c\in A)$
    \item b größtes Element in A $\leftrightarrow b\in A$ und $(c\leq b f.a. c\in A)$
    \item b untere Schranke von A $\leftrightarrow b\leq c f.a. c\in A$
    \item b obere Schranke von A $\leftrightarrow c\leq b f.a. c\in A$
    \item b kleinste obere Schranke von A $\leftrightarrow$ b ist kleinstes Element von $(b'\in X: \text{b' obere Schranke von A})$ auch Supremum von A: $\lor A = b$
    \item b größte untere Schranke von A $\leftrightarrow$ b ist das größte Element von $(b'\in X: \text{ b' untere Schranke von A} )$ auch Infinum von A; $\land A = b$
\end{itemize}
kleinstes und größtes Element sind jew. eindeutig bestimmt (falls existent)

\paragraph{Wohlordnungssatz}
Jede Menge lässt sich durch eine Ordnung so ordnen, dass jede nichtleere Teilmenge von X darin ein kleinstes Element ist.

\section{Induktion}
Menge M heißt induktiv $:\leftrightarrow \emptyset \in M \wedge \forall X \in M, \{X^+ \in M\}$.\
Ist O eine Menge von induktiven Mengen, $O\pm O$ dann ist auch $\bigcap O$ induktiv. Insbesondere ist der Durchschnitt zweier induktiver Mengen induktiv.

\paragraph{Induktion I}
Sei $p(n)\in \mathbb{N}$. Gelte $p(0)$ und $p(n)\rightarrow p(n^{+})$ f.a. $n\in \mathbb{N}$ dann ist $p(n)$ wahr f.a. $n \in \mathbb{N}$. 

\paragraph{Induktion II}
Sei $p(n)\in \mathbb{N}$, gelte $\{\forall x < n: p(x)\} \rightarrow p(n)$ f.a. $n\in \mathbb{N}$. Damit ist $p(n)$ wahr für alle $n\in \mathbb{N}$.



\section{Funktionen}
Eine Relation $f\subseteq A x B$ heißt Funktion $f:A\rightarrow B$ ("A nach B") falls es zu jedem $x\in A$ genau ein $y\in B$ mit $(x,y)\in f$ gibt.\
Satz: $f:A\rightarrow B, g:A\rightarrow B$, dann gilt $f=g \leftrightarrow f(x)=g(x)$.\
Sei $f:A\rightarrow B$ Funktion, f heißt:
\begin{itemize}
    \item injektiv $\leftrightarrow$ jedes y aus B hat höchstens ein Urbild $(f(x)=f(y)\rightarrow x=y)$
    \item subjektiv $\leftrightarrow$ jedes y aus B hat wenigstens ein Urbild $f(x)=y$
    \item bijektiv $\leftrightarrow$ jedes y aus B hat genau ein Urbild; injektiv und surjektiv
\end{itemize}
Ist $f:A\rightarrow B$ bijektiv, dann ist auch $f^{-1}\subseteq BxA$ bijektiv, die Umkehrfunktion von f.\
Mit $f:A\rightarrow B$, $g:B\rightarrow C$, wird durch $(g \circ f)(x):=g(f(x))$ eine Funktion $g \circ f: A \rightarrow C$ definiert.\

Satz: ist $f:A\rightarrow B$ bijektiv, so ist $f^{-1}$ eine Funktion B nach A.\
Mengen A,B, heißen gleichmächtig ($|A|=|B| \equiv A\cong B$) falls Bijektion von A nach B.\
Eine Menge A heißt endlich, wenn sie gleichmächtig zu einer natürlichen Zahl ist; sonst heißt A unendlich.\
Eine Menge A heißt Deckend-unendlich, falls es eine Injektion $f:A\rightarrow B$ gibt die nicht surjektiv ist.\
A heißt höchstens so mächtig wie B, falls es eine Injektion von A nach B gibt: $|A|\leq |B|$ bzw $A\preceq B$ (Quasiordnung).

Für zwei Mengen A,B gilt $|A|\leq |B|$ oder $|B| \leq |A|$. Eine Relation f heißt partielle Bijektion (oder Matching), falls es Teilmengen $A'\subseteq A$ und $B'\subseteq B$ gibt sodass f eine Bijektion von A' nach B' gibt.

\paragraph{Kontinuitätshypothese}
Aus $|\mathbb{N}|\leq |A| \leq |\mathbb{R}|$ folgt $|A|=|\mathbb{N}|$ oder $|A|=|\mathbb{R}|$ (keine Zwischengrößen).

\section{Gruppen, Ringe, Körper}
Eine Operation auf eine Menge A ist eine Funktion $f:AxA\rightarrow A$; schreibweise $xfy$. Eine Menge G mit einer Operation $\circ$ auf G heißt Gruppe, falls gilt:
\begin{itemize}
    \item $a\circ (b\circ c) = (a\circ b)\circ c$ freie Auswertungsfolge
    \item es gibt ein neutrales Element $e\in G$ mit $a\circ e=a$ und $e\circ a=a$ f.a. $a\in G$
    \item $\forall a\in G \exists b\in G: \{a\circ b=e\} \vee \{b\circ a=e\}; b=a^{-1}$
\end{itemize}
kommutativ/abelsch, falls neben 1.,2. und 3. außerdem gilt:
\begin{itemize}
    \item $a\circ b = b\circ a$ f.a. $a,b \in G$
\end{itemize}
Eine Bijektion von X nach X heißt Permutation von X. $(S_X, \circ)$ ist eine Gruppe.

Zwei Gruppen $(G, \circ_G)$ und $(H,\circ_H)$ heißen isomorph, falls es einen Isomorphismus $(G,\circ_G)\cong (H,\circ_H)$ von $(G,\circ_G)$ nach $(H,\circ_H)$ gibt.

\paragraph{Addition von $\mathbb{N}$}
$+: \mathbb{N} x \mathbb{N} \rightarrow \mathbb{N}$ wird definiert durch:
\begin{itemize}
    \item $m+0:=m$ f.a. $m\in \mathbb{N}$ (0 ist neutral)
    \item $m+n$ sei schon definiert f.a. $m\in \mathbb{N}$ und $n\in \mathbb{N}$
    \item $m+n^+:=(m+n)^+$ f.a. $m,n \in \mathbb{N}$
\end{itemize}

\paragraph{Multiplikation}
$*: \mathbb{N} x \mathbb{N} \rightarrow \mathbb{N}$ wird definiert durch:
\begin{itemize}
    \item $m*0:=0$ f.a. $m\in \mathbb{N}$
    \item $m*n^+=m*n+m$ f.a. $n\in\mathbb{N}$
\end{itemize}

\paragraph{ganze Zahlen $\mathbb{Z}$}
Durch $(a,b)\sim (c,d)\leftrightarrow a+d=b+c$ wird eine Äquivalenzrelation auf $\mathbb{N} x\mathbb{N}$ definiert.
Die Äquivalenzklassen bzgl $\sim$ heißen ganze Zahlen (Bezeichnung $\mathbb{Z}$. Wir definieren Operationen +, * auf $\mathbb{Z}$ durch:
\begin{itemize}
    \item $[(a,b)]_{/\sim } + [(c,d)]_{/\sim } = [(a+c, b+d)]_{/\sim }$
    \item $[(a,b)]_{/\sim } * [(c,d)]_{/\sim } = [(ac+bd, ad+bc)]_{/\sim }$
\end{itemize}
Satz: $\mathbb{Z}$ ist eine abelsche Gruppe (+ assoziativ, enthält neutrales Element, additiv Invers).

Ein Ring R ist eine Menge mit zwei Operationen $+,*: \mathbb{R} x \mathbb{R} \rightarrow \mathbb{R}$ mit:
\begin{itemize}
    \item $a+(b+c) = (a+b)+c$ f.a. $a,b,c\in \mathbb{R}$
    \item Es gibt ein neutrales Element $O\in \mathbb{R}$ mit $O+a=a+O=O$ f.a. $a\in\mathbb{R}$
    \item zu jedem $a\in \mathbb{R}$ gibt es ein $-a\in \mathbb{R}$ mit $a+(-a)=-a+a=0$
    \item $a+b=b+a$ f.a. $a,b\in\mathbb{R}$
    \item $a*(b*c)=(a*b)*c)$ f.a. $a,b,c\in\mathbb{R}$
    \item $a*(b+c)=a*b+a*c$ f.a. $a,b,c\in\mathbb{R}$
\end{itemize}
R heißt Ring mit 1, falls:
\begin{itemize} 
    \item es gibt ein $1\in\mathbb{R}$ mit $a*1=1*a=a$ f.a. $a\in\mathbb{R}$
\end{itemize}
R heißt kommutativ, falls:
\begin{itemize}
    \item $a*b=b*a$ f.a. $a,b\in\mathbb{R}$
\end{itemize}
Ein kommutativer Ring mit $1\not=O$ heißt Körper, falls:
\begin{itemize}
    \item zu jedem $a\in\mathbb{R}$ gibt es ein $a^{-1}\in\mathbb{R}$ mit $a*a^{-1}=1$
\end{itemize}

\begin{itemize}
    \item Ist $\mathbb{R}$ ein Körper, so ist $\mathbb{R}*=\mathbb{R} /(0)$ mit $*$ eine abelsche Gruppe.
    \item $\mathbb{Z}$ mit + und * ist ein kommutativer Ring mit $1 \not= 0$ aber kein Körper
    \item $\mathbb{Q}, \mathbb{C}, \mathbb{R}$ mit + und * ist ein Körper
\end{itemize}

\paragraph{Zerlegen in primäre Elemente}
Jede ganze Zahl $n>0$ lässt sich bis auf die Reihenfolge der Faktoren eindeutig als Produkt von Primzahlen darstellen.

\paragraph{Konstruktion von rationalen Zahlen aus $\mathbb{Z}$}
Sei $M=\mathbb{Z} x(\mathbb{Z} /{0})$ die Menge von Brüchen. Durch $(a,b)\sim (c,d)\leftrightarrow ad=bc$ wird Äquivalenzrelation auf M durchgeführt. Definiere Operationen +,* auf $\mathbb{Q}$ wie folgt:
\begin{itemize}
    \item $\frac{a}{b}+\frac{c}{d} = \frac{ad+bc}{b*d}$ (wohldefiniert)
    \item $\frac{a}{b}*\frac{c}{d} = \frac{a*c}{b*d}$
\end{itemize}
Satz: $\mathbb{Q}$ mit +,* ist ein Körper.

\paragraph{Ring der formalen Potenzreihe}
Sei k ein Körper. Eine Folge $(a_0, a_1,...,a:n)\in K^{\mathbb{N}}$ mit Einträgen aus K heißt formale Potenzreihe $K[[x]]$. Die Menge aller Polynome wird mit $K[x]$ bezeichnet. $K[[x]]$ wird mit +,* zu einem kommutativen Ring mit $1\not=0$
\begin{itemize}
    \item +: $(a_0,a_1,...) + (b_0,b_1,...) = (a_o+b_0, a_1+b_1, ...)$
    \item *: $(a_0,a_1,...) + (b_0,b_1,...) = (c_0, c_1,...)$ mit $c_K=\sum_{j=a}^{k} a_j*b_{k-j}$
\end{itemize}


B mit $\vee, \wedge, \bar{ }$ seien boolesche Algebren. Sie heißen isomorph, falls es einen Isomorphismus von B nach $\dot{B}$ gibt, d.h. eine Bijektion $\phi: B \rightarrow \dot{B}$ mit:
\begin{itemize}
    \item $\phi(a\vee b) =\phi(a)\dot{\vee}\phi(b)$
    \item $\phi(a\wedge b)=\phi(a)\dot{\wedge}\phi(b)$
    \item $\phi(\bar{a}) = \dot{\bar{\phi(a)}}$
\end{itemize}

Lemma: Sei B mit $\vee, \wedge, \bar{}$ eine boolesche Algebra, dann gilt:
\begin{itemize}
    \item $a\vee T = T$ f.a. $a\in B$
    \item $a\wedge \perp = \perp$ f.a. $a\in B$
    \item $a\vee b$ ist obere Schranke von ${a,b}$, d.h. $a\leq a\vee b$, dann $a\vee(a\vee b)=a\vee b$
    \item $a\vee b$ ist kleinste obere Schranke, d.h. $a\leq z$ und $b\leq z$ folgt $a\vee b \leq z$
\end{itemize}


\section{Diskrete Wahrscheinlichkeitsräume}
Ein Wahrscheinlichkeitsraum ist ein Paar $(\Omega, p)$ bestehend aus einer endlichen Menge $\Omega$ und einer Funktion $p:\Omega \rightarrow [0,1]\in \mathbb{R}$ mit $\sum_{\omega \in \Omega} p(\omega)=1$. Jeder derartige p heißt Verteilung auf $\Omega$. Die Elemente aus $\Omega$ heißen Elementarereignis, eine Teilmenge A von $\Omega$ heißt ein Ereignis; seine Wahrscheinlichkeit ist definiert durch $p(A):=\sum_{\omega in A} p(\omega)$.\
$A=\emptyset$ und jede andere Menge $A\subseteq \Omega$ mit $p(A)=0$ heißt unmöglich (unmögliches Ereignis).\
$A=\Omega$ und jede andere Menge $A\subseteq \Omega$ mit $p(A)=1$ heißt sicher (sicheres Ereignis).\
Es gilt für Ereignisse $A,B,A_1,...,A_k$:
\begin{itemize}
    \item $A\subseteq B \rightarrow p(A)\leq p(B)$
    \item $p(A\cup B) \rightarrow p(A)+p(B)-p(A\cap B)$
    \item disjunkt($A_i \cap A_J=\emptyset$ für $i\not =j$) so gilt $p(A_1 \cup ... \cup A_k)= p(A_1)+...+p(A_k)$
    \item $p(\Omega / A):=$ Gegenereignis von $A=1-p(A)$
    \item $p(A_1,...,A_k) \leq p(A_1)+...+p(A_k)$
\end{itemize}
$(\Omega, p)$ heißt Produktraum von $(\Omega_1, p_1),...$.\
$A,B\in \Omega$ heißen (stochastisch) unabhängig, falls $p(A\cap B) = p(A)*p(B)$.

\paragraph{Bedingte Wahrscheinlichkeiten}
$B\subseteq \Omega$ ("bedingtes Ereignis") mit $p(B)>0$, dann ist $p_B:B\rightarrow [0,1]; p_B(\omega)=\frac{p(\omega)}{p(B)}$ eine Verteilung auf B. Für $A\subseteq \Omega$ gilt $p_B(A\cap B)=\sum p_B(\omega)=\sum\frac{p(\omega)}{p(B)}= \frac{p(A\cap B)}{p(B)}:= p(A|B)$ bedingte Wahrscheinlichkeit von A unter B. $p(A|B)=\frac{p(B|A)*p(A)}{p(B)}$\


\paragraph{Erwartung, Varianz, Covarianz}
Erwartungswert $E(X) = \sum_{\omega \in \Omega} X(\omega)p(\omega)$\
Linearität von E: $E(x+y)=E(x)+E(y)$ und $E(\alpha x)=\alpha E(x)$.\
Varianz von X: $Var(X)=E((X^2)-E(X))^2)$\
Covarianz: $Cov(X,Y)=E((X-E(X))*(Y-E(Y)))$\
Verschiebungssatz: $Cov(X,Y)=E(X*Y)-E(X)*E(Y)$\
$Var(X)=Cov(X,X)=E(X*X)-E(X)E(X)=E(X^2)-(E(X))^2$\
Sind X,Y stochastisch unabhängig ZVA, so ist $E(X)*E(Y)=E(X*Y)$; folglich $Cov(X,Y)=0$\
Bernoulliverteilt falls $p(X=1)=p$ und $p(X=0)=1-p$, $p\in [0,1]$. $E(X)=\sum x*p(X=x)= 1*p(X=1)=p$

\paragraph{Binominalkoeffizienten}
N sei Menge, dann ist $\binom{N}{k} := (x \subseteq N: \text{x hat genau k Elemente } (|x|=k) )$ für $k\in \mathbb{N}$.
$\binom{N}{0}=(\emptyset)$, $\binom{N}{n}={N}\rightarrow \binom{n}{0}=\binom{n}{n}=1$ $\binom{n}{0}=1$, $\binom{n}{k}=\binom{n-1}{k-1}+\binom{n-1}{k}=\frac{n!}{k!(n-k)!}$

\paragraph{Hypergeometrische Verteilung}
Beispiel: Urne mit zwei Sorten Kugeln; N Gesamtzahl der Kugeln, M Gesamtzahl Kugeln Sorte 1, N-M Gesamtzahl Kugeln Sorte 2, $n\leq N$ Anzahl Elemente einer Stichprobe. X Anzahl der Kugeln Sorte 1 in einer zufälligen n-elementigen Stichprobe.
$p(X=k)=\frac{\binom{M}{k}\binom{N-M}{n-k}}{\binom{N}{n}}$\
$E(X)=\sum_{x=0}^M \frac{\binom{M}{k}\binom{N-M}{n-k}}{\binom{N}{n}}=...=n*\frac{M}{N}$\
$Var(X)=E(X^2)-E(X)^2 = n*\frac{M}{N}(1-\frac{M}{N})\binom{N-n}{N-1}$

\section{Elementare Graphentheorie}
$G=(V,E)$ heißt Graph mit Eckenmenge $V(G)=V$ und Kantenmenge $E(G)=E\subseteq {{x,y}:x\not=y \in V}$.\
Für $(a,b)\in V(G)$ heißt $d_G(a,b)=min(l: \text{ es gibt einen a,b-Weg der Länge l} )$ Abstand von a nach b.\
G heißt zusammenhängend, wenn G höchstens eine Komponente besitzt. 
\begin{itemize}
    \item $d_G(x,y)=0 \leftrightarrow x=y$ f.a. $x,y \in V(G)$
    \item $d_G(x,y)=d_G(y,x)$ f.a. $x,y\in V(F)$
    \item $d_G(x,z)\leq d_G(x,y) + d_G(y,z))$ f.a. $x,y,z \in V(G)$
\end{itemize}

$\leq$ ist Ordnung, denn:
\begin{itemize}
    \item $G\leq G$
    \item $H\leq G \wedge G\leq H \rightarrow H=G$
    \item $H\leq G \wedge G=L \rightarrow H\leq L$
\end{itemize}

Ein Teilgraph H des Graphen G heißt aufspannend, falls $V(H)=V(G)$. Weiter $N_G(x):={x\in V(G): xy \in E(G)}$ die Menge der nachbarn von x in G. Hier gilt: $|N_G(x)=d_G(x)|$. In jedem Graph G gilt $\sum_{x\in V(G)} d_G(x)=2|E(G)|$. Der Durchschnittsgrad von G ist somit $\bar{d(G)}=\frac{1}{|V(G)|}\sum d_G(x)=\frac{2|E(G)|}{|V(G)|}$.

Ein Graph ist ein Baum wenn "G ist minimal zusammenhängend und kreisfrei"
\begin{itemize}
    \item G ist kreisfrei und zusammenhängend
    \item G kreisfrei und $|E(G)|=|V(G)|-1$
    \item G zusammenhängend und $|E(G)|=|V(G)|-1$
\end{itemize}
Breitensuchbaum von G falls $d_F(z,x)=d_G(z,x)$ f.a. $z\in V(G)$.\
Tiefensuchbaum von G falls für jede Kante zy gilt: z liegt auf dem y,x-Weg in T oder y liegt auf dem z,t-Weg in T.

Satz: Sei G zusammenhängender Graph $x\in V(G)$.
(X) sind $x_0,...,x_{e-1}$ schon gewählt und gibt es ein $+ \in (0,..., e-1)$ so, dass $x_{+}$ einen Nachbarn y in $V(G)\ (x_0,...,x_{e-1} )$, so setze $x_e=y$ und $f(e):=t$; iteriere mit $e+1$ statt e.
Dann ist $T:=({x_0,...,x_e},{x_j*x_{f(j)}: j\in {1,...,e}})$ ein Spannbaum
\begin{itemize}
    \item f(e) wird in + stets kleinstmöglich gewählt, so ist T ein Breitensuchbaum
    \item f(e) wird in + stets größtmöglich gewählt, so ist T ein Tiefensuchbaum
\end{itemize}

\paragraph{Spannbäume minimaler Gewichte}
Sei G zuständiger Graph, $\omega:E(G)\rightarrow \mathbb{R}$; Setze $F=\emptyset$. Solange es eine Kante $e\in E(G)/F$ gibt so, dass $F \vee (e)$ kreisfrei ist, wähle e mit minimalem Gewicht $\omega(e)$, setzte $F=F\vee {e}$, iterieren. Das Verfahren endet mit einem Spannbaum $T=G(F)$ minimalen Gewichts.

\paragraph{Das Traveling Salesman Problem}
Konstruiere eine Folge$x_0,...,x_m$ mit der Eigenschaft, dass jede Kante von T genau zweimal zum Übergang benutzt wird, d.h. zu $e\in E(T)$ existieren $i\not = j$ mit $e=x_i x_{i+1}$ und $e=x_j x_{j+1}$ und zu jedem k existieren $e\in E(T)$ mit $e=x_k x_{k+1}$. Das Gewicht dieser Folge sei $\sum \omega(x_i x_{i+1})= 2\omega(T)$.
Eliminiere Mehrfachnennungen in der Folge. Durch iteration erhält man einen aufspannenden Kreis mit $\omega(X) \leq 2 \omega(T)$.

\paragraph{Färbung \& bipartit}
Eine Funktion $f:V(G)\rightarrow C$ mit $|C|\leq k$ heißt k-Färbung, falls $f(x)\not = f(y)$ für $xy\in E(G)$.
Ein Graph heißt bipartit mit den Klassen A,B falls $(x\in A \wedge y\in B)\vee (x\in B \wedge y\in A)$. Mit Bipartitheit gilt G hat ein Matching von A $\leftrightarrow |N_G(X)|\leq |X|$ für alle $X\subseteq A$.

\end{multicols}
\end{document}