% das Paket "flashcards" erzeugt Karteikarten zum lernen
% auf der Vorderseite steht das Buzzword oder die Frage
% auf der Rückseite steht die Antwort
% beim ausdrucken auf doppelseitiges Drucken achten
%
\documentclass[avery5371, frame]{flashcards}
\usepackage[utf8]{inputenc}
\usepackage[]{amsmath} 
\usepackage[]{amssymb}
\usepackage{bussproofs} % prooftrees
\usepackage{mdwlist} % less space for lists
\cardfrontstyle{headings}
\begin{document}
%%%%%%%%%%%%%%%%%%%%%%%%%%%%%%%%%%%%%

\begin{flashcard}[Kryptosysteme]{Ein Kryptosystem ist ein Tupel $S=(X,K,Y,e,d)$, wobei}
    \begin{itemize*}
      \item X nicht leere endliche Menge als Klartext
      \item K nicht leere endliche Menge als Schlüssel
      \item Y eine Menge als Chiffretexte
      \item $e:X\times K\rightarrow Y$ Verschlüsselungsfunktion
      \item $d:Y\times K\rightarrow X$ Entschlüsselungsfunktion
    \end{itemize*}
\end{flashcard}

\end{document}