\documentclass[10pt, a4paper]{exam}
\printanswers			    % Comment this line to hide the answers 
\usepackage[utf8]{inputenc}
\usepackage[T1]{fontenc}
\usepackage[ngerman]{babel}
\usepackage{listings}
\usepackage{float}
\usepackage{graphicx}
\usepackage{color}
\usepackage{listings}
\usepackage[dvipsnames]{xcolor}
\usepackage{tabularx}
\usepackage{geometry}
\usepackage{color,graphicx,overpic}
\usepackage{amsmath,amsthm,amsfonts,amssymb}
\usepackage{tabularx}
\usepackage{listings}
\usepackage[many]{tcolorbox}
\usepackage{multicol}
\usepackage{hyperref}
\usepackage{pgfplots}
\usepackage{bussproofs}

\pdfinfo{
    /Title (Kryptographie - Übung)
    /Creator (TeX)
    /Producer (pdfTeX 1.40.0)
    /Author (Robert Jeutter)
    /Subject ()
}
\title{Kryptographie - Übung}
\author{}
\date{}

% Don't print section numbers
\setcounter{secnumdepth}{0}

\newtcolorbox{myboxii}[1][]{
  breakable,
  freelance,
  title=#1,
  colback=white,
  colbacktitle=white,
  coltitle=black,
  fonttitle=\bfseries,
  bottomrule=0pt,
  boxrule=0pt,
  colframe=white,
  overlay unbroken and first={
  \draw[red!75!black,line width=3pt]
    ([xshift=5pt]frame.north west) -- 
    (frame.north west) -- 
    (frame.south west);
  \draw[red!75!black,line width=3pt]
    ([xshift=-5pt]frame.north east) -- 
    (frame.north east) -- 
    (frame.south east);
  },
  overlay unbroken app={
  \draw[red!75!black,line width=3pt,line cap=rect]
    (frame.south west) -- 
    ([xshift=5pt]frame.south west);
  \draw[red!75!black,line width=3pt,line cap=rect]
    (frame.south east) -- 
    ([xshift=-5pt]frame.south east);
  },
  overlay middle and last={
  \draw[red!75!black,line width=3pt]
    (frame.north west) -- 
    (frame.south west);
  \draw[red!75!black,line width=3pt]
    (frame.north east) -- 
    (frame.south east);
  },
  overlay last app={
  \draw[red!75!black,line width=3pt,line cap=rect]
    (frame.south west) --
    ([xshift=5pt]frame.south west);
  \draw[red!75!black,line width=3pt,line cap=rect]
    (frame.south east) --
    ([xshift=-5pt]frame.south east);
  },
}

\begin{document}
\begin{myboxii}[Disclaimer]
  Die Übungen die hier gezeigt werden stammen aus der Vorlesung \textit{Kryptographie}! Für die Korrektheit der Lösungen wird keine Gewähr gegeben.
\end{myboxii}

%##########################################
\begin{questions}
  \question Possibilistisch sichere Kryptosysteme

  Bestimmen Sie alle possibilistisch sicheren Kryptosysteme $S=(X,K,Y,e,d)$ mit $X=\{a,b\}$ und $K=\{1,2\}$ (bis auf das Umbenennen von Chiffretexten).
  \begin{solution}
  \end{solution}

  \question Possibilistische Sicherheit: Eine alternative Definition?
  Beweisen oder widerlegen Sie: Ein Kryptosystem $S=(X,K,Y,e,d)$ ist possibilistisch sicher genau dann, wenn Folgendes gilt: $\forall x\in  X\forall y\in  Y\exists k\in K:d(y,k)=x$.
  \begin{solution}
  \end{solution}

  Bemerkung: Im Gegensatz zur Definition der possibilistischen Sicherheit wird hier eine Aussage über die Entschlüsselungsfunktion gemacht.

  \question Possibilistische Sicherheit bei komponentenweiser Verschlüsselung

  Gegeben seien ein Kryptosystem $S=(X,K,Y,e,d)$ und $l\in\mathbb{N}^+$. Wir können $S$ benutzen, um längere Klartexte (Elemente aus $X^l$) zu verschlüsseln.

  Das Kryptosystem $S'=(X^l,K,Y^l,e',d')$ mit $e'((x_1,...,x^l),k)=(e(x_1,k),...,e(x_l,k))$ verschlüsselt komponentenweise unter Verwendung eines einzigen Schlüssels $k$.
  \begin{parts}
    \part Definieren Sie $d'$ so, dass $S'$ tatsächlich ein Kryptosystem ist.
    \begin{solution}
    \end{solution}
    \part Zeigen Sie, dass $S'$ für $|X|,l\geq 2$ nicht possibilistisch sicher ist. (Dies gilt auch dann, wenn S selber possibilistisch sicher ist!)
    \begin{solution}
    \end{solution}
  \end{parts}

  Das Kryptosystem $S^*=(X^l,K^l,Y^l,e^*,d^*)$ mit $e^*((x_1,...,x_l),(k_1,...,k_l))=(e(x_1,k_1),..., e(x_l,k_l))$ verschlüsselt komponentenweise unter Verwendung mehrerer Schlüssel $k_1,...,k_l$.
  \begin{parts}
    \part Definieren Sie $d^*$ so, dass $S^*$ tatsächlich ein Kryptosystem ist.
    \begin{solution}
    \end{solution}
    \part Zeigen Sie, dass $S^*$ genau dann possibilistisch sicher ist, wenn $S$ possibilistisch sicher ist.
    \begin{solution}
    \end{solution}
  \end{parts}

  Notation: Für eine natürliche Zahl $n\geq 2$ sei $Z_n$ die Menge der Zahlen $\{0,1,...,n-1\}$. Die Addition $+_n$ und Multiplikation $*_n$ auf $Z_n$ sind wie folgt definiert:  $a +_n b =(a+b)\ mod\ n$ und $a *_n b =(a*b)\ mod\ n$, wobei $x\ mod\ n$ der Rest von $x$ bei Division durch $n$ ist.

  \question Verschiebe- und affines Kryptosystem

  Für $n\in N^+$ betrachten wir zwei Kryptosysteme, um Elemente aus $Z_n$ zu verschlüsseln.
  Das Verschiebekryptosystem (Cäsar-Chiffre) mit Parameter $n$ ist gegeben durch $C_n=(Z_n,Z_n,Z_n,e_n,d_n)$ mit $e_n(x,k)=x +_n k$.
  \begin{parts}
    \part Wie muss $d_n$ definiert werden, damit $C_n$ tatsächlich ein Kryptosystem ist?
    \begin{solution}
    \end{solution}
    \part Zeigen Sie, dass $C_n$ possibilistisch sicher ist.
    \begin{solution}
    \end{solution}
  \end{parts}

  Das affine Kryptosystem mit Parameter $n\geq 2$ ist gegeben durch $A_n=(Z_n,A_n\times Z_n,Z_n,e_n',d_n')$ mit $A_n=\{a\in Z_n| ggT(a, n) = 1\}$ und $e_n'(x,(a,b)=a *_n x +_n b$.
  Hinweis: Falls $ggT(a,n)=1$, d.h., $a$ und $n$ teilerfremd sind, dann gilt: Es existert genau ein $b\in A_n\subseteq Z_n\backslash\{0\}$, so dass $a*_n b=b*_n a=1$. Dieses Element $b$ heißt ,,multiplikatives Inverses von a modulo n''.
  \begin{parts}
    \part Definieren Sie $d_n'$ so, dass $A_n$ tatsächlich ein Kryptosystem ist.
    \begin{solution}
    \end{solution}
    \part Zeigen Sie, dass $A_n$ possibilistisch sicher ist.
    \begin{solution}
    \end{solution}
  \end{parts}

\end{questions}
\end{document}
