\documentclass[a4paper,12pt,titlepage]{scrartcl}
\usepackage[sc]{mathpazo} % Schrift - wie Funcky und in PDF zu Fonts beschrieben
\usepackage[T1]{fontenc}
\usepackage[utf8]{inputenc}
\usepackage[a-1b]{pdfx}
\usepackage[ngerman]{babel}
\usepackage[amssymb]{SIunits} 
\usepackage{graphicx} 
\usepackage{subfigure}                         
\usepackage{float}
\usepackage[iso,german]{isodate} %his package provides commands to switch between different date formats
\usepackage{hyperref}
\usepackage{listings}

\usepackage{fancyhdr}
\renewcommand{\headrulewidth}{0.5pt}
\renewcommand{\footrulewidth}{0.5pt}
%Abstand zwischen Absätzen, Zeilenabstände
\voffset26pt 
\parskip6pt
%\parindent1cm  %Rückt erste Zeile eines neuen Absatzes ein
\usepackage{setspace}
\onehalfspacing

\begin{document}
\pagenumbering{roman}
\titlehead
{
    \small
    {
        Technische Universität Ilmenau\\
        Fakulät IA\\
        Institut für Biomedizinische Technik und Informatik\\

        Praktikum Deep Learning in der Biomedizintechnik\\
        WS 2021/22}
}

\title {Versuchsprotokoll}
\author{}
\date{16.12.2021\\*[60pt]}
\maketitle

%Einstellungen zur Kopf- und Fußzeile
\pagestyle{fancy}
\fancyhead[R]{Deep Learning}
\pagenumbering{arabic}
\newpage

\section{Kontrollfragen}
\begin{enumerate}
    \item Erklären Sie die Rechenschritte in einem neuronalen Netz.
          \begin{itemize}
              \item Signaleingang über Axon an Eingang des Neurons
              \item Summe aller Eingänge mti Aktivierungsfunktion (Soma) verrechnen
              \item Ausgabe der Aktivierungsfunktion mit Ausgabefunktion (Axonhügel) berechnen
              \item Ergebnis der Ausgabefunktion auf Ausgang legen 
          \end{itemize}
    \item Nennen Sie drei Aktivierungsfunktionen von neuronalen Netzen.
          \begin{itemize}
              \item Skalarprodukt $\sum_{j=1}^n w_{ij} * x_J$
              \item Sigma-Pi $\sum_{j=1}^n (w_{ij} * \prod_{w=1}^p x_{jw})$
              \item Manhatten $\sum_{j=1}^n |x_j-w_{ji}|$
              \item Euklidische Distanz $\sqrt{\sum_{j=1}^n (x_j-w_{ji})^2}$
              \item Mahalanobis $\sqrt{(x-w_i)^T *C_i^{-1} * (x-w_i)}$ mit $C_i=\frac{1}{n} \sum_{p=1}^N (x^p-w_i)*(x^p-w_i)^T$
              \item Maximum-Distanz: $max_{1\leq j\leq n} |x_j-w_{ij}|$
              \item Minimum-Distanz: $min_{1\leq j\leq n} |x_j-w_{ij}|$
          \end{itemize}
    \item Nennen Sie verschiedene Arten von Layern in neuronalen Netzen.
          \begin{itemize}
              \item Input-Layer: Neuronen, die von der Aussenwelt Signale empfangen
              \item Hidden-Layer: Neuronen, die sich im inneren des neuronalen Netzes befinden und eine interne Repräsentation der Aussenwelt enthalten
              \item Output-Layer: Neuronen, die Signale an die Aussenwelt abgeben
          \end{itemize}
    \item Warum ist es nicht sinnvoll eine lineare Funktion $(y=\alpha x+b)$ als Aktivierungsfunktion in den verdeckten Schichten eines neuronalen Netzes zu verwenden?
          \begin{itemize}
              \item
          \end{itemize}
    \item Was verstehen Sie unter Backpropagation?
          \begin{itemize}
              \item Im Wesentlichen ist Backpropagation ein Algorithmus, der zur schnellen Berechnung von Ableitungen verwendet wird
              \item auch Fehlerrückführung oder Rückwärtspropagierung
              \item um einen Gradientenabstieg in Bezug auf Gewichtungen zu berechnen
              \item gewünschte Ausgaben werden mit erreichten Systemausgaben verglichen, und dann werden die Systeme durch Anpassung der Verbindungsgewichte so eingestellt, dass der Unterschied zwischen den beiden so gering wie möglich ist
              \item der Algorithmus hat seinen Namen daher, dass die Gewichtungen rückwärts aktualisiert werden, von der Ausgabe zur Eingabe
          \end{itemize}
    \item Warum ist eine Stufenfunktion (Rosenblatt-Perceptron) ungünstig für den Backpropagation-Algorithmus?
          \begin{itemize}
              \item Anpassung schlecht möglich
          \end{itemize}
    \item Was ist die Learning Rate? Was passiert, wenn sie zu hoch oder niedrig gewählt wird?
          \begin{itemize}
              \item Sein Wert bestimmt, wie schnell das Neuronale Netz zu Minima konvergieren würde
              \item wenn er zu niedrig ist, ist der Konvergenzprozess sehr langsam
              \item wenn er zu hoch ist, ist die Konvergenz schnell, aber es besteht die Möglichkeit, dass der Verlust überschreitet
          \end{itemize}
    \item Was verstehen Sie unter Augmentation? Nennen Sie Beispiele für Augmentation.
          \begin{itemize}
              \item mit Hilfe von verschiedenen Prozessen die Originalbilddaten verändern
              \item Beispielsweise kann ein Bild gedreht werden oder es ist möglich einen Filter über das Bild zu legen
          \end{itemize}
    \item Warum ist es bei neuronalen Netzen besonders wichtig, die Testdaten beim Training außen vor zu lassen?
          \begin{itemize}
              \item Um einen korrekten Vergleich mit anderen Netzen und unabhängigkeit vom Training nachzuweisen
          \end{itemize}
    \item Wie können Sie die Güte eines neuronalen Netzes bewerten?
          \begin{itemize}
              \item 
          \end{itemize}
    \item Warum ist es potentiell kritisch, wenn mit einem neuronalen Netz ein unscharfes Bild scharf und hochaufgelöst gemacht wird?
          \begin{itemize}
              \item kleine fehlerhafte oder ungenaue Bild-Sektionen können zu größeren Fehlern und Abweichungen von dem Original führen die durch die Hochauflösung überdeckt oder verschoben werden
          \end{itemize}
\end{enumerate}

\section{Versuchsdurchführung}
\subsection{Grundkenntnisse zur Anwendung von Deep Learning}
\subsubsection{Erstellen eines einfachen neuronalen Netzes}
Berechnung der Parameter eines neuronalen Netzes per Hand

Erstellen und Anwenden eines neuronalen Netzes in Python

\subsubsection{Neuronales Netz zur Funktionsapproximation}
Approximieren eines QRS-Komplexes 6 durch manuelles und automatisches Setzen der Parameter

\subsubsection{Neuronales Netz zur Erkennung handschriftlicher Ziffern}
Erweiterung des neuronalen Netzes aus Aufgabe 1a zur Verarbeitung des MNIST-Datensatzes

\subsection{Anwendung von Deep Learning in der Biomedizintechnik}
\subsubsection{Data Sanitization mit Hilfe von Pandas}
Nutzen der Pandas-Bibliothek zur Vorverarbeitung von Daten für das Training eines neuronalen Netzes

\subsubsection{Neuronales Netz zur Klassifikation von OCT-Aufnahmen}
Manuelle und automatische Klassifikation von OCT-Aufnahmen

Optimierung der Accuracy des neuronalen Netzes durch Nutzung von Data Augmentation

\subsubsection{Aufgabe 3: Grenzen von Deep Learning}
Kennenlernen, worin die Herausforderung bei der Interpretation von neuronalen Netzen besteht

Anwendung von Class Activation Maps

\end{document}

