\documentclass[a4paper,12pt,titlepage]{scrartcl}
\usepackage[sc]{mathpazo} % Schrift - wie Funcky und in PDF zu Fonts beschrieben
\usepackage[T1]{fontenc}
\usepackage[utf8]{inputenc}
\usepackage[a-1b]{pdfx}
\usepackage[ngerman]{babel}
\usepackage[amssymb]{SIunits} 
\usepackage{graphicx} 
\usepackage{subfigure}                         
\usepackage{float}
\usepackage[iso,german]{isodate} %his package provides commands to switch between different date formats
\usepackage{hyperref}
\usepackage{listings}

\usepackage{fancyhdr}
\renewcommand{\headrulewidth}{0.5pt}
\renewcommand{\footrulewidth}{0.5pt}
%Abstand zwischen Absätzen, Zeilenabstände
\voffset26pt 
\parskip6pt
%\parindent1cm  %Rückt erste Zeile eines neuen Absatzes ein
\usepackage{setspace}
\onehalfspacing

\begin{document}
\pagenumbering{roman}
\titlehead
{
    \small
    {
        Technische Universität Ilmenau\\
        Fakulät IA\\
        Institut für Biomedizinische Technik und Informatik\\

        Praktikum Deep Learning in der Biomedizintechnik\\
        WS 2021/22}
}

\title {Versuchsprotokoll}
\author{}
\date{\today\\*[60pt]}
\maketitle  %Erstellt das Titelblatt wie oben definiert

%Einstellungen zur Kopf- und Fußzeile
\pagestyle{fancy}
\fancyhead[R]{Deep Learning}
\pagenumbering{arabic}
\newpage

\section{Kontrollfragen}
\begin{itemize}
    \item Erklären Sie die Rechenschritte in einem neuronalen Netz.
    \item Nennen Sie drei Aktivierungsfunktionen von neuronalen Netzen.
    \item Nennen Sie verschiedene Arten von Layern in neuronalen Netzen.
    \item Warum ist es nicht sinnvoll eine lineare Funktion $(y=\alpha x+b)$ als Aktivierungsfunktion in den verdeckten Schichten eines neuronalen Netzes zu verwenden?
    \item Was verstehen Sie unter Backpropagation?
    \item Warum ist eine Stufenfunktion (Rosenblatt-Perceptron) ungünstig für den Backpropagation-Algorithmus?
    \item Was ist die Learning Rate? Was passiert, wenn sie zu hoch oder niedrig gewählt wird?
    \item Was verstehen Sie unter Augmentation? Nennen Sie Beispiele für Augmentation.
    \item Warum ist es bei neuronalen Netzen besonders wichtig, die Testdaten beim Training außen vor zu lassen?
    \item Wie können Sie die Güte eines neuronalen Netzes bewerten?
    \item Warum ist es potentiell kritisch, wenn mit einem neuronalen Netz ein unscharfes Bild scharf und hochaufgelöst gemacht wird?
\end{itemize}

\section{Versuchsdurchführung}
\subsection{Grundkenntnisse zur Anwendung von Deep Learning}
\subsubsection{Erstellen eines einfachen neuronalen Netzes}
Berechnung der Parameter eines neuronalen Netzes per Hand

Erstellen und Anwenden eines neuronalen Netzes in Python

\subsubsection{Neuronales Netz zur Funktionsapproximation}
Approximieren eines QRS-Komplexes 6 durch manuelles und automatisches Setzen der Parameter

\subsubsection{Neuronales Netz zur Erkennung handschriftlicher Ziffern}
Erweiterung des neuronalen Netzes aus Aufgabe 1a zur Verarbeitung des MNIST-Datensatzes

\subsection{Anwendung von Deep Learning in der Biomedizintechnik}
\subsubsection{Data Sanitization mit Hilfe von Pandas}
Nutzen der Pandas-Bibliothek zur Vorverarbeitung von Daten für das Training eines neuronalen Netzes

\subsubsection{Neuronales Netz zur Klassifikation von OCT-Aufnahmen}
Manuelle und automatische Klassifikation von OCT-Aufnahmen

Optimierung der Accuracy des neuronalen Netzes durch Nutzung von Data Augmentation

\subsubsection{Aufgabe 3: Grenzen von Deep Learning}
Kennenlernen, worin die Herausforderung bei der Interpretation von neuronalen Netzen besteht

Anwendung von Class Activation Maps

\end{document}

