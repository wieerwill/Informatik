\documentclass[a4paper,12pt,titlepage]{scrartcl}
\usepackage[sc]{mathpazo} % Schrift - wie Funcky und in PDF zu Fonts beschrieben
\usepackage[T1]{fontenc}
\usepackage[utf8]{inputenc}
\usepackage[a-1b]{pdfx}
\usepackage[ngerman]{babel}
\usepackage[amssymb]{SIunits} 
\usepackage{graphicx} 
\usepackage{subfigure}                         
\usepackage{float}
\usepackage[iso,german]{isodate} %his package provides commands to switch between different date formats
\usepackage{hyperref}
\usepackage{listings}

\usepackage{fancyhdr}
\renewcommand{\headrulewidth}{0.5pt}
\renewcommand{\footrulewidth}{0.5pt}
%Abstand zwischen Absätzen, Zeilenabstände
\voffset26pt 
\parskip6pt
%\parindent1cm  %Rückt erste Zeile eines neuen Absatzes ein
\usepackage{setspace}
\onehalfspacing

\begin{document}
\pagenumbering{roman}
\titlehead
{
    \small
    {
        Technische Universität Ilmenau\\
        Fakulät IA\\
        Fachgebiet Biosignalverarbeitung\\

        Praktikum EKG-Signalanalyse\\
        WS 2021/22}
}

\title {Versuchsprotokoll}
\subtitle{EKG-Signalanalyse}
\author{}
\date{\today\\*[60pt]}
\maketitle  %Erstellt das Titelblatt wie oben definiert

%Einstellungen zur Kopf- und Fußzeile
\pagestyle{fancy}
\fancyhead[R]{Praktikumsbericht: EKG-Signalanalyse}
\pagenumbering{arabic}
\newpage

\section{Vorbereitungsaufgaben}
\subsection{EKG-Vorverarbeitung}
Entwickeln Sie eine Strategie zur EKG-Vorverarbeitung (Filterung)! Bedenken Sie dabei, dass die EKG-Vorverarbeitung maßgeblichen Einfluss auf die Qualität der QRS-Detektion hat.

\subsection{QRS-Detektion}
\subsubsection{}
Entwickeln Sie einen Algorithmus zur adaptiven QRS-Detektion, von dem Sie ein möglichst gutes Detektionsergebnis erwarten! Achten Sie dabei besonders auf dessen prinzipielle Online-Fähigkeit. 250 ms Zeitvorlauf sollen dabei nicht überschritten werden, d.h. zur Entscheidungsfindung, ob ein Abtastwert zum Zeitpunkt t eine R-Zacke darstellt oder nicht, können maximal die Abtastwerte der nächsten 250 ms einbezogen werden. Notieren und erklären Sie die verwendeten Operatoren/Formeln und beschreiben Sie den Block der Entscheidungsfindung in einem Programmablaufplan bzw. Entscheidungsbaum!

\subsubsection{}
Entwickeln Sie zu diesem Algorithmus die zugehörige(n) MATLAB-Funktion(en) und bringen Sie den Quelltext schriftlich zum Praktikum mit! Benutzen Sie folgende Funktionsschnittstelle:
\begin{lstlisting}[basicstyle=\tiny]
function [R_Positionen, Entscheidungssignal, Schwellwertverlauf, Lernphase] = QRS_Detektion (EKG_Signal, fa);
\end{lstlisting}

\section{Praktikumsaufgaben}
\subsection{EKG-Ableitung mit Hilfe des Biosignalverstärkers ,,g.BSamp'' und ,,g.ECGbox''}
Leiten Sie jeweils ein 5 minütiges EKG eines Studenten innerhalb der folgenden vier Phasen ab. Achten Sie dabei auf die Auswahl der Kanäle!
\begin{itemize}
    \item Lagetyp-Phase: Proband liegt und atmet normal (alle Kanäle)
    \item Ruhe-Phase: Proband liegt und atmet normal (Kanal mit größter R-Zacke)
    \item RESP-Phase: Proband liegt und atmet langsam tief ein und tief aus (Kanal mit größter R- Zacke)
    \item STEH-Phase: Proband steht und atmet normal (Kanal mit größter R-Zacke)
\end{itemize}

\subsection{Bestimmung der elektrischen Herzachse}
Mit Hilfe des Matlab-GUI „Datenanzeigen“ können die abgeleiteten Signale dargestellt und ausgewertet werden. Für die Bestimmung des Lagetyps gehen Sie wie folgt vor:
\begin{itemize}
    \item Laden Sie die abgeleiteten Daten und filtern Sie diese, falls nötig
    \item Wählen Sie einen Artefakt-freien Signalabschnitt
    \item Suchen Sie die höchste der R-Zacke
    \item Bestimmen Sie die Amplitude der Zacken zu einem Zeitpunkt in den Ableitungen I, II und III,benutzen Sie dazu den „Data-Cursor“ von Matlab.
    \item Tragen Sie die Funktionswerte in das vorgefertigte Protokoll ein und bilden Sie von mindestens zwei Vektoren graphisch den Summenvektor
    \item Mit Hilfe des Cabrera-Kreises können Sie nun den Lagetyp bestimmen.
\end{itemize}

\subsection{EKG-Vorverarbeitung}
Unter dem Menü-Punkt EKG-Vorverarbeitung stehen Ihnen Methoden zur Signal-Vorverarbeitung (Filterung) bereits fertig zur Verfügung.
Überprüfen Sie die Wirksamkeit der von Ihnen vorgeschlagenen Vorverarbeitungsmethoden und deren Parametereinstellungen anhand der EKG-Signale der MIT-Datenbank: 100, 106, 107, 208 und 222 visuell! Überlagern Sie die Signale mit einem Drift und einer 50 Hz-Sinusschwingung!

\subsection{QRS-Detektion mit Hilfe von MATLAB}
\subsubsection{}
Implementieren Sie Ihren entwickelten Algorithmus zur QRS-Detektion in die Funktion QRS\_Detektion.m

\subsubsection{}
Evaluieren Sie Ihren QRS-Detektor anhand der folgenden EKG-Signale der MIT-Datenbank: 100, 106, 107, 208 und 222! Notieren Sie die Detektionsquote! Wo liegen die Stärken bzw. die Schwächen Ihres Detektors?

\subsubsection{}
Versuchen Sie anhand der Ergebnisse dieses ersten Detektionstests den Detektionsalgorithmus bzw. (falls angebracht) auch Ihre EKG-Vorverarbeitung zu optimieren! Wiederholen Sie den Detektionstest! Inwieweit konnten die Detektionseigenschaften verbessert werden?

\subsection{Analyse der Herzfrequenzvariabilität}
\subsubsection{}
Analysieren Sie die aufgezeichneten EKG-Signale während der drei Phasen (RUHE, RESP und STEH)!

Interpretieren Sie die Unterschiede in den Ergebnissen der einzelnen Phasen der HRV-Analyse!

\subsubsection{}
Vergleichen Sie die HRV-Ergebnisse des Praktikumsprobanden mit den Ergebnissen eines Polyneuropathie-Patienten (Datei: zwickau.dat).

\end{document}

