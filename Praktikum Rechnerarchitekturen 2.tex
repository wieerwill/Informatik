\documentclass[a4paper,12pt,titlepage]{scrartcl}
\usepackage[sc]{mathpazo} % Schrift - wie Funcky und in PDF zu Fonts beschrieben
\usepackage[T1]{fontenc}
\usepackage[utf8]{inputenc}
\usepackage[a-1b]{pdfx}
\usepackage[ngerman]{babel}
\usepackage[amssymb]{SIunits} 
\usepackage{graphicx} 
\usepackage{subfigure}                         
\usepackage{float}
\usepackage[iso,german]{isodate} %his package provides commands to switch between different date formats
\usepackage{hyperref}
\usepackage{listings}

\usepackage{fancyhdr}
\renewcommand{\headrulewidth}{0.5pt}
\renewcommand{\footrulewidth}{0.5pt}
%Abstand zwischen Absätzen, Zeilenabstände
\voffset26pt 
\parskip6pt
%\parindent1cm  %Rückt erste Zeile eines neuen Absatzes ein
\usepackage{setspace}
\onehalfspacing

\begin{document}
\pagenumbering{roman}
\titlehead
{
    \small
    {
        Technische Universität Ilmenau\\
        Fakulät IA\\
        Fachgebiet Rechnerarchitektur\\

        Praktikum Rechnerarchitektur 2\\
        WS 2021/22}
}

\title {Versuchsprotokoll}
\subtitle{Versuche Befehlsausführung und Mikrocontroller}
\author{}
\date{\today\\*[60pt]}
\maketitle  %Erstellt das Titelblatt wie oben definiert

%Einstellungen zur Kopf- und Fußzeile
\pagestyle{fancy}
\fancyhead[R]{Praktikumsbericht: RA2}
\pagenumbering{arabic}
\newpage

\section{Versuch B: Befehlsausführung}
Simulative Untersuchung der Ausführung von Maschinenbefehlen in unterschiedlichen Pipeline-Architekturen  

\section{Versuch M: Mikrocontroller}
Assemblerprogrammierung mit dem 8-Bit-Mikrocontroller ATtiny25  

\end{document}

