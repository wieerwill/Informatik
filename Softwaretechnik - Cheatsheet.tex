\documentclass[a4paper]{article}
\usepackage[ngerman]{babel}
\usepackage{multicol}
\usepackage{calc}
\usepackage{ifthen}
\usepackage[landscape,left=1cm,top=1cm,right=1cm,nohead,nofoot]{geometry}
\usepackage{amsmath,amsthm,amsfonts,amssymb}
\usepackage{color,graphicx,overpic}
\usepackage{listings}
\usepackage[compact]{titlesec} %less space for headers
\usepackage{mdwlist} %less space for lists
\usepackage[utf8]{inputenc}
\usepackage{tikz}
\usepackage{pdflscape}
\usepackage{verbatim}
\usetikzlibrary{mindmap, arrows,shapes,positioning,shadows,trees}
\tikzstyle{every node}=[draw=black,thin,anchor=west, minimum height=2em]
\usepackage[hidelinks,pdfencoding=auto]{hyperref}

\pdfinfo{
    /Title (Softwaretechnik - Cheatsheet)
    /Creator (TeX)
    /Producer (pdfTeX 1.40.0)
    /Author (Robert Jeutter)
    /Subject ()
}
% Information boxes
\newcommand*{\info}[4][16.3]{
  \node [ annotation, #3, scale=0.65, text width = #1em, inner sep = 2mm ] at (#2) {
  \list{$\bullet$}{\topsep=0pt\itemsep=0pt\parsep=0pt
    \parskip=0pt\labelwidth=8pt\leftmargin=8pt
    \itemindent=0pt\labelsep=2pt}
    #4
  \endlist
  };
}

% This sets page margins to .5 inch if using letter paper, and to 1cm
% if using A4 paper. (This probably isn't strictly necessary.)
% If using another size paper, use default 1cm margins.
\ifthenelse{\lengthtest { \paperwidth = 11in}}
    { \geometry{top=.5in,left=.5in,right=.5in,bottom=.5in} }
    {\ifthenelse{ \lengthtest{ \paperwidth = 297mm}}
        {\geometry{top=1cm,left=1cm,right=1cm,bottom=1cm} }
        {\geometry{top=1cm,left=1cm,right=1cm,bottom=1cm} }
    }

% Redefine section commands to use less space
\makeatletter
\renewcommand{\section}{\@startsection{section}{1}{0mm}%
                                {-1ex plus -.5ex minus -.2ex}%
                                {0.5ex plus .2ex}%x
                                {\normalfont\large\bfseries}}
\renewcommand{\subsection}{\@startsection{subsection}{2}{0mm}%
                                {-1explus -.5ex minus -.2ex}%
                                {0.5ex plus .2ex}%
                                {\normalfont\normalsize\bfseries}}
\renewcommand{\subsubsection}{\@startsection{subsubsection}{3}{0mm}%
                                {-1ex plus -.5ex minus -.2ex}%
                                {1ex plus .2ex}%
                                {\normalfont\small\bfseries}}
\makeatother

% Define BibTeX command
\def\BibTeX{{\rm B\kern-.05em{\sc i\kern-.025em b}\kern-.08em
    T\kern-.1667em\lower.7ex\hbox{E}\kern-.125emX}}

% Don't print section numbers
\setcounter{secnumdepth}{0}

\setlength{\parindent}{0pt}
\setlength{\parskip}{0pt plus 0.5ex}    
% compress space
\setlength\abovedisplayskip{0pt}
\setlength{\parskip}{0pt}
\setlength{\parsep}{0pt}
\setlength{\topskip}{0pt}
\setlength{\topsep}{0pt}
\setlength{\partopsep}{0pt}
\linespread{0.5}
\titlespacing{\section}{0pt}{*0}{*0}
\titlespacing{\subsection}{0pt}{*0}{*0}
\titlespacing{\subsubsection}{0pt}{*0}{*0}

% Turn off header and footer
\pagestyle{empty}
\begin{document}

\raggedright
\begin{multicols}{3}\scriptsize
  % multicol parameters
  % These lengths are set only within the two main columns
  %\setlength{\columnseprule}{0.25pt}
  \setlength{\premulticols}{1pt}
  \setlength{\postmulticols}{1pt}
  \setlength{\multicolsep}{1pt}
  \setlength{\columnsep}{2pt}

  Wie kann man Software besser entwickeln?
  \begin{itemize*}
    \item Ingenieursmäßige Herangehensweise
    \begin{itemize*}
      \item Verwendung bekannter Prinzipien und Methoden
      \item Systematische Vorgehensweise
    \end{itemize*}
    \item Verwendung von:
    \begin{itemize*}
      \item Abstraktion, Modelle, Notation, Simulation
      \item Wiederverwendung:Muster, Komponenten, Framework
    \end{itemize*}
    \item Organisation
    \begin{itemize*}
      \item Arbeitsteilung, Integration, Planung
    \end{itemize*}
    \item Verwendung von Werkzeugen
    \begin{itemize*}
      \item IDE (Integrated Development Environment)
      \item Versionierung, Bugtracker, Modellierungswerkzeug
    \end{itemize*}
  \end{itemize*}

  \section{Modellierungskonzepte}
  Modell: ist eine Abstraktion eines Systems mit der Zielsetzung, das Nachdenken über ein System zu vereinfachen, indem irrelevante Details ausgelassen werden

  \begin{itemize*}
    \item erstellen einer Abstraktion
    \item abbilden signifikanter Eigenschaften
    \item Deskriptiv/präskriptiv (real oder geplant)
    \item Sichtweise auf ein System (Struktur, Verhalten, Zustand,...)
    \item setzt Verstehen voraus
    \item ist nicht automatisierbar
  \end{itemize*}

  \subsection{Objektorientierung}
  \begin{itemize*}
    \item Grundprinzip: Teile und Herrsche
    \item ein System besteht aus vielen Objekten, mit:
    \begin{itemize*}
      \item definiertes Verhalten
      \begin{itemize*}
        \item Menge genau definierter Operationen
        \item Operation beim Empfang einer Nachricht ausgeführt
      \end{itemize*}
      \item inneren Zustand
      \begin{itemize*}
        \item Zustand des Objekts ist Privatsache
        \item Resultat hängt vom aktuellen Zustand ab
      \end{itemize*}
      \item eindeutige Identität
      \begin{itemize*}
        \item Identität ist unabhängig von anderen Eigenschaften
        \item verschiedene Objekte mit identischem Verhalten im gleichen System möglich
      \end{itemize*}
    \end{itemize*}
    \item Klasse
    \begin{itemize*}
      \item Gleichartige Objekte mit ggf. verschiedenen Zuständen
      \item Verhaltensschema - Operationen
      \item Innere Struktur - Attribute
    \end{itemize*}
    \item Vorteile
    \begin{description*}
      \item[Zuständigkeitsbereiche] Daten, Operationen und Zustand: lokal und gekapselt
      \item[Klare Schnittstellen] Definiertes Objektverhalten, Nachrichten
      \item[Hierarchie] Vererbung und Polymorphie (Spezialisierung), Klassenschachtelung
      \item[Baukastenprinzip] Benutzung vorgefertigter Klassenbibliotheken, Anpassung durch Spezialisierung
    \end{description*}
  \end{itemize*}

  \subsection{Unified Modeling Language}
  \begin{itemize*}
    \item Grafisches Beschreibungsmittel für Aspekte des Softwareentwurfs diskreter Systeme
    \item Kombination von Struktur-, Verhaltens-, Interaktions-, und Verteilungsmodellen
    \item Für Analyse, Entwurf, Implementierung und Test einsetzbar
    \item Gute Werkzeugunterstützung für Editieren, Versionierung, Codegenerierung
    \item Erweiterbarkeit der UML mit Stereotypen und Tags
  \end{itemize*}

  Nachteile UML
  \begin{itemize*}
    \item UML ist in vielen Facetten nicht präzise festgelegt
    \item Werkzeuge für Transformation, Analyse etc. fehlen noch
    \item UML ist keine „kleine Sprache“: Lernaufwand notwendig
    \item Komponenten sind nicht adäquat darstellbar
  \end{itemize*}

  \subsection{Klassische Modelle}
  \begin{itemize*}
    \item Funktionen
    \begin{itemize*}
      \item Funktionsbaum
      \item Blockschaltbild
    \end{itemize*}
    \item Daten
    \begin{itemize*}
      \item Data Dictionary (Daten mit Strukturinformationen)
      \item Entity Relationship Diagram
    \end{itemize*}
    \item Systemumgebung (Datenflussdiagramm)
    \item Algorithmen
    \begin{itemize*}
      \item Entscheidungstabelle
      \item Pseudocode
      \item Programmablaufplan
      \item Struktogramm
    \end{itemize*}
    \item Dynamisches Verhalten (diskrete Zustände und atomare zustandübergänge)
    \begin{itemize*}
      \item Zustandsautomat
      \item Flow-Chart
      \item Ereignisgesteuerte Prozesskette (EPK)
      \item Petri-Netz (ggf. mit Zeitmodell)
    \end{itemize*}
    \item Objektorientierte Modelle
    \begin{itemize*}
      \item Klassendiagramme
      \item UML
    \end{itemize*}
    \item Struktur-Diagramme
    \begin{itemize*}
      \item Klassen-, Objekt-, Komponenten-, Kompositions-Struktur-,
      \item Paket- und Verteilungsdiagramm
    \end{itemize*}
    \item Verhaltens-Diagramme
    \begin{itemize*}
      \item Use-Case-, Aktivitäts- und Zustandsdiagramms
      \item Interaktionsdiagramme: Sequenz-, Kommunikations-, Timing- und Interaktionsübersichts-Diagramm
    \end{itemize*}
  \end{itemize*}

  \paragraph{Use-Case-Diagramm}
  \begin{itemize*}
    \item Beschreiben Systemfunktion aus Benutzersicht (Was, nicht Wie)
    \item Erste Anforderungsspezifikation (requirements)
    \item Planbare Einheiten als Inkremente für die Entwicklung
    \item Keine Modellierung eines Ablaufs!
    \item Erstellen von Testfällen (test case generation)
  \end{itemize*}

  \paragraph{Klassendiagramm}
  \begin{itemize*}
    \item Modellierung der Struktur (Aufbau) eines Systems
    \item Modellierung von statischen Aspekten und Struktur von Daten
    \item Klasse im Mittelpunkt (Attribute, Operationen, Beziehung)
  \end{itemize*}

  \paragraph{Objektdiagramm}
  \begin{itemize*}
    \item Struktur des Systems zur Laufzeit zu einem Zeitpunkt
    \item detaillierte Sicht auf einen Aspekt
    \item Keine Exemplare von Operationen
    \item Kein Verlauf der Wertebelegung über die Zeit
  \end{itemize*}

  \paragraph{Paketdiagramm}
  \begin{itemize*}
    \item Gliederung des Systems in Teile (Pakete)
    \item Zuordnung von Elementen zu einem Paket
    \item Bildung von Hierarchien (Enthält-Beziehung)
    \item Abhängigkeiten zwischen den Paketen
    \item Zum Grobentwurf von Systemen
  \end{itemize*}

  \paragraph{Komponentendiagramm}
  \begin{itemize*}
    \item Strukturierung des Systems durch Komponenten
    \item Komponente: Modulare, austauschbare Einheit (Substitution)
    \item Modellierung der Abhängigkeiten zwischen Komponenten
    \item Modellierung der inneren Struktur von Komponenten
    \item Definition von Schnittstellen
  \end{itemize*}

  \paragraph{Kompositionsstrukturdiagramm}
  \begin{itemize*}
    \item Teile-Ganzes-Strukturen $\rightarrow$ Kompositionsstruktur
    \item Strukturell statische Kompositionsstrukturen
    \item Strukturell dynamische Kompositionsstrukturen
    \item Spezialisierte Kompositionsbeziehung $\rightarrow$ erweiterte Semantik
  \end{itemize*}

  \paragraph{Aktivitätsdiagramm}
  \begin{itemize*}
    \item Modellierung von Kontrollflüssen, Datenflüssen, Parallelem Verhalten
    \item Geschäftsprozessmodellierung möglich
    \item Abstrakte und detaillierte Verhaltensbeschreibung möglich
    \item Grundlage zur Codegenerierung
    \item Zur Verfeinerung von
    \begin{itemize*}
      \item Use-Cases
      \item Operationen / Interaktionen
    \end{itemize*}
  \end{itemize*}

  \paragraph{Interaktionsdiagramme}
  \begin{itemize*}
    \item Modellierung von Kommunikation \& Operationen
    \item Sehr detaillierte Diagramme
    \item Meist nicht zur vollständigen Beschreibung eines Systems
    \item Betrachtung eines wichtigen Teilaspekts
    \item Grundlage zur Codegenerierung
    \item Ereignis des Sendens bzw. Empfangens von Nachrichten
    \item Typen:
    \begin{itemize*}
      \item Operationsaufruf (synchron / asynchron)
      \item Antwort Nachricht
      \item Signal (asynchron), Create-/ Delete Message
    \end{itemize*}
  \end{itemize*}

  \paragraph{Sequenzdiagramm}
  \begin{itemize*}
    \item Genaue zeitliche Abfolge von Nachrichten
    \item Umfangreichstes Interaktionsdiagramm
    \item Kontrollelemente möglich (Schleifen, Verzweigungen)
    \item Weitere Elemente des Sequenzdiagramms
    \begin{itemize*}
      \item Nachrichten ohne Sender z.B. am Beginn einer Interaktion
      \item Verlorene Nachrichten (ohne Empfänger) z. B. am Ende einer Interaktion
      \item Erzeugen von Lebenslinien
      \item Zerstören von Lebenslinien
    \end{itemize*}
  \end{itemize*}

  \paragraph{Kommunikationsdiagramm}
  \begin{itemize*}
    \item Kommunikationsbeziehungen der Kommunikationspartner stehen im Vordergrund
    \item Welche Komponenten arbeiten wie zusammen, um eine Funktion zu erfüllen
  \end{itemize*}

  \paragraph{Timing-Diagramm}
  \begin{itemize*}
    \item Genaue zeitliche Darstellung von Zustandsübergängen
    \item Kommunikation abhängiger Zustandsautomaten
    \item Modellierung einzelner Interaktion
    \item Zeitlicher Verlauf senkrecht
    \item Kommunikationspartner waagerecht (unsortiert)
    \item Ereignisspezifikation
    \item Trace: Folge von Sende- und Empfangsereignissen
  \end{itemize*}

  \paragraph{Zustandsdiagramm}
  \begin{itemize*}
    \item Modellierung des (vollständigen) Verhaltens
    \begin{itemize*}
      \item Zustände von Klassen / Objekten / Komponenten
      \item Übergänge zwischen den Zuständen
      \item Ereignisse, die Zustandswechsel auslösen
    \end{itemize*}
    \item Modellierung von endlichen Automaten (Zustandsmaschinen)
    \begin{itemize*}
      \item Deterministische
      \item Nichtdeterministische
    \end{itemize*}
    \item Modellierung von verteilten Systemen / parallelem Verhalten
    \item Grundlage zur Codegenerierung
  \end{itemize*}

  \newpage
  \section{Analyse}

  Funktionale Anforderungen - Was soll es tun?
  \begin{itemize*}
    \item Was leistet das System
    \item Welche Funktionen bietet es
    \item Wie interagiert es mit der Umgebung
    \item Anforderungen an: Verhalten, Struktur
  \end{itemize*}

  Nichtfunktionale Anforderungen - Wie?
  \begin{itemize*}
    \item Quality of Service/Qualitätsanforderungen
    \item Hängen oft von Verhalten ab: komplex und nicht direkt sichtbar
    \item Nicht direkt „by construction“ zu realisieren
    \item Untersuchung der Projektrisiken bereits in der Analysephase
    \item Modellbasierter Entwurf des Systems und seiner Umwelt
    \item Arten - FURPS (ISO 9126):
    \begin{description*}
      \item[Functionality] (Funktionalität) Angemessen, Genauigkeit, Sicherheit
      \item[Usability] (Benutzbarkeit) Verständlichkeit, Erlernbarkeit, Bedienbarkeit, Attraktivität
      \item[Reliability] (Zuverlässigkeit) Reife (Fehler-Anzahl), Fehlertoleranz, Wiederherstellbarkeit
      \item[Performance] (Effizienz/Portability) Zeitverhalten, Verbrauchsverhalten, Wirtschaftlichkeit, Koexistenz
      \item[Supportability] (Änderbarkeit/Wartbarkeit) Analysierbarkeit, Änder- und Erweiterbarkeit, Stabilität, Testbarkeit
    \end{description*}
    \item weitere
    \begin{itemize*}
      \item Konformität zu Konventionen und Bestimmungen
      \item Interoperabilität zu anderen Systemen
      \item Implementierungsanforderungen
      \item Schnittstellenanforderungen
      \item Skalierbarkeit (Änderungen des Problemumfangs)
      \item Betriebliche und rechtliche Rahmenbedingungen
      \item Liefer- und Verpackungsanforderungen
    \end{itemize*}
  \end{itemize*}

  Randbedingungen
  \begin{itemize*}
    \item Werden nicht umgesetzt
    \item Schränken Lösungsraum ein
      \item Kosten, Durchlaufzeit, Technische Randbedingungen
  \end{itemize*}

  %![Analysebaum von Sommerville](Assets/Softwaretechnik1_Analyseformen1.png)
  %![Analysebaum von Sommerville](Assets/Softwaretechnik1_Analyseformen2.png)

  Geforderte (Meta-)Eigenschaften
  \begin{description*}
    \item[Vollständig] alle Szenarien sind beschrieben
    \item[Konsistent] keine Widersprüche
    \item[Eindeutig] nur eine Interpretation möglich
    \item[Korrekt] genaue und richtige Darstellung
    \item[Realistisch] unter geg. Einschränkungen implementierbar
    \item[Überprüfbar] durch Tests am Endprodukt nachweisbar
    \item[Rückverfolgbar] Auswirkungen bis Implementierung nachvollziehbar
    \item[Klassifizierbar] Risiko, Priorität, Dringlichkeit, Nutzen ...
    \item[Validierung] mit dem Kunden
  \end{description*}

  %![Anforderungsentwicklung von Balzert](Assets/Softwaretechnik1_Anforderungsentwicklung.png)
  %| | Anforderungsermittlung | Systemmodellierung |
  %| -\item | -\item | -\item |
  %| Ergebnis | Anforderungsspezifikation im Lastenheft, Glossar, Lastenheft | funktionale Spezifikation in Produktdefinition, Analysemodell, Pflichtenheft |
  %| Notation | Text | Text + (semi-) formales Modell |
  %| Kommunikation | mit dem Kunden | zwischen Entwicklern |
  %| Sichtweise | des Anwenders | äußere Systemaspekte |
  %Vor allem: Kommunikationsleistung!

  \subsection{Ermitteln von Anforderungen}
  \begin{itemize*}
    \item Ausgangspunkt
    \begin{itemize*}
      \item Projektidee, schriftliche Skizze
      \item Kurz und knapp
      \item Stichpunkte der wichtigsten Funktionen
      \item Lastenheft (falls schon existiert)
    \end{itemize*}
    \item Beteiligte Rollen
    \begin{description*}
      \item[Endbenutzer] Aufnahme Ist-Zustand, Domänenwissen, Anforderungen
      \item[Kunde] Definiert Ziel des Systems, Vertragsverhandlung
      \item[Konfigurationsmanager] Revisionsgeschichte der Dokumente, Nachvollziehbarkeit
      \item[Architekt] Integration von Anwendungsfall- und Objektmodellen
      \item[Analytiker] Modelliert das System und erstellt Anwendungsfälle
      \item[Redakteur]
      \item[Prüfer]
    \end{description*}
    \item Wie ermittelt man Anforderungen?
    \begin{itemize*}
      \item Kommunikation mit Kunden
      \item Geschäftsprozess: fachlicher Ablauf, der Wert oder Kosten verursacht
      \begin{description*}
        \item[Akteur] Benutzer, Schnittstelle nach außen
        \item[Szenario] Interaktion mit System als Ablauf
        \item[Anwendungsfall] Automatisierter Arbeitsschritt, vom System ausgeführt
        \item[Interviews mit Fachanwendern] Mitschrift, später strukturierter Text und Tabelle
        \item[Strukturierte Spezifikation] Vorlagen, Formulare, Reduzierung sprachlicher Mehrdeutigkeiten
      \end{description*}
      \item Anwendungsfalldiagramm (Use-Case-Diagramm)
      \begin{itemize*}
        \item Arbeitsschritt eines Geschäftsprozesses
        \item Anforderungen an das System modellieren
        \item Systemgrenzen / Systemkontext festlegen
        \item Systembeteiligte modellieren
        \item Planbare Einheiten als Schritte für die Entwicklung
        \item Verwendung bereits ab Projektbeginn
      \end{itemize*}
      \item Umgang mit Szenarien und Anwendungsfällen
      \begin{itemize*}
        \item Systemgrenze definieren
        \item Beschreibungen verfeinern
        \item Änderungen mit Kunden abstimmen
        \item Prototypen nur zur visuellen Unterstützung
      \end{itemize*}
    \end{itemize*}
    \item Typische Probleme
    \begin{itemize*}
      \item Kommunikations- und Verständnisprobleme
      \item Viele verschiedene Beteiligte
      \item Widersprüchliche Anforderungen, verschiedene Interessen
      \item Nicht-technische organisatorische, historische oder rechtliche Rahmenbedingungen
      \item Anforderungen ändern sich während der Entwicklung
    \end{itemize*}
    \item Tätigkeiten der Anforderungsanalyse
    \begin{itemize*}
      \item Anforderungen strukturieren
      \begin{itemize*}
        \item Unterteilung (Funktional, Nichtfunktional)
        \item Hierarchische Zerlegung (Unterteilen, Verfeinern)
        \item Ordnung festlegen, eindeutig Nummerieren
        \item Beziehungen festhalten
        \item Verwendung von Werkzeugen
      \end{itemize*}
      \item Eigenschaften der Anforderungen bestimmen
      \item Anforderungen priorisieren
      \begin{itemize*}
        \item Muss-, Kann-, Optional, Nicht
        \item Ad-hoc: Stakeholder priorisiert Anforderungen
        \item Priorisierungsmatrix / Kosten-Wert-Analyse
        \item Priorität berechnen $\text{Prioritäten} = \frac{\text{Nutzen} - \text{Nachteil}}{\text{Kosten} + \text{Risiko}}$
      \end{itemize*}
      \item Kano-Klassifikation
      \begin{itemize*}
        \item Basiseigenschaften: Werden vorausgesetzt
        \item Leistungseigenschaften: Sonderwünsche
        \item Begeisterungseigenschaften: Wird nicht erwartet
        \item Abfragen per Fragenkatalog
      \end{itemize*}
      \item Reihenfolge festlegen
      \item Anforderungen dokumentieren
      \item Anforderungen modellieren
      \item Anforderungen auf inhaltliche Qualität prüfen
      \item Auf Übereinstimmung mit den Zielen prüfen
    \end{itemize*}
  \end{itemize*}

  %![Kano Klassifikation von Balzert](Assets/Softwaretechnik1_Kano1.png)
  %![Kano Klassifikation von Balzert](Assets/Softwaretechnik1_Kano2.png)

  \subsection{Objektorientierte Analyse und Modellierung}
  \begin{itemize*}
    \item Systemmodell erstellen, funktionale Spezifikation (Was, nicht wie)
    \item Analysemodell
    \begin{itemize*}
      \item Korrekt, vollständig, konsistent und nachprüfbar
      \item Struktur und Verhalten
    \end{itemize*}
    \item Eingangsdokumente (Lastenheft, Anforderungsspezifikation)
    \item Typische Ergebnisse
    \begin{itemize*}
      \item Funktionales Modell (Anwendungsfälle)
      \item Dynamisches Modell - Systemverhalten (Zustandsdiagramme)
      \item Vor- und Nachbedingungen von Systemoperationen
      \item Pflichtenheft
    \end{itemize*}
    \item Verdeutlicht iterativen Ablauf
    \item Unterteilung des Analysemodells
    \begin{itemize*}
      \item Funktionales Modell (Anwendungsfälle)
      \item Objektmodell (Klassen und Objektdiagramme)
      \item Dynamisches Modell (Zustands- und Sequenzdiagramme)
      \item Unterscheidung der Objekttypen
    \end{itemize*}
  %![Analyse nach Brügge/Dutoit](Assets/Softwaretechnik1_Bruegge1.png)
  %![Analyse nach Brügge/Dutoit](Assets/Softwaretechnik1_Bruegge2.png)
  \item Objektarten im Systemmodell
  \begin{description*}
    \item[Entitätsobjekte] vom System verwaltete Informationen
    \item[Grenzobjekte] Interaktion zwischen System und Akteuren
    \item[Steuerungsobjekte] Durchführung der Anwendungsfälle
  \end{description*}
  \item Identifizierung von Entitätsobjekten
  \begin{itemize*}
    \item Begriffe, die klargestellt werden müssen
    \item Wiederkehrende Substantive in Anwendungsfällen
    \item Reale Objekte, die das System kennen muss
    \item Reale Prozesse, die das System verfolgen muss
    \item Anwendungsfälle
    \item Datenquellen und -senken
  \end{itemize*}
  \item Identifizierung von Grenzobjekten
  \begin{itemize*}
    \item Elemente der Benutzungsschnittstelle
    \item Formulare für Eingaben
    \item Nachrichten, Rückmeldungen
    \item Endgeräte
  \end{itemize*}
  \item Identifizierung von Steuerungsobjekten
    \item Abläufe der Anwendungsfälle modellieren
    \item Use Case durch Interaktion verfeinern
    \begin{description*}
      \item[datengetriebener Ablauf mit Verzweigungen] Aktivitätsdiagramm
      \item[Interaktion zwischen den Objekten wichtig] Kommunikations-, Aktivitäts-, Sequenzdiagramm
      \item[zeitliche Abfolge steht im Mittelpunkt] Sequenzdiagramm
      \item[Zustandswechsel / zeitliche Abfolge von Zuständen] Zustands-/Timing-Diagramm
      \item[komplexe Abläufe mit Verzweigungen und Parallelitäten] Interaktionsübersichtsdiagramm
      \item[komplexer strukturierter Ablauf] Kollaboration aus Kompositionsstrukturdiagramm
    \end{description*}
    \item Verfeinerung der Aktionen durch Aktivitäten
    \item Aktion durch Interaktionen verfeinern (Detaillierte Diagramme)
    \item Verfeinerung der Aktionen durch StateChart
    \item Objekte zusammenstellen und klassifizieren
    \begin{itemize*}
      \item Toolunterstützung (Möglichkeiten stark toolabhängig)
      \item Objekte Ergebnis der Verhaltensmodellierung
      \item Ergebnis Verhaltensdiagramm: Operationen der Klassen
      \item Klassen generalisieren / spezialisieren $\rightarrow$ Klassenhierarchie
    \end{itemize*}
    \item Übergang zum Entwurf: Klassenstruktur festlegen
    \item Spezifikation von Benutzungsschnittstellen
    \begin{itemize*}
      \item Skizzieren, Prototyp generieren, Spezialwerkzeuge
      \item Klassen und Operationen in Funktionen
      \item Gestaltung MMI, style guides, Standards
    \end{itemize*}
  \end{itemize*}

  \subsection{Dokumentation von Anforderungen}
  \begin{itemize*}
    \item Lastenheft
    \begin{itemize*}
      \item Forderungen eines Auftraggebers (AG) an Lieferungen und Leistungen eines Auftragnehmers (AN)
      \item Muss-Kriterien, Kann-Kriterien, Abgrenzungskriterien
    \end{itemize*}
    \item Pflichtenheft: Entwurf aus AN-Sicht, Umsetzung des Lastenhefts
    \item Anforderungsspezifikation
    \begin{description*}
      \item[Zielsetzung]
      \item[Allgemeine Beschreibung] Umgebung, Funktion, Benutzer
      \item[Spezifische funktionale Anforderungen] quantitativ + identifizierbar
      \item[Spezifische nicht-funktionale Anforderungen] Entwicklungs- und Produkt-Standards
      \item[Qualitäts-Zielbestimmung]
      \item[Zu erwartende Evolution] des Systems, Versionen
      \item[Abkürzungsverzeichnis] Glossar, Index, Referenzen
    \end{description*}
  \end{itemize*}


  \newpage
  \section{Grobentwurf}
  \subsection{Einführung}
  Systementwurf - Aufgabe
  \begin{itemize*}
    \item Sicht des geplanten Systems von innen (Entwickler)
    \item Wie sollen vereinbartes Verhalten und Funktionen (Analysemodell) intern realisiert werden?
    \item Von Spezifikation von Anforderungen und Funktionen $\rightarrow$ Vorbereitung der Implementierung
    \item Formal: Transformation des Analysemodells in ein Systementwurfsmodell
    \item System(grob)entwurf, Feinentwurf/Objektentwurf
  \end{itemize*}

  Teile und herrsche
  \begin{itemize*}
    \item Grobentwurf
    \begin{itemize*}
      \item Entwurfsziele identifizieren
      \item Grobe Systemstruktur festlegen (Architektur)
      \item Zerlegung in Subsysteme, Spezifikation
      \begin{itemize*}
        \item Schichten, Pakete, Komponenten
      \end{itemize*}
      \item Bewerten der Zerlegung anhand der Entwurfsziele
      \item Schnittstellen festlegen
    \end{itemize*}
    \item Feinentwurf
    \begin{itemize*}
      \item Subsysteme im Detail entwerfen
      \begin{itemize*}
        \item Strukturierung der Komponenten
        \item Klassen, Objekte, Funktionen, Datenstrukturen
        \item Verhalten, Algorithmen - Teillösungen
      \end{itemize*}
    \end{itemize*}
  \end{itemize*}


  \subsection{Systemzerlegung}
  Vorgehen
  \begin{itemize*}
    \item Zerlegung eines Systems in Subsysteme
    \item Betrachtung der Lösungsdomäne!
    \item Subsysteme weiter zerlegen bis Komplexität ausreichend klein ist z.B. für Arbeitspakete
  \end{itemize*}

  Was macht ein Subsystem aus?
  \begin{itemize*}
    \item Schnittstellen, Funktionen, „Verantwortung“
    \item Was bietet es an?
    \item Was benutzt es?
    \item Was tut es intern?
  \end{itemize*}

  Operation
  \begin{itemize*}
    \item Name und Parameter
    \item Funktion, Prozedur, Methode, Eintrittspunkt ...
  \end{itemize*}

  Dienst: Satz von Operationen, die bereitgestellt werden

  Abhängigkeiten von Subsystemen
  \begin{itemize*}
    \item Subsysteme untereinander: Kopplung (coupling)
    \item Maß für die Abhängigkeit von Subsystemen
  \end{itemize*}

  Möglichst lose Kopplung
  \begin{itemize*}
    \item Änderungen in einem beteiligten Subsystem haben geringe Auswirkungen (Stabilität)
    \item Erleichtert Wartbarkeit und Arbeitsteilung
  \end{itemize*}

  Mittel zur Verringerung der Kopplung
  \begin{itemize*}
    \item Zusätzliche Unterteilung in Subsysteme
    \item Aber: dann größere Komplexität!
  \end{itemize*}

  Abhängigkeiten von Subsystemen
  %| Kopplungsart | Bemerkung |
  %| -\item | -\item |
  %| Datenkopplung (gemeinsame Daten) | Möglichst vermeiden! Wenn nicht möglich, Verwaltung zentralisieren und Zugriff über Schnittstelle |
  %| Schnittstellenkopplung (gegenseitiger Aufruf) | Akzeptabel |
  %| Strukturkopplung (gemeinsame Strukturelemente) | Vermeiden! (z.B. keine Vererbung über Paketgrenzen hinweg) |

  \begin{itemize*}
    \item Elemente eines Subsystems: Kohäsion (cohesion)
    \item Maß für Zusammengehörigkeit der Elemente
    \item Möglichst hohe Kohäsion
    \begin{itemize*}
      \item Enge Beziehung oder ähnliche Aufgaben der Elemente
      \item Erleichtert Verständnis, Wartung und Anpassung
    \end{itemize*}
    \item Mittel zum Erreichen hoher Kohäsion
    \begin{itemize*}
      \item Datenkapselung, Objektorientierung
      \item Benutzung geeigneter Patterns (Kapitel 5)
    \end{itemize*}
  \end{itemize*}

  Metriken für modulare Entwürfe
  \begin{itemize*}
    \item Fan-in / fan-out-Metrik [S.Henry, D. Kafura 1981]:
    \begin{itemize*}
      \item Fan-in: Anzahl der Stellen, wo Kontrollfluss auf das betrachtete Modul M übergeht (Aufrufe von Funktionen / Prozeduren in M) + Anzahl globaler Variablen, die in M zugänglich sind
      \item Fan-out: Anzahl von Stellen, an denen M andere Module aufruft + Anzahl der globalen Variablen, die von M verändert werden
    \end{itemize*}
    \item Heuristik Kopplung / Kohäsion
    \begin{itemize*}
      \item Hoher Fan-out bedeutet hohe Kopplung, minimieren
      \item Hoher Fan-in kann auf geringe Kohäsion von M hindeuten
    \end{itemize*}
  \end{itemize*}

  Komplexität beherrschen: "Wenn Du es nicht in fünf Minuten erklären kannst, hast Du es entweder selbst nicht verstanden oder es funktioniert nicht." [Rechtin, Maier: The Art of Systems Architecting 2000]

  Vorgehen: Heuristiken und Erfahrungen
  \begin{itemize*}
    \item „Erfahrung ist die härteste Lehrerin. Sie gibt Dir zuerst den Test und anschließend den Unterricht.“ [Ruth 1993]
    \item „Ein Architekt der zu Beginn seiner Arbeit vollständige und konsistente Anforderungen benötigt, mag ein brillanter Entwickler sein - aber er ist kein Architekt“ [Rechtin 2000]
    \item „Das Leben von Software-Architekten besteht aus einer langen und schnellen Abfolge suboptimaler Entwurfs-entscheidungen, die teilweise im Dunkeln getroffen werden.“ [Kruchten2001]
  \end{itemize*}

  Wie organisiert man Subsysteme?
  \begin{itemize*}
    \item Innerhalb einer Verfeinerungsstufe: fachlich orientierte Zerlegung
    \item Mehrfache Zerlegung: Hierarchie-Graph der Verfeinerung
  \end{itemize*}

  Schicht
  \begin{itemize*}
    \item Gruppe von Subsystemen in der Zerlegungshierarchie
    \item Verwandte Dienste
    \item Ähnlicher Abstraktionsgrad
    \item Abhängigkeit nur von darunter liegenden!
  \end{itemize*}

  \begin{itemize*}
    \item Geschlossene Schichtenarchitektur
    \begin{itemize*}
      \item Beispiel: OSI-Modell für Kommunikationssysteme
    \end{itemize*}
    \item Offene Schichtenarchitektur
    \begin{itemize*}
      \item Beispiel: Java Swing auf X11-Plattform
    \end{itemize*}
  \end{itemize*}

  Prinzipien des OO-Entwurfs
  \begin{itemize*}
    \item So-einfach-wie-möglich-Prinzip (KISS)
    \item Fehler berücksichtigen (Strukturierung, Kapselung, Modularisierung, Wiederverwendung)
    \item Entwerfen nach Verantwortlichkeiten
    \item Hohe Kohäsion / Geringe Kopplung
    \item Zyklische Abhängigkeiten vermeiden
    \item Auf Schnittstellen konzentrieren
    \begin{itemize*}
      \item Abhängigkeiten nur von Schnittstellen
      \item Abtrennung von Schnittstellen (eher viele kleine als eine große)
      \item Umkehr der Abhängigkeiten (dependency inversion-Prinzip)
    \end{itemize*}
    \item Offen / Geschlossen Prinzip
  \end{itemize*}

  Zyklische Abhängigkeiten vermeiden
  \begin{itemize*}
    \item Änderungen wirken sich auf beide Komponenten aus
    \item Probleme beim Löschen und Initialisieren
    \item Auflösen durch
    \begin{itemize*}
      \item Gemeinsame Klassen in separates Paket
      \item Gemeinsame Schnittstellen definieren
    \end{itemize*}
  \end{itemize*}

  Symptome schlechten Designs
  \begin{itemize*}
    \item Starrheit
    \begin{itemize*}
      \item Einfache Änderungen schwierig realisierbar
      \item Einfache Änderungen führen zur Modifikation einer Vielzahl von Komponenten
    \end{itemize*}
    \item Zerbrechlichkeit
    \begin{itemize*}
      \item Änderungen an einer Stelle führen zu Fehlern an völlig anderer Stelle
    \end{itemize*}
    \item Schlechte Wiederverwendbarkeit
    \begin{itemize*}
      \item Komponenten können Aufgrund spezieller Anhängigkeiten kaum wiederverwendet werden
    \end{itemize*}
  \end{itemize*}

  Wann ist ein Entwurf „gut“?
  \begin{itemize*}
    \item Korrekt
    \begin{itemize*}
      \item Erfüllung der Anforderungen
      \item Wiedergabe aller Funktionen des Systemmodells
      \item Sicherstellung der nichtfunktionalen Anforderungen
    \end{itemize*}
    \item Verständlich und präzise, gut dokumentiert
    \item Anpassbar
    \item Hohe Kohäsion innerhalb der Komponenten
    \item Schwache Kopplung zwischen den Komponenten
    \item Wiederverwendung
    \item Kriterien gelten auf allen Ebenen des Entwurfs! (Architektur, Subsysteme, Komponenten)
  \end{itemize*}

  \subsection{Architekturmodelle}
  \begin{itemize*}
    \item Modellierung mit UML
    \begin{itemize*}
      \item Bisher: logische Sicht
      \item Technisch: Organisation in Paketen, Namensraum, Import
    \end{itemize*}
    \item Paketdiagramm
    \begin{itemize*}
      \item Gliederung (Strukturierung) des Systems in Teile
      \item Zuordnung von Elementen zu einem Paket
      \item Hierarchien und Abhängigkeiten zwischen den Paketen
      \item Anwendung: Definition von Schichten
    \end{itemize*}
    \item Enthält-Beziehung
    \begin{itemize*}
      \item Definiert, in welchem Paket ein Element enthalten ist
      \item Ermöglicht qualifizierten Zugriff auf enthaltene Elemente
      \item Löschen des Pakets bewirkt Löschen beinhalteter Elemente
      \item Definition von Sichtbarkeit / Zugriffsrechte
      \begin{itemize*}
        \item Auswirkung auf weitere Enthält-Beziehung
        \item '+' - public (default)
        \item '-' - private
      \end{itemize*}
    \end{itemize*}
    \item Paket- / Element-Import
    \begin{itemize*}
      \item Unqualifizierter Zugriff auf Elemente eines anderen Namensraums (Paketes)
    \end{itemize*}
    \item Komponentendiagramm
    \begin{itemize*}
      \item Komponente - modulare, austauschbare Einheit
      \item Strukturierung des Systems durch Komponenten
      \item Modellierung der
      \begin{itemize*}
        \item Abhängigkeiten zwischen Komponenten
        \item inneren Struktur von Komponenten
      \end{itemize*}
      \item Definition von Schnittstellen
      \item Verwendung von Elementen aus Klassen- und Objektdiagramm
      \item Stärkere dynamische Sicht $\rightarrow$ kein Verhalten
      \item Komponente <<component>>
      \begin{itemize*}
        \item Kapselt Funktionalitäten (Physisch gruppierte Klassen)
        \item „Spezialisierte“ Klasse (Vererbung, Exemplare möglich)
        \item Stellt Funktionalitäten über Schnittstellen bereit
        \item Definiert benötigte Schnittstellen
        \item Enthält Klassen oder weitere Komponenten
        \item Modulares Element: Substitution (Austauschbarkeit) steht im Vordergrund
      \end{itemize*}
      \item Black-Box-Darstellung
      \begin{itemize*}
        \item Zur Verfügung gestellte Funktionalität `<<provided interfaces>>`
        \item Benötigte Funktionalität `<<required interfaces>>`‚
      \end{itemize*}
      \item White-Box-Darstellung
      \begin{itemize*}
        \item Interner Aufbau der Komponente `<<realization>>`
        \item Artefakte `<<artifacts>>`‚ Realisierende physische Einheit (z.B.: .dll)
      \end{itemize*}
    \end{itemize*}
  \end{itemize*}

  Schnittstellen / Interfaces
  \begin{itemize*}
    \item Definition Diagrammunabhängig
    \begin{itemize*}
      \item Meist Klassendiagramm
    \end{itemize*}
    \item Ähnlich Semantik einer Klasse
    \begin{itemize*}
      \item Nur public-Attribute und Operationen
    \end{itemize*}
    \item Definiert Verpflichtung zur Implementierung von
    \begin{itemize*}
      \item Operationen
      \item Merkmale $\rightarrow$ Attribute dürfen definiert werden
      \item Verpflichtungen (z.B.: Vor- / Nachbedingungen)
    \end{itemize*}
    \item Meist abstrakte Klassen mit abstrakten Operationen
    \item Abstrakt - muss überschrieben werden
    \item Notation
    \begin{itemize*}
      \item Stereotyp: <<Interface>>
      \item Meist kursiv geschrieben, da abstrakte Klasse
    \end{itemize*}
  \end{itemize*}

  Schnittstellenrealisierung, Implementierungsbeziehung
  \begin{itemize*}
    \item Schnittstellen werden realisiert, nicht instanziiert
    \item Schnittstellenkonform
    \begin{itemize*}
      \item Klasse realisiert alle Attribute und Operationen
    \end{itemize*}
    \item Schnittstelle kann von anderen Schnittstellen erben
    \item Keine Schnittstellenrealisierung zwischen zwei Interface-Klassen $\rightarrow$ Generalisierung verwenden
    \item Darstellung
    \begin{itemize*}
      \item Gestrichelte Linie mit nicht gefülltem Dreieck an der Seite der Superklasse
      \item Alternativ: Lollipop-Darstellung
    \end{itemize*}
  \end{itemize*}


  \subsection{Softwarearchitekturmuster}
  \begin{itemize*}
    \item Wiederverwendung auf sehr hoher Abstraktionsstufe
    \item Falls geplante Anwendung passt, anwenden!
  \end{itemize*}

  Schichten-Architektur (layers)
  \begin{itemize*}
    \item Problem
    \begin{itemize*}
      \item Komplexität: Strukturierung des Systems, unterschiedliche Abstraktionsebenen
      \item Änderungen sollen möglichst lokal bleiben
      \item Teilsysteme sollen austauschbar, wiederverwendbar und getrennt entwickelbar sein
      \item Schnittstellen sollen stabil sein
    \end{itemize*}
    \item Lösung
    \begin{itemize*}
      \item Zuordnung von Subsystemen zu horizontalen Schichten gleicher Abstraktionsebene
      \item Komponenten einer Schicht bieten Dienste der darüber liegenden Schicht an
    \end{itemize*}
  \end{itemize*}

  Client-Server (Klient/Anbieter)
  \begin{itemize*}
    \item Client (front-end)
    \begin{itemize*}
      \item Benutzungsschnittstelle
      \item Einbindung in Geschäftsprozesse
      \item Entkoppelt von Netztechnologie und Datenhaltung
    \end{itemize*}
    \item Server (back-end)
    \begin{itemize*}
      \item Datenhaltung, evtl. Fachlogik
    \end{itemize*}
    \item Genauer: Two-tier client/server architecture
    \item Asynchroner Kontrollfluss
    \item Aufteilung Funktionen Client / Server
    \begin{itemize*}
      \item Mehr Funktionen im Server:
      \begin{itemize*}
        \item zentrale Verwaltung, Wartungsaufwand geringer, Portabilität, einfache Client-Hardware (Net PC)
        \item „Thin Client“ - nur GUI
      \end{itemize*}
      \item Mehr Funktionen im Client: Flaschenhals Server wird entlastet, individuellere Client-Funktionen
      \begin{itemize*}
        \item „Fat Client“ - Teil der Anwendung im Client
      \end{itemize*}
      \item Entscheidet mit über Umsetzung (Java Script, ...)
    \end{itemize*}
  \end{itemize*}

  Three-Tier / Four-Tier Architecture: Client/Server mit weiterer Aufteilung ähnlich Repository

  Bewertung Client-Server
  \begin{itemize*}
    \item Vorteile
    \begin{itemize*}
      \item Leicht verständlich
      \item Änderungen bleiben lokal
      \item Geringere Kopplung zwischen den Schichten
      \item Schichten austauschbar und wiederverwendbar
      \item Getrennte Entwicklung der Schichten möglich
      \item Vorhandene / stabilere Schnittstellen
    \end{itemize*}
    \item Nachteile
    \begin{itemize*}
      \item Geringere Performance
      \item Zusätzlicher Verwaltungs- oder Datenoverhead
      \item Manche Änderungen führen zu Änderungen in allen Schichten (z.B. neues Datenfeld)
    \end{itemize*}
  \end{itemize*}

  Pipes and Filters
  \begin{itemize*}
    \item Datenstrom- oder Kontrollflussorientiertes System
    \item Lose verbundene Berechnungskomponenten
    \item Kombination der Berechnungskomponenten nur vom Typ der Ein- und Ausgabedaten abhängig
    \item Leicht erweiterbar System gewünscht
    \item Parallele Verarbeitung vorteilhaft
    \item Verwendung von globalen Steuerungskontrollstrukturen (Parallelisierung, Verzweigung, Schleifen) gewünscht
    \item Vorteile
    \begin{itemize*}
      \item Stark entkoppelte Komponenten
      \item Hohe Flexibilität gegenüber Änderungen \& Erweiterungen
      \item Hoher Wiederverwendungsgrad der Komponenten
      \item Unabhängige Entwicklung der Komponenten
      \item Leichte Parallelisierung der Berechnungen möglich
      \item Überprüfung der Datenkompatibilität dynamisch / statisch
    \end{itemize*}
    \item Nachteile
    \begin{itemize*}
      \item Schwierige Fehlerbehandlung, kein expliziter Kontrollfluss
      \item Fehler durch inkompatible Datentypfehler erst zur Laufzeit
      \item Häufig zusätzliche Datenkonvertierungen notwendig
    \end{itemize*}
  \end{itemize*}


  Plug-In Architektur (Microkernel)
  \begin{itemize*}
    \item Zielstellung
    \begin{itemize*}
      \item Stabile, verbreitete Standard-Anwendung (Kern)
      \item Funktionalität soll durch Komponenten leicht erweiterbar sein
      \item Dritte sollen Komponenten leicht erstellen können
    \end{itemize*}
    \item Lösung
    \begin{itemize*}
      \item Möglichst schlanker zentraler Kern
      \item Plugin-Manager verwaltet Komponenten: Laden, Entladen, Zugriffskontrolle, Konfiguration
    \end{itemize*}
    \item Plugin
    \begin{itemize*}
      \item Komponente mit Standard-Schnittstelle
      \item Erweitert Funktionalität (extension point)
    \end{itemize*}
    \item Vorteile
    \begin{itemize*}
      \item Robustes Verhalten
      \item Trennung der Zuständigkeiten
      \item Erweiterbar, Austauschbar, Wiederverwendbar
      \item Geringe Kopplung zu den Komponenten
      \item Anpassung an eigene Bedürfnisse möglich
      \item Leichte Aufteilung der Entwicklung der Arbeitspakete
    \end{itemize*}
    \item Nachteile
    \begin{itemize*}
      \item Höherer initialer Aufwand
      \item Verwaltungsoverhead zur Laufzeit
      \item Versionsverwaltung der Komponenten nötig
      \item Abhängigkeiten unter den Komponenten schwierig realisierbar
      \item Geschickte Definition der Extension Points nötig
    \end{itemize*}
  \end{itemize*}

  Repository (Depot, blackboard)
  \begin{itemize*}
    \item Zentrale Datenhaltung
    \begin{itemize*}
      \item Datenbankmanagementsystem, Dateisystem
    \end{itemize*}
    \item Anwendungen tauschen Daten nur über Repository aus
    \item Kontrollfluss z.B. über Signale oder Semaphore
    \item Gut für datenintensive Verarbeitungsaufgaben geeignet
  \end{itemize*}

  Peer-to-peer
  \begin{itemize*}
    \item Gleichberechtigte Partner, “Föderation”
    \item Verteilte kommunizierende Subsysteme
    \item Orts- und Umgebungsunabhängigkeit
  \end{itemize*}

  Model-View-Controller (MVC)
  \begin{itemize*}
    \item Modell / Sicht / Steuerung
    \item Trennung verschiedener Aufgabengebiete:
    \begin{itemize*}
      \item Model: verwaltet Domänenwissen, Daten und Zustand; häufig Datenbank
      \item View: Darstellung, Anzeige, GUI
      \item Controller: Steuerung der Interaktion, Nutzerbefehle
    \end{itemize*}
    \item Erlauben Austausch von Anzeige- und Speichersystem
    \item Kontrollfluss
    \begin{itemize*}
      \item Controller steuert
      \item View wird über Datenänderungen benachrichtigt (callback)
    \end{itemize*}
    \item Geeignet für interaktive Systeme
    \item Problem
    \begin{itemize*}
      \item Lose Kopplung zwischen verschiedenen Komponenten
      \item Daten werden in verschiedenen Sichten dargestellt
      \item Realisierung von GUI‘s
    \end{itemize*}
    \item Lösung durch drei Komponenten
    \begin{itemize*}
      \item Daten (Model) enthält die Kernfunktionalität / Durchführung der Geschäftsprozesse, kapselt und Speichert die Daten
      \item Sichten bzw. Dialoge (View) stellt die Daten für den Anwender in unterschiedlicher Art dar
      \item Logik bzw. Steuerung (Controller) Realisiert die Interaktion mit dem Benutzer, übernimmt die Eingaben vom View und ändert die Daten im Modell, legt die Darstellungsart der Sichten fest
    \end{itemize*}
    \item Vorteile
    \begin{itemize*}
      \item Unabhängige Entwicklung der Komponenten
      \item Änderung der Oberfläche ohne Änderung des Modells
      \item Unterschiedliche Oberflächen für das selbe Modell
    \end{itemize*}
    \item Nachteile
    \begin{itemize*}
      \item Performance
      \item Erhöhter initialer Entwicklungsaufwand
    \end{itemize*}
  \end{itemize*}


  \subsection{Frameworks}
  Was ist ein Framework?
  \begin{itemize*}
    \item A framework is a set of prefabricated software building blocks that programmers can use, extend, or customize for specific computing solutions [Taligent]
    \item Ein framework (Rahmenwerk, Anwendungsgerüst) ist eine Menge von zusammengehörigen Klassen, die einen abstrakten Entwurf für eine Problemfamilie darstellen [nach Pomberger/Blaschek]
  \end{itemize*}

  Ziele
  \begin{itemize*}
    \item Wiederverwendung von Code, Architektur, Entwurfsprinzipien und Verhaltensschema
    \item Ähnliche Benutzungsschnittstelle
  \end{itemize*}

  Klassifikation I
  \begin{itemize*}
    \item Anwendungs-Framework (application framework)
    \begin{itemize*}
      \item Gibt Systemarchitektur für typische Anwendungsstruktur vor
      \item GUI-Framework: Motif, Qt, Swing, ...
    \end{itemize*}
    \item Bereichsspezifisches Framework (domain framework)
    \begin{itemize*}
      \item Expertenwissen für Anwendungsbereich
      \item für typische Anwendungen u.a. in den Bereichen Luftfahrt, Produktion, Finanzwesen, Automotive, ...
      \item Beispiel: AUTOSAR
    \end{itemize*}
    \item Infrastrukturgerüst (support framework)
    \begin{itemize*}
      \item Gerätetreiber, Anpassung an Hardware
      \item Middleware: DCOM, Java RMI, CORBA, WebSphere, ...
    \end{itemize*}
  \end{itemize*}

  Klassifikation II
  \begin{itemize*}
    \item Offene Programmgerüste (white box)
    \begin{itemize*}
      \item Erweiterbarkeit durch Vererbung und dynamische Bindung
      \item Funktionen konkretisieren durch Ableitung von Basisklassen des Programmgerüsts und Überschreiben vordefinierter Methoden
    \end{itemize*}
    \item Geschlossene Programmgerüste (black box)
    \begin{itemize*}
      \item Erweiterbarkeit durch Definition von Schnittstellen für Module, die für eine konkrete Anwendung in das Gerüst eingesetzt werden können
      \item Wiederverwendung durch Komponenten, die sich an Schnittstellen halten; Aufruf über Delegation
    \end{itemize*}
  \end{itemize*}


  Webframeworks - Angular JS
  \begin{itemize*}
    \item Clientseitiges Webframework von Google
    \item Frei verwendbar (Open Source)
    \item Erstellung von Single-Page-Webanwendungen
    \item Model View Prinzip
  \end{itemize*}

  \begin{itemize*}
    \item Vorteile
    \begin{itemize*}
      \item Weitergabe von Expertenwissen
      \item Durchdachtes Design: langfristige Aufwandsersparnis
      \item Wartungsaufwand reduziert, systematische Tests möglich
      \item Prinzipiell sehr hohe Produktivität möglich
      \item Erleichtert Integration und Konsistenz verwandter Anforderungen
    \end{itemize*}
    \item Nachteile
    \begin{itemize*}
      \item Erstellung und Einarbeitung aufwändig
      \item Zusätzlicher Dokumentations- und Wartungsaufwand
      \item Fehlersuche erschwert durch Overhead des Frameworks
      \item Kombination verschiedener Frameworks sehr schwierig
    \end{itemize*}
  \end{itemize*}

  \subsection{Systemarchitektur und Verteilung}
  Systemarchitektur
  \begin{itemize*}
    \item Aufbau und Elemente der Ablaufumgebung, Hardware
    \item Häufig enger Zusammenhang mit Softwarearchitektur
    \begin{itemize*}
      \item Architekturmuster
      \item Ablaufmuster
    \end{itemize*}
    \item Besonders bei eingebetteten Systemen
    \item Systemarchitektur hat Einfluss auf Softwarearchitektur
    \begin{itemize*}
      \item Grenzobjekte, Schnittstellen, ...
    \end{itemize*}
    \item Gute Systemarchitektur?
    \begin{itemize*}
      \item Nichtfunktionale Anforderungen
      \item Modellierung und Simulation, Lastmessungen
    \end{itemize*}
  \end{itemize*}

  Typische Strukturen
  \begin{itemize*}
    \item Zentralrechner (mainframe) mit Terminals
    \item Server und einfache Stationen
    \item PCs und Server
    \item Kommunikationsverbindungen, Sensoren, Speicher, ...
  \end{itemize*}


  \begin{itemize*}
    \item Blockdiagramm
    \begin{itemize*}
      \item Klassisches Beschreibungsmittel für Systemaufbau
      \item Nicht Teil von UML
    \end{itemize*}
    \item Konfigurationsdiagramm
    \begin{itemize*}
      \item meistverbreitetes Hilfsmittel zur Beschreibung der physikalischen Verteilung von System-Komponenten
      \item Nicht Teil von UML
    \end{itemize*}
    \item Verteilungsdiagramm (UML deployment diagram)
    \begin{itemize*}
      \item Darstellung der Hardwaretopologie
      \item Zuordnung von Artefakten zu Hardwareeinheiten (Knoten)
      \begin{itemize*}
        \item Verteilung von Systembestandteilen auf Hardware
      \end{itemize*}
      \item Kommunikationsverbindung und Abhängigkeiten zwischen Knoten
      \item Relativ spät im Projekt Installation / Wartung des Systems
    \end{itemize*}
  \end{itemize*}

  \subsection{Globaler Kontrollfluss}
  Globaler Kontrollfluss
  \begin{itemize*}
    \item Ablaufsicht der Architektur
    \begin{itemize*}
      \item Definition nebenläufiger Systemeinheiten (z.B. Prozesse)
      \item Steuerung der Abfolge von Einzelfunktionen
      \item Synchronisation und Koordination
      \item Reaktion auf externe Ereignisse
      \item Darstellung z.B. durch Sequenzdiagramme
    \end{itemize*}
    \item Nebenläufigkeit auf Architekturebene
    \begin{itemize*}
      \item Threads , Prozesse, verteiltes System
      \item Asynchroner Nachrichtenaustausch
    \end{itemize*}
    \item Einfluss auf Architektur / abhängig von Architektur!
    \item Ablaufmuster
    \begin{itemize*}
      \item Zentral
      \begin{itemize*}
        \item Call/Return (prozedural, synchron)
        \item Master/Slave (nebenläufig mit zentraler Steuerung)
      \end{itemize*}
      \item Dezentral
      \begin{itemize*}
        \item Ereignisgesteuert (event-driven)
        \item interrupts
        \item publish-subscribe (ähnlich observer)
        \item (selective) broadcast
        \item Datenflussgesteuert (data flow architecture)
      \end{itemize*}
    \end{itemize*}
  \end{itemize*}

  \subsection{Sonstiges}
  Ablauf des OO-Systementwurfs [B. Oesterreich]
  \begin{itemize*}
    \item Schichtenmodell definieren
    \item Verteilungsmodell definieren
    \item Fachliches Subsystemmodell definieren
    \item Ablaufverantwortlichkeiten definieren
    \item Komponentenspezifisches Klassenmodell entwickeln
    \item Komponentenschnittstelle entwerfen
    \item Zustandsmodelle weiterentwickeln
    \item Objektfluss modellieren
    \item Interaktionsmodelle entwickeln, Attribute definieren
    \item Dialoge spezifizieren
    \item Design-Diskurs
    \item Testgetriebene Entwicklung
  \end{itemize*}

  Weitere Aufgaben beim Grobentwurf
  \begin{itemize*}
    \item Entwurf einer persistenten Datenverwaltung
    \begin{itemize*}
      \item Dateisystem, Datenbank
    \end{itemize*}
    \item Sicherheit
    \begin{itemize*}
      \item Zugriffskontrolle
      \item Fehlertoleranz (Daten und Hardware)
      \item Protokollfunktionen
    \end{itemize*}
    \item Kontrollfluss
    \begin{itemize*}
      \item Ausnahmen
      \item Starten, Initialisieren und Beenden der Anwendung
      \item „Randanwendungsfälle“
    \end{itemize*}
  \end{itemize*}

  Notwendigkeit der Architekturdokumentation
  \begin{itemize*}
    \item Quellcode aufgrund niedrigen Abstraktionsniveaus ungünstig für Dokumentation
    \item Überblick und Arbeitsteilung
    \item Lebensdauer von Systemen länger als geplant
    \item Fehler und Probleme leichter finden und beseitigen
    \item Neue Anforderungen mit angemessenem Aufwand erfüllen
    \item Vereinfachung der Wartung, Pflege, Erweiterung, Wiederverwendung
  \end{itemize*}

  \subsection{Dokumentation}
  \begin{itemize*}
    \item Grundprinzipien
    \begin{itemize*}
      \item Verständlich aus Sicht des Lesers formulieren (Glossar)
      \item Das Warum beschreiben (Entwurfsentscheidungen)
      \item Annahmen, Voraussetzungen, Randbedingungen dokumentieren
      \item Wiederholungen vermeiden
      \item Notation erklären oder Standards verwenden (UML)
      \begin{itemize*}
        \item Legende hinzufügen
      \end{itemize*}
      \item Auf Zweckdienlichkeit prüfen, Reviews durchführen (Inhalt, Qualität)
      \item Verschiedene Sichten für verschiedene Zielgruppen
    \end{itemize*}
  \end{itemize*}

  \newpage
  \section{Feinentwurf}

  %| Analyse-Modell | Entwurfs-Modell |
  %| -\item | -\item |
  %| Fachliche Domäne  | Lösungsdomäne |
  %| Teilweise unvollständig in Attributen und Operationen | Vollständige Angabe aller Attribute und Operationen |
  %| Datentypen und Parameter können noch fehlen | Vollständige Angabe von Datentypen und Parametern |
  %| Noch kaum Bezug zur Realisierungssprache | Auf Umsetzung in gewählter Programmiersprache bezogen |
  %| Keine Überlegungen zur Realisierung von Assoziationen | Navigationsangaben, Qualifikation, Ordnung, Verwaltungsklassen |
  %| | Entscheidung über Datenstrukturen, Anbindung GUI |

  Schließen der Lücke zwischen Grobentwurf und Implementierung
  \begin{itemize*}
    \item Identifizieren und Entwerfen von Klassen der Lösungsdomäne
    \item Identifikation und Verwendung von Entwurfsmustern
    \item Detaillierte Beschreibung der Klassen
    \item Beschreibung von Schnittstellen
    \item Iterativer Prozess!
    \begin{itemize*}
      \item Verbesserung des Entwurfs - Refactoring
      \item Optimieren des Entwurfsmodells zur Erfüllung nichtfunktionaler Anforderungen
    \end{itemize*}
  \end{itemize*}

  Objektorientierter Feinentwurf
  \begin{itemize*}
    \item Ausgangspunkt
    \begin{itemize*}
      \item Grobdefinition der Architektur, Zerlegung in Subsysteme (evtl. unter Verwendung von Standardarchitekturen)
      \item Verteilungskonzept
      \item Ablaufmodell
    \end{itemize*}
    \item Ergebnis
    \begin{itemize*}
      \item OO-Modell für jedes Subsystem der Architektur
      \item OO-Modell für unterstützende Subsysteme unter Berücksichtigung gewählter Technologien
      \item Spezifikationen der Klassen
      \item Spezifikationen von externen Schnittstellen
    \end{itemize*}
  \end{itemize*}

  \subsection{Klassen- und Objektentwurf}
  \begin{itemize*}
    \item Klassen der Lösungsdomäne
    \begin{itemize*}
      \item Klassen, die nicht durch objektorientierte Analyse der Anwendungsdomäne entstehen
    \end{itemize*}
    \item Entstehungsgründe
    \begin{itemize*}
      \item Architektur von Software und System
      \item nichtfunktionale Anforderungen
      \item Beispiele: Kommunikation, Fehlertoleranz, Adapter, Datenhaltung, Effizienz, Benutzerschnittstellenobjekte, Middleware, ...
      \item Sichtbare (Grenz- und Steuerungsobjekte) werden schon in der Analyse identifiziert
    \end{itemize*}
  \end{itemize*}

  Klassen identifizieren (responsibility-driven design (Wirfs-Brock, McKean))
  > Verantwortlichkeits-Prinzip: Sichtweise: Objekte und Klassen sind nicht nur Behälter für Verhalten und Daten, sondern erfüllen in Zusammenarbeit mit anderen Objekten bestimmte Aufgaben eigenverantwortlich

  Responsibility-Driven Design - Begriffe
  \begin{itemize*}
    \item Sichtweise auf Softwaresystem
    \item Application = set of interacting objects
    \item Object = implementation of role(s)
    \item Role = set of related responsibilities
    \item Responsibility = obligation to perform a task or know information
    \item Collaboration = interaction of objects or roles
    \item Contract = agreement outlining collaboration terms
  \end{itemize*}

  Arten von Rollen

  \begin{itemize*}
    \item Information holder  | knows and provides information |
    \item Structurer | maintains relationship between objects and information about relationships |
    \item Service provider | performs work, offers computing services |
    \item Coordinator | reacts to events by delegating tasks to others |
    \item Controller | makes decisions and directs other’s actions |
    \item Interfacer | transforms information and requests between system parts |
  \end{itemize*}

  Hilfsmittel: CRC-Karten
  \begin{itemize*}
    \item Candidate (or class), Responsibility, Collaboration
    \item Informelles Mittel zum
    \begin{itemize*}
      \item Finden,
      \item Beschreiben und
      \item iterativen Verändern von Klassen
    \end{itemize*}
  \end{itemize*}

  Ein Objekt
  \begin{itemize*}
    \item implementiert eine Schnittstelle und beeinflusst andere Objekte
    \item wird in drei Teilen entworfen
    \begin{itemize*}
      \item Öffentliche Schnittstelle
      \item Art und Weise der Benutzung
      \item Innere Details der Funktionsweise
    \end{itemize*}
    \item Kohärenz: zusammengehörende Verantwortlichkeiten in einer Klasse konzentrieren!
  \end{itemize*}


  \subsection{Entwurfsprinzipien}
  \begin{itemize*}
    \item Kapselung
    \begin{itemize*}
      \item Probleme: Zugriff auf private oder ebenen-fremde Attribute
      \item Verwenden von get- und set-Operationen
      \item Zusicherungen einhalten
      \item Zugriffe zentralisieren
      \item Verbalisierung
      \item Zugriffsbeschränkung
    \end{itemize*}
    \item Zerlegung
    \begin{itemize*}
      \item Teile und Herrsche
      \item Zerlegen in Komponenten
      \item Verantwortlichkeitsprinzip: Komponente ist klar für eine Aufgabe verantwortlich
      \item Eigenschaften und Schnittstellen im Klassendiagramm
      \item Beziehungen zwischen Klassen: Assoziationen
      \item Aggregation
      \begin{itemize*}
        \item „besteht aus“, „ist Teil von“ oder „Ganzes-/Teile-Beziehung“
        \item Schwache Bindung der Teile mit dem Ganzen
        \item Notation: ungefüllte Raute am Ganzen
      \end{itemize*}
      \item Komposition
      \begin{itemize*}
        \item Wie Aggregation, jedoch stärkere Bindung
        \item Teil nur einem Ganzen zugeordnet
        \item Nur Multiplizität von 1 oder 0..1 möglich!
        \item Gefüllte Raute am Ganzen
      \end{itemize*}
      \item Polymorphie
      \begin{itemize*}
        \item Reaktion auf eine Nachricht abhängig vom Typ des Objektes
        \item Variablen können Objekte verschiedener Klassen aufnehmen (Voraussetzung: Typ der Variablen ist eine gemeinsame Basisklasse der (davon) abgeleiteten Klasse(n) der Objekte)
        \item Überladen von Operationen
        \item gleicher Operationsname, unterschiedliche Signatur
        \item abstrakte Operationen: Virtuelle Operationen ohne Implementierung
        \item abstrakte Klasse: Klasse mit abstrakten Operationen
        \item Folgen:
        \item von abstrakten Klassen können keine Objekte angelegt werden (Implementierung fehlt)
        \item Abgeleitete Klassen müssen Operation implementieren, damit Objekte angelegt werden können
      \end{itemize*}
    \end{itemize*}
  \end{itemize*}

  Vererbung im Entwurf
  \begin{itemize*}
    \item In der Analyse: Klassifikation von Objekten, Taxonomie, Spezialisierung/Verallgemeinerung, Organisation von Klassen in Hierarchien
    \item Verringerung von Redundanz und damit Inkonsistenzen
    \begin{itemize*}
      \item Funktionalität nur einmal implementieren!
      \item Spezifikations-Wiederverwendung
      \item Implementierungs-Wiederverwendung
    \end{itemize*}
    \item Verbesserung der Erweiterbarkeit
    \begin{itemize*}
      \item Abstrakte Schnittstellen einsetzen!
    \end{itemize*}
  \end{itemize*}

  Vererbung oder Assoziation
  \begin{itemize*}
    \item Schlüsselwort Vererbung: ist ein
    \item Schlüsselwort Assoziation: besteht aus, ist Teil, hat,...
    \item Vererbung: Unterscheidungsmerkmal definierbar (Diskriminator)
    \item Vermeide Vererbung, wenn es Alternativen gibt
    \item Mehrfachvererbung
    \begin{itemize*}
      \item Problem: Unabhängige Aspekte der Vererbungshierarchie
      \item Vermeidung: abstrakte Klassen oder Komposition
    \end{itemize*}
  \end{itemize*}

  Abstrakte Klassen
  \begin{itemize*}
    \item Nur Unterklassen, keine Instanzen
    \item Attribute in Unterklassen füllen
    \item Notation: Kursiv oder Stereotyp <<abstract>>
  \end{itemize*}

  Offen / Geschlossen-Prinzip [Meyer 1988]
  \begin{itemize*}
    \item Erweiterbarkeit eines Entwurfs
    \item Offen für Erweiterungen,
    \begin{itemize*}
      \item z.B. durch Vererbung / Polymorphie
      \item Virtuelle Operationen verwenden
      \item Verändert vorhandenes Verhalten nicht
      \item Erweiterung um zusätzliche Funktionen oder Daten
    \end{itemize*}
    \item Geschlossen für Änderungen
    \begin{itemize*}
      \item private Attribute
      \item Möglichst protected Operationen
    \end{itemize*}
    \item Beschränkung der Erweiterbarkeit
    \begin{itemize*}
      \item Keine Einschränkungen der Funktionalität der Basisklasse!
    \end{itemize*}
  \end{itemize*}

  Liskovsches Ersetzungsprinzip
  \begin{itemize*}
    \item Wenn S eine Unterklasse von T ist, dann können Objekte des Typs T in einem Programm durch Objekte des Typs S ersetzt werden, ohne die Funktion des Programms zu verändern. [Barbara Liskov 1987]
    \item Engere Definition als „ist-ein“-Beziehung
    \item Kein unerwartetes Verhalten eines Objektes eines Subtyps
    \item Methoden, die Objekte der Basisklasse erwarten, müssen auch mit Objekten der abgeleiteten Klasse funktionieren
    \item Zusicherungen der Basisklasse müssen von der abgeleiteten Klasse erfüllt werden!
  \end{itemize*}

  Gesetz von Demeter (LoD)
  \begin{itemize*}
    \item Gesetz von „schüchternen“ Objekten
    \item Objekte sollen nur mit Objekten in ihrer unmittelbaren Umgebung kommunizieren
    \item Aus einer Methode M dürfen (sollten) nur Nachrichten an Objekte gesendet werden, die ...
    \begin{itemize*}
      \item unmittelbarer Bestandteil des Objekts von M sind (super)
      \item M als Argument übergeben wurden
      \item direkt in M erzeugt wurden
      \item (oder sich in globalen Variablen befinden)
    \end{itemize*}
    \item Als Metrik überprüfbar
  \end{itemize*}

  Ein Objekt sollte
  \begin{itemize*}
    \item Nur Methoden aufrufen, die zur eigenen Klasse gehören
    \item Nur Methoden von Objekten aufrufen, die:
    \begin{itemize*}
      \item Von Attributen referenziert werden
      \item Als Parameter übergeben wurden
      \item Selbst erzeugt wurden
    \end{itemize*}
  \end{itemize*}


  \subsection{Entwurfsmodelle}
  Klassendiagramm
  \begin{itemize*}
    \item Eigenschaften
    \begin{itemize*}
      \item Modellierung der statischen Struktur (Aufbau)
      \item Modellierung der Struktur von Daten
      \item Klasse im Mittelpunkt (Aufbau, Beziehungen zueinander)
      \item Wichtigstes und bekanntestes Diagramm der UML!
    \end{itemize*}
    \item Elemente des Klassendiagramms
    \begin{itemize*}
      \item Klasse (Attribute, Operationen)
      \item Vererbung / Realisierung
      \item Assoziationen
      \item Beziehungen / Abhängigkeiten
    \end{itemize*}
    \item Attribute
    \begin{itemize*}
      \item Klassenattribut: "X" static - statisch, nur einmal pro Klasse vorhanden
      \item Sichtbarkeit
      \item "+" public - im Namensraum sichtbar
      \item "\#" protected - nur in abgeleiteten Klassen sichtbar
      \item "~" package - im Paket sichtbar
      \item "-" private - nur in der Klasse selbst sichtbar
      \item Ableitung "/" derived - abgeleitetes Attribut
    \end{itemize*}
    \item Weitere Eigenschaften
    \begin{itemize*}
      \item readOnly - nach Initialisierung nicht änderbar
      \item composite - Aggregation: Composition
      \item redefines X - überschreibe Attr. der Oberklasse
      \item subsets X - Teilmenge
      \item union - Attribut ist Vereinigung der subsets
      \item unique - Elemente eindeutig (Schlüsselattribut)
      \item ordered - Elemente sind geordnet (unordered)
      \item sequence - Speicherung der Elemente als Liste
      \item bag - Elemente sind Multimenge
    \end{itemize*}
    \item Parameterlisten
    \begin{itemize*}
      \item in: Eingangsparameter
      \item out: Ausgangsparameter
      \item inout: Eingangs- und Ausgangsparameter
      \item return: Rückgabewert
    \end{itemize*}
    \item Beziehungen
    \begin{itemize*}
      \item navigierbar/unspezifiziert/nicht-navigierbar
      \item ungerichtete/gerichtete Relation/assoziation
    \end{itemize*}
  \end{itemize*}

  Aktive Klassen
  \begin{itemize*}
    \item Reagieren nicht nur, sondern werden von sich aus aktiv
    \item Z.B. Steuerobjekte
    \item Als Thread oder Prozess realisiert
  \end{itemize*}

  \subsection{Schnittstellen}
  \begin{itemize*}
    \item Vereinbarung über Art des Aufrufs
    \begin{itemize*}
      \item Homogenität gleicher Funktionen
      \item Enthält:
      \item Spezifikation von Operationen
      \item keine Implementierung ( Java, nicht UML!)
      \item keine Attribute
      \item In Java außerdem anstelle von Mehrfachvererbung
    \end{itemize*}
    \item Schnittstellen in UML
    \begin{itemize*}
      \item Funktion ähnlich abstrakter Klasse
      \item Meist für technische Aspekte
      \item Notation: Stereotyp <<interface>> oder grafisch (lollipop notation)
    \end{itemize*}
    \item Verträge („design by contract“)
    \begin{itemize*}
      \item Schnittstelle sagt bisher nichts über Effekt der Klasse aus
      \item Vollständige Beschreibung wäre Programm?
      \item Vereinfachte Beschreibung für Abfolgen:
      \item Vorbedingung: Prädikat, das vor Aufruf gelten muss <<precondition>>
      \item Nachbedingung: Prädikat, das nach Aufruf gelten muss <<postcondition>>
      \item Invariante: Prädikat, das immer gilt <<invariant>>
      \item Jeweils Einschränkungen!
    \end{itemize*}
  \end{itemize*}

  Protokollrollen - Dynamisches Verhalten von Schnittstellen
  \begin{itemize*}
    \item Ohne Sicht auf innere Implementierung (anders als beim Objektlebenszyklus)
    \item Protokoll = Kollaboration von Protokollrollen (protocol, protocol role)
    \item Modell: Zustandsautomat
    \begin{itemize*}
      \item Genauer: Spezialisierung
      \item Beschreibung der Synchronisation von Objekten
    \end{itemize*}
  \end{itemize*}

  \subsection{Entwurfsmuster}
  \begin{itemize*}
    \item Warum Wiederverwendung?
    \begin{itemize*}
      \item Geringerer Aufwand
      \item Das Rad nicht noch einmal neu erfinden
      \item Verwenden üblicher, aus Erfahrung gewachsener Strukturen
    \end{itemize*}
    \item ... und warum nicht?
    \begin{itemize*}
      \item Aufwand für Anpassung kann hoch sein!
      \item Einarbeiten in teilweise komplexe Schnittstellen
      \item Abhängigkeit von externen Komponenten, Zwang zu späterer Portierung
    \end{itemize*}
  \end{itemize*}

  > Was ist ein Entwurfsmuster? Eine schematische Lösung für eine Klasse verwandter Probleme (Höhere Ebene: Architekturmuster)
  \begin{itemize*}
    \item Wie helfen Muster im Entwurf?
    \begin{itemize*}
      \item Identifizieren von Klassen (Anwendungs- und Lösungsdomäne)
      \item Regeln sind abstrakt oder an realen Objekten orientiert
      \item Muster: Arten von Rollen bzw. Lösungshinweise für typische Strukturierungsaufgaben
      \item Änderbarkeit und Lesbarkeit des Entwurfs verbessern
    \end{itemize*}
    \item Arten von Entwurfsmustern
    \begin{itemize*}
      \item Erzeugungsmuster
      \item Strukturmuster
      \item Verhaltensmuster
    \end{itemize*}
    \item Erzeugungsmuster
    \begin{itemize*}
      \item Factory Method, Fabrikmethode:
      \item Implementierungsvarianten; Erzeugung von Objekten wird an Unterklassen delegiert
      \item Abstract Factory, Abstrakte Fabrik:
      \item Schnittstelle zur Erzeugung von Familien verwandter Objekte
      \item Prototype, Prototyp:
      \item Objekterzeugung durch Vorlage und Kopie
      \item Builder, Erbauer:
      \item Trennung von Erzeugung und Repräsentation komplexer Objekte, für Erzeugung unterschiedlicher Repräsentationen
      \item Singleton:
      \item Sicherstellung, dass nur ein Objekt einer Klasse erzeugt wird, die einen globalen Zugriff bietet
    \end{itemize*}
  \end{itemize*}

  Strukturmuster
  \begin{itemize*}
    \item Adapter
    \begin{itemize*}
      \item Anpassung der (inkompatiblen) Schnittstelle einer Klasse oder eines Objekts an eine erwartete Schnittstelle
    \end{itemize*}
    \item Bridge, Brücke
    \begin{itemize*}
      \item Abstraktion (Schnittstelle) von Implementierung entkoppeln, um beide unabhängig zu ändern; Impl.-Klasse nur als Verweis
    \end{itemize*}
    \item Decorator, Dekorierer
    \begin{itemize*}
      \item Objekt dynamisch um Zuständigkeiten erweitern (Alternative zur Bildung von Unterklassen)
    \end{itemize*}
    \item Facade, Fassade
    \begin{itemize*}
      \item Einheitliche Schnittstelle zu einer Schnittstellenmenge, vereinfacht Zugriff
    \end{itemize*}
    \item Flyweight, Fliegengewicht
    \begin{itemize*}
      \item Gemeinsame Nutzung kleiner Objekte zur effizienten Verwendung großer Mengen davon (Speicheraufwand)
    \end{itemize*}
    \item Composite, Verbund, Kompositum
    \begin{itemize*}
      \item Zusammenfügen verschiedener Objekte zur Repräsentation von Teil-Ganzes-Beziehungen; Objekte und Kompositionen können einheitlich behandelt werden, Baumstruktur
    \end{itemize*}
    \item Proxy, Stellvertreter
    \begin{itemize*}
      \item Kontrollierter Zugriff auf Objekt durch vorgeschaltetes Stellvertreterobjekt
      \item Gründe: Schutz, entfernter Zugriff (remote proxy), smart pointer, Erzeugung on demand
    \end{itemize*}
  \end{itemize*}

  Adapter
  \begin{itemize*}
    \item Vorteile
    \begin{itemize*}
      \item Kommunikation unabhängiger Softwarekomponenten
      \item Einfache Erweiterung um zusätzliche Funktionalität
      \item Austausch der Komponente durch Änderung des Adapters leicht möglich
    \end{itemize*}
    \item Nachteile
    \begin{itemize*}
      \item Zusätzlicher Adaptierungsschritt benötigt Zeit
      \item Schlechte Wiederverwendbarkeit der Adapter
    \end{itemize*}
    \item Bekannte Verwendung, Spezialfälle
    \begin{itemize*}
      \item Fassade: Adapter eines Teilsystems
      \item Proxy: erweitert die Funktionalität bei gleicher Schnittstelle
      \item Brücke: keine Anpassung, sondern vorherige Strukturierung
    \end{itemize*}
  \end{itemize*}

  Verhaltensmuster
  \begin{itemize*}
    \item Command, Befehl
    \begin{itemize*}
      \item Befehl / Operation als Objekt kapseln (Parameterübergabe, Operations-Warteschlangen, logging, Rückgängig machen)
    \end{itemize*}
    \item Observer, Beobachter
    \begin{itemize*}
      \item 1-zu-n-Beziehung zwischen Objekten, so dass die Änderung des zentralen Objekts zu einer Benachrichtigung und Aktualisierung der n (abhängigen) Zustände führt
    \end{itemize*}
    \item Visitor, Besucher
    \begin{itemize*}
      \item Beschreibung und Kapselung einer zu definierenden Operation, die auf einer Objektmenge ausgeführt wird
    \end{itemize*}
    \item Interpreter
    \begin{itemize*}
      \item Repräsentation der Grammatik einer Sprache sowie Interpreter zur Analyse von Sätzen der Sprache
    \end{itemize*}
    \item Iterator
    \begin{itemize*}
      \item Sequentieller Zugriff auf die Elemente einer Sammlung ohne Kenntnis der Implementierung der Sammlung
    \end{itemize*}
    \item Memento
    \begin{itemize*}
      \item Internen Zustand eines Objekts erfassen und speichern, um Objektzustand wiederherstellen zu können
    \end{itemize*}
    \item Template Method, Schablonenmethode
    \begin{itemize*}
      \item Beschreibung des Skeletts eines Algorithmus mit Delegation der Einzelschritte an Unterklassen; Teilschritte können von Unterklassen geändert werden
    \end{itemize*}
    \item Strategy, Strategie
    \begin{itemize*}
      \item Ermöglicht Austausch verschiedener Implementierungen einer Aufgabe ohne Beeinflussung der sie benutzenden Objekte
    \end{itemize*}
    \item Mediator, Vermittler
    \begin{itemize*}
      \item Objekt, welches das Zusammenspiel einer lose gekoppelten Objektmenge in sich kapselt. Vermeidet direkten Bezug der Objekte untereinander und ermöglicht unabhängige Änderung des Zusammenspiels
    \end{itemize*}
    \item State, Zustand
    \begin{itemize*}
      \item Ermöglicht Objekt, sein Verhalten abhängig von seinem inneren Zustand zu ändern, als ob es die Klasse wechselt
    \end{itemize*}
    \item Chain of Responsibility, Zuständigkeitskette
    \begin{itemize*}
      \item Vermeidet direkte Kopplung von Auslöser und Empfänger einer Anfrage bzw. Operation. Mehrere Objekte werden nacheinander benachrichtigt, bis die Anfrage erledigt ist
    \end{itemize*}
  \end{itemize*}

  Bewertung Observer
  \begin{itemize*}
    \item Vorteile
    \begin{itemize*}
      \item Entkopplung von Komponenten und Schichten möglich
      \item Broadcast und selective Broadcast möglich
    \end{itemize*}
    \item Nachteile
    \begin{itemize*}
      \item Bei vielen Beobachtern: Benachrichtigung aufwendig
      \item Unerwartete Änderung, Änderungskaskaden und Rekursion
      \item Abmelden der Beobachter vor dem Löschen
    \end{itemize*}
    \item Bekannte Verwendung, Spezialfälle
    \begin{itemize*}
      \item Verwendung im Model-View-Controller Muster
      \item Qt: Signal / Slot-Prinzip ähnlich
    \end{itemize*}
  \end{itemize*}

  Anwendung von Entwurfsmustern
  \begin{itemize*}
    \item Untersuche Anwendbarkeit und Konsequenzen
    \item Analysiere Struktur, Teilnehmer und Kollaborationen
    \item Wähle aus dem Anwendungskontext Namen für Teilnehmer
    \item Spezifiziere die teilnehmenden Klassen
    \begin{itemize*}
      \item Deklariere Schnittstellen, Vererbung und Variablen
      \item Identifiziere existierende Entwurfsklassen, die durch das Muster beeinflusst werden
    \end{itemize*}
    \item Wähle anwendungsspezifische Namen für Operationen
    \item Implementiere Operationen entsprechend den Verantwortlichkeiten und Kollaborationen des Musters
  \end{itemize*}

  \subsection{Klassenbibliotheken und Komponenten}
  Klassenbibliotheken
  \begin{itemize*}
    \item Zusammenfassung von Modulen, Klassen, etc.
    \item Mit einem bestimmten (abstrakten) Zweck
    \begin{itemize*}
      \item Abstrakte Datenverwaltung, Templates
      \item Grundlegende System-Aufgaben
      \item Untere Kapselungs-Schicht des Laufzeitsystems oder der Programmierumgebung
      \item Numerische Routinen, Simulation, ...
    \end{itemize*}
    \item Wird in Anwendung eingebunden (importiert), API
    \begin{itemize*}
      \item Objekte instanziieren oder Klassen ableiten
    \end{itemize*}
    \item Meist passiv: Kontrollfluss wird von Anwendung gesteuert
    \item Beispiele: stdlib, MFC, GNU scientific library, Java 3D, IPP
  \end{itemize*}

  Komponentenbasierte Entwicklung
  \begin{itemize*}
    \item Bausteinorientierte Programmierung (component-ware)
    \item Softwareentwicklung: Konstruktion aus vorgegebenen Bausteinen
    \item Entsprechung für Wiederverwendung: Generische Bausteine (components)
    \begin{itemize*}
      \item Anpassbar, zusammensetzbar
    \end{itemize*}
    \item Werkzeuggestützte bzw. grafische Kompositionsmechanismen
    \item Beispiele: Java Beans, Enterprise Java Beans (EJBs), Microsoft COM+
    \item Komponenten-Entwicklung oft auch projektspezifisch
    \item Warum Komponenten
    \begin{itemize*}
      \item Monolithische, proprietäre Software führt zunehmend zu Problemen
      \item Zunehmend verteilte Anwendungen mit offener Struktur und Internet-Anbindung
      \item Zusammensetzen der Funktionalität aus standardisierten Elementen, die über offene Schnittstellen kommunizieren
      \item Komponenten sollen Flexibilität bei sich ändernden Anforderungen erhöhen
      \item Weg aus der „Software-Krise“?
    \end{itemize*}
    \item Eigenschaften von Komponenten
    \begin{itemize*}
      \item müssen von ihrer Umgebung und anderen Komponenten unabhängig und getrennt sein
      \item Kontextabhängigkeiten: benötigte Komponenten-Infrastruktur und Systemressourcen
      \item Kapseln ihre angebotenen Funktionen
      \item Werden immer als ganze Einheit eingesetzt; alle Bestandteile sind enthalten (Archiv-Datei)
      \item Sind nicht von Kopien ihrer selbst unterscheidbar
      \item Klare Spezifikation der Schnittstelle nötig; explizit definierte Interaktionen mit Komponenten und Umgebung
      \item Komposition durch Dritte: Endbenutzer, Komponenten-Hersteller und Komponenten-Integrator; meist nur kompilierter Code verfügbar
    \end{itemize*}
  \end{itemize*}

  Komponenten für Client/Server-Architekturen
  \begin{itemize*}
    \item Wichtige Aspekte
    \begin{itemize*}
      \item Transaktionen
      \item Sicherheit
      \item Ressourcenverwaltung
      \item Persistenz
    \end{itemize*}
    \item Komponentenkonzept für Server-Komponenten
    \begin{itemize*}
      \item meist unsichtbare Komponenten
      \item standardisierte Realisierung der wichtigen Eigenschaften für Client/Server-Anwendungen
      \item Realisierung: Enterprise Java Beans (EJBs) innerhalb eines Java Enterprise Edition Servers
    \end{itemize*}
  \end{itemize*}

  \subsection{Dokumentation}
  Dokumentation des Feinentwurfs
  \begin{itemize*}
    \item Möglichkeiten
    \begin{itemize*}
      \item Eigenständiges Dokument
      \item Erweiterung des Lastenhefts / Grobkonzepts
      \item Eingebettet in den Quellcode (Werkzeug, z.B. Javadoc)
    \end{itemize*}
    \item Inhalt
    \begin{itemize*}
      \item Ähnlich Grobkonzept
      \item Zusätzlich detaillierte Modelle
      \item Abwägungen des Objektentwurfs
      \item Klassenschnittstellen
    \end{itemize*}
  \end{itemize*}

  \newpage
  \section{Implementierung}
  Aufgaben der Implementierung
  \begin{itemize*}
    \item Aus Spezifikationen Programm(code) erzeugen
    \item Aufbauend auf Ergebnissen des Feinentwurfs
    \begin{itemize*}
      \item Algorithmen konzipieren
      \item Datenstrukturen realisieren
      \item Umsetzen in konkreter Programmiersprache
      \item Dokumentation
      \item Untersuchung des Zeit- und Speicherbedarfs
      \item Test und Verifikation
    \end{itemize*}
    \item „Programmieren im Kleinen“
  \end{itemize*}

  \subsection{Konventionen und Werkzeuge}
  Konventionen beim Programmieren
  \begin{itemize*}
    \item (Coding Rules, -conventions, -standards)
    \item Regeln für verständliche Programme
    \begin{itemize*}
      \item „wie“ sollte Quellcode formal und strukturell gestaltet sein
      \item Bezeichner, Einrückungen, Dokumentation, Dateien, ...
      \item Strukturierung: Block, Methode, Klasse, Package
    \end{itemize*}
    \item Firmenspezifische Regeln
    \begin{itemize*}
      \item Festlegung Entwurfsprinzipien (z.B. keine Mehrfachvererbung)
    \end{itemize*}
  \end{itemize*}

  Namenskonventionen
  \begin{itemize*}
    \item Klasse
    \begin{itemize*}
      \item (mit) Substantiv, „UpperCamelCase“
      \item Beispiele: Account, StandardTemplate
    \end{itemize*}
    \item Methode
    \begin{itemize*}
      \item (mit) Verb, Imperativ (Aufforderung), „lowerCamelCase“
      \item Beispiele: checkAvailability(), getDate()
    \end{itemize*}
    \item Attribut, Variable
    \begin{itemize*}
      \item (mit) Substantiv, „lowerCamelCase“
      \item Beispiele: anzahlAutos, fensterBreite
    \end{itemize*}
    \item Konstante
    \begin{itemize*}
      \item Nur Großbuchstaben, Worte mit "\_" zusammengesetzt
      \item Standardpräfixe: "MIN\_", "MAX\_", "DEFAULT\_", ...
      \item Beispiele: NORTH, BLUE, MIN\_WIDTH, DEFAULT\_SIZE
    \end{itemize*}
  \end{itemize*}

  Formatierungs-Richtlinien
  \begin{itemize*}
    \item Entsprechend Schachtelungstiefe einrücken, aber nicht zu weit
    \item Einheitliche Verwendung von Leerzeilen und Leerzeichen
    \item Einheitliche Dateistruktur verwenden
    \begin{itemize*}
      \item Eine .java-Datei pro Klasse
      \item Ein Verzeichnis für jedes package
    \end{itemize*}
    \item Werkzeuge: source beautifier, oft in IDEs enthalten
    \item Editor: syntax highlighting
    \item Navigationswerkzeuge
    \begin{itemize*}
      \item Auf- und Zuklappen, Inhaltsverzeichnis, tagging
      \item doxygen, Eclipse etc.
    \end{itemize*}
  \end{itemize*}

  Änderungsfreundlicher Code
  \begin{itemize*}
    \item Wahl von Variablen, Konstanten und Typen orientiert an der fachlichen Aufgabe, nicht an der Implementierung:
    \begin{itemize*}
      \item `typedef char name [NAME\_LENGTH]`
      \item `typedef char firstName [FIRST\_NAME\_LENGTH]`
    \end{itemize*}
    \item Symbolische Konstanten statt literaler Werte verwenden, wenn spätere Änderung denkbar
    \item Algorithmen, Formeln, Standardkonzepte in Methoden/Prozeduren kapseln
    \item Übersichtlichkeit: Zusammenhängende Einheit nicht größer als Editorfenster (40-60 Zeilen, 70 Zeichen breit)
    \item Strukturierte Programmierung (Regeln je nach Schärfe)
    \begin{itemize*}
      \item Kein goto verwenden (in anderen Sprachen als Java)
      \item switch nur mit break-Anweisung nach jedem Fall
      \item break nur in switch-Anweisungen verwenden
      \item continue nicht verwenden (Effekt ähnlich goto)
      \item return nur am Ende zur Rückgabe des Werts
    \end{itemize*}
    \item Übersichtliche Ausdrücke
    \begin{itemize*}
      \item Seiteneffektfreie Ausdrücke, schlecht: y += 12*x++;
    \end{itemize*}
    \item Variablen möglichst lokal und immer private deklarieren
    \item Wiederverwendung "äußerer" Namen vermeiden
  \end{itemize*}

  Werkzeuge
  \begin{itemize*}
    \item Integrated Development Environments (Eclipse, KDevelop)
    \item Compiler, Linker; Build / Make; Versionskontrolle (git, svn)
  \end{itemize*}

  \subsection{Code-Qualität}
  Portierbarer Code
  \begin{itemize*}
    \item Code, den man ohne Änderungen in ein anderes System (Compiler, Betriebssystem, Rechner) übertragen kann
    \begin{itemize*}
      \item Kein implementierungsabhängiges Verhalten!
    \end{itemize*}
    \item ANSI C++ Standard ist nicht vollständig definiert
    \begin{itemize*}
      \item Ist das Verhalten nicht festgelegt, unterscheidet der ANSI C++ Standard zwischen:
      \item Implementierungsabhängigem, unspezifiziertem oder undefiniertem Verhalten
      \item Code, welcher auf implementierungsabhängigem, unspezifiziertem oder undefiniertem Verhalten basiert, ist
      \item Nicht portabel und somit häufig verboten
      \item Wird unter Umständen ungewollt wegoptimiert
    \end{itemize*}
  \end{itemize*}

  Implementierungsabhängiges Verhalten
  \begin{itemize*}
    \item Compiler übersetzen bestimmte Sprachkonstrukte unterschiedlich, Ergebnis unterscheidet sich
    \item Voraussetzung:
    \item Verhalten ist konsistent festgelegt und dokumentiert
    \item Kompilierung von standardkonformem Code ist erfolgreich
    \item Beispiel: Speichergröße von Integer-Typen
    \item char kann signed oder unsigned sein: Nicht damit rechnen!
    \item 32 Bit System ist wie erwartet
    \item 16 Bit System: Multiplikation wird mit int durchgeführt $\rightarrow$ Überlauf $\rightarrow$ undefiniertes Verhalten
  \end{itemize*}

  \begin{itemize*}
    \item Unspezifiziertes Verhalten:
    \item Wie implementierungsabhängiges Verhalten
    \item Compiler muss sich für ein bestimmtes Verhalten entscheiden
    \item Muss nicht dokumentiert sein
    \item Beispiel: Evaluierungsreihenfolge von Funktionsargumenten `tuWas(zuerstDas(),oderDochLieberDas());`
    \item Undefiniertes Verhalten:
    \item Keinerlei Vorgaben
    \item Compiler muss mögliches Problem nicht melden
    \item Keine Voraussage welches Resultat eintritt
    \item Bereits die Kompilierung kann fehlschlagen
    \item Oder das laufende Programm kann falsche Resultate liefern.
    \item Effekt: „Bei mir läuft es aber!?“
    \item „undefiniertes Verhalten nutzen grenzt an Sabotage!“
  \end{itemize*}

  Sicherer Code mit const
  \begin{itemize*}
    \item Const Variable - Konstante
    \begin{itemize*}
      \item Stellt sicher, dass sich der Wert nicht verändert
    \end{itemize*}
    \item Const Parameter
    \begin{itemize*}
      \item Übergabeparameter ändert sich nicht innerhalb der Operation
      \item Z.B. bei Übergabe komplexer Daten als Referenz bzw. Zeiger `long calcMeanValue(const image \&i){...}`
    \end{itemize*}
    \item Const Operationen
    \begin{itemize*}
      \item Sicherstellen, dass Operation das Exemplar nicht ändert
      \item Aufruf der const Operation bei const Variablen möglich
    \end{itemize*}
    \item Verwende const wenn möglich
  \end{itemize*}


  \subsection{Dokumentation}
  \begin{itemize*}
    \item Selbstdokumentierende Programme?
    \begin{itemize*}
      \item 2001 Int. Obfuscated C Code Contest Winner, Short Program
    \end{itemize*}
  \end{itemize*}

  Integrierte Dokumentation
  \begin{itemize*}
    \item Verständlichkeit, Wartbarkeit - auch für Programmierer!
    \item Code selbst sollte möglichst verständlich sein
    \item Dokumentation in Programm schreiben und aktualisieren
    \item Beschreibung der Bedeutung des Codes!
    \item Als Konventionen festschreiben
    \item Programmvorspann
    \item Kurzbeschreibung Datei / Klasse / Funktion ...
    \item Verwaltungsinformationen
    \begin{itemize*}
      \item Autor, Datum, Version, Projekt, ToDo, FixMe, ...
      \item Zustand: geplant, in Bearbeitung, vorgelegt, akzeptiert
    \end{itemize*}
    \item Laufende Kommentare im Quellcode
  \end{itemize*}

  Programmierer-Dokumentation
  \begin{itemize*}
    \item Als eigenes Dokument elektronisch oder gedruckt
    \item Einstieg in Programmverständnis (z.B. Bachelor-Arbeit)
    \item Konsistenz mit Quelltext? Verweise?
    \item Technische Unterstützung: JavaDoc (Java), doxygen (C++)
    \item Ergänztes Java-Programm: Dokumentation HTML, PDF,
  \end{itemize*}
  %  ```js
  %  /** @author name Mustermann */
  %  /** @param name description */
  %  /** @return description */
  %  ```

  Benutzerdokumentation
  \begin{itemize*}
    \item Benutzer-Handbuch, Online-Dokumentation
    \item Unterstützung ohne Support?
    \item Vollständige und fehlerfreie Beschreibung der Benutzung
    \begin{itemize*}
      \item Beispiele, screen shots
    \end{itemize*}
    \item Arten: Tutorial, Beschreibung,
  \end{itemize*}

  Benutzer-Unterstützungssysteme
  \begin{itemize*}
    \item Integrierte Hilfe (Suchfunktion, balloon help / tool tips)
    \item Assistenz-System (Zustandsabhängige Anleitung)
    \item Tutor-System zum Erlernen
    \item Bug-Listen, Mailinglisten, Diskussionsforen
  \end{itemize*}

  \subsection{Codegenerierung}
  Bezug zwischen Modell und Programmcode
  \begin{itemize*}
    \item Vorwärtsmodellierung: Modell - Code
    \item Rückwärtsmodellierung: Code - Modell
    \begin{itemize*}
      \item Außerdem: Modelltransformation, Refaktorisierung
    \end{itemize*}
    \item Idealfall: Automatische Übersetzung durch SW-Werkzeug (in beiden Richtungen)
    \begin{itemize*}
      \item „Modellbasierte Entwicklung“
    \end{itemize*}
    \item Statisch: Beispiel Klassendiagramm - Quelltext der Klassen mit allen Vererbungsbeziehungen, Attributen und Methodensignaturen (Klassen-Stümpfe mit leeren Methodenrümpfen zum Ausfüllen)
    \item Dynamisch: Beispiel Zustandsdiagramm - Quelltext der Zustandssteuerung einer Klasse
  \end{itemize*}

  %![Codegenerierung](Assets/Softwaretechnik1_Codegenerierung.png)

  Weitere statische Transformationen
  \begin{itemize*}
    \item Abbildung von Assoziationen auf Sammlungen
    \item Abbildung von Verträgen auf Ausnahmen
    \item Abbildung von Objektmodellen auf Datenbankschemata
    \item Abbildung von Entwurfsmustern auf Codefragmente
  \end{itemize*}

  Optimierung des Entwurfsmodells
  \begin{itemize*}
    \item Grund: nichtfunktionale Eigenschaften
    \item Zugriffspfade
    \item Klassen in Attribute umwandeln
    \item Verzögerung von Berechnungen
    \item Zwischenspeicherung aufwändiger Ergebnisse
  \end{itemize*}

  Codegenerierung aus StateCharts
  \begin{itemize*}
    \item Einfachste Möglichkeit: Switch (Case) Statement
    \item Zustände werden durch Datenwerte repräsentiert
    \begin{itemize*}
      \item Aktueller Zustand: einzelne skalare Variable
    \end{itemize*}
    \item Jedes Ereignis wird durch Methode implementiert
    \item Ausgehend von aktivem Zustand wird bei Eintreffen eines Ereignisses der entsprechende Programmcode ausgeführt
    \item Abhängig von Zustandsvariable wird Aktion ausgeführt und der Folgezustand eingestellt
    \item Wird in einer Klasse realisiert
    \item Sinnvoll für einfache, “flache” Modelle
    \begin{itemize*}
      \item Sonst Logik für Hierarchie nötig
    \end{itemize*}
  \end{itemize*}

  Anpassung der Generierung
  \begin{itemize*}
    \item Verschiedene Zielsprachen (Java, C++, ...)
    \item Model2Text-Transformationen
    \begin{itemize*}
      \item Verschiedene Generatoren, z.B. Eclipse Modelling Project
    \end{itemize*}
    \item Generierung aus dem Modellierungswerkzeug
    \begin{itemize*}
      \item Parametrisierung der Codegenerierung
      \item Generierungsvorlagen
    \end{itemize*}
  \end{itemize*}

  Weitere Werkzeuge
  \begin{itemize*}
    \item Compiler-Compiler: Syntaxbeschreibung wird in lexikalische Analyse (tokenizer) und Syntaxanalyse-Programm transformiert (lex \& yacc / flex \& bison / antlr)
    \item Codegenerierung für grafische Benutzungsoberflächen aus grafischer Beschreibung: GUI toolkits
    \item XML-Parser
    \begin{itemize*}
      \item XSLT, DOM, SAX, ...
    \end{itemize*}
  \end{itemize*}

  \subsection{Implementierung aktiver Objekte}
  Realisierung aktiver Entwurfsobjekte
  \begin{itemize*}
    \item Reagieren nicht nur (Methodenaufruf), sondern implementieren eigenes Verhalten
    \item Aktive Klassen, z.B. Steuerobjekte
  \end{itemize*}

  Arten von Programmabarbeitung
  \begin{itemize*}
    \item Sequentiell: es gibt immer genau einen nächsten Schritt, alle Schritte werden nacheinander ausgeführt
    \item Parallel: Spezielle Hardware bzw. Mehrkernprozessor, mehrere Befehlsfolgen werden echt parallel bearbeitet
    \item Quasi-parallel: Ein Prozessor arbeitet mehrere Befehlsfolgen in freier Einteilung ab
    \item Nebenläufig: Oberbegriff für Parallel und Quasi-parallel
    \begin{itemize*}
      \item concurrent
    \end{itemize*}
  \end{itemize*}

  Vorteile
  \begin{itemize*}
    \item Höhere Geschwindigkeit
    \item Kein aktives Warten auf Ereignisse
    \item Getrennte Implementierung unabhängiger Aspekte
  \end{itemize*}

  Ergebnisse eines Programms
  \begin{itemize*}
    \item Ein Programm, dessen Ablauf eindeutig vorherbestimmt ist, nennt man deterministisch (deterministic)
    \item Ein Programm, das bei gleichen Eingaben gleiche Ausgaben produziert, heißt determiniert (determined)
    \item Programme in üblichen Programmiersprachen sind sequentiell, deterministisch und determiniert
    \item Grund: Herkömmliche Programmiersprachen sind durch das von-Neumann-Modell geprägt
    \item Determinismus nicht notwendig für Determiniertheit!
    \begin{itemize*}
      \item Determiniertheit nebenläufiger Programme: Synchronisation
      \item Vermeidung von Schreib/Schreib und Schreib/Lese-Konflikten
    \end{itemize*}
  \end{itemize*}

  Java Threads
  \begin{itemize*}
    \item Verwaltung durch die Java Virtuelle Maschine (JVM)
    \item Realisierung der Threads ist je nach Implementierung der JVM unterschiedlich
    \begin{itemize*}
      \item Abbildung auf Betriebssystem-Threads (z.B. unter Windows weitverbreitet)
      \item Realisierung durch die JVM (z.B. unter Unix und in Java-fähigen Browsern)
      \item Nachteile: Keine Ausnutzung von Multiprozessorsystemen durch die VM; Zuteilungsstrategie für Threads ist in derzeitigen Implementierungen unterschiedlich
    \end{itemize*}
    \item Threads arbeiten immer im Adressraum der JVM (eigener Prozess) und sind außerhalb dieser nicht sichtbar
  \end{itemize*}

  Erzeugung eines Threads
  \begin{itemize*}
    \item Unterklasse der Basisklasse „Thread“ bilden `class MyThread extends Thread`
    \item Problem: keine Mehrfachvererbung, daher Alternative nötig (Beispiel: Applet):
    \begin{itemize*}
      \item Schnittstelle „Runnable“ implementieren
      \item `class MyThread implements Runnable`
    \end{itemize*}
    \item Die vordefinierte Schnittstelle Runnable ist definiert als
    % ```java
    % public interface Runnable
    % { public abstract void run(); }
    % ```
  \end{itemize*}

  Starten eines Threads
  \begin{itemize*}
    \item Eine Klasse, die Runnable implementiert, muss wie Unterklassen von Thread immer eine run()-Methode definieren
    \item Seiteneffekt der Runnable-Schnittstelle
    \begin{itemize*}
      \item Instanzen der Klasse werden nebenläufig zu den anderen laufenden Threads ausgeführt
      \item Ausführung beginnt mit der Methode run ()
    \end{itemize*}
    \item Ablauf
    \begin{itemize*}
      \item Thread-Objekt erzeugen
      \item Thread starten mit t.start()
      \item start() ruft implizit run() auf
    \end{itemize*}
  \end{itemize*}

  Synchronisation von Threads
  \begin{itemize*}
    \item Gezielte Einschränkung der Nebenläufigkeit
    \item Gründe
    \begin{itemize*}
      \item Zugriffsbeschränkung, gegenseitiger Ausschluss
      \item Abhängigkeiten, einseitige Synchronisation
    \end{itemize*}
    \item Methoden: Semaphore, Monitore, Schlossvariablen, ...
  \end{itemize*}

  Java: Monitore
  \begin{itemize*}
    \item Zugriffsoperationen werden in Klassen zusammengefasst
    \item Gegenseitiger Ausschluss: Spezifikation der betroffenen Zugriffsoperation als synchronized
  \end{itemize*}

  \subsection{Verifikation und Testen}
  Wie erreicht man qualitativ hochwertige Software?
  \begin{itemize*}
    \item Wissen, Erfahrung und Methodenkompetenz der Programmierer
    \item Projektstruktur, klare Verantwortlichkeiten
    \item Kosten- und Zeitdruck? Änderungen?
    \item Programmier- und Testmethoden
    \begin{itemize*}
      \item pair programming, code reading etc.
      \item Qualitätsverantwortlicher, automatisiertes Testen
    \end{itemize*}
    \item Technische Unterstützung
    \begin{itemize*}
      \item Z.B. Versionierung, Dokumentation, Testen, Entwicklungsumgebung
    \end{itemize*}
  \end{itemize*}

  Begriffe
  \begin{itemize*}
    \item Zuverlässigkeit: Maß für Übereinstimmung des Systemverhaltens mit Spezifikation
    \item Grund für Unzuverlässigkeit:
    \begin{itemize*}
      \item Fehler (bug, fault): fehlerhafter Programmcode o.ä.
      \item Der Begriff „Bug“:
      \begin{itemize*}
        \item Schon vor Computern als Begriff für Fehler benutzt
        \item Motte im Relais des Computers Mark II Aiken (1947)
      \end{itemize*}
    \end{itemize*}
    \item Fehlerhafter Zustand (error): Fehler hat zur Laufzeit zu einem internen fehlerhaften Zustand geführt, der möglicherweise zu einem Ausfall führt
    \item Störfall, Ausfall (failure): Abweichung vom spezifizierten Verhalten, meist mit negativen Folgen
  \end{itemize*}

  Vergleich System / Systemmodell
  \begin{itemize*}
    \item Anspruch guter Software: System entspricht Systemmodell (Korrektheit)
    \item Problem: System nicht vollständig automatisch erzeugbar!
    \item Auswege
    \begin{itemize*}
      \item Fehlervermeidung (Inspektion, pair programming, ...)
      \item Nachweis, dass System dem Modell entspricht - Verifikation
      \item Überprüfen, ob System dem Modell entspricht - Testen
      \item Fehlertoleranz (durch Redundanz)
    \end{itemize*}
  \end{itemize*}

  Verifikation
  \begin{itemize*}
    \item Mathematisch formaler Beweis, dass ein Programm einer Spezifikation genügt
    \item Vorteil: wenn anwendbar, dann vollständiger Beweis
    \item Problem: für viele (realistisch große) Fälle nicht anwendbar
    \begin{itemize*}
      \item Zu aufwändig
      \item Umgebung muss ebenfalls verifiziert werden
      \item Auch in der Theorie nicht immer entscheidbar: Halteproblem, Gödelscher Unvollständigkeitssatz
    \end{itemize*}
    \item Theoretische Informatik: Berechenbarkeitstheorie, formale Semantik; aktives Forschungsgebiet
    \begin{itemize*}
      \item model checking
    \end{itemize*}
  \end{itemize*}

  Testen
  \begin{itemize*}
    \item Systematischer Versuch, Defekte in der Software zu finden
    \item Ingenieurtechnik zur Erhöhung des Vertrauens in Softwaresysteme, aber: unvollständig!
    \begin{itemize*}
      \item Kann nur die Anwesenheit von Fehlern nachweisen, aber nicht Korrektheit (Abwesenheit von Fehlern)!
    \end{itemize*}
    \item Aufgabe: Unterschiede zwischen Modell und System finden
    \item Destruktiv im Gegensatz zu sonstigen SWE-Aufgaben
    \begin{itemize*}
      \item Daher sollten nicht (nur) Entwickler selbst testen
    \end{itemize*}
  \end{itemize*}

  Testplanung
  \begin{itemize*}
    \item Testen ist aufwändig, deshalb ist gute Planung nötig!
    \item Testplanung sollte bereits mit der Anforderungsanalyse beginnen und im Entwurf verfeinert werden (V-Modell, Test-First-Ansatz)!
    \item Typische Bestandteile einer Test-Spezifikation (Testdrehbuch)
    \begin{itemize*}
      \item Phasenmodell des Testprozesses
      \item Zusammenhang zur Anforderungsspezifikation, z.B. dort festgelegte Qualitätsziele
      \item Zu testende Produkte
      \item Zeitplan für die Tests
      \item Abhängigkeiten der Testphasen
      \item Aufzeichnung der Testergebnisse
      \item Hardware\item und Softwareanforderungen
    \end{itemize*}
  \end{itemize*}

  Arten von Tests
  \begin{itemize*}
    \item Komponententest: Fehler in einzelnen Objekten oder Subsystemen, losgelöst vom umgebenden System
    \begin{itemize*}
      \item Umgebung muss nachgebildet werden
    \end{itemize*}
    \item Integrationstest: Zusammenspiel von Komponenten
    \begin{itemize*}
      \item Vollständiges System: Systemtest; Szenarios
    \end{itemize*}
    \item Strukturtest: innere Zustände, Interaktionen
    \item Funktionstest: Anforderungen aus Lastenheft
    \item Leistungstest: nichtfunktionale Anforderungen
    \item Benutzbarkeitstest: Fehler in der Benutzungsschnittstelle, Verständlichkeit, Akzeptanz bei Anwendern
    \begin{itemize*}
      \item Prototypen
    \end{itemize*}
    \item Akzeptanztest, Installationstest: Kunde, Abnahme
  \end{itemize*}

  Komponententests
  \begin{itemize*}
    \item Überprüft Verhalten einer Systemkomponenten im Vergleich zur Spezifikation
    \item Da Tests bereits frühzeitig stattfinden sollten, ist Umgebung meist nicht vollständig implementiert
    \begin{itemize*}
      \item Teststumpf (stub, dummy) simuliert aufgerufene Komponenten
      \item Testtreiber simuliert aufrufende Komponenten
    \end{itemize*}
    \item Vorgehensweisen
    \begin{itemize*}
      \item Bottom-up
      \item Top-down
      \item Sandwich
      \item Schichtenweises Testen
    \end{itemize*}
  \end{itemize*}

  Systematisches Testen
  \begin{itemize*}
    \item Testfall
    \begin{itemize*}
      \item Beschreibung, Name
      \item Zu testende Komponente, Testgegenstand (Pfad, Aufrufart)
      \item Eingabedaten (Testdaten)
      \item Erwartete Ergebnisse („Orakel“)
      \item Protokoll (erzeugte Ausgaben)
      \item Bewertung des Ergebnisses
    \end{itemize*}
    \item Weitere Begriffe
    \begin{itemize*}
      \item Regressionstest: erneute Durchführung eines Tests anhand einer geänderten Version des Testgegenstands
      \item Alphatest: Test eines Prototypen durch Benutzer
      \item Betatest: Test der vollständigen Software durch Benutzer
    \end{itemize*}
  \end{itemize*}

  Funktionaler Test (black box test)
  \begin{itemize*}
    \item Testfallauswahl beruht auf Spezifikation
    \item Ohne Wissen über inneren Aufbau
    \item E/A-Zusammenhang
  \end{itemize*}

  Äquivalenzklassen im funktionalen Test
  \begin{itemize*}
    \item Problem: alle Kombinationsmöglichkeiten der Eingangsdaten sind zu umfangreich für vollständigen Test
    \item Mögliche Einschränkung: Bildung von Äquivalenzklassen der Eingangsdaten, für die ähnliches Verhalten erwartet wird
    \item Basierend auf Anwendungsdomäne
    \item Äquivalenzklasse = Teilmenge der möglichen Datenwerte der Eingabeparameter
    \item Test je eines Repräsentanten jeder Äquivalenzklasse
    \item Finden von Äquivalenzklassen
    \begin{itemize*}
      \item Zulässige / unzulässige Teilbereiche der Datenwerte
      \item Unterteilung der Bereiche nach erwarteten Ausgabewerten
    \end{itemize*}
  \end{itemize*}

  Grenztests
  \begin{itemize*}
    \item Ergänzung von Äquivalenztests: Spezialfälle
    \item Rand der Äquivalenzklasse
    \item Außerdem: Sonderfälle, erwartete Problemfälle (technisch)
  \end{itemize*}

  Strukturtest (white box test, glass box test)
  \begin{itemize*}
    \item Testfallauswahl beruht auf Programmstruktur
    \item Wie erreicht man möglichst vollständige Abdeckung?
    \item Kontrollflussorientiert
    \begin{itemize*}
      \item Anweisungsüberdeckung anhand Quellcode
      \item Zweigüberdeckung und
      \item Pfadüberdeckung anhand des Flussgraphen reduzierte Variante: bounded interior Pfadtest
    \end{itemize*}
    \item Datenflussorientiert
    \begin{itemize*}
      \item defines / uses-Verfahren: Abarbeitungspfade von Definition zu jeder Verwendung von Variable oder Objekt durchlaufen
    \end{itemize*}
    \item Zustandsorientiert
  \end{itemize*}

  \subsection{Testaktivitäten und Werkzeuge}
  Wann wird getestet?
  \begin{itemize*}
    \item Während der Implementierung!
    \begin{itemize*}
      \item Auch wenn Schreiben von Tests scheinbar unproduktiv ist
      \item Tests sind unbeliebt, da Probleme aufgezeigt werden
      \item Aber: spätes Testen erhöht Aufwand!
    \end{itemize*}
    \item Inkrementell, d.h. für jede überschaubare Änderung
    \item Spezielle Vorgehensweise: test first-Ansatz
  \end{itemize*}

  Wie wird getestet? Geplant, systematisch, möglichst automatisiert, dokumentiert, wiederholbar

  Testplanung
  \begin{itemize*}
    \item Iterative Erstellung eines Testplans / Testdrehbuchs
    \item Aktivitäten - Festlegen von
    \begin{itemize*}
      \item Teststrategie (Testumfang, Testabdeckung, Risikoabschätzung)
      \item Testziele und Kriterien für Testbeginn, -ende und -abbruch
      \item Vorgehensweise (Testarten)
      \item Testumgebung (Beschreibung)
      \item Hilfsmittel und Werkzeuge zum Testen
      \item Testspezifikation
      \item Testorganisation (Termine, Rollen, Ressourcen)
      \item Fehlermanagement (Werkzeugunterstützung)
    \end{itemize*}
    \item Ergebnis: Inhalt des Testplans
  \end{itemize*}

  Testspezifikation
  \begin{itemize*}
    \item Auswahl der zu testenden Testfälle
    \item Definition einzelner Testfälle
    \begin{itemize*}
      \item Grundlage: Anwendungsfälle, Anforderungen, Fehlermeldungen (Zuordnung notwendig)
    \end{itemize*}
    \item Aktivitäten
    \begin{itemize*}
      \item Testfallfindung und Testfalloptimierung
      \item Testfälle beschreiben (was genau ist zu testen)
      \item Randbedingungen finden (Abhängigkeiten zu anderen Testfällen)
      \item Festlegen und Erstellen der Eingabedaten
      \item Festlegungen zum Testablauf und zur Testreihenfolge
      \item Festlegen Soll-Ergebnis
      \item Festlegung der Bedingung(en) für 'Test erfüllt'; ...
    \end{itemize*}
  \end{itemize*}

  Testvorbereitung
  \begin{itemize*}
    \item Tätigkeiten im Vorfeld des Tests
    \item Aktivitäten
    \begin{itemize*}
      \item Vorbereitung der Tests entsprechend Testplan
      \item Bereitstellen der Dokumente (Testspezifikation)
      \item Verfügbar machen von Werkzeugen (Fehlermanagement)
      \item Aufbauen der Testumgebung(en)
      \item Integration der Entwicklungsergebnisse in die Testumgebung (Software installieren, konfigurieren, ...)
      \item Bereitstellung von Testdaten bzw. Eingabedaten in die Testumgebung
      \item Benutzer und Benutzerrechte anlegen
    \end{itemize*}
  \end{itemize*}

  Testdurchführung
  \begin{itemize*}
    \item Durchführung der spezifizierten Tests
    \begin{itemize*}
      \item Manuell
      \item Automatisiert
    \end{itemize*}
    \item Aktivitäten
    \begin{itemize*}
      \item Auswählen der zu testenden Testfälle
      \item Bereitstellen der Testdaten
      \item Starten, Überwachen, Abbrechen, Beenden
      \item Erfassung des Ist-Ergebnisses zur Auswertung
      \item Besondere Vorkommnisse dokumentieren
      \item Umgebungsinformationen für den Testlauf archivieren, ...
    \end{itemize*}
  \end{itemize*}

  Testauswertung
  \begin{itemize*}
    \item Überprüfung der Ergebnisse
    \item Vergleich Ist-Ergebnis mit Soll-Ergebnis
    \item Entscheidung Testergebnis (ok / Fehler)
    \item Bei Fehler:
    \begin{itemize*}
      \item Klassifizierung (z. B.: Fehlerursache, Fehlerschwere etc.)
      \item Fehler reproduzieren!
      \item Angemessene Fehlerbeschreibung und –erläuterung: Nur ausreichend dokumentierte Fehler sind gültig und können bearbeitet werden!
      \item Fehler im Fehlermanagement eintragen (Bug report)
      \item Testfall bleibt offen
    \end{itemize*}
  \end{itemize*}

  Bugbeschreibung
  \begin{itemize*}
    \item Möglichst genaue Beschreibung des Fehlers
    \item Ziel ist die Reproduzierbarkeit des Fehlers
    \item Zuordnung zu Projektplan: Meilenstein, Version
    \item Fehlerklassifikation
    \begin{itemize*}
      \item defect: Fehler in einer bestehenden Funktionalität
      \item enhancement / feature: Funktionale Anforderung oder Erweiterung der bestehenden Funktionalität
      \item task: Allgemeine Aufgabe
    \end{itemize*}
    \item Priorität festlegen
    \begin{itemize*}
      \item Unterschiedliche Stufen
      \item Festlegung innerhalb eines Unternehmens / Projektes
    \end{itemize*}
    \item Prioritäten von Fehlern (bugs)
  \end{itemize*}

  %| Stufe | Bedeutung | Verhalten |
  %| -\item | -\item | -\item |
  %| blocker | Schwerwiegender Fehler. Führt zur Arbeitsunfähigkeit anderer (Entwicklung, QA, Produktion,...). | Verdrängt alle anderen Aufgaben! Muss sofort gelöst werden, um Arbeitsfähigkeit wieder herzustellen. |
  %| critical | Schwerer Fehler. Eine grundlegende Funktionalität des Systems ist nicht gegeben. Fehler muss bis zur nächsten Version behoben werden. System kann nicht verwendet werden. | Alle anderen Fehler (außer blocker) werden später gelöst. Bis zum nächsten Release muss dieser Bug gelöst sein. |
  %| major | „normaler“ Fehler. Es tritt ein Fehler innerhalb einer Funktion des Systems auf. Das System kann jedoch weiterhin verwendet werden. | Fehler wird normal eingeplant. |
  %| minor | Kleiner Fehler. Fehler behindert nicht weiter die Funktionsfähigkeit des Systems. Einfacher Work-Around vorhanden? | Je nach Aufwand und Zeitplanung erledigen. |
  %| trivial | Schönheitsfehler. Fehler stört während des Betrieb des Systems nicht. | Je nach Aufwand „zwischendrin“ erledigen. Alle anderen Bugs bevorzugt lösen. |

  Testabschluss
  \begin{itemize*}
    \item Zusammenfassen der Tests
    \item Gesamtstatus dokumentieren und kommunizieren
    \item Entscheidungen herbeiführen z.B.: Auslieferung?
    \begin{itemize*}
      \item Ziele erreicht - nächste Schritte (Auslieferung)
      \item Tests vorzeitig beenden oder unterbrechen
      \item Gründe dokumentieren
      \item Vollständiger Nachtest oder Teiltest möglich?
    \end{itemize*}
    \item Unterlagen archivieren
  \end{itemize*}

  Testautomatisierung
  \begin{itemize*}
    \item Automatische Code-Validierung
    \begin{itemize*}
      \item Statisch: lint (C), Compiler
      \item Dynamisch: run-time checking, memory leaks etc.
    \end{itemize*}
    \item Beispiel: Test-Framework JUnit
  \end{itemize*}

  Behebung funktionaler Fehler
  \begin{itemize*}
    \item Log-Ausschriften bzw. Signale
    \item Debugger: Vorwärts / Rückwärts
    \item Haltepunkte setzen: Bedingte Haltepunkte für Schleifen
    \item Darstellung der Variablenbelegung, Werte setzen
    \item Analyse des Aufruf-Stacks
  \end{itemize*}

  Behebung nichtfunktionaler Fehler
  \begin{itemize*}
    \item Geschwindigkeit: profiling z.B. mit Eclipse TPTP
    \item Aufrufe, Zeitverbrauch in Methoden usw.
  \end{itemize*}

  Memory Leaks, JProbe
  \begin{itemize*}
    \item Runtime heap summary: Welche Objekte verbrauchen Speicher?
    \item Reference graph: Wer referenziert diese Objekte, so dass sie nicht per garbage collection gelöscht werden?
  \end{itemize*}

  \subsection{Softwareverteilung}
  Softwareverteilung (deployment)
  \begin{itemize*}
    \item Prozess zur Installation von Software auf
    \begin{itemize*}
      \item Anwender-PC‘s, Servern, Maschinen in Produktion ...
    \end{itemize*}
    \item Steuerung der Installation der Software
    \item Voraussetzungen für die Software schaffen
    \begin{itemize*}
      \item Schulungen planen und durchführen
    \end{itemize*}
    \item Softwareverteilung ist ein kritischer Prozess!
    \begin{itemize*}
      \item Fehler können zu vielen Störungen und Ausfällen führen
    \end{itemize*}
    \item Ziele
    \begin{itemize*}
      \item Automatische Installation, Konfiguration und Wartung einer großen Anzahl von Systemen mit geringem Aufwand
      \item Erreichen eines störungsarmen und sicheren Betriebs
      \item Möglichst einheitliche Softwareversionen auf allen Systemen
    \end{itemize*}
  \end{itemize*}

  Installationsarten
  \begin{itemize*}
    \item Erstinstallation
    \item Software-Update (Software-Aktualisierung)
    \begin{itemize*}
      \item Aktualisierung der Software, Daten oder Konfiguration
    \end{itemize*}
    \item Hotfixes und Service Packs
    \begin{itemize*}
      \item Nur bestimmte Teile der Software werden aktualisiert
      \item Meist nur Fehlerbehebung, keine neuen Features
    \end{itemize*}
    \item Upgrade
    \begin{itemize*}
      \item Erweitert eine Software deutlich um neue Funktionen
    \end{itemize*}
    \item Unbeaufsichtigte (automatische) Installation
    \begin{itemize*}
      \item Installation erfolgt ohne Benutzereingriff
      \item Periodisch, durch Aufruf des Anwenders, beim Programmstart
      \item Einstellungen in einem Skript festgelegt oder werden als Parameter übergeben
    \end{itemize*}
  \end{itemize*}

  Installationshilfsmittel
  \begin{itemize*}
    \item Installationsprogramm (Installer)
    \begin{itemize*}
      \item Windows: Windows Installer, InstallShield
      \item Linux: RPM, Port, APT
      \item MAC-OS: Installer, WarpIn
    \end{itemize*}
    \item Installations-Script
    \item Installationsanweisung
  \end{itemize*}

  Software-Rollout
  \begin{itemize*}
    \item Vorgang des Veröffentlichens und Verteilens von Softwareprodukten auf entsprechende Clients
    \item Anzahl der Clients kann weit über 10.000 liegen!
    \item Rollout abhängig von verschiedenen Kriterien (Vorherige Installation, Hardwarekonfiguration, Zeit, Kunde)
    \item Rollout-Varianten
    \begin{itemize*}
      \item Zentral / Dezentral
      \item Manuell (Benutzer löst Installation aus)
      \item Automatisiert (ohne Benutzerinteraktion)
    \end{itemize*}
  \end{itemize*}

  Vollständige Verteilung (big bang)
  \begin{itemize*}
    \item Alle Installationen werden mit einem Mal installiert
    \item Sehr hohes Risiko
    \item Eventuelle Fehler führen zu vielen fehlerhaften Zuständen oder Störfällen - führt zu hohem Druck
    \item Eventuelle Fehler müssen schnell gelöst werden (Provisorium)
    \item Sehr kurze Einführungsphase
    \item Rollback-Mechanismus sehr empfohlen
  \end{itemize*}

  Rollback
  \begin{itemize*}
    \item Wiederherstellung des Ursprungszustands
    \item Technische Realisierung muss ermöglicht werden
  \end{itemize*}

  Pilotierte Einführung
  \begin{itemize*}
    \item Einführung für wenige ausgewählte Installationen
    \item Sehr hohe Sicherheit
    \item Festlegung der späteren Rollout-Schritte
    \item Benötigt zusätzliche Zeit
    \item Geringere Auftrittswahrscheinlichkeit von Fehlern
  \end{itemize*}

  Schrittweise (iterative) Einführung
  \begin{itemize*}
    \item Einführung erfolgt schrittweise mit einer definierten Anzahl
    \item von Installationen (Rampe - ramp-up)
    \item Höhere Fehlerwahrscheinlichkeit $\rightarrow$ Bessere Reproduzierbarkeit $\rightarrow$ schnelleres Finden von Fehlern $\rightarrow$ Erfahrungsgewinn
    \item Begrenztes Risiko, mittlerer Zeitaufwand
  \end{itemize*}

  \newpage
  \section{Vorgehensmodelle}
  \subsection{Einführung}
  Wie läuft Softwareerstellung ab?
  \begin{itemize*}
    \item (oder besser, wie sollte sie ablaufen?)
    \item Aufgaben und Phasen siehe vorangegangene Kapitel
    \item Wann wird was getan? Abhängigkeiten?
    \begin{itemize*}
      \item Sequentiell / nebenläufig,
    \end{itemize*}
    \item Prozessmodelle der Softwareentwicklung
    \begin{itemize*}
      \item Regelwerke, Erfahrungen, best practices für große Projekte
      \item Aktives Entwicklungsgebiet
    \end{itemize*}
    \item Erweiterbar zum Software-Lebenszyklus mit Inbetriebnahme, Wartung, Außerdienststellung usw.
  \end{itemize*}

  Softwareentwicklungsprozess/Vorgehensmodell
  \begin{itemize*}
    \item Methode zur Erstellung von Softwaresystemen
    \item Systematisch, rational und schrittweise erfolgender Weg vom Problem zur Lösung
    \item Ziel: Softwareentwicklungsprozess übersichtlich, plan- und strukturierbar
    \item Zerlegung des Softwareentwicklungsprozesses in überschaubare Einheiten
    \item Unternehmensspezifisch, anpassbar, erweiterbar
    \begin{itemize*}
      \item Eigene Varianten, evtl. projektabhängig
      \item An spezielle Bedürfnisse des Informationsbereichs angepasst
      \item Kein allgemeingültiges Vorgehen
      \item Einsetzbar in verschiedenartigen Projekten
    \end{itemize*}
  \end{itemize*}

  Phasen
  \begin{itemize*}
    \item Überwiegend zeitliche Zerlegung
    \item Zeitlich begrenzte Phasen
    \item Auch inhaltlich und organisatorisch begrenzte Phasen möglich
    \item Teilprozesse / Aktivitäten
    \begin{itemize*}
      \item Inhaltliche Zerlegung
      \item Satz von Aufgaben
      \item Verteilung der Teilprozesse / Aktivitäten auf verschiedene Phasen
      \item Begleitet von unterstützenden, übergreifenden Aktivitäten
    \end{itemize*}
  \end{itemize*}

  Aufgabe
  \begin{itemize*}
    \item Erzeugt Arbeitsergebnis (Artefakt)
    \item Verbraucht Ressourcen (z.B. Arbeitskraft, Zeit, Ausrüstung)
  \end{itemize*}

  Arbeitsergebnis (oder Artefakt)
  \begin{itemize*}
    \item Dokument, Modell, System, Programmcode Lastenheft, Spezifikation, Glossar, Handbuch usw.
    \item Intern zu lieferndes Ergebnis
  \end{itemize*}

  Teilnehmer und Rollen
  \begin{itemize*}
    \item Verantwortungsbereich eines Teilnehmers (z.B. Kunde, Projektmanager, Entwickler, Architekt)
    \item Rolle bearbeitet / enthält Satz von Aufgaben
  \end{itemize*}

  Unterstützungsprozesse / -Aktivitäten
  \begin{itemize*}
    \item Projektmanagement
    \begin{itemize*}
      \item Projektplanung, -verfolgung und -steuerung
      \item Risikomanagement
    \end{itemize*}
    \item Anforderungsmanagement
    \begin{itemize*}
      \item Im Gegensatz zur Anforderungsaufnahme und -analyse
    \end{itemize*}
    \item Qualitätsmanagement
    \begin{itemize*}
      \item Problem Management
      \item Softwaremetriken (Messung von Softwareeigenschaften)
      \item Statische + dynamische Analyse (Bestimmung von Schwachstellen)
    \end{itemize*}
    \item Konfigurationsmanagement
    \begin{itemize*}
      \item Versionsverwaltung, Änderungsmanagement
    \end{itemize*}
    \item Dokumentation
  \end{itemize*}

  \subsection{Sequenzielle Modelle}
  \begin{itemize*}
    \item Wasserfallmodell
    \begin{itemize*}
      \item Abhängigkeiten zwischen Teilergebnissen
      \item Ursprung in System- Hardwareentwicklung
      \begin{itemize*}
        \item Wurde für die SW-Entwicklung übernommen
        \item Auch heute verbreitetes Vorgehen bei HW-Entwicklung
      \end{itemize*}
      \item Sequenzielles Phasenmodell (Abschluss der Phasen)
      \item Stark Dokumentengetrieben (Ergebnisse der Phasen)
      \item Unterteilung in abgeschlossene Phasen:
      \begin{itemize*}
        \item Analyse
        \item Design/Entwurf
        \item Implementierung
        \item Test \& Integration
        \item Einführung, Betrieb \& Wartung
      \end{itemize*}
      \item Alternativ:
      \begin{itemize*}
        \item Planung
        \item Definition
        \item Entwurf
        \item Implementierung
        \item Test \& Integration
        \item Einsatz und Wartung
      \end{itemize*}
      \item Vorteile
      \begin{itemize*}
        \item Einfach und verständlich, bekannt und verbreitet
        \item Erleichterte Planung und Steuerung
      \end{itemize*}
      \item Nachteile
      \begin{itemize*}
        \item Idealisierte Annahme rein sequentiellen Ablaufs
        \item Starke Abhängigkeiten zwischen Teilergebnissen
        \item Ungeeignet, falls Anforderungen zu Beginn unklar
        \item Unflexibel gegenüber Änderungen
        \item Erst sehr spät greifbare Ergebnisse
      \end{itemize*}
    \end{itemize*}
    \item Erweitertes Wasserfallmodell
    \begin{itemize*}
      \item Verbesserung für Änderungen und Fehler - Rückschritte
    \end{itemize*}
    \item Alternative Arten von Phasenmodellen
    \begin{itemize*}
      \item Sequenziell
      \begin{itemize*}
        \item Phasen strikt nacheinander, Dokumenten-orientiert
        \item Keine Änderungen abgeschlossener Artefakte
      \end{itemize*}
      \item Nebenläufig
      \begin{itemize*}
        \item Phasen laufen teilweise parallel für bessere Zeitnutzung
        \item Weiterhin keine Änderungen fertiger Dokumente
      \end{itemize*}
      \item Inkrementell
      \begin{itemize*}
        \item Unterteilung des Produkts in Teile
        \item Schnellere Auslieferung von Teilfunktionalität nach vollständiger Aufnahme der Anforderungen
      \end{itemize*}
      \item Alternative: Evolutionär
      \begin{itemize*}
        \item Anforderungen entwickeln sich im Projekt
        \item Ausgelieferte Versionen ergeben neue Anforderungen
      \end{itemize*}
      \item V-Modell [Boehm]
      \begin{itemize*}
        \item Explizite Adressierung der Qualitätssicherung
        %\item %![VModell](Assets/Softwaretechnik1_VModell.png)
        \item Entwicklung des V-Modells in Deutschland
        \begin{itemize*}
          \item Leitfaden, in Bundesbehörden ab 1990er verbindlich
          \item Version V-Modell 97 erweitert um Werkzeuganforderungen, Rollen und Submodelle der Beteiligten
          \item Kritik: schlecht skalier\item und anpassbar, zu unflexibel,
          \item bürokratisch, nicht an moderner OO-SWEntw. orientiert
        \end{itemize*}
        \item V-Modell XT (extreme tailoring)
        \begin{itemize*}
          \item Aktuelle Version ab 2004/05
          \item Einfache projektspezifische Anpassbarkeit
          \item Assistent www.v-modell-xt.de (kleines Projekt ~40 Dok.!)
          \item Überprüfbarer Projektfortschritt
          \item AG/AN-Sichten und Schnittstellen, Ausschreibungen
          \item Gesamter SW-Lebenszyklus
        \end{itemize*}
      \end{itemize*}
    \end{itemize*}
  \end{itemize*}


  \subsection{Iterative Modelle}
  Iterativer Entwicklungsprozess
  \begin{itemize*}
    \item Spezifikation, Entwurf und Implementierung müssen immer wieder verändert und angepasst werden
    \item Häufiges Integrieren, Validieren und Testen
    \item "You should use iterative development only on projects that you want to succeed." [Fowler]
  \end{itemize*}

  Inkrementelle Entwicklung
  \begin{itemize*}
    \item Wenn möglich, sollte es immer einen lauffähigen (unvollständigen) Prototypen geben
    \item Neue Funktionen sofort integrieren
    \item Neue Versionen gegenüber Anforderungen Validieren
  \end{itemize*}

  %![Spiralmodell](Assets/Softwaretechnik1_Spiralmodell.png)
  Spiralmodell
  \begin{itemize*}
    \item Risikogetrieben: Größte Projektrisiken identifizieren und als erstes bearbeiten (Prototyp?)
    \item Spirale = iterativer Zyklus durch dieselben Schritte
    \item Ziele jedes Durchlaufs aus alten Ergebnissen ableiten
    \item Kosten/Nutzen abwägen
    \item Regelmäßige Überprüfung des Prozessmodells
    \item Anpassbar
    \item Hoher Management-Overhead, große Projekte
  \end{itemize*}

  Unified Process
  \begin{itemize*}
    \item Vorgehensmodelle zur objektorientierten Softwareentwicklung
    \begin{itemize*}
      \item Ivar Jacobson, Grady Booch and James Rumbaugh: The Unified Software Development Process, (Rational/IBM), 1999
    \end{itemize*}
    \item Phasen der Entwicklung
    \begin{itemize*}
      \item Anfang, Ausarbeitung, Erstellung, Überleitung
      \item Unterteilung in Iterationen
      \item Definition von Meilensteinen
    \end{itemize*}
    \item Definition von Kernprozessen, die in den Phasen ablaufen
    \begin{itemize*}
      \item Geschäftsprozessmodellierung, Anforderungsanalyse, Analyse \& Design, Implementierung, Test, Auslieferung
      \item In unterschiedlichen Anteilen parallel ablaufend!
    \end{itemize*}
  \end{itemize*}

  Prinzipieller Ablauf des Unified Process
  %![Unified Process](Assets/Softwaretechnik1_UnifiedProcess.png)
  \begin{itemize*}
    \item Haupteigenschaften des UP
    \begin{itemize*}
      \item Inkrementell, iterativ, evolutionär
      \item Anwendungsgetrieben
      \item Architekturzentriert
      \item Risikoorientiert
      \item Ereignisorientiert
      \item Nutzt die UML als Notationssprache
      \item Erweiterbar / Anpassbar
    \end{itemize*}
    \item Verbreitete Vertreter
    \begin{itemize*}
      \item Rational Unified Process - RUP (Rational / IBM)
      \item Open Unified Process - OpenUP (Eclipse Foundation)
      \item Object Engineering Process - OEP (oose GmbH)
    \end{itemize*}
    \item Vorteile
    \begin{itemize*}
      \item Vorteile der Eigenschaften (iterativ, inkrementell, anpassbar, ...)
      \item Berücksichtigung des Risikos
      \item Passend für objektorientiertes Paradigmas und UML
      \item Tool-Unterstützung
      \item Reaktion auf Änderungen möglich
      \item Industriestandard
    \end{itemize*}
    \item Nachteile
    \begin{itemize*}
      \item Hoher Bürokratischer Aufwand $\rightarrow$ Tailoring notwendig!
      \item Relativ hohe „Lernkurve“ aller Beteiligten
      \item Keine schnelle Reaktion auf Änderungen
      \item Keine integrierte Qualitätssicherung
    \end{itemize*}
  \end{itemize*}

  \subsection{Agile Methoden}
  \begin{itemize*}
    \item Ausgangspunkt
    \begin{itemize*}
      \item Ziv's Unsicherheitsprinzip des Software Engineering:  "Unsicherheit ist im Software-Entwicklungsprozess und den Produkten inhärent und unvermeidlich." (Ziv, 1996)
      \item Humphrey's Prinzip der Anforderungsunsicherheit: "In einem neuen Software System werden die Anforderungen solange nicht komplett bekannt sein, bis die Anwender damit arbeiten." (Humphrey, 1995)
      \item Wegner's Lemma: "Es ist unmöglich, ein interaktives System komplett zu spezifizieren." (Wegner, 1995)
    \end{itemize*}
    \item Ziele
    \begin{itemize*}
      \item geringer bürokratischer Aufwand
      \item Hauptziel ist die Softwareentwicklung
      \item nur wenige Regeln bzw. Verhalten definiert
      \item sehr flexibel gehaltenes Vorgehen
      \item stark Anwendungs- und Ereignisorientiert
      \item iterativ / inkrementell / evolutionär
      \item sehr schnelle Entwicklungsiterationen
      \item meist Architekturzentriert
      \item auch testgetriebenes Vorgehen möglich
      \item Berücksichtigung sozialer Aspekte
      \item Softwareentwicklung: Kreative Arbeit von Kreativen
    \end{itemize*}
  \end{itemize*}

  Das Agile Manifest (2001)
  \begin{itemize*}
    \item Individuen und Interaktionen bedeutender als Prozesse und Tools
    \item Funktionierende Software bedeutender als übermäßige Dokumentation
    \item Stetige Zusammenarbeit mit dem Kunden bedeutender als Vertragsverhandlung
    \item Mut und Offenheit für Änderungen bedeutender als Befolgen eines Plans
  \end{itemize*}

  Eigenschaften agiler Vorgehensmodelle
  \begin{itemize*}
    \item Team ist für Ergebnis verantwortlich und organisiert sich selbst
    \item Kleine Teams 5-8 Personen
    \item Definition von Richtlinien, Werten und Prinzipien
    \item Beispiele für Werte
    \begin{itemize*}
      \item Kommunikation (Kommunikation statt Dokumentation)
      \item Einfachheit (KISS „Keep it small and simple“)
      \item Feedback
      \item Mut
      \item Respekt
    \end{itemize*}
    \item Beispiele für Prinzipien
    \begin{itemize*}
      \item Beidseitiger Vorteil
      \item Fehlschläge hinnehmen
      \item Ständige Verbesserungen
      \item Ständige Lauffähigkeit des Codes
      \item Kleine Schritte
      \item Wiederverwendung bestehender / bewährter Lösungen
    \end{itemize*}
    \item Beispiele für Praktiken
    \begin{itemize*}
      \item Pair-Programing, Coding Rules
      \item Kollektives Eigentum / Gemeinsamer Codebesitz
      \item Testgetriebene Entwicklung
      \item Ständiges Refactoring
      \item Keine Überstunden
    \end{itemize*}
    \item Vorteile agiler Methoden
    \begin{itemize*}
      \item Geringer bürokratischer Aufwand
      \item Besseres Arbeitsklima (Berücksichtigung Sozialer Aspekte)
      \item Ständige Verfügbarkeit einer lauffähigen Version
      \item Mögliche / nötige Einflussnahme des Kunden
      \item „Freie“ Wahl der Prinzipien/Regeln
      \item Vermeidung von Spezialistentum und individuellem Besitz
    \end{itemize*}
    \item Nachteile
    \begin{itemize*}
      \item Schwierigeres Projektmanagement
      \begin{itemize*}
        \item Chaotische Vorgehen
        \item Schwere Planbarkeit des Ergebnisses
      \end{itemize*}
      \item Notwendige Beteiligung des Kunden
      \item Ergebnis ist schwer vorherzusagen
    \end{itemize*}
  \end{itemize*}

  \paragraph{eXtreme Programming (XP)}
  \begin{itemize*}
    \item Beck 1999, aus Kritik an „monumentalen Modellen“
    \item Evolutionäre Entwicklung in kleinen Schritten
    \begin{itemize*}
      \item Möglichst einfaches Design
    \end{itemize*}
    \item Konzentration auf Programmcode als Analyseergebnis, Entwurfsdokument und Dokumentation
    \item Weglassen von explizitem Design, ausführlicher Dokumentation und Reviews
    \item Code wird permanent lauffähig gehalten (täglich)
    \item Schnell und flexibel
    \item Erfordert Disziplin der Teilnehmer
    \item Rollen: Projektleiter, Kunde (verfügbar), Entwickler
    \begin{itemize*}
      \item Max. 5-10 Entwickler
    \end{itemize*}
    \item Kunde bestimmt Anforderung und Prioritäten
    \begin{itemize*}
      \item planning game; story cards (use cases)
    \end{itemize*}
    \item Implementierung in kleinen Schritten
    \begin{itemize*}
      \item pair programming, collective code ownership
      \item Häufige Releases inkl. Integration
      \item Refactoring bei Designänderungen
      \item Programmier-Konventionen
    \end{itemize*}
    \item Regelmäßiges automatisiertes Testen
    \begin{itemize*}
      \item test-first Ansatz
    \end{itemize*}
    \item Morgendliches Meeting im Stehen ohne Diskussionen
    \item 40h-Woche
    \item XP
    \begin{itemize*}
      \item Sammlung von 12 "best practices"
      \item Test-getrieben
      \item Flexibel, effizient
      \item Kleine Teams
      \item Erfordert Disziplin der Teilnehmer
      %\item %![XP](Assets/Softwaretechnik1_XP.png)
    \end{itemize*}
  \end{itemize*}

  \paragraph{Scrum}
  \begin{itemize*}
    \item [Ken Schwaber, Jeff Sutherland und Mike Beedle]
    \item Haupteigenschaften
    \begin{itemize*}
      \item Iterativ / Inkrementell, Evolutionär
      \item stark Anwendungs- und Ereignisorientiert
      \item schnelle Entwicklungsiterationen
    \end{itemize*}
    \item Sprint
    \begin{itemize*}
      \item eine schnelle Iteration: Dauer ca. 30 Tage
      \item Festlegung welche Features umgesetzt werden
    \end{itemize*}
    \item Product Backlog
    \begin{itemize*}
      \item Liste der gewünschten Features des Produkts
      \item Vom Produkt-Owner priorisiert / Aufwand vom Team geschätzt
      \item Jeder kann Einträge beisteuern
    \end{itemize*}
    \item Rollen
    \begin{itemize*}
      \item Product Owner
      \begin{itemize*}
        \item Erfasst Bedürfnisse der Kunden und Stakeholder
        \item Pflegt Backlog, definiert, priorisiert Features pro Sprint
      \end{itemize*}
      \item Scrum Master
      \begin{itemize*}
        \item Berät das Team, Überprüft Einhaltung von Werten und Techniken, moderiert die Meetings
        \item Schützt das Team vor äußeren Störungen
        \item Repräsentiert Team gegenüber Management
      \end{itemize*}
      \item Scrum Team (ca. 5-9 Personen)
      \begin{itemize*}
        \item Team organisiert sich und die Aufgaben selbst
        \item Team bedeutet: Zielgerichtet und funktionsübergreifend arbeiten, gemeinsames Ziel verfolgen, selbstloses Handeln, Teamentscheidungen vertreten
      \end{itemize*}
    \end{itemize*}
  \end{itemize*}

  Sprint Backlog
  \begin{itemize*}
    \item Für die aktuelle Iteration ausgewählte Aufgaben
    \begin{itemize*}
      \item Aufgabe nicht länger als 2 Tage Aufwand
    \end{itemize*}
    \item Team-Mitglieder wählen Tasks aus - keine Zuweisung
    \item Restaufwand wird täglich aktualisiert - Burndown Chart
    \item Team-Mitglied kann Tasks hinzufügen, löschen, ändern
    \item Darstellung an prominenter Stelle
  \end{itemize*}

  Daily Meeting
  \begin{itemize*}
    \item ca. 15 Minuten
    \item Kurze Statusmeldung, Was wurde geschafft? Was ist zu tun? Was behindert den Fortschritt?
    \item Weiterführende Diskussionen erst im Anschluss
  \end{itemize*}

  %![Scrum Vorgehen](Assets/Softwaretechnik1_ScrumVorgehen.png)

  Sprint Review-Meeting
  \begin{itemize*}
    \item Präsentation des Erreichten (Feature Demo)
    \item Product Owner, Kunde usw. geben Feedback
    \begin{itemize*}
      \item Neue Anforderungen hinzufügen / Neu priorisieren
      \item Qualitätsansprüche ändern
    \end{itemize*}
  \end{itemize*}

  Sprint-Retrospektive
  \begin{itemize*}
    \item Rückkopplungsschleife
    \begin{itemize*}
      \item Was war gut und was hat nicht funktioniert?
      \item Was kann verbessert werden?
    \end{itemize*}
    \item Nach jedem Sprint
    \item Diskussion der identifizierten Probleme
    \item Identifikation von wenigen „Action Items“
  \end{itemize*}

  Burndown Chart
  \begin{itemize*}
    \item Darstellung der offenen und erledigten Aufwände / Tasks
    %\item %![Burndown Chart](Assets/Softwaretechnik1_Burndownchart.png)
  \end{itemize*}

  \subsection{Zusammenfassung}
  \begin{itemize*}
    \item Software-Entwicklungsmethode
    \begin{itemize*}
      \item Elemente
      \begin{itemize*}
        \item Darstellung - Notation und Semantik für Modelle, Diagrammtypen, Dokumentvorlagen (Artefakte)
        \item Vorgehensmodell - Phasen, Arbeitsschritte
        \item Verfahren - Regeln, Anweisungen, Aktivitäten (+Rollen)
        \item Werkzeuge
      \end{itemize*}
      \item Industriestandards: RUP + UML
      \item Öffentliche Auftraggeber: V-Modell
      \item Firmenintern: eigene Varianten, evtl. projektabhängig
      \item Weitere Themen
      \begin{itemize*}
        \item Reifegradbeurteilung CMMI, SPICE, ISO 9000
      \end{itemize*}
    \end{itemize*}
    \item Charakterisierung von Vorgehensmodellen
    \begin{itemize*}
      \item Sequenziell
      \begin{itemize*}
        \item Teilprozesse strikt nacheinander
        \item Keine Änderungen abgeschlossener Artefakte
      \end{itemize*}
      \item Nebenläufig
      \begin{itemize*}
        \item Teilprozesse laufen teilweise parallel für bessere Zeitnutzung
      \end{itemize*}
      \item Dokumentgetrieben
      \begin{itemize*}
        \item Erstellung von Dokumenten (Artefakte) im Vordergrund
        \item Festlegung der Dokumente pro Phase
      \end{itemize*}
    \end{itemize*}
    \item Charakterisierung von Vorgehensmodellen
    \begin{itemize*}
      \item Iterativ
      \begin{itemize*}
        \item Definition einer sich wiederholenden Abfolge von Teil-Prozessen bzw. Aktivitäten
        \item Schnelles Wiederholen dieser Abfolgen
      \end{itemize*}
      \item Inkrementell
      \begin{itemize*}
        \item Definition und Kontrolle des Fortschritts pro Iteration
        \item Kleine Erweiterungen
      \end{itemize*}
      \item Evolutionäres Vorgehen
      \begin{itemize*}
        \item Schnelle Prototypen
        \item Lauffähiger Prototyp jederzeit vorhanden
        \item Toolunterstützung (Versionierung)
      \end{itemize*}
    \end{itemize*}
    \item Charakterisierung von Vorgehensmodellen
    \begin{itemize*}
      \item Ereignisorientiert
      \begin{itemize*}
        \item Schnelle Reaktion auf Anforderungsänderungen
        \item Keine starre Abfolge von Tätigkeiten / Prozessen
        \item Voraussetzung: Prozesse laufen parallel ab
      \end{itemize*}
      \item Architekturzentriert
      \begin{itemize*}
        \item Starke Gewichtung der Architektur
        \item Verwendung von Modellen, Mustern und vorhandenem Wissen
      \end{itemize*}
      \item Anwendungsgetrieben
      \begin{itemize*}
        \item Orientierung an den Anwendungsfällen
        \item Umsetzen, was einem Anwendungsfall zugeordnet werden kann
        \item Anwender steht im Mittelpunkt (User Stories)
      \end{itemize*}
      \item Risikoorientiert
      \begin{itemize*}
        \item Risiko der Entwicklung wird in Planung berücksichtigt
        \item Risiko- / Nutzen-Analyse
      \end{itemize*}
      \item Test- / Qualitätsgetrieben
      \begin{itemize*}
        \item Qualität steht im Vordergrund
        \item Test wird während oder sogar vor der Implementierungs-phase erstellt
      \end{itemize*}
      \item Erweiterbar / Anpassbar (tailoring)
      \begin{itemize*}
        \item Nur Rahmen des Vorgehens festgelegt
        \item Konkretes Vorgehen wird an die Bedürfnisse angepasst
        \item Grundlegende Eigenschaft von Vorgehensmodellen
      \end{itemize*}
    \end{itemize*}
    \item Softwareprojekt im Sommersemester
    \begin{itemize*}
      \item Auswahl aus 3 Vorgehensmodellen
      \begin{itemize*}
        \item Klassisches Vorgehen
        \item Unified Process
        \item Agiles Vorgehen
      \end{itemize*}
    \end{itemize*}
  \end{itemize*}

  \newpage
  \section{Projektmanagement}

  Was ist ein Projekt?
  \begin{itemize*}
    \item Merkmale von Projekten
    \begin{itemize*}
      \item Zielgerichtetes Vorhaben
      \item Einmaligkeit
      \item Zeitliche, finanzielle und personelle Rahmenbedingungen
      \item Abgrenzung zu anderen Vorhaben
      \item Projektspezifische Organisation
      \item Komplexität (Unterteilung in abhängige Teilaufgaben)
    \end{itemize*}
    \item Unsicherheit vor allem in den Frühphasen
    \item Risiko durch unbekannte Aufgabe
  \end{itemize*}

  Was ist Projektmanagement?
  \begin{itemize*}
    \item Überbegriff für planende und durchsetzende Aktivitäten zur Vorbereitung und Durchführung eines Projekts
    \item Management des Problemlösungsprozesses
    \item Aufgaben
    \begin{itemize*}
      \item Problemabgrenzung
      \item Zielfestlegung, Ablaufplanung
      \item Planung und Bereitstellung personeller, finanzieller und sachlicher Ressourcen
      \item Führen der Projektgruppe und Koordination der Aktivitäten
      \item Steuerung und Überwachung des Projektablaufes
      \item Zum großen Teil Planungs- und Kommunikationsleistung
    \end{itemize*}
  \end{itemize*}

  \subsection{Projektplanung}
  Planung des Projektablaufs
  \begin{itemize*}
    \item Zunächst wieder: Teile und Herrsche!
    \item Projektstruktur
    \begin{itemize*}
      \item Teilung der Projektaufgabe in Arbeitspakete (work packages) und darin enthaltenen Aktivitäten (activities)
      \item Einteilung anhand Produktstruktur, fachlicher Struktur oder Phasenmodell
    \end{itemize*}
    \item Überblick weiterer Planungsaufgaben
    \begin{itemize*}
      \item Bestimmen der Abhängigkeiten
      \item Ermitteln der nötigen Ressourcen
      \item Schätzen der Aufwände
      \item Zeitplan aufstellen
      \item Meilensteine definieren
    \end{itemize*}
  \end{itemize*}
  %![Projektstruktur](Assets/Softwaretechnik1_Projektstruktur.png)

  Ablaufplanung
  \begin{itemize*}
    \item Abhängigkeiten zwischen Vorgängen A und B
    \begin{itemize*}
      \item Ende-Anfang (Normalfolge)
      \begin{itemize*}
        \item B kann begonnen werden, sobald A beendet worden
      \end{itemize*}
      \item Anfang-Anfang (Anfangsfolge)
      \begin{itemize*}
        \item B kann begonnen werden, sobald A begonnen worden
      \end{itemize*}
      \item Anfang-Ende (Sprungfolge)
      \begin{itemize*}
        \item B kann beendet werden, sobald A begonnen worden
      \end{itemize*}
      \item Ende-Ende (Endfolge)
      \begin{itemize*}
        \item B kann beendet werden, sobald A beendet worden
      \end{itemize*}
    \end{itemize*}
    \item Netzplantechnik
    \begin{itemize*}
      \item Planungsarten
      \begin{itemize*}
        \item Vorwärtsplanung (ab Startzeitpunkt)
        \item Rückwärtsplanung (ab gewünschtem Projektende)
      \end{itemize*}
      \item Berechnete Daten für Vorgänge
      \begin{itemize*}
        \item Frühester und spätester Anfangszeitpunkt (FAZ/SAZ)
        \item Frühester und spätester Endzeitpunkt (FEZ/SEZ)
        \item Pufferzeiten, Gesamtpuffer
        \item Notation unterschiedlich
      \end{itemize*}
      \item Allgemein
      \begin{itemize*}
        \item Kritischer Pfad: Verzögerung der Projektdauer
        \item Kritische Vorgänge: Teil des kritischen Pfades
      \end{itemize*}
    \end{itemize*}
  \end{itemize*}

  Aufwandsschätzung
  \begin{itemize*}
    \item Aus Erfahrungswerten systematisiert
    \item Versuch, wichtige Einflussfaktoren zu erfassen
    \begin{itemize*}
      \item Metriken für Spezifikationen
      \item Komplexität von Teilfunktionen
      \item Menge der Funktionen
      \item Anpassung durch individuelle Faktoren
    \end{itemize*}
  \end{itemize*}

  \paragraph{Function Point Analyse}
  \begin{itemize*}
    \item Ursprung IBM Ende 1970er Jahre
    \item Funktionsumfang und Schwierigkeitsgrad von Software
    \item Verschiedene Verfahren
    \item Jeweils Anzahl x Gewicht
    \item Summe aller Werte = Unadjusted Function Points (UFP)
    \item Function Points = UFP x EG
    \item Einflussgrad EG = 1 + 0.01 x SummeEinflussfaktoren
    \item Einflussfaktoren: Einfluss auf Anwendungsentwicklung
    %\item %![FPA2](Assets/Softwaretechnik1_FPA2.png)
    \item Berechnung der Personen-Monate aus Erfahrungen vorangegangener Projekte
    \begin{itemize*}
      \item Aufwand, der von einer Person in einem Monat unter Idealbedingungen erledigt werden kann
    \end{itemize*}
    \item Umrechnung mit Tabelle (nichtlinearer Verlauf)
    %\item %![FPA3](Assets/Softwaretechnik1_FPA3.png)
  \end{itemize*}

  \paragraph{CoCoMo II}
  \begin{itemize*}
    \item Constructive Cost Model [Boehm2000]
    \item Ausgangspunkt: geschätzte Anzahl Zeilen Quellcode
    \begin{itemize*}
      \item SLOC, source lines of code (zB. aus UFP schätzen)
    \end{itemize*}
    \item Aufwand Personen-Monate (PM) $PM = A * Size^E \prod_{i=1}^n EM$ mit $E=B+0,01*\sum_{j=1}^5 SF_j$
    \item Faktoren $A=2.94$ und $B=0.91$ (anpassbare Koeffizienten)
    \item Effort multiplier $EM: n=6..16$, Tabelle nach Boehm
    \item Scale factor SF: Fünf Einflüsse auf Rentabilität der Softwareentwicklung
    \item Notwendige Entwicklungszeit (time to develop) $TDEV = C * PM^F$ mit $F = D + 0,2 * ( E - B )$
    \item Faktoren $C=3,67$ und $D=0,28$, anpassbar
    \item Ressourcenplanung %![Ressourcenplanung](Assets/Softwaretechnik1_Ressourcenplanung.png)
    \item Zeitplanung (Gantt-Chart / Balkendiagramm)
  \end{itemize*}

  \subsection{Projektdurchführung}
  Projektorganisation
  \begin{itemize*}
    \item Teilnehmer: Personen, Rollen, Verantwortung, Teams
    \item Linienorganisation:
    \begin{itemize*}
      \item hierarchisch %![hierarchisch](Assets/Softwaretechnik1_Organisation_hierarchisch.png)
      \item Matrixorganisation %![matrix](Assets/Softwaretechnik1_Organisation_matrix.png)
      \item Reine Projektorganisation: Mitarbeiter werden aus Organisation herausgelöst und Projektleiter unterstellt
    \end{itemize*}
  \end{itemize*}

  Projektmanager - Rolle und Aufgaben
  \begin{itemize*}
    \item Planung, Start, Kontrolle und Beenden des Projekts
    \item Schnittstelle zur Umgebung des Projekts
    \begin{itemize*}
      \item Kunden, Unterauftragnehmer, interne Kontakte, Verträge
    \end{itemize*}
    \item Team zusammenstellen und steuern
    \begin{itemize*}
      \item 5-7 Mitglieder gemischter Qualifikation
      \item Team von äußeren Aufgaben abschirmen
      \item Teilaufgaben definieren, vergeben und koordinieren
      \item Fortschritt kontrollieren und Probleme beseitigen
    \end{itemize*}
    \item Weitere Ressourcen bereitstellen
    \item Notwendige Planänderungen erkennen und reagieren
  \end{itemize*}

  Projektstart
  \begin{itemize*}
    \item Nach Abschluss der Planungsphase
    \item Festlegung von ...
    \begin{itemize*}
      \item Arbeitsstil und interne Organisation
      \item Aufgabenverteilung und Themen-Verantwortung
      \item Erste Aufgaben, Verantwortliche und Termine
      \item Einigung über Meilensteine und Termine
      \item Art und Termine der Projekttreffen
      \item Informationen und Kommunikationswege
      \item Technische Infrastruktur
    \end{itemize*}
    \item Starttreffen des Projekts (kick-off meeting)
  \end{itemize*}

  Meetings / Projekttreffen
  \begin{itemize*}
    \item Regelmäßige Abstimmung der Projektteilnehmer
    \item Inhalt und Ablauf: geplant (Tagesordnung), Moderator
    \begin{itemize*}
      \item Aktueller Stand
      \item Bericht über individuelle Aufgaben
      \item Planung des nächsten Termins
    \end{itemize*}
    \item Protokoll
    \begin{itemize*}
      \item Datum, Zeit, Ort, Teilnehmer, Moderator
      \item Bezeichnung
      \item Tagesordnung mit einzelnen Punkten
      \item Kurz und knapp, neutral bei Diskussionen
    \end{itemize*}
  \end{itemize*}

  Fortschrittskontrolle
  \begin{itemize*}
    \item Meilensteine: Klar definiertes Zwischenresultat zur Beurteilung des Projektfortschritts
    \item Besprechung in Projekttreffen
    \begin{itemize*}
      \item Besprechung des Status jedes Meilensteins / jeder Aufgabe
      \item Welche Probleme sind aufgetreten / gelöst?
      \item Verbleibender Aufwand - Terminverschiebung nötig?
      \item Planung der nächsten Schritte (Aufgabe, Termin)
    \end{itemize*}
  \end{itemize*}

  Meilenstein-Trendanalyse
  \begin{itemize*}
    \item Technik zur Fortschrittskontrolle
    \item Überwachung des Projektfortschritts zur Erkennung von Terminverzögerungen
    \item Bei Verzögerungen:
    \begin{itemize*}
      \item Ressourcen erhöhen
      \item Termine verschieben
      \item Funktionen reduzieren
    \end{itemize*}
  \end{itemize*}

  Wie viel Planung?
  \begin{itemize*}
    \item Aufwand und Detaillierungsgrad der Planung an Projektgröße und „echten“ Aufwand anpassen
    \item Pläne müssen sich ändern können!
    \item Projekte sind einmalig und daher unvorhersehbar
    \item Adaptiv planen: nächste Aufgaben genau, spätere grob
    \item Einsatz von Projektmanagement-Software
    \item Projektende
    \begin{itemize*}
      \item Abschlusstreffen
      \item Bewertung von Ergebnis und Organisation
    \end{itemize*}
  \end{itemize*}

\end{multicols}

\end{document}