\documentclass[a4paper]{article}
\usepackage[ngerman]{babel}
\usepackage[utf8]{inputenc}
\usepackage{multicol}
\usepackage{calc}
\usepackage{ifthen}
\usepackage[landscape,left=1cm,top=1cm,right=1cm,nohead,nofoot]{geometry}
\usepackage{amsmath,amsthm,amsfonts,amssymb}
\usepackage{color,graphicx,overpic}
\usepackage[compact]{titlesec} %less space for headers
\usepackage{mdwlist} %less space for lists
\usepackage[hidelinks,pdfencoding=auto]{hyperref}

\pdfinfo{
    /Title (Stochastik - Cheatsheet)
    /Creator (TeX)
    /Producer (pdfTeX 1.40.0)
    /Author (Robert Jeutter)
    /Subject ()
}

% This sets page margins to .5 inch if using letter paper, and to 1cm
% if using A4 paper. (This probably isn't strictly necessary.)
% If using another size paper, use default 1cm margins.
\ifthenelse{\lengthtest { \paperwidth = 11in}}
    { \geometry{top=.5in,left=.5in,right=.5in,bottom=.5in} }
    {\ifthenelse{ \lengthtest{ \paperwidth = 297mm}}
        {\geometry{top=1cm,left=1cm,right=1cm,bottom=1cm} }
        {\geometry{top=1cm,left=1cm,right=1cm,bottom=1cm} }
    }

% Redefine section commands to use less space
\makeatletter
\renewcommand{\section}{\@startsection{section}{1}{0mm}%
                                {-1ex plus -.5ex minus -.2ex}%
                                {0.5ex plus .2ex}%x
                                {\normalfont\large\bfseries}}
\renewcommand{\subsection}{\@startsection{subsection}{2}{0mm}%
                                {-1explus -.5ex minus -.2ex}%
                                {0.5ex plus .2ex}%
                                {\normalfont\normalsize\bfseries}}
\renewcommand{\subsubsection}{\@startsection{subsubsection}{3}{0mm}%
                                {-1ex plus -.5ex minus -.2ex}%
                                {1ex plus .2ex}%
                                {\normalfont\small\bfseries}}
\makeatother

% Don't print section numbers
\setcounter{secnumdepth}{0}

\setlength{\parindent}{0pt}
\setlength{\parskip}{0pt plus 0.5ex}    
% compress space
\setlength\abovedisplayskip{0pt}
\setlength{\parskip}{0pt}
\setlength{\parsep}{0pt}
\setlength{\topskip}{0pt}
\setlength{\topsep}{0pt}
\setlength{\partopsep}{0pt}
\linespread{0.5}
\titlespacing{\section}{0pt}{*0}{*0}
\titlespacing{\subsection}{0pt}{*0}{*0}
\titlespacing{\subsubsection}{0pt}{*0}{*0}

% Turn off header and footer
\pagestyle{empty}

\begin{document}
\raggedright
\begin{multicols}{3}\scriptsize
  % multicol parameters
  % These lengths are set only within the two main columns
  %\setlength{\columnseprule}{0.25pt}
  \setlength{\premulticols}{1pt}
  \setlength{\postmulticols}{1pt}
  \setlength{\multicolsep}{1pt}
  \setlength{\columnsep}{2pt}

  \subsection{Wahrscheinlichkeitsraum $(\Omega , P)$}
  \begin{itemize*}
    \item Ergebnis-/Grundraum $\Omega$, Menge aller Elementarereignisse
    \item Ergebnis/Ausgang $\omega \in \Omega$
    \item Ereignis $A \subseteq \Omega$
    \item Ereignisraum, die Menge aller Ereignisse, $P(\Omega)$
    \item Wahrscheinlichkeit, dass A eintritt: $\Omega \supseteq A \rightarrow P(A) \in [0,1]$
    \item $\sigma$-Additivität: $P(U_{k\in N} A_k)= \sum_{k\in N} P(A_k)$
  \end{itemize*}

  \subsection{Ereignisalgebra}
  \begin{itemize*}
    \item mit $A=\{1,2\}, B=\{2,3\}$
    \item Vereinigung: $A\cup B=\{1,2,3\}$
    \item Durchschnitt: $A\wedge B=\{2\}$
    \item Gegenereignis: $\Omega=\{1,2,3,4\}, \bar{A}=\{3,4\}$
    \item Differenz $A\backslash B=\{1\}$
    \item Symmetrische Differenz $A\cup B=\{1,3\}$
    \item disjunkte (unvereinbar) Ereignisse $A\cap B = \varnothing$
  \end{itemize*}

  \subsection{Rechengesetze}
  \begin{itemize*}
    \item Kommutativ $A\cup B = B\cup A$
    \item Assoziativ $(A\cup B)\cup C = A\cup(B\cup C)$
    \item Distributiv $A\cap(B\cup C)=(A\cap B)\cup(A\cap C)$
    \item Absorption $A\cap(A\cup B)=A$
    \item Idempotenz $A\cap A=A$
    \item De-Morgan-Gesetz $\bar{A}\cap\bar{B}=\overline{A\cup B}$
    \item Neutrale Elemente $A\cap \Omega=A$
    \item Dominante Elemente $A\cap \varnothing = \varnothing$
    \item Komplemente
    \begin{itemize*}
      \item $A\cap \bar{A} = \varnothing$
      \item $A\cup \bar{A} = \Omega$
      \item $\bar{\bar{A}} = A$
    \end{itemize*}
  \end{itemize*}

  \subsection{Vierfeldertafel}
  Alle vier Felder zusammen entsprechen dem Ergebnisraum $\Omega$
  \begin{center}
    \begin{tabular}{c | c c | c}
      $\Omega$  & $B$             & $\bar{B}$            &   \\\hline
      $A$       & $A\cap B$       & $A\cap \bar{B}$      &   \\
      $\bar{A}$ & $\bar{A}\cap B$ & $\bar{A}\cap\bar{B}$ &   \\\hline
                &                 &                      & 1 \\
    \end{tabular}
  \end{center}

  \subsection{Absolute Häufigkeit}
  wie oft das Ereignis E innerhalb eines Zufallsexperiments, welches n-mal ausgeführt wird, aufgetreten ist.
  Die Summe der absoluten Häufigkeiten ergibt n. Bsp $H_{20}(Kopf)=8$

  \subsection{Relative Häufigkeit}
  Tritt ein Ereignis $E$ bei $n$ Versuchen $k$-mal ein, so heißt die Zahl $h_n(E)=\frac{k}{n}=\frac{H_n(E)}{n}$,
  Bsp: $h_{20}(Kopf)=\frac{8}{20}=0,4$

  \begin{itemize*}
    \item die relative Häufigkeit nimmt Werte zwischen 0 und 1 an
    \item die relative Häufigkeit des sicheren Ereignisses ist 1 $h_n(\Omega)=1$
    \item die relative Häufigkeit des unmöglichen Ereignisses ist 0
    \item $h_n(\bar{E})=1-h_n(E)$
    \item $h_n(A\cup B)= h_n(A)+h_n(B)-h_n(A\cap B)$
    \item $H_n(E)=h_n(E)*n$
  \end{itemize*}

  \subsection{Baumdiagramm}
  \begin{enumerate*}
    \item (UND) Die Wahrscheinlichkeit eines Elementarereignisses ist gleich dem Produkt der Wahrscheinlichkeiten des zugehörigen Pfades. Bsp: $P(\{SS\})=\frac{1}{2}*\frac{1}{2}$
    \item (ODER) Die Wahrscheinlichkeit eines Ereignisses ist gleich der Summe der Wahrscheinlichkeiten aller Pfade, die zu diesem Ereignis führen. Bsp: $P(\{SW, WS\})=\frac{1}{2}*\frac{1}{3} + \frac{1}{3}*\frac{1}{2}$
  \end{enumerate*}

  \subsection{Kombinatorik}
  \begin{tabular}{l | c c c c}
    Kombinatorik & Wdh  &                                            & Menge   & Reihenfolge \\\hline
    Permutation  & ohne & $n!$                                       & n aus n & beachtet    \\
    Permutation  & mit  & $\frac{n!}{k!}$,$\frac{n!}{k_1!*k_2!*...}$ & n aus n & beachtet    \\
    Variation    & ohne & $\frac{n!}{(n-k)!}$                        & k aus n & beachtet    \\
    Variation    & mit  & $n^k$                                      & k aus n & beachtet    \\
    Kombination  & ohne & $\binom{n}{k}$                             & k aus n & nein        \\
    Kombination  & mit  & $\binom{n+k-1}{k}$                         & k aus n & nein
  \end{tabular}

  \subsection{Laplace Expriment}
  alle Elementarereignisse gleiche Wahrscheinlichkeit $P(E)=\frac{|E|}{|\Omega|}$ \newline
  $\Omega$ endlich; $P(\omega)=\frac{1}{\Omega} \rightarrow$ Laplace-Verteilung/diskrete Gleichverteilung 
  $$P(A)=\sum_{\omega \in A} P(\omega)=\frac{*A}{*\Omega}=\frac{\text{Anzahl günstige Ausgänge}}{\text{Anzahl alle Ausgänge}}$$
  Satz von de Moivre-Laplace: Für eine binomialverteilte Zufallsgröße X gilt $P(a\leq X \leq b)= \int_{a-0,5}^{b+0,5} \varphi_{\mu_i \delta} (x) dx_i$ wobei $\mu = n*p$ und $\delta = \sqrt{n*p*(1-p)}$ ist.

  \subsection{Stochastische Unabhängigkeit}
  Ereignisse sind stochastisch unabhängig, wenn das Eintreten eines Ereignisses das Eintreten des anderen Ereignisses nicht beeinflusst. Bsp:
  \begin{itemize*}
    \item Ziehen mit Zurücklegen (unabhängig)
    \item Ziehen ohne Zurücklegen (abhängig)
  \end{itemize*}
  also unabhängig, wenn gilt: $P(A \cap B)=P(A)*P(B)$.

  Bei stochastischer Unabhängigkeit zweier Ereignisse ist die Wahrscheinlichkeit eines Feldes in der Vierfeldertafel gleich dem Produkt der Wahrscheinlichkeiten der zugehörigen Zeile und zugehörigen Spalte.

  \subsection{Multiplikationssatz}
  Die Wahrscheinlichkeit eines Elementarereignisses ist gleich dem Produkt der Wahrscheinlichkeiten des zugehörigen Pfades.
  Bsp: $P(A\cap B)= P(B)*P_B(A)$

  \subsection{Bedingte Wahrscheinlichkeiten}
  $P_B(A)=P(A|B)=\frac{P(A \cap B)}{P(B)}$ ist die Wahrscheinlichkeit von A unter der Bedingung B.

  \subsection{Totale Wahrscheinlichkeit}
  Die Wahrscheinlichkeit eines Ereignisses ist gleich der Summe der Wahrscheinlichkeiten aller Pfade, die zu diesem Ereignis führen. 
  Bsp: $P(A)=\sum_{i=1}^n P(A|B_i)P(B_i)$ $P(A) = P(A\cap B) + P(A\cap \bar{B}) = P(B)*P_B(A)+P(\bar{B})*P_{\bar{B}}(A)$

  \subsection{Satz von Bayes} 
  Umkehren von Schlussfolgerungen $P_B(A)=\frac{P(A)*P_A(B)}{P(B)}$

  \subsection{Diskrete Zufallsvariable}
  wenn sie nur endlich viele oder abzählbar unendlich viele Werte annimmt; meist durch einen Zählvorgang.
  \begin{itemize*}
    \item Erwartungswert :$\mu_x =E(X)=\sum_i x_i*P(X=x_i)$\\
    \item Varianz: $\omega^2_X = Var(X) = \sum_i(x_i-\mu_X)^2 *P(X=x_i)$\\
    \item Standardabweichung: $\omega_X = \sqrt{Var(x)}$
  \end{itemize*}

  \subsection{Stetige Zufallsvariable}
  wenn sie überabzählbar unendlich viele Werte annimmt; meist durch einen Messvorgang.
  \begin{itemize*}
    \item Erwartungswert: $\mu_X= E(X)=\int_{-\infty}^{\infty} x*f(x)dx$\\
    \item Varianz: $\omega_X^2 =Var(X) = \int_{-\infty}^{\infty} (x-\mu_X)^2 *f(x)dx$\\
    \item Standardabweichung: $\omega_X= \sqrt{Var(X)}$
  \end{itemize*}

  \subsection{Wahrscheinlichkeitsverteilung}
  Eine Wahrscheinlichkeitsverteilung gibt an, wie sich die Wahrscheinlichkeiten auf die möglichen Werte einer Zufallsvariablen verteilen.
  Eine Wahrscheinlichkeitsverteilung lässt sich entweder
  \begin{itemize*}
    \item durch die Verteilungsfunktion oder
    \item die Wahrscheinlichkeitsfunktion (bei diskreten Zufallsvariablen)
    \item bzw. die Dichtefunktion (bei stetigen Zufallsvariablen)
  \end{itemize*}
  vollständig beschreiben.

  \subsection{Wahrscheinlichkeitsfunktion}
  Eine Funktion f, die jedem x einer Zufallsvariablen X genau ein p aus [0;1] zuordnet, heißt Wahrscheinlichkeitsfunktion. Kurz: $f:x\rightarrow p$

  $$f(x)=P(X=x)= \begin{cases} p_i \quad\text{für } x=x_i (i=1,2,...,n) \\ 0 \quad\text{ sonst} \end{cases}$$
  Für die Summe der Wahrscheinlichkeiten gilt $\sum_{i=1}^n p_i=1$

  \subsection{Dichtefunktion}
  zur Beschreibung einer stetigen Wahrscheinlichkeitsverteilung
  \begin{itemize*}
    \item kann nur positive Werte annehmen. $f(x) \geq 0$
    \item Fläche unter der Dichtefunktion hat den Inhalt 1
  \end{itemize*}
  Die Verteilungsfunktion ergibt sich durch Integration der Dichtefunktion:
  $F(X)=P(X\leq x)=\int_{-\infty}^x f(u)du$

  \subsection{Verteilungsfunktion}
  Eine Funktion F, die jedem x einer Zufallsvariablen X genau eine Wahrscheinlichkeit $P(X\leq x)$ zuordnet, heißt Verteilungsfunktion: $F:x \rightarrow P(X\leq x$). $P(X\leq x)$ gibt die Wahrscheinlichkeit dafür an, dass die Zufallsvariable X höchstens den Wert x annimmt.
  \begin{itemize*}
    \item die Verteilungsfunktion F ist eine Treppenfunktion
    \item $F(x)$ ist monoton steigend
    \item $F(x)$ ist rechtssteitig stetig
    \item $lim_{x\rightarrow -\infty} F(x)=0$ und $lim_{x\rightarrow +\infty} F(x) =1$
  \end{itemize*}

  \subsection{Diskrete Verteilungsfunktionen}
  \begin{itemize*}
    \item $P(X\leq a)=F(a)$
    \item $P(X<a)= F(a)-P(X=a)$
    \item $P(X>a)= 1-F(a)$
    \item $P(X\geq a)=1-F(a)+P(X=a)$
    \item $P(a<X\leq b)=F(b)-F(a)$
    \item $P(a\leq X \leq b)=F(b)-F(a)+P(X=a)$
    \item $P(a<X<b) = F(b)-F(a)-P(X=b)$
    \item $P(a\leq X < b)=F(b)-F(a)+P(X=a)-P(X=b)$
  \end{itemize*}

  \subsection{Stetige Verteilungsfunktion}
  \begin{itemize*}
    \item $P(X=x)=0$
    \item $P(X\leq a)=F(a)$
    \item $P(a<X\leq b)=F(b)-F(a)$
    \item $P(X>a)=1-F(a)$
  \end{itemize*}

  \begin{tabular}{c c}
    diskret                      & stetig                   \\\hline
    Binominalverteilung          & Normalverteilung         \\
    Hypergeometrische Verteilung & Stetige Gleichverteilung \\
    Poisson Verteilung           & Exponentialverteilung
  \end{tabular}

  \subsection{Erwartungswert}
  zentrale Lage einer Verteilung
  \begin{itemize*}
    \item diskret: $\mu_x = E(X)=\sum_i x_i*P(X=x_i)$
    \item stetig: $\mu_x=E(X)=\int_{-\infty}^{\infty} x*f(x) dx$
  \end{itemize*}

  \subsection{Varianz}
  erwartete quadratische Abweichung vom Erwartungswert.
  \begin{itemize*}
    \item diskret: $\delta_x^2 = Var(X) = \sum_i (x_i-\mu_x)^2*P(X=x_i)$
    \item stetig: $\delta_x^2 = Var(X)= \int_{-\infty}^{\infty} (x-\mu x)^2*f(x) dx$
  \end{itemize*}
  Verschiebungssatz: $Var(X)=E(X^2)-(E(X))^2$

  \subsection{Standardabweichung}
  erwartete Abweichung vom Erwartungswert $\delta_x = \sqrt{Var(X)}$

  \section{Deskriptive Statistik}
  Die Menge aller Elemente, auf die ein Untersuchungsziel in der Statistik gerichtet ist, heißt Grundgesamtheit. Eine Datenerhebung der Grundgesamtheit nennt man Vollerhebung, wohingegen man eine Datenerhebung einer Stichprobe als Stichprobenerhebung bezeichnet. Die in einer Stichprobe beobachteten Werte heißen Stichprobenwerte oder Beobachtungswerte.

  \subsection{Merkmale}
  Eigenschaften, die bei einer Datenerhebung untersucht werden
  \begin{itemize*}
    \item Qualitative Merkmale lassen sich artmäßig erfassen
    \begin{itemize*}
      \item nominale Merkmale (Bsp. Geschlecht): Einzelne Ausprägungen des Merkmals lassen sich feststellen und willkürlich nebeneinander aufreihen. Es lässt sich keine Aussage über eine Reihenfolge oder über Abstände einzelner Ausprägungen machen.
      \item ordinale Merkmale (Bsp. Schulnoten): Einzelne Merkmale lassen sich zwar nicht im üblichen Sinne messen, wohl aber in eine Reihenfolge bringen. Eine Aussage über den Abstand der Ränge lässt sich dagegen nicht machen.
    \end{itemize*}
    \item Quantitative Merkmale lassen sich zahlenmäßig erfassen
    \begin{itemize*}
      \item diskrete Merkmale (Bsp. Schülerzahl): Es gibt nur bestimmte Ausprägungen, die sich abzählen lassen. Die Merkmalsausprägungen diskreter Merkmale sind also ganze, meist nichtnegative Zahlen.
      \item stetige Merkmale (Bsp. Gewicht): Einzelne Ausprägungen eines Merkmals können jeden beliebigen Wert innerhalb eines gewissen Intervalls annehmen.
    \end{itemize*}
  \end{itemize*}

  \subsection{Lageparamter}
  alle statistischen Maßzahlen zusammengefasst, die eine Aussage über die Lage einer Verteilung machen
  \begin{itemize}
    \item Arithmetisches Mittel $x=\frac{x_1+x_2+\dots+x_n}{n}=\frac{1}{n}*\sum_{i=1}^n x_i$
    \item Geometrisches Mittel $\bar{x}_{geom} = \sqrt[n]{x_1*x_2*\dots*x_n}$
    \item Harmonisches Mittel $\bar{x}_{harm} = \frac{n}{\frac{1}{x_1}+\dots+\frac{1}{x_n}}$
    \item Median: Wert, welcher größer oder gleich 50\% aller Werte ist
    \item Modus: $\bar{x}_d=$ Häufigster Beobachtungswert
  \end{itemize}

  \subsection{Streuungsparameter}
  alle statistischen Maßzahlen zusammengefasst, die eine Aussage über die Verteilung von einzelnen Werten um den Mittelwert machen
  \begin{itemize*}
    \item Spannweite: $R=x_{max}-x_{min}$
    \item Interquartilsabstand: $IQR=Q_{0,75}-Q_{0,25}$
    \item Mittlere absolute Abweichung: $D=\frac{1}{n} * \sum_{i=1}^{n} \|x_i-\bar{x}\|$
    \item $Q_{0,75}$ entspricht dem Wert, welcher $\geq 75\%$ aller Werte ist
    \item $Q_{0,25}$ entspricht dem Wert, welcher $\geq 25\%$ aller Werte ist
  \end{itemize*}

  \section{Schätzer}
  Zusammenfassung gesammelter Stichprobe mit einer bestimmten Formel.
  Als Beispiele können wir die Schätzfunktionen für den Anteilswert p betrachten - der Schätzer wird dann meist $\hat{p}$ („p-Dach“) genannt: $\hat{p}=\frac{\sum_{i=1}^n x_i}{n}$

  Beispiel Schätzer für Variant $\sigma^2$ in der Grundgesamtheit: $\hat{\sigma}^2=\frac{1}{n-1}\sum_{i=1}^n (x_i - \bar{x})^2$

  \subsection{Schätzfunktionen für den Mittelwert}
  Der Erwartungswert $\mu$ wird in der Regel mit dem arithmetischen Mittel der Stichprobe geschätzt:
  \begin{itemize*}
    \item Schätzfunktion $\bar{X}=\frac{1}{n}\sum_{i=1}^n X_i$
    \item Schätzwert $\hat{\mu}=\bar{x}=\frac{1}{n}\sum_{i=1}^n x_i$
  \end{itemize*}
  Ist die Verteilung symmetrisch, kann auch der Median der Stichprobe als Schätzwert für den Erwartungswert verwendet werden.

  \subsection{Schätzfunktionen für die Varianz}
  \begin{itemize*}
    \item Schätzfunktion $S_n^2= \frac{1}{n-1} \sum_{i=1}^n (X_i-\bar{X})^2$
    \item Schätzwert $\hat{\sigma}^2=s_n^2=\frac{1}{n-1}\sum_{i=1}^n (x_i-\bar{x})^2$
  \end{itemize*}

  \subsection{Schätzfunktionen für den Anteilswert}
  \begin{itemize*}
    \item Schätzfunktion $\prod=\frac{X}{n}=\frac{1}{n}\sum_{i=1}^n X_i$
    \item Schätzwert $\pi^2=\frac{1}{n}\sum_{i=1}^n x_i$
  \end{itemize*}

  \subsection{Gütekriterien}
  Eine erwartungstreue Schätzfunktion ist im Mittel gleich dem wahren Parameter $\gamma$: $E(g_n)=\gamma$.

  Verzerrung eines Schätzers $Bias(g_n)=E(g_n)-\gamma = E(g_n - \gamma)$

  Mittl. quad. Fehler $MSE(g_n)=E[(g_n-\gamma)^2]=(Bias(g_n))^2 + Var(g_n)$

\end{multicols}

\begin{tabular}{r | l | l | l | l }
                               & Dichtefunktion                                                               & Verteilungsfunktion                                                                               & Erwartungswert                                      & Varianz                                          \\\hline
  Normalverteilung             & $f(x)=\frac{1}{\sigma*\sqrt{2\pi}}*e^{-\frac{1}{2}(\frac{x-\mu}{\sigma})^2}$ & $F(x)=\frac{1}{1-\sigma*\sqrt{2\pi}}\int_{-\infty}^{x}e^{-\frac{1}{2}(\frac{u-\mu}{\sigma})^2}du$ & $E(Y)=\mu$                                          & $Var(Y)=\sigma^2$                                \\
  Stetige Verteilung           & $f(x)=\begin{cases}0 \quad\text{ für } x<a \\ \frac{1}{b-a} \quad\text{ für } a\leq x \leq b \\ 0 \quad\text{ für } x>b \end{cases}$                                            & $F(x)=\begin{cases} 0 \quad\text{ für } x\leq a \\ \frac{x-a}{b-a} \quad\text{ für } a< x < b \\ 1 \quad\text{ für } x\geq b\end{cases}$                                                                 & $E(X)=\frac{a+b}{2}$                                & $Var(X)=\frac{1}{12}(b-a)^2$                     \\
  Exponentialverteilung        & $f(x)=\begin{cases}0 \quad\text{ f+r } x<0 \\ \frac{1}{\mu}e^{-\frac{x}{\mu}} \quad\text{ für } x\geq 0 \end{cases}$                                            & $F(x)=\begin{cases} 0 \quad\text{ für } x<0 \\ 1-e^{-\frac{x}{\mu}} \quad\text{ für } x\geq 0 \end{cases}$                                                                 & $E(X)=\frac{1}{\lambda}$                            & -                                                \\
  Binomialverteilung           & $f(x)=\binom{n}{x}p^x(1-p)^{n-x}$                                            & -                                                                                                 & $E(X)=np$                                           & $Var(X)=np(1-p)$                                 \\
  Geometrische Verteilung      & $f(x)=(1-p)^{x-1}*p$                                                         & -                                                                                                 & $E(X)=\frac{1}{p}$ bzw. $E(Y)=E(X)-1=\frac{1-p}{p}$ & -                                                \\
  Hypergeometrische Verteilung & $f(x)=\frac{\binom{X}{x}*\binom{W}{w}}{\binom{N}{n}}$                        & -                                                                                                 & -                                                   & -                                                \\
  Poisson-Verteilung           & $f(x)=e^{-\lambda}*\frac{\lambda^x}{x!}$                                     & -                                                                                                 & $E(X)=\lambda$                                      & $Var(X)=\lambda$                                 \\
  empirische Verteilung        & $f(x)=\frac{1}{n}$                                                           & $P=\frac{1}{n} \sum_{i=1}^{n} \sigma_{x_i}$                                                       & $E(X)=\frac{1}{n}_{i=1}^n x_i$                      & $Var(X)=\frac{1}{n}\sum_{i=1}^n (x_i-\bar{x})^2$ \\
  Laplace Verteilung           & $f(x)=\frac{1}{2\sigma}e^{-\frac{\| x-\mu \|}{\sigma}}$                      & -                                                                                                 & -                                                   & -                                                \\
  Dirac Maß                    & -                                                                            & $F(X)=\begin{cases} 1 \quad\text{ falls } x< b \\0 \quad\text{ falls } b \leq x \end{cases}$                                                                 & $E(X)=b$                                            & $Var(X)=0$                                       \\
\end{tabular}


\subsection{Skalenniveaus}
\begin{tabular}{r c c l l}
  Skalen          & diskret & qualitiativ &                                                                       & für                     \\\hline
  Nominalskala    &         & Y           & Klassifikation, Kategorien                                            & Geschlecht, Studiengang \\
  Ordinalskala    &         & Y           & Rangordnung ist definiert                                             & Schulnoten              \\
  Intervallskala  &         &             & Rangordnung und Abstände sind definiert                               & Temperatur              \\
  Verhältnisskala &         &             & Rangordnung, Abstände und natürlicher Nullpunkt definiert             & Gehalt, Gewicht         \\
  Absolutskala    & Y       & Y           & Rangordnung, Abstände, natürlicher Nullpunkt und natürliche Einheiten & Anzahl Fachsemester
\end{tabular}

\end{document}