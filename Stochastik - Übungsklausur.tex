\documentclass[10pt, a4paper]{article}
\usepackage[ngerman]{babel}
\usepackage{multicol}
\usepackage{enumitem,amsmath,amsthm,amsfonts,amssymb}
\usepackage[top=1in, bottom=1in, left=0.75in, right=0.75in]{geometry}
\usepackage{color,graphicx,overpic}
\usepackage{hyperref}
\usepackage{mdwlist} %less space for lists
\usepackage{tikz}
% Turn off header and footer
\pagestyle{empty}
% Don't print section numbers
\setcounter{secnumdepth}{0}

\newlist{todolist}{itemize}{2}
\setlist[todolist]{label=$\square$}

\pdfinfo{
  /Title (Stochastik - Übungsklausur)
  /Creator (TeX)
  /Producer (pdfTeX 1.40.0)
  /Subject ()
}
\title{Stochastik - Übungsklausur}
\author{}
\date{}
\begin{document}
\maketitle
\textbf{Für die Klausur sind keinerlei Hilfsmittel wie Skript, Bücher oder Taschenrechner zulässig. Die Aufgaben sind an Übungen und Vorlesung orientiert. Für Richtigkeit der Lösungen besteht keine Garantie.}

\section{Aufgabe 1: Laplace-Verteilung}
Sie haben zwei Tetraeder zur Verfügung, deren Flächen jeweils mit den Augenzahlen 1, 2, 3 und 4 beschriftet sind. Beide Tetraeder werden geworfen und aus den beiden geworfenen Augenzahlen wird der Absolutbetrag ihrer Differenz, diesen nennen wir $D$, ermittelt.$D$ kann also die Werte 0, 1, 2 und 3 annehmen.

\paragraph{a)} Geben Sie einen Wahrscheinlichkeitsraum für ein geeignetes Laplace-Experiment an und definieren Sie auf diesem Wahrscheinlichkeitsraum $D$ durch Angabe der Abbildungsvorschrift als Zufallsvariable. Vergessen Sie nicht zu begründen, warum es sich bei Ihrem gewählten Experiment um ein Laplace-Experiment handelt. \textbf{(5 Punkte)}
\begin{center}
  \framebox(450,100){}
\end{center}

\paragraph{b)} Berechnen Sie die Verteilung von $D$. Ist $D$ Laplaceverteilt? Begründen Sie Ihre Antwort.  \textbf{(3 Punkte)}
\begin{center}
  \framebox(450,100){}
\end{center}

\paragraph{c)} Berechnen Sie den Erwartungswert von $D$.  \textbf{(2 Punkte)}
\begin{center}
  \framebox(450,100){}
\end{center}


\section{Aufgabe 2: Binomial-Verteilung}
\paragraph{a)} Geben Sie mit Hilfe Bernoulliverteilter Zufallsvariablen $Z_1,Z_2,...$ eine Zufallsvariable $X$ an, welche Binom $(n,p)$-verteilt ist. Welche Voraussetzungen müssen $Z_1,Z_2,...$ erfüllen? Was modelliert Binom $(n,p)$ anschaulich? \textbf{(3 Punkte)}
\begin{center}
  \framebox(450,100){}
\end{center}

\paragraph{b)} Bestimmen Sie basierend auf $X$ den Maximum-Likelihood-Schätzer $\hat{p}$ für $p$.Existiert dieser stets eindeutig? \textbf{(5 Punkte)}
\begin{center}
  \framebox(450,100){}
\end{center}

\paragraph{c)} Berechnen Sie den MSE (mean squared error) von $\hat{p}$. Ist $\hat{p}$ unverzerrt? \textbf{(2 Punkte)}
\begin{center}
  \framebox(450,100){}
\end{center}


\section{Aufgabe 3: Geometrische Verteilung}
\paragraph{a)} Geben Sie die Wahrscheinlichkeitsfunktion zur geometrischen Verteilung mit Parameterpan. Welche Werte darf $p$ annehmen? \textbf{(2 Punkte)}
\begin{center}
  \framebox(450,100){}
\end{center}

\paragraph{b)} Was modelliert die geometrische Verteilung mit Parameter $p$? \textbf{(1 Punkt)}
\begin{center}
  \framebox(450,100){}
\end{center}

\paragraph{c)} Sie beobachten $X_1,X_2,...,X_n$ unabhängig und identisch verteilte Zufallsgrößen, die jeweils eine Geom(p)-Verteilung besitzen. Bestimmen Sie den Momentenschätzer $\hat{p}$ für $p$. \textbf{(3 Punkte)}
\begin{center}
  \framebox(450,100){}
\end{center}

\paragraph{d)} Ist $\hat{p}$ unverzerrt? \textbf{(2 Punkte)}
\begin{center}
  \framebox(450,100){}
\end{center}

\paragraph{e)} Ist $\hat{p}$ konsistent? \textbf{(2 Punkte)}
\begin{center}
  \framebox(450,100){}
\end{center}

\section{Aufgabe 4: Exponentialverteilung}
Gegeben sei eine Zufallsvariable $X \sim Exp(2)$.
\paragraph{a)} Bestimmen Sie den Median und geben Sie den Erwartungswert von $X$ an. Vergleichen Sie die beiden Werte und erklären Sie einen eventuell vorhandenen Unterschied zwischen den beiden Ergebnissen. \textbf{(3 Punkte)}
\begin{center}
  \framebox(450,100){}
\end{center}

Für die weitere Rechnung dürfen Sie ohne Nachweis benutzen, dass eine Zufallsvariable $V\sim Exp(p)$ die Varianz $Var(V) =\frac{1}{p^2}$ besitzt. Seien nun $X_{i,i}\in N$ unabhängig und identisch $Exp(2)$-verteilte Zufallsgrößen und $\bar{X}_n \frac{1}{n}\sum_{i=1}^{n} X_{i,n}\in N$ die zugehörige Folge der Mittelwerte.
\paragraph{b)} Bestimmen Sie $E(\bar{X}_n)$ und $Var(\bar{X}_n)$. \textbf{(2 Punkte)}
\begin{center}
  \framebox(450,100){}
\end{center}

\paragraph{c)} Wie ist $Z_n= \cos(\pi*\bar{X}_n)$ für großes $n$ approximativ verteilt? \textbf{(5 Punkte)}
\begin{center}
  \framebox(450,100){}
\end{center}

\section{Aufgabe 5: Uniforme Verteilung}
Gegeben sei eine Zufallsvariable $X\sim Unif(-0.5,0.5)$.
\paragraph{a)} Geben Sie die Wahrscheinlichkeitsdichte und die Verteilungsfunktion von $X$ an. \textbf{(2 Punkte)}
\begin{center}
  \framebox(450,100){}
\end{center}

\paragraph{b)} Geben Sie den Median, den Erwartungswert und die Varianz von $X$ an. \textbf{(3 Punkte)}
\begin{center}
  \framebox(450,100){}
\end{center}

Berechnungen im Rahmen von Bankgeschäften ergeben oft Ergebnisse mit gebrochenen Centanteilen, welche für Buchungsvorgänge gerundet werden müssen. Zum Beispiel würde man einen Betrag von $7,35...$ Cent auf 7 Cent abrunden und einen Betrag von $15.87...$ Cent auf 16 Cent aufrunden. Den Rundungsfehler kann man als uniform-verteilt auf $(-0.5,0.5)$ (eigentlich $(-0.5,0.5]$) modellieren. Nehmen Sie nun an, dass in einer Bank 106 unabhängig und identisch $Unif(-0.5,0.5)$-verteilte Rundungsvorgänge stattfinden.
\paragraph{c)} Geben Sie mit Hilfe der Ungleichung von Tschebyscheff eine sinnvolle obere Abschätzung für die Wahrscheinlichkeit an, dass der Absolutbetrag der Summe der Rundungsfehler mindestens 10 Euro (1000 Cent) beträgt. \textbf{(2 Punkte)}
\begin{center}
  \framebox(450,100){}
\end{center}

\paragraph{d)} Mit welcher Wahrscheinlichkeit verliert die Bank mindestens einen Euro (100 Cent)?Nutzen Sie den zentralen Grenzwertsatz, um diese Wahrscheinlichkeit geeignet zu approximieren. Runden Sie Ihr Ergebnis mit Hilfe der unten angegebenen Tabelle auf volle 10\%. \textbf{(3 Punkte)}
\begin{center}
  \framebox(450,100){}
\end{center}

Quantile der Standardnormalverteilung
\begin{tabular}{c | c | c | c | c | c | c | c | c | c}
  $\alpha$   & 10\%  & 20\%  & 30\%  & 40\%  & 50\% & 60\%  & 70\%  & 80\%  & 90\%  \\\hline
  $q_{\alpha}$ & $-1,28$ & $-0,84$ & $-0,52$ & $-0,25$ & $0$ & $0,25$ & $0,52$ & $0,84$ & $1,28$
\end{tabular}

\section{Aufgabe 6: Normalverteilung}
Gegeben seien die unabhängigen Zufallsvariablen $X_1,X_2,...,X_n$ mit den Verteilungen $X_i \sim \mathcal{N}(\mu,9)$ für $i= 1,2,...,n$ und einem reellen Parameter $\mu$.
\paragraph{a)} Geben Sie einen erwartungstreuen Schätzer $\hat{\mu}$ für $\mu$ an. \textbf{(1 Punkt)}
\begin{center}
  \framebox(450,100){}
\end{center}

\paragraph{b)} Wie ist $\hat{\mu}$ verteilt? \textbf{(2 Punkte)}
\begin{center}
  \framebox(450,100){}
\end{center}

\paragraph{c)} Geben Sie mit Hilfe von $\hat{\mu}$ und der Faustformel ein 95\%-Konfidenzintervall für $\mu$ an. \textbf{(2 Punkte)}
\begin{center}
  \framebox(450,100){}
\end{center}

\paragraph{d)} Wie würden Sie die Hypothese $\mu= 1$ zum Niveau 5\% testen? \textbf{(2 Punkte)}
\begin{center}
  \framebox(450,100){}
\end{center}

\paragraph{e)} Berechnen Sie $Cov(X_1+X_2,X_2+X_3)$. \textbf{(2 Punkte)}
\begin{center}
  \framebox(450,100){}
\end{center}

\paragraph{f)} Sind $X_1+X_2$ und $X_2+X_3$ stochastisch unabhängig? \textbf{(1 Punkt)}
\begin{center}
  \framebox(450,100){}
\end{center}

\section{Aufgabe 7: Sinus-Verteilung}
\paragraph{a)} Wann ist eine Funktion $f:R\rightarrow R$ eine Wahrscheinlichkeitsdichte? \textbf{(2 Punkte)}
\begin{center}
  \framebox(450,100){}
\end{center}

Gegeben sei $f:R\rightarrow R$ mit $f(x) =a*\sin(x)·1_{(0,\pi)}(x)$, wobei a ein reeller Parameter ist.
\paragraph{b)} Bestimmen Sieaso, dassfeine Wahrscheinlichkeitsdichte ist. \textbf{(1 Punkt)}
\begin{center}
  \framebox(450,100){}
\end{center}

Sei nunXeine Zufallsvariable, die die Wahrscheinlichkeitsdichtefbesitzt.
\paragraph{c)} Berechnen Sie $P(X >\frac{\pi}{2})$. \textbf{(2 Punkte)}
\begin{center}
  \framebox(450,100){}
\end{center}

\paragraph{d)} Berechnen Sie $E X$. \textbf{(2 Punkte)}
\begin{center}
  \framebox(450,100){}
\end{center}

\paragraph{e)} Begründen Sie, dass hier die Markoff-Ungleichung angewendet werden kann und verwenden Sie sie, um $P(X >\frac{\pi}{2})$ abzuschätzen. Wie erklären Sie den Unterschied zu c)? \textbf{(3 Punkte)}
\begin{center}
  \framebox(450,100){}
\end{center}

\section{Aufgabe 8: Deskriptive Statistik}
Aus einer Charge von Fäden werden 5 Stück entnommen, um sie auf Reißfestigkeit zu testen. Notiert werden die erreichten Dehnungslängen $L_i,i= 1,...,5$ in cm zum Zeitpunkt des Reißens. Die Ergebnisse lauten:
\begin{tabular}{c | c | c | c | c}
  $L_1$ & $L_2$ & $L_3$ & $L_4$ & $L_5$ \\
  4   & 11  & 1   & 6   & 3
\end{tabular}

\paragraph{a)} Befinden sich diese Daten auf einer Verhältnisskala? \textbf{(1 Punkt)}
\begin{center}
  \framebox(450,100){}
\end{center}

\paragraph{b)} Skizzieren Sie die empirische Verteilungsfunktion zu diesem Datensatz. \textbf{(1 Punkt)}
\begin{center}
  \framebox(450,100){}
\end{center}

\paragraph{c)} Bestimmen Sie den Mittelwert, den Median, die Quartile und den Interquartilsabstand der Daten. Wie erklärt sich der Unterschied zwischen Median und Mittelwert? \textbf{(4 Punkte)}
\begin{center}
  \framebox(450,100){}
\end{center}

\paragraph{d)} Skizzieren Sie einen Boxplot zu diesem Datensatz. \textbf{(3 Punkte)}
\begin{center}
  \framebox(450,100){}
\end{center}



\end{document}